%%%%%%%%%%%%%%%%%%%%%%%%%%%%%%%%%%%%%%%%%
% kaobook
% LaTeX Template
% Version 1.3 (December 9, 2021)
%
% This template originates from:
% https://www.LaTeXTemplates.com
%
% For the latest template development version and to make contributions:
% https://github.com/fmarotta/kaobook
%
% Authors:
% Federico Marotta (federicomarotta@mail.com)
% Based on the doctoral thesis of Ken Arroyo Ohori (https://3d.bk.tudelft.nl/ken/en)
% and on the Tufte-LaTeX class.
% Modified for LaTeX Templates by Vel (vel@latextemplates.com)
%
% License:
% CC0 1.0 Universal (see included MANIFEST.md file)
%
%%%%%%%%%%%%%%%%%%%%%%%%%%%%%%%%%%%%%%%%%

%---------------
%	宏包及文档配置
%--------
\PassOptionsToPackage{svgnames}{xcolor}
\documentclass[a4paper,fontsize=10pt,open=any,twoside=false,numbers=noenddot,UTF-8,]{kaobook}

% Choose the language
\ifxetexorluatex
\usepackage{polyglossia}
\setmainlanguage{english}
\else
\usepackage[english]{babel} % Load characters and hyphenation
\fi
\usepackage[english=british]{csquotes}	% English quotes
\usepackage{ctex}
% Load packages for testing
\usepackage{tikz-feynman}
\usepackage{fixdif}
\usepackage{ulem}
\usetikzlibrary{tikzmark}
\usetikzlibrary{backgrounds}
\usetikzlibrary{arrows,shapes}
\usetikzlibrary{calc}
% Commands for Highlighting text -- non tikz method
\newcommand{\highlight}[2]{\colorbox{#1!17}{$\displaystyle #2$}}
%\newcommand{\highlight}[2]{\colorbox{#1!17}{$#2$}}
\newcommand{\highlightdark}[2]{\colorbox{#1!47}{$\displaystyle #2$}}
% Commands for Highlighting text -- non tikz method
\renewcommand{\highlight}[2]{\colorbox{#1!17}{#2}}
\renewcommand{\highlightdark}[2]{\colorbox{#1!47}{#2}}
\usepackage{blindtext}
\usepackage{kaobiblio}
\addbibresource{main.bib}
\usepackage[framed=true]{kaotheorems}
\usepackage{kaorefs}

\graphicspath{{examples/documentation/images/}{images/}} % Paths in which to look for images

\makeindex[columns=3, title=按字母排序的索引, intoc] % Create an index

\makeglossaries % Make LaTeX produce the files required to compile the glossary
\input{glossary.tex} % Include the glossary definitions

\makenomenclature % Make LaTeX produce the files required to compile the nomenclature
%%%%%%%%%%%%%%%%%%%%%%%%%%%%%%%%%%%%%%%%%%%%%%%%%%
%新定义的命令!
%%%%%%%%%%%%%%%%%%%%%%%%%%%%%%%%%%%%%%%%%%%%%%%%%%
\newcommand{\used}{\text{此处使用}}
\newcommand{\blue}{\color{blue}}
\newcommand{\red}{\color{red}}
\newcommand\mb[1]{\mathbf{#1}}
\newcommand{\dt}{\d t}
\newcommand{\dx}{\d x}
\newcommand{\dy}{\d y}
\newcommand{\dz}{\d z}
\newcommand{\dq}{\d q}
\newcommand{\da}{\d a}

\newcommand{\ddx}{\d \mathbf{x}}
\newcommand{\ra}{\rangle}
\newcommand{\la}{\langle}
% 重定义\to
\makeatletter
\def\minusfill{\mathord-\!\!
    \cleaders\hbox{$\!\!\mathord-\!\!$}\hfill
    \!\!\mathord-}
\makeatother
\def\longto{\mathop{\minusfill\!\!\!\rightarrow}\limits}
\def\to{\futurelet\next\toAux}
\def\toAux{\ifx\next^%
    \expandafter\longto%
    \else%
    \ifx\next_%
    \expandafter\longto%
    \else%
    \rightarrow%
    \fi%
    \fi%
}












\setmathfont{Latin Modern Math}





%----------------------------------------------------------------------------------------
\renewcommand\proofname{证明}

%----------------------------------------------------------------------------------------
%自定义的一段无意义文字,用以代替没有中文内容的\blindtext命令。
%简单的直接代替,未来有机会再写成标准的函数命令发布
%----------------------------------------------------------------------------------------
\newcommand{\cblindtext}{你好,这里有一些没有意义的文本。这段文本应该展示在这个地方打印的文本会是什么样子。如果你阅读这段文本,你将不会获得任何信息。真的吗?真的没有信息吗?这段文本和一些无意义的东西,比如 \enquote{Huardest 天不},有什么区别吗?Kjift - 完全没有!像这样的盲文给你提供了关于所选字体的信息,字母是如何书写的,以及整体外观的印象。这段文本应该包含字母表中的所有字母,并且应该用原始语言书写。内容不需要特别,但单词的长度应该与语言相匹配。}
\usepackage{qftcover}
\coverset
{
    title    = 凝聚态场论:讲义,
    subtitle = Condensed Matter Field theory : Notes,
    bioinfo  = {当前进度:尚未完成.},
    color    = DarkGreen,
    head     = background/4,
    logo     = cat/sixpointed stars,
}
\begin{document}
    
    \maketitle
    
    %----------------------------------------------------------------------------------
    %	PREFACE
    %----------------------------------------------------------------------------------
    
    \chapter*{前言}
\addcontentsline{toc}{chapter}{Preface} % Add the preface to the table of contents as a chapter



\textbf{2025/4/20编: 更改了模板,彻底调整了之前的一些遗留问题}\\

本书第一章提供了大部分数学内容的概述,主要是使初学者意识到:\textbf{一味的学数学打基础是有害无益的},读者可以参阅后续章节思考究竟实际用到了多少数学内容.对于想要正常学习的读者,建议从第二章开始正文,可以将第一章当做对相应数学领域的粗浅了解,同时,对于真的想要深入学习下去的读者,也提供了相应的推荐书目来参考.

\marginnote[]{无论我怎么强调数学不过是辅助,但总是有人不由自主的去钻牛角尖,于是我在第一章准备了多到离谱的数学内容供这些人阅读,\textbf{如果想要正常思路阅读下去,请从第二章开始.}}
很多初学者在初次自学较高深的物理的时候往往会犯一个错误:被同等级的高深的数学所迷惑,认为学好这些物理离不开这些数学.但事实上,学好物理确实离不开数学,但仅仅是一小部分的数学.例如,学好量子力学离不开线性代数,但经常有很多物理系的学生拐去学习泛函分析,李代数,辛几何之类的内容,而这些数学内容往往只有一小部分应用在物理上面,打着先学完这些数学再开始学习物理的想法,只会让物理的学习一拖再拖.我们应当意识到,数学不过是物理的工具,\textbf{切勿舍本逐末}.

但这并不意味着数学不重要,离开数学的物理不亚于纸上谈兵,毫无意义.而对于本讲义,前面的部分并不十分紧要,仅作为开启凝聚态场论的先备知识,将重要的部分提取出来,作为先导知识.%第三章是为后续的大量使用做铺垫,第四章开始是文章的主体部分,最后的几章为目前在做的研究方向的总结.

前言仅供参考,不代表最终前言内容和书中内容,该前言已经是第四版.




\begin{flushright}
	\textit{530 Group}
\end{flushright}

    \index{preface}
    
    %----------------------------------------------------------------------------------
    %	目录
    %----------------------------------------------------------------------------------
    
    \begingroup % Local scope for the following commands
    \setlength{\textheight}{230\hscale} % Manually adjust the height of the ToC pages
    \etocclasstocstyle % "toc display" as if etoc was not loaded
    \etocstandardlines % "toc lines" as if etoc was not loaded
    \tableofcontents % Output the table of contents
    %\listoffigures % Output the list of figures
    \let\cleardoublepage\bigskip
    \let\clearpage\bigskip
    %\listoftables % Output the list of tables
    \endgroup
    
    %----------------------------------------------------------------------------------
    %	文档主体
    %----------------------------------------------------------------------------------
    
    \mainmatter % Denotes the start of the main document content, resets page numbering and uses arabic numbers
    \setchapterstyle{kao} % Choose the default chapter heading style
    
    \setchapterpreamble[u]{\margintoc}
\chapter{李群(Lie Group)与李代数(Lie Algebra)}
\labch{intro}


\section*{章节导论:对称性与对称性破缺}
仍在画饼,可以参考这篇知乎文章,写的很不错:\href{https://zhuanlan.zhihu.com/p/338221764}{知乎专栏}




\section{李群(Lie Group)初步}
\subsection{群与李群(Lie Group)}
我们或许听说过一个说法:``物理学的关键是对称和守恒",而诺特定理给出了对称性与守恒性的联系,例如,时间平移不变性意味着能量守恒;空间平移不变性意味着动量守恒;转动不变性意味着角动量守恒;电势和向量势的规范不变性得出电荷的守恒等.而描述对称性的语言就是群论.


我们可以认为群是一类拥有特殊结构的集合,即满足如下关系的集合\sidenote[][]{当然,这里会忽略对于主线无用的那些群论内容,所以如果和数学系的抽象代数对比,你甚至可能会感觉到学的不是同一个东西}:
\marginnote[]{这一章虽然名字比较吓人,但并没有太过深入讲解李群与李代数,主要为了帮助物理系学生来快速对这一门学科建立印象.当然,我们不能保证在后续的不断更新中是否会改变这一点,在最初的版本中,这一章被置于第三章,也就是初量的后续部分,以及这一章的进度只会跟进后续章节的需要,也就是说这一章的全部已完成内容足以支撑起后续章节的学习.}
\begin{definition}[群的定义]
    设$G$是一个集合,若满足下面4个条件,则称$G$为一个群(Group)
    \begin{enumerate}
        \item $G$中存在一种运算规则,对$G$中的任意两个元素$g,h\in G$,存在对应$G$中的一个元素,记为
        \begin{equation}
            k=g\circ h(k=gh)
        \end{equation}
        \item 运算规则满足交换律,对$G$中的任意三个元素$g,h,k\in G$,存在
        \begin{equation}
            (gh)k=g(hk)
        \end{equation}
        \item $G$中存在一个幺元$e$(有时也称单位元),使得对于$G$中任意元素$g$,均有
        \begin{equation}
            ge=eg=g
        \end{equation}
        \item $G$中每一个元素$g$,均存在一逆元$g^{-1}$,使得
        \begin{equation}
            gg^{-1}=g^{-1}g=e
        \end{equation}
    \end{enumerate}
\end{definition}
我们可以发现,群的运算规则通常不满足交换律,特殊的,我们把满足交换律的群称为阿贝尔群(Abel Group)\sidenote[][]{关于这个有一个经典笑话:一位美国数学教授来到法国,见路边有一小孩,遂上前问到:“小朋友,你知道1+2等于几吗?”小孩摇摇头说:“不知道.”教授正想感叹法国数学教育如此之落后,却听到小孩接着说:“虽然我不知道1+2等于几,但是我知道1+2等于2+1,因为整数加法群是阿贝尔群!”}.
\begin{definition}[子群的定义]
    设$G$是一个群,$H$为$G$的一个子集($H\subseteq G$),若$H$按照$G$的运算规则仍是一个群,则称$H$是$G$的子群.
\end{definition}
下面我们将给出一系列典型的群的实例,请根据定义思考它们是怎么构成的,并且尝试找到一些共性.
\begin{example}
    全体实数$\mathbb{R}$(或复数$\mathbb{C}$),对加法构成一个阿贝尔群.\\
    我们知道有理数全体是$\mathbb{R}$的子群,而全体偶数是有理数的子群,自然也是$\mathbb{R}$的子群,那么存在一个问题:无理数全体,或奇数全体是否是$\mathbb{R}$的子群?
    \sidenote[][]{都不是,首先对于无理数我们注意到$\pi+(-\pi)=0$,而0不是无理数,故无理数不构成加法群.同样的,我们注意到$1+(-1)=0$,0同样也不是奇数,故奇数也不构成加法群.}
\end{example}
\begin{example}
    全体实数除去零$\mathbb{R}/0$或全体复数除去零$\mathbb{C}/0$对乘法构成阿贝尔群.\\
    同样的,我们有个问题:为什么要除去0?\sidenote{答案是显然的,群中幺元为1,但$0/0$无意义.}
\end{example}
\begin{example}
    $G =\{1,-1,i,-i\}$对复数乘法运算构成一有限阿贝尔群.这里1是$G$的幺元,而-1的逆元就是-1,$i$与$-i$互为逆元.
\end{example}
\begin{example}
    行列式不为零的n阶实矩阵全体对矩阵乘法构成一个群,$n$阶全线性群,其记为$GL(n,R)$,它的元素由$n^2$个独立实参数所确定.其是一个$n^2$维(不可交换)李群,在后面我们会再次讨论它.
\end{example}
\begin{example}
    行列式为 1 的 2 阶实矩阵全体对矩阵乘法构成一个群:二阶(实)特殊线性群$SL(2,R)$.因为二阶实矩阵$\begin{bmatrix}a&b\\c&d\end{bmatrix}$由四个实数$a,b,c,d$ 构成,由于行列式为1的要求,使他们必须满足条件:$ad-bc=1$.所以$SL(2,R)$ 中的元素由3个独立的实参数所确定.按照下面将要给出的定义可见$SL(2,R)$ 是一个三维(不可交换)李群,而且它是$GL(2,R)$的子群.
\end{example}
\begin{example}
    行列式为 1 的 n 阶实矩阵全体对矩阵乘法构成一个群:n阶特殊线性群$SL(n,R)$,这是一个$n^2-1$维(不可交换)李群,而且它是$GL(n,R)$的子群.
\end{example}
\begin{example}
    行列式不为0的 n 阶复矩阵全体对矩阵乘法构成一个群:n阶(复)全线性群$GL(n,C)$,行列式为 1 的 n 阶复矩阵全体对矩阵乘法构成一个群:n阶(复)特殊线性群$SL(n,C)$.\\
    $GL(n,C)$是一个$2n^2$维(不可交换)李群,$SL(n,C)$是一个$2n^2-12$维(不可交换)李群.
\end{example}

我们发现,所举的例子(除第一个外)都存在共同点:元素都是矩阵(实数和复数可看作一阶矩阵),群的运算法则都是矩阵乘法.我们把这类群统称为\textbf{线性群},线性群也是最具代表性的一类李群,今后所使用的李群基本上都是线性群.
\marginnote[]{事实上,从现在开始,我们就已经走上物理的道路上了,实际上,哪怕你掌握了这一章的全部内容,可能对于数学上的抽象代数那一套仍非常陌生,但早已足够应付物理上的内容了.在上一段,我们给出了一个断言:``今后所使用的李群基本上都是线性群",实际上,我们完全可以这么说,如果不去碰高能和那些fancy的理论(例如弦论,Ads/CFT等),哪怕仅掌握$U(1),SU(2),SU(4)$,$SO(2),SO(3)$这几个群和其表示论对于凝聚态学习就远远足够了.}

下面我们正式进入李群这一部分的内容.
\begin{definition}[Lie群的定义]
    设$G$是一个$r$维流形,同时$G$又是一个群,并将其幺元记为$e$,因$e$又是流形$G$中的一点,所以可取定一个包含$e$的局部坐标邻域$U$;在$U$中取定坐标系$\{U,\varphi\}$.设取$e$为坐标原点,有
    \begin{equation}
        \varphi(e)=(0,0,\cdots,0)
    \end{equation}
    任取$U$中的三个元素$g,h,k$,并设其坐标为
    \begin{equation}
        \begin{aligned}
            \varphi(g)&=(x_1,x_2,\cdots,x_r)\\
            \varphi(h)&=(y_1,y_2,\cdots,y_r)\\
            \varphi(k)&=(z_1,z_2,\cdots,z_r)
        \end{aligned}
    \end{equation}
    而群乘法$k=gh$则可以被定义为以下形式:
    \begin{equation}
        \begin{aligned}
            &z_{1}=f_{1}(x_{1},\cdots,x_{r};y_{1},\cdots,y_{r})\\
            &z_{2}=f_{2}(x_{1},\cdots,x_{r};y_{1},\cdots,y_{r})\\
            &z_{r} =f_r(x_1,\cdots,x_r;y_1,\cdots,y_r) 
        \end{aligned}
    \end{equation}
    我们要求这$r$个函数$f_1,f_2,\cdots,f_r$是无限次可导的(光滑的).我们把这$r$个函数$f_1,f_2,\cdots,f_r$称为$G$的\textbf{乘法函数}.其完全确定了群$G$的结构.我们把这样的群$G$叫做一个$r$维李群.
\end{definition}
我们现在根据群的定义来给出几个自然性质
\begin{enumerate}
    \item 第一个定义是显然的,因为李群的定义建立在这种运算规则上,我们只需要对另外3个条件进行讨论.
    \item 我们现在给出交换律所导出的性质,为方便表述,我们简记群乘法关系为$z=f(x,y)$:
    \begin{equation}
        f(f(x,y),z)=f(x,f(y,z))
    \end{equation}
    \item 对于幺元$e$,其坐标为$(0,0,\cdots,0)$,所以有$ex=xe=x$.
    \begin{equation}
        f(x,0)=f(0,x)=x
    \end{equation}
    \item 对于逆元$g^{-1}$,我们设其坐标为$(\tilde{x}_1,\cdots\tilde{x}_r)$,于是有
    \begin{equation}
        f(x,\tilde{x})=f(\tilde{x},x)=0
    \end{equation}
\end{enumerate}
我们很容易看出,乘法函数是很抽象的,只有乘法函数来研究李群往往是无处下手的(更何况我们是学物理的),于是有了李代数的理论.不过在展开李代数之前,我们使用几个实际的李群的例子来帮助建立对于李群的理解.
\begin{example}
    $T_2=\Big\{\begin{bmatrix}e^{x_1}&x_2\\0&1\end{bmatrix}\Big|x_1,x_2\in\mathbb{R}\Big\}$.这个群的元素由两个独立实参数$x_1,x_2$决定.所以,$T_2$是一个二维流形\sidenote[][]{流形:一句话来表述是将一个空间的局部近似为一个欧氏空间,我们把这个欧氏空间称为流形(manifold),你可以把它当做一种空间.}.我们现在来逐个验证其满足群的要求.
\end{example}
\begin{proof}
    \begin{enumerate}
        \item 首先我们验证其封闭性
        \begin{equation}
            \begin{gathered}\begin{bmatrix}e^{x_1}&x_2\\0&1\end{bmatrix}\begin{bmatrix}e^{y_1}&y_2\\0&1\end{bmatrix}=\begin{bmatrix}e^{x_1}e^{y_1}&e^{x_1}y_2+x_2\\0&1\end{bmatrix}\\=\begin{bmatrix}e^{x_1+y_1}&e^{x_1}y_2+x_2\\0&1\end{bmatrix}=\begin{bmatrix}e^{z_1}&z_2\\0&1\end{bmatrix}\in T_2\end{gathered}
        \end{equation}
        并同时写出其乘法函数,不难发现其乘法函数是无限次可微的.
        \begin{equation}
            \begin{aligned}&z_{1}=f_{1}(x_{1},x_{2};y_{1},y_{2})=x_{1}+y_{1},\\&z_{2}=f_{2}(x_{1},x_{2};y_{1},y_{2})=e^{x_{1}}y_{2}+x_{2}.\end{aligned}
        \end{equation}
        \item $T_2$的乘法运算为矩阵乘法,自然满足结合律.
        \item 对于幺元,我们注意到
        \begin{equation}
            \begin{bmatrix}e^0&0\\0&1\end{bmatrix}=\begin{bmatrix}1&0\\0&1\end{bmatrix}
        \end{equation}
        \item 我们注意到有
        \begin{equation}
            \begin{aligned}\begin{bmatrix}e^{-x_1}&-x_2e^{-x_1}\\0&1\end{bmatrix}\begin{bmatrix}e^{x_1}&x_2\\0&1\end{bmatrix}&=\begin{bmatrix}e^{x_1}&x_2\\0&1\end{bmatrix}\begin{bmatrix}e^{-x_1}&-x_2e^{-x_1}\\0&1\end{bmatrix}\\&=\begin{bmatrix}1&0\\0&1\end{bmatrix}\end{aligned}
        \end{equation}
        所以逆元为$\begin{bmatrix}e^{-x_1}&-x_2e^{-x_1}\\0&1\end{bmatrix}$并容易验证其不满足交换律.
    \end{enumerate}
\end{proof}
\begin{example}
    我们的下一个实例是绕定轴转动的旋转群SO(2),显然,我们只需要一个变量(转动角$\theta$)就可以表述一个转动变换,所以我们表示群元为$g(\theta)$,其中$\theta$的取值范围是$[0,2\pi)$.而群的运算法则可以被规定为相继的两个转动,即转动角相加,但需要保持转动角始终在取值范围内.我们可以使用公式表达:
    \begin{equation}
        g(\theta_1)g(\theta_2)=g(\theta_{12}),\qquad\theta_{12}=(\theta_1+\theta_2)\mod 2\pi
    \end{equation}
    我们容易验证其满足对应的4条性质.不过我们在关于线性代数的学习中,我们知道:我们也可以使用旋转矩阵来表述定轴转动.
    \begin{equation}
        \begin{bmatrix}x\\y\end{bmatrix}\xrightarrow{g(\theta)}\begin{bmatrix}x'\\y'\end{bmatrix}=\begin{bmatrix}\cos\theta&-\sin\theta\\\sin\theta&\cos\theta\end{bmatrix}\begin{bmatrix}x\\y\end{bmatrix}=\begin{bmatrix}x\cos\theta-y\sin\theta\\x\sin\theta+y\cos\theta\end{bmatrix}
    \end{equation}
    我们发现,旋转矩阵是SO(2)群的群元.我们在下一个例子会更加深入讨论这部分内容.
\end{example}
\begin{example}
    现在我们需要讨论三维旋转群SO(3),其群元表示三维空间中绕某个固定点的一个转动$g\in SO(3)$,为了方便表述SO(3),我们使用如图所示的欧拉(Euler)角$(\alpha,\beta,\gamma)$来表示一个转动.
    \begin{marginfigure}
        \centering
        \includegraphics[width=0.4\linewidth]{"images/Euler angle"}
        \caption{欧拉角}
        \label{fig:euler-angle}
    \end{marginfigure}
    \marginnote[*6]{我们依次写出绕$z$轴旋转$\alpha$角;绕$x$轴旋转$\beta$角;绕$z$轴旋转$\gamma$角的三个群元的矩阵表示:
    $$g_z^\alpha={\begin{bmatrix}\cos \alpha &-\sin \alpha &0\\\sin \alpha &\cos \alpha &0\\0&0&1\end{bmatrix}}$$
    $$g_x^\beta={\begin{bmatrix}1&0&0\\0&\cos \beta &-\sin \beta \\0&\sin \beta &\cos \beta \end{bmatrix}}$$
    $$g_z^\gamma={\begin{bmatrix}\cos \gamma &-\sin \gamma &0\\\sin \gamma &\cos \gamma &0\\0&0&1\end{bmatrix}}$$
    }

    我们给出最终群元的表示:
    \begin{equation}
        g=g_z^\alpha g_x^\beta g_z^\gamma
    \end{equation}
    因此,SO(3)的元素可以通过三个独立参量$\alpha,\beta,\gamma$来确定,因此不难验证SO(3)是一个三维李群.
    
    现在我们给出另一种表述SO(3)的方法.\\
    我们对于两个矢量$x=(x_1,x_2,x_3),y=(y_1,y_2,y_3)$给出三维欧式空间$\mathbb{R}_3$的内积:
    \begin{equation}
        \langle x,y\rangle=\sum_{j=1}^{3}x_j y_j=x_1y_1+x_2y_2+x_3y_3
    \end{equation}
    我们定义一个线性变换算符$g=(g_{ij})$,存在关系
    \begin{equation}
        \begin{aligned}
            x\xrightarrow{g}x'=gx=\begin{bmatrix}g_{11}&g_{12}&g_{13}\\g_{21}&g_{22}&g_{23}\\g_{31}&g_{32}&g_{33}\end{bmatrix}\begin{bmatrix}x_1\\x_2\\x_3\end{bmatrix}\\ g\in SO(3)\Leftrightarrow\langle gs,gy\rangle=\langle x,y\rangle\quad\forall x,y\in\mathbb{R}_3\quad\text{且}\det g>0
        \end{aligned}
    \end{equation}
    而且我们发现$\langle gs,gy\rangle=\langle x,g^Tgy\rangle$,其中$g^T$表示$g$的转置,即$g_{ij}^T=g_{ij}$,我们根据刚才所给出的关系发现$\langle x,g^Tgy\rangle=\langle x,y\rangle$,即$g^Tg=\textbf{1}$,我们把满足关系$g^Tg=gg^T$的线性变换构成的群称为正交群.
    
    对于3阶矩阵$g$,存在9个元素,但为了满足特殊正交群的特殊性($\det g=1$)和正交性($g^Tg=gg^T$),共有6个方程需要满足.所以,我们可以拿出3个作为独立参数,这再次证明了$g$可以表述SO(3)这个三维李群.
\end{example}
\subsection{指数映射与对数映射}
\marginnote[]{在前面的部分,我们强调了群的乘法一般不可交换,这直接导致了刻画李群的乘法函数变得非常复杂,这意味着想通过研究乘法函数来研究李群是不现实的.而为了研究李群的各种结构,我们可以对李群在幺元处的无穷小变换进行研究,而Lie证明了李群的主要特征可以通过无穷小变换来得到,这就是为什么现在称这类群为李群的原因.}对于无穷小变换,它是一个拥有特殊结构的线性空间,我们称它构成的代数结构为\textbf{李代数}.

这里,我们再次强调,后面默认所有的群都是\textbf{线性群},群的运算规则都是\textbf{矩阵乘法}!

我们回到这一节前面所提到的,由于矩阵乘法不可交换,导致运算变得复杂,那么有没有一种办法,可以让复杂的运算变为较简单的运算呢(最好还是物理中最喜欢的线性运算)?

答案是肯定的,我们高中就学过一种特殊的运算:指数运算,它可以把较为复杂的乘法变为较为简单的加法,即对于给出的$y_1=e^{x_1},y_2=e^{x_2}$,我们有
\begin{equation}
    y_1y_2=e^{x_1}e^{x_2}=e^{x_1+x_2}
\end{equation}
这样,我们就实现了运算的``降级",并且还是线性运算.现在,问题变为,我们能否同样应用这种方式,将矩阵乘法转变为某种加法呢?

答案同样是肯定的,但由于矩阵乘法比代数乘法更为复杂,相应的``加法"自然也更加复杂.而为了得到这种简单的``加法"运算,我们首先对$n$阶矩阵$A$定义幂指数:
\begin{equation}
    e^A=\exp(x)=\textbf{1}+A+\frac{A^2}{2!}+\frac{A^3}{3!}+\cdots 
\end{equation}
易证此级数对于任意的矩阵$A$都是收敛的.并且对于零矩阵$O$,显然有
\begin{equation}
    e^O=\textbf{1}
\end{equation}
并且,前面我们多次提到矩阵乘法相比代数乘法是不可交换的,那么,反应到对应的``加法"运算上,自然也有区分加法的性质,即当且仅当$A,B$对易的时候,才有$e^Ae^B=e^{A+B}$,这是主要的困难点,那么我们的主要问题就集中在$e^Ae^B=e^{?}$上,这也就是我们将要学习的李代数的内容.不过,在正式开始李代数的内容之前,我们先讨论一下其他同样有价值的内容.

我们仅了解了和指数函数对应的运算,而我们高中还知道,指数函数的逆运算是对数函数,接下来我们效仿之前的内容,对对数函数应用同样的定义方法:

同样对于$n$阶矩阵$A$,我们定义\sidenote[][]{在物理的语境中,$\log$通常仅指$\ln$,同样的,本文中采取该写法.}
\begin{equation}
    \log A=(A-\textbf{1})-\frac{(A-\textbf{1})^2}2+\frac{(A-\textbf{1})^3}3-\frac{(A-\textbf{1})^4}4+\cdots 
\end{equation}
不同于指数函数,为了保证级数收敛,我们要求$A-\textbf{1}$的每一个元素的绝对值均小于$\dfrac{1}{n}$,即要求$A$是与幺元邻近的元素.并且指数函数与对数函数互为逆运算的关系对于这个定义同样适用(仅需泰勒展开即可证明,留给读者自行尝试).
\subsection{单参数子群}
在前面,我们通过代数方法初步建立了一些对应关系,这一小节,我们通过几何的角度再次考虑这个对应关系.

\begin{definition}[单参数子群]
    设$G$为一个李群,并且$\gamma(t)(-\infty<t<\infty)$为$g$中过幺元$e$的一条曲线,则对每一取定的$t_0\in \mathbb{R},\gamma(t_0)$是$G$中的一个元素.并设参数$t$满足:
    \begin{equation}
        \gamma(t_1+t_2)=\gamma(t_1)\gamma(t_2)
    \end{equation}
    则称$\gamma(t)$是$G$中的一个单参数李群.
\end{definition}

现在我们通过几何的角度思考问题.

我们将李群$G$的一个单参数子群看成流形$G$ (对二维Lie 群,可将$G$ 看成为一张曲面)中过$e$处的一条曲线.从微积分知道这只要对$\gamma\left(t\right)$在$t=0$处求导即得$\gamma\left(t\right)$在$t=0$处的切向量,$\gamma^\prime(0)=\frac{\d\gamma(t)}{\d t}$.(因为我们只讨论线性群,所以$\gamma\left(t\right)$是矩阵,其元素是$t$ 的函数,$\gamma^{\prime}(t)$ 表示对$\gamma(t)$的每一元素求导所得的矩阵).由于:
\begin{equation}
    \gamma (t+s)=\gamma (t)\gamma (s)
\end{equation}
两边对$s$求导,同时令$s=0$,得到一个微分方程
\begin{equation}
    \gamma' (t)=\gamma(t)\gamma' (0)
\end{equation}
对于这类微分方程,我们知道其解为
\begin{equation}
    \gamma(t)=\exp(t\gamma'(0))
\end{equation}
由此,我们知道,单参数子群必能表达为指数映射的形式.\\
\begin{marginfigure}
    \centering
    \caption{二维李群与其单参数子群}
    
    
    \begin{tikzpicture}
        
        % Draw the manifold G
        \draw[thick] plot [smooth,tension=1] 
        coordinates {(0.2,1.4) (3.2,0.8) (4,2.5) (5,4) (1,5) (-0.5,3.7) (0.2,1.4)};
        %\draw[thick] (0,0) to[out=30, in=180] (4,2.5) to[out=0, in=270] (5,4) to[out=90, in=30] (1,5) to[out=210, in=150] cycle;
        
        % Label G
        \node at (5.5,3.5) {$G$};
        
        % Draw the tangent space T_eG
        \draw[thick] (1.5, 2) -- (4, 3) -- (3, 4.5) -- (0.5, 3.5) -- cycle;
        
        % Label TeG
        \node at (1.65, 2.3) {$T_e G$};
        
        % Draw the curve \gamma(t)
        \draw[thick] (1.8, 2.8) .. controls (2.5, 3.5) .. (3.3, 3.7);
        
        % Label points and vectors
        \node at (0.5,4.7) {$\gamma(0)$};
        \draw[thick, ->] (0.5,4.4) .. controls (1.5,3.8) .. (2.3,3.55);
        \node at (4.8, 2) {$\gamma(t)$};
        \draw[thick, ->] (4.8,2.4) .. controls (3.5,2.8) .. (1.9,2.8);
        \filldraw[black] (2.5,3.45) circle (1pt) node[anchor=north] {$e$};
        \draw[thick, ->] (1.9, 3.15) -- (3.1, 3.75);
        \node at (3.6, 5.7) {$\gamma'(0)$};
        \draw[thick, ->] (3.6, 5.4) .. controls (3.4,4.3) .. (3.1,3.95);
        
    \end{tikzpicture}
\end{marginfigure}
由解的形式可见,对于$G$在幺元处的切空间$T_eG$的任意一个向量$A=\gamma'(0)$,就有$G$中的一元素$\gamma(1)=\exp(\gamma'(0))$与之对应.现在我们稍微总结一下:对于李群$G$这个流形,我们可以找到其单参数子群$T_eG$作为其切空间,并且我们可以找到一个指数映射从切空间到原空间,很快,当我们学会李代数的时候,我们会再次使用李代数的语言来总结:``李代数就是李群的切空间所导出的代数".

我们再次回到群$G$和它的单参数子群,我们发现,如果给定$G$中与幺元邻近的一个元素\sidenote[][]{当然,由于我们研究线性群,幺元为单位矩阵.}$g$,并定义向量$A=\log g$,则由$e^A=g$ 知,$A$ 为$G$ 在$e$ 处之切向量,$e^tA$为以 $A$ 为单位切向量的单参数子群.因此,对$G$ 中与幺元$e$ 邻近的一个元素就有$T_{e}(G)$($G$ 在幺元处的切空间)中一向量$A$ 与之对应,也就是说,设$U\subset G$ 中包含$e$ 的一个适当邻域,我们建立了一种对应关系
\begin{equation}
    \begin{aligned}
        G\supset U&\overunderset{\log}{\exp}{\longleftrightarrow}T_{e}(G)\\g&\to A=\log g\\e^{A}&\leftarrow A
    \end{aligned}
\end{equation}
这种对应关系可以使我们对李群的研究转化到与其对应的在幺元$e$处的切空间$T_e(G)$.而我们知道,$T_e(G)$是由向量组成的线性空间,其线性结构具有先天优势,拥有远比李群简单的结构和运算.但由于我们前面所提到的,由于矩阵乘法相比于数乘的不可交互性,自然由此导出的切空间的运算自然也不能简单用普通加减法来表述,即右侧关系图
\marginnote[]{$$\begin{aligned}&T_{e}(G)\qquad G\\&A\quad\longrightarrow\quad e^{A}\\&B\quad\longrightarrow\quad e^{B}\\&A+B\to e^{A+B}\ne e^{A}e^{B}\end{aligned}$$}

为此,我们迫切需要引入一种新的代数结构来反映$G$中的不可交换性,而具有这种新结构的线性空间$T_e(G)$
,就是我们下一节所要讲的\textbf{李代数}.
\section{李群与李代数}
\subsection{李代数}
由上面的讨论,我们现在知道$e^Ae^B\ne e^{A+B}$,那么,问题自然变为:$e^Ae^B=e^{?}$,或者表述为,$G$的单参数子群的代数结构是什么样的?

为了解决这个问题,我们设$A,B\in T_e(G)$,取一个参数$t$,并要求$|t|$适当小,从而能够保证$e^{tA}$与$e^{tB}$均为李群$G$中与幺元$e$邻近的元素\sidenote[][]{这个要求是必要的,我们需要满足后续使用级数的收敛性.}.现在我们构造一个函数:
\begin{equation}
    g(t)=e^{tA}e^{tB}e^{-tA}e^{-tB}
\end{equation}
显然,对于特例,即如果$e^{tA}$与$e^{tB}$可交换,$g(t)=e=\textbf{1}$,对于不可交换的情况,$g(t)$与幺元$e$的偏离程度反映了$e^{tA}$与$e^{tB}$的乘法与可交换的乘法之间的差异大小.现在我们具体分析$g(t)$.
\marginnote[*-2]{展开的计算过程
$$\begin{aligned}
    g(t)& =e^{tA}e^{tB}e^{-tA}e^{-tB} \\
    &=(\textbf{1}+tA+\frac{t^{2}}{2!}A^{2}+\frac{t^{3}}{3!}A^{3}+\cdots)(\textbf{1}+tB+\frac{t^{2}}{2!}B^{2}+\frac{t^{3}}{3!}B^{3}+\cdots) \\
    &(\textbf{1}-tA+\frac{t^2}{2!}A^2-\frac{t^3}{3!}A^3+\cdots)(\textbf{1}-tB+\frac{t^2}{2!}B^2-\frac{t^3}{3!}B^3+\cdots) \\
    &=\{\textbf{1}+t(A+B)+t^{2}(\frac{A^{2}}{2}+AB+\frac{B^{2}}{2})\\&+t^{3}(\frac{A^{3}}{6}+\frac{A^{2}B}{2}+\frac{AB^{2}}{2}+\frac{B^{3}}{6})+O(t^{4})\} \\
    &\{\textbf{1}-t(A+B)+t^2(\frac{A^2}{2}+AB+\frac{B^2}{2})\\&-t^3(\frac{A^3}{6}+\frac{A^2B}{2}+\frac{AB^2}{2}+\frac{B^3}{6})+O(t^4)\} \\
    &=\textbf{1}+t(A+B-A-B)+t^{2}(AB-BA)+t^{3}(\frac{A^{2}B}{2}-\frac{AB^{2}}{2} \\
    &-\frac{B^2A}{2}+\frac{BA^2}{2}-ABA+BAB)+O(t^4) \\
    &=\textbf{1}+t^{2}[A,B]+\frac{t^{3}}{2}([A,[A,B]]-[B,[B,A]])+O(t^{4})
\end{aligned}$$
}
\begin{equation}
    \begin{aligned}
        g(t)& =e^{tA}e^{tB}e^{-tA}e^{-tB} \\
        &=\textbf{1}+t^{2}[A,B]+\frac{t^{3}}{2}([A,[A,B]]-[B,[B,A]])+O(t^{4})
    \end{aligned}
\end{equation}
这里我们使用了对易子记号$[,]$,不过对于李代数,它也称为李括号,李乘法\sidenote[][*5]{事实上,对于线性群它等同于对易子,后面会加以区分的使用对易子和李括号.}.对于函数$g(t)$,我们有
\begin{equation}
    \frac{g(t)-\textbf{1}}{t^2}=[A,B]+O(t)
\end{equation}
因此,考虑极限$t\to0$时,
\begin{equation}
    \lim_{t\to0}\frac{g(t)-\textbf{1}}{t^{2}}=[A,B]
\end{equation}
由此,我们发现李群$G$的元素$e^{tA}$与$e^{tB}$的乘法的不可交换程度在$|t|$很小时主要取决于$[A,B]$

现在我们做变量代换$t=\sqrt{s}$,则
\begin{equation}
    \frac{g(\sqrt{s})-g(0)}{s}=[A,B]+O(\sqrt{s})
\end{equation}
并因此
\begin{equation}
    \lim_{s\to0}\frac{g(\sqrt{s})-g(0)}{s}=[A,B]
\end{equation}
这也说明$[A,B]$是李群$G$中过幺元的曲线$g(\sqrt{s})$在幺元处的\textbf{切向量},即$[A,B]\in T_e(G)$,这也意味着我们证明了如下关系
\begin{equation}
    \forall A,B\in T_{e}(G) , [A,B]\in T_{e}(G)
\end{equation}
即对易子(李括号)对向量空间$T_e(G)$的封闭性.


此时,我们可以回答开头所提到的问题了,不妨设$e^{tA}e^{tB}=e^{tC}$,则
\marginnote[]{展开的计算过程
$$\begin{aligned}
    tC&=\log e^{tC}=\log e^{tA}e^{tB}\\
    &=\log\{(\textbf{1}+t(A+B)+\frac{t^{2}}{2}(A^{2}+2AB+B^{2})+ \\
    &\quad+\frac{t^{3}}{6}(A^{3}+3A^{2}B+3AB^{2}+B^{3})+O(t^{4})\} \\
    &=t(A+B)+\frac{t^{2}}{2}(A^{2}+2AB+B^{2})+\frac{t^{3}}{6}(A^{3}+3A^{2}B+3AB^{2}+B^{3}) \\
    &\quad+O(t^{4})-\{t(A+B)+\frac{t^2}{2}(A^2+2AB+B^2)+O(t^3)\}^2/2+ \\
    &\quad+\{t(A+B)+\frac{t^{2}}{2}(A^{2}+2AB+B^{2})+O(t^{3})\}^{3}/3+O(t^{4}) \\
    &=t(A+B)+\frac{t^{2}}{2}(AB-BA)+\frac{t^{3}}{12}(A^{2}B-ABA-ABA+BA^{2} \\
    &\quad-B^2A+BAB+BAB-AB^2)+O(t^4) \\
    &=(tA+tB)+\frac{1}{2}[tA,tB]+\frac{1}{12}[tA,[tA,tB]-tB,[tB,tA]]+O(t^{4}) 
\end{aligned}$$}
\begin{equation}
\begin{aligned}
    tC&=\log e^{tC}=\log e^{tA}e^{tB}\\
    &=(tA+tB)+\frac{1}{2}[tA,tB]+\frac{1}{12}[tA,[tA,tB]-tB,[tB,tA]]+O(t^{4}) 
\end{aligned}
\end{equation}
由此可见,只要给出李括号,$T_e(G)$中知道了与$e^{tA},e^{tB}$相对应的元素$tA,tB$即可求得$T_e(G)$中与$e^{tA}e^{tB}$相对应的元素.因此,我们认为李括号可以表述李群切空间的代数结构,并对于李括号有如右侧所示的性质.
\marginnote[]{李括号的性质$$\begin{gathered}
        \left\lbrack {A, A}\right\rbrack = 0\\
        \left\lbrack {A, B}\right\rbrack = - \left\lbrack {B, A}\right\rbrack\\
        \left\lbrack {A, c}\right\rbrack = 0\;\left( {c\text{ 只是一个数 }}\right)\\
        \left\lbrack {A + B, C}\right\rbrack = \left\lbrack {A, C}\right\rbrack + \left\lbrack {B, C}\right\rbrack\\
        \left\lbrack {A,{BC}}\right\rbrack = \left\lbrack {A, B}\right\rbrack C + B\left\lbrack {A, C}\right\rbrack\\
        \left\lbrack {A,\left\lbrack {B, C}\right\rbrack }\right\rbrack + \left\lbrack {B,\left\lbrack {C, A}\right\rbrack }\right\rbrack + \left\lbrack {C,\left\lbrack {A, B}\right\rbrack }\right\rbrack = 0
    \end{gathered}$$}

我们称有了李括号的向量空间$T_e(G)$构成一个李代数,更准确的来讲是李群$G$的李代数,并记为$\mathfrak{g}$.\\
李群的李代数完全刻画了李群在幺元附近的结构,而要研究李群在幺元附近的性质只需要研究李代数即可,但是需要注意的是,李代数\textbf{仅}刻画了李群在幺元附近的\textbf{局部}性质,\textbf{不能}反映其整体性质,一个李群对应一个李代数,而一个李代数可以对应多个李群.
\begin{definition}[结构常数]
    设$G$是一个$r$维李群,取定幺元$e$的一个邻域$U$,在$U$中取定坐标系$\{U,\varphi\}$并取$e$为坐标原点:
    \begin{equation}
        \varphi(e)=(0,0,\cdots,0)
    \end{equation}
    对于其的单参数子群,我们有
    \begin{equation}
        \begin{cases}\gamma_j(t),&j=1,2,\cdots,r\\\varphi(\gamma_j(t))=(\underbrace{0,\cdots,0}_{(j-1)\text{个零}},t,0,\cdots,0)\end{cases}
    \end{equation}
    为其$r$条坐标曲线.
    
    以$X_{j}= \gamma _{j}^{\prime }(0),j= 1, 2, \cdots , r$记为其在幺元处的切向量.即$X_j\in T_e(G) = \mathfrak{g}, j= 1, 2, \cdots , r$.显然,$\{X_1,X_2,\cdots,X_r\}$可取作为向量空间$T_{_e}(G)$的基——$T_{e}(G)$中任一向量可用它们的线性组合表出.由于$[X_i,X_j]\in \mathfrak{g}$,所以
    \begin{equation}
        [X_{i},X_{j}]=\sum_{j,k=1}^{n}C_{ij}^{k}X_{k}\quad i,j=1,2,\cdots,r
    \end{equation}
    这$r^3$个数$\{C_{ij}^k\}k,i,j=1,2,\cdots,r$称为李群以$\{X_1,X_2,\cdots,X_r\}$,为基的\underline{结构常数}.
\end{definition}
对于李代数$\mathfrak{g}$中任意向量$X,Y$,有
\begin{equation}
    \begin{aligned}
        X&=\sum_{j=1}^{r}\xi^{j}X_{j}&Y=\sum_{j=1}^{r}\eta^{j}X_{j}\\
        X&\sim(\xi^{1},\xi^{2},\cdots,\xi^{r})&Y=(\eta^{1},\eta^{2},\cdots\eta^{r})\\
        Z&=[X,Y]=\left[\sum_{j=1}^{r}\xi^{j}X_{j},\sum_{k=1}^{r}\eta^{k}X_{k}\right]&=\sum_{i=1}^{r}\sum_{j,k=1}^{r}C_{jk}^{i_{k}}\xi^{j}\eta^{k}X_{i}
    \end{aligned}
\end{equation}
将$Z$也用坐标表示
\begin{equation}
    Z=\sum_{i=1}^{r}\zeta^{ i}X_{i},\quad Z\sim(\zeta^{ 1},\zeta^{ 2},\cdots\zeta^{ r})
\end{equation}
此时易求出结构常数
\begin{equation}
    \zeta^{i}=\sum_{j,k=1}^{r}C_{jk}^{i}\xi^{j}\eta^{k},\quad i=1,2,\cdots,r
\end{equation}
由此可见,结构常数可以完全确定一个李代数.

需要强调的是,结构常数与基的选取有关,而李代数的一个重要的问题就是如何选取适当的基使结构常数最简单.\\
或许到此,你可能还对结构常数一头雾水,在再次讲解结构常数之前,我们还是先给出一些基本性质,并实际算一下结构常数.
\begin{equation}
    \begin{aligned}
        &C_{ij}^{ k}=-C_{ji}^{ k}&i,j,k=1,2,\cdots,r\\
        &\sum_{l=1}^{r}\left(C_{ij}^{l}C_{lk}^{m}+C_{jk}^{l}C_{li}^{m}+C_{ki}^{l}C_{lj}^{m}\right)=0&i,j,k,m=1,2,\cdots,r.\end{aligned}
\end{equation}
\begin{example}
    我们再次考虑由例题3.8给出的群$T_2=\Big\{\begin{bmatrix}e^{x_1}&x_2\\0&1\end{bmatrix}\Big|x_1,x_2\in\mathbb{R}\Big\}$,我们知道$\gamma_1(t)=\begin{bmatrix}e^t&0\\0&1\end{bmatrix} -\infty<t<\infty $与$\gamma_2(t)=\begin{bmatrix}1&t\\0&1\end{bmatrix} -\infty<t<\infty $是$T_2$的两个单参数子群,同时也是过幺元的两条曲线,我们给出在幺元处的切向量$X_1=\gamma_1'(0)=\begin{bmatrix}1&0\\0&0\end{bmatrix},X_2=\gamma_2'(0)=\begin{bmatrix}0&1\\0&0\end{bmatrix}$,因此李群$T_2$的李代数$\mathfrak{t}_2$的基由$X_1,X_2$组成,现在来求结构常数.
    \begin{equation}
        \begin{aligned}&[X_1,X_1]=\begin{bmatrix}0&0\\0&0\end{bmatrix}\quad[X_2,X_2]=\begin{bmatrix}0&0\\0&0\end{bmatrix}\\&[X_1,X_2]=\begin{bmatrix}1&0\\0&0\end{bmatrix}\begin{bmatrix}0&1\\0&0\end{bmatrix}-\begin{bmatrix}0&1\\0&0\end{bmatrix}\begin{bmatrix}1&0\\0&0\end{bmatrix}=\begin{bmatrix}0&1\\0&0\end{bmatrix}=X_2.\end{aligned}
    \end{equation}
    所以,$C_{11}^1=C_{11}^2=C_{22}^1=C_{22}^2=0,\quad C_{12}^1=-C_{21}^1=0,\quad C_{12}^2=1,\quad C_{21}^2=-1.$
\end{example}
\begin{example}
    我们再次回到$SO(3)$群,现在我们来求其李代数$\mathfrak{so}(3)$及其结构常数.\\
    我们列出其群元(绕$x,y,z$的转动),即旋转矩阵,如右侧所示
        \marginnote[*-3]{三个群元的矩阵表示:
        $$g_{x}(t)=\begin{bmatrix}1&0&0\\0&\cos t&-\sin t\\0&\sin t&\cos t\end{bmatrix}$$
        $$g_y(t)=\begin{bmatrix}\cos t&0&\sin t\\0&1&0\\-\sin t&0&\cos t\end{bmatrix}$$
        $$g_{z}(t)=\begin{bmatrix}\cos t&-\sin t&0\\\sin t&\cos t&0\\0&0&1\end{bmatrix}$$
    }

    此时,我们发现,这是其的三个单参数子群,而它们在幺元处的切向量一并给出
    \begin{equation}
        I_{1}=g_{x}'(0),\quad I_{2}=g_{y}'(0),\quad I_{3}=g_{z}'(0).
    \end{equation}
    于是$\{I_1,I_2,I_3\}$构成$SO(3)$的李代数$\mathfrak{so}(3)$的一组基,其李括号为:
    \begin{equation}
        [I_1,I_2]=I_1I_2-I_2I_1=I_3,[I_2,I_3]=I_1,[I_3,I_1]=I_2
    \end{equation}
    同时得出结构常数$C_{12}^{ 1}=0,C_{12}^{ 2}=0,C_{12}^{ 3}=1,\cdots.$
    
    我们可以发现,对于这个李括号,其还等价于三维欧式空间的向量乘法,我们就得到了简单情况下的李括号的退化情况.
\end{example}

%\begin{example}
%	我们这次要讨论的是在物理上极为重要的一类李群:$SU(2)$
%\end{example}
\subsection{李氏三定理和无穷小变换}
1
\subsection{李群的无穷小生成元}
1
\subsection{典型李群和李代数}
1
\section{李代数进阶}
\subsection{张量,不可约张量}
1
\subsection{卡西米尔算符(Casimir operator)}
1
\subsection{典型李代数的二阶卡西米尔算符}
1



    \setchapterpreamble[u]{\margintoc}
\chapter{场论初步}
\labch{intro}

\section*{引子:最小作用量原理与路径积分}
\textit{我们都知道,数学上很讲究“公理”,希望一切都可以由公理自然给出,无论是欧几里得的平面几何几大公设还是近现代的ZFC公理系统,都试图为数学大厦建立一个夯实的根基,而这群搞物理的和搞数学又常常客串,自然而然,这些人也开始思考物理有没有所谓的公设,牛顿三定律?麦克斯韦方程组?薛定谔方程?爱因斯坦场方程?虽然这些定律都是正确的,但都在对应的能标下成立,谈不上“公理”.而最后,弄出来一些似是而非的东西,我们这一篇便讲其中之一:最小作用量原理.}\\
    \marginnote[]{\textbf{这里给出来自于陈童老师的从经典力学出发的视角:}\\
        最小作用量原理(principle of least action).按照最小作用量原理,粒子在相空间中不是按照哈密顿正则方程这样的微分方程演化,粒子是按“代价”最小的相空间路径演化.即是说,粒子的演化路径有无穷多种可能性,每一条可能的演化路径都要付出一个相应的“代价”,而粒子的真实演化路径是所有可能路径中“代价”最小的那条,严格一点说应该是“代价”取极值的那条(更严格地说应该是“代价”取驻值的那条,但是驻值这个概念比较微妙,为了使得读者更好理解,我们统一称作极值).每一条路径的“代价”就叫做这条路径的作用量.\\
        很显然,最小作用量原理看起来与哈密顿正则方程完全不同,结果却可以证明,这两种描述粒子在相空间中如何演化的方式物理上完全等价.哈密顿正则方程是一种局部视角的描述方式,每一个时刻都只需用到当前相点(相空间点) 局部邻域内的信息,因为微分方程中的求导运算只涉及邻域.而最小作用量原理是一种全局视角的描述方式,需要知道每一条可能路径的作用量这种全局信息.奇妙的是,这两种不同视角在物理上却是等价的.人们有时候将微分方程这样的局部视角称作蚂蚁视角(蚂蚁太小,每一只蚂蚁都只能看到一个很小的邻域),而将最小作用量原理这样的全局视角称作上帝视角.看待物理的蚂蚁视角源自于牛顿,正是牛顿想到用微分方程来描述物理规律.而上帝视角则源自于几何光学中的费马原理,然后经过莫培督(Maupertuis) 和哈密顿等人推广到力学里来.实际上,最小作用量原理有时候也称作哈密顿原理.不可思议的是,我们可以用这样两种完全不同的视角来看待同样的物理.\\不仅如此,这种局部视角和全局视角也都可以延伸到量子物理中,在量子力学中,局部视角大致会导致算符描述,而全局视角会导致路径积分描述.}
    \textit{而什么是最小作用量原理呢?正所谓水往低处流,水不会无缘无故的克服重力向上做功,它永远沿着做功最少的路径前进.对于我们中学就学过的电路来讲,导线会把电阻短路,因为电流自发的按照对外做功最小的路径前进.而这种按对外做功最小的行为,我们统称为最小作用量原理.作用量是系统拉格朗日量对时间的泛函(即函数的函数),而Wick转动告诉我们,实轴上的最小作用量原理就是虚时上的最小能量原理.\\
    我们知道,量子态和经典态是截然不同的,一个量子态经由某条路径(区分实际空间中的路径和概念上的路径)到达一个结果的量子态的传播过程并不能像经典的时候那样直接表述出来:我们只知道起始时刻的位置和终点时刻的位置,中间的过程也不是传统意义上的移动,而是黑箱一般,经过一定时间之后就转移到终点,中间的过程是未知的.\\
    于是,我们的首要目的就是破解这个黑箱,对于初态和末态,我们定义了一个新的算符:传播子来代表这个黑箱过程.而下一刻粒子出现在末态的概率幅依赖于始末位置间的所有路径,但是我们知道,两点间可能有无数种方式路径,但那又如何?费曼提出了路径积分,将所有可能的路径全部积分一遍.其遵循了一些简朴的思想:对于过多扭曲的路径,我们可以认为反复的部分的平均为0,自然也不必考虑扭曲转圈的情况,路径被统一为单行线.而始末位置的连线则是经典路径,所有扭曲的其他路径都要以经典路径为核心,偏离太远的也被近似为0.\\
    我们此时从最小作用量原理来解释这个朴素的思想:存在无穷多可能的路线,但是大部分都是高阶项,这说明我们可以取最少的路径来表述这个过程,作用量相当于不同路径的权重,最小作用量原理要求总体权重和最小,这意味着需要有效路径最少,也对应着之前所遵循的原理.\\
    另外需要强调的是,一次量子化(量子力学)告诉我们应当使用波来描述粒子,二次量子化又重新使用粒子来描述波,这些虽然符合实验现象,但是对于某些情况就捉襟见肘了.路径积分时区别于那两种的表述方法,它认为需要使用场来表述粒子,其构建了场论形式的泛函积分方法.}
\section{路径积分}
在初等量子力学的学习中,我们在经典量子化的框架内进行表述,在本节,我们将初步探索另一种表述方法:\textbf{路径积分法}.
\subsection{量子系统与经典系统中的路径积分}
我们采取通过一个最基本物理图像的方式来引入路径积分:考虑一维空间内的一个有质量$ m $的粒子,其动力学可以通过拉格朗日量来表述:\sidenote[][]{本章开始采取这类更加清晰的公式标注方法.}\\

\begin{equation}
    \vspace{\baselineskip}
    L(q,\dot{q})=\tikzmarknode{1eq1}{\highlight{red}{$\frac12m\dot{q}^2$}}-\tikzmarknode{1eq2}{\highlight{blue}{$V(q)$}}
\end{equation}
\begin{tikzpicture}[overlay,remember picture,>=stealth,nodes={align=left,inner ysep=1pt},<-]
    % 对于 "1" 定位点
    \path (1eq1.north) ++ (0,2em) node[anchor=south east,color=red!67] (eq1/1){\textbf{动能项}};
    \draw [color=red!87](1eq1.north) |- ([xshift=-0.3ex,color=red]eq1/1.south west);
    % 对于 "2" 定位点
    \path (1eq2.south) ++ (0,-1.5em) node[anchor=north west,color=blue!67] (eq1/2){\textbf{势能项}};
    \draw [color=blue!57](1eq2.south) |- ([xshift=-0.3ex,color=blue]eq1/2.south east);
\end{tikzpicture}

并假设粒子位于与时间无关的势$ V(q) $中.我们使用广义坐标$ q $来表述粒子的位置\sidenote[][]{广义坐标往往指一组无关联约束的坐标,即对于三维坐标表述,如果$x,y,z$之间没有约束方程,那么$x,y,z$就可以认为是一组广义坐标.不过,目前可以仅把它们当做特殊的坐标/动量(对于广义动量).},对于广义坐标,同样有$\dot{q}=\partial_t q$.于是,根据我们在理论力学中所学到的欧拉-拉格朗日方程
\begin{equation}
    \frac{\d}{\dt}\frac{\partial L(q,\dot{q})}{\partial\dot{q}}=\frac{\partial L(q,\dot{q})}{\partial q}\quad\text{也就是}\quad m\ddot{q}=-\frac{\partial V(q)}{\partial q}.
\end{equation}
同时,我们考虑由下式定义的\textbf{哈密顿量}$H(p,q)$
\begin{equation}
    H(p,q)=p\dot{q}-L(q,\dot{q})=\frac{p^2}{2m}+V(q)\quad\text{其中广义坐标被定义为}\quad p=\frac{\partial L(q,\dot{q})}{\partial\dot{q}}=m\dot{q}
\end{equation}
我们于是通过哈密顿量给出该粒子的经典动力学关系(特别强调的一点是,在对哈密顿量偏导时,我们认为广义坐标和广义动量是无关联的.):
\begin{equation}
    \dot{p}=-\frac{\partial H(p,q)}{\partial q}=-\frac{\partial V(q)}{\partial q},\dot{q}=\frac{\partial H(p,q)}{\partial p}=\frac{p}{m}
\end{equation}

在量子力学中,我们把广义坐标和广义动量的关系上升至对易关系(后续我们把广义坐标和广义动量简称为坐标和动量).
\begin{equation}
    [\hat{q},\hat{p}]=i\hbar
\end{equation}
同时经典变量$A(p,q)$同时上升为算符$\hat{A}\equiv A(\hat{p},\hat{q})$自然经典哈密顿量也变为量子哈密顿量方程.
\begin{equation}
    \hat{H}=\frac{\hat{p}^2}{2m}+V(\hat{q})
\end{equation}


我们知道系统的物理态由希尔伯特空间$ \mathcal H $中的态矢量$|\psi(t)\ra$所描述,我们使用薛定谔方程来描述态矢量的时间演化:\\

\begin{equation}
    \vspace{\baselineskip}
    %公式编号: 7
    \tikzmarknode{7eq1}{\highlight{red}{$i\hbar\partial_t$}}|\psi(t)\rangle=\tikzmarknode{7eq2}{\highlight{blue}{$\hat{H}$}}|\psi(t)\rangle
\end{equation}
\begin{tikzpicture}[overlay,remember picture,>=stealth,nodes={align=left,inner ysep=1pt},<-]
    % 对于 "1" 定位点
    \path (7eq1.north) ++ (0,2em) node[anchor=south east,color=red!67] (eq7/1){\textbf{能量$ E $}};
    \draw [color=red!87](7eq1.north) |- ([xshift=-0.3ex,color=red]eq7/1.south west);
    % 对于 "2" 定位点
    \path (7eq2.south) ++ (0,-1.5em) node[anchor=north west,color=blue!67] (eq7/2){\textbf{哈密顿量算符}};
    \draw [color=blue!57](7eq2.south) |- ([xshift=-0.3ex,color=blue]eq7/2.south east);
\end{tikzpicture}

我们熟知我们可以利用时间演化算符$\hat U$写成方程的解:
\begin{equation}
    |\psi(t)\rangle=\hat{U}(t)|\psi(t=0)\rangle,\quad i\hbar\partial_t\hat{U}(t)=\hat{H}\hat{U}(t).
\end{equation}
由于哈密顿量与时间无关,所以我们可以写成时间演化算符的表示
\begin{equation}
    \hat{U}(t)=e^{-\frac i\hbar\hat{H}t}
\end{equation}
在这里,我们重申一点:给定时空位置的波函数代表了该粒子位于该时空位置的概率振幅.\\
同时我们写出\\\\\\

\begin{equation}
    \vspace{\baselineskip}
    %公式编号: 10
    \tikzmarknode{10eq1}{\highlight{red}{$\psi(q_f,t_f)$}}=\langle q_f|\psi(t_f)\rangle=\langle q_f|\hat{U}(t_f-t_i)|\psi(t_i)\rangle=\int \dq_iU(q_f,q_i;t_f-t_i)\tikzmarknode{10eq2}{\highlight{blue}{$\psi(q_i,t_i)$}}
\end{equation}
\begin{tikzpicture}[overlay,remember picture,>=stealth,nodes={align=left,inner ysep=1pt},<-]
    % 对于 "1" 定位点
    \path (10eq1.north) ++ (0,2em) node[anchor=south west,color=red!67] (eq10/1){\textbf{$ t_f $时刻时粒子位于$q_f$的概率}};
    \draw [color=red!87](10eq1.north) |- ([xshift=-0.3ex,color=red]eq10/1.south east);
    % 对于 "2" 定位点
    \path (10eq2.south) ++ (0,-1.5em) node[anchor=north east,color=blue!67] (eq10/2){\textbf{$ t_i $时刻时粒子位于$q_i$的概率}};
    \draw [color=blue!57](10eq2.south) |- ([xshift=-0.3ex,color=blue]eq10/2.south west);
\end{tikzpicture}

其中$U(q_f,q_i;t_f-t_i) = \langle q_f|\hat{U}(t_f-t_i)|q_i\rangle $被称为传播子,其表示了这个粒子在$t_f-t_i$时间内从位置$q_i$传播到位置$q_f$的概率振幅.并且若已知哈密顿量$\hat{H}$本征态$\{|n\ra,\epsilon_n\}$,那么我们以此可以把传播子写作
\begin{equation}
    \begin{aligned}U(q_f,q_i;t_f-t_i)&=\langle q_f|e^{-\frac{i}{\hbar}\hat{H}(t_f-t_i)}|q_i\rangle=\sum_n\langle q_f|n\rangle e^{-\frac{i}{\hbar}\epsilon_n(t_f-t_i)}\langle n|q_i\rangle\\[2ex]&=\sum_ne^{-\frac{i}{\hbar}\epsilon_n(t_f-t_i)}\varphi_n(q_f)\varphi_n^*(q_i)\end{aligned}
\end{equation}
其中$\varphi_n(q)=\la q|n\ra$为坐标表象下的波函数.我们发现传播子给出了关于这个哈密顿量的波函数和能级的全部信息,这也意味着我们可以把求解波函数的问题变为求解这个哈密顿量所对应的传播子的问题.
\subsection{路径积分初步}
我们刚刚发现了通过求解传播子可以间接求解波函数,而现在的问题是:如何求出传播子? 这里我们用到费曼的路径积分方法,我们先来首先说明\textbf{传播子可以写为路径积分的形式}.

我们首先考虑一个充分短的时间$\epsilon$

\begin{equation}
    \vspace{\baselineskip}
    %公式编号: 12
    U(q_f,q_i;\epsilon)=\langle q_f|e^{-i\hat{H}\epsilon}|q_i\rangle \tikzmarknode{12eq2}{\highlight{blue}{$\simeq$}}\langle q_f|e^{-i\epsilon\frac{\hat{p}^2}{2m}}e^{-i\epsilon V(\hat{q})}|q_i\rangle
\end{equation}
\begin{tikzpicture}[overlay,remember picture,>=stealth,nodes={align=left,inner ysep=1pt},<-]
    % 对于 "2" 定位点
    \path (12eq2.south) ++ (0,-1.5em) node[anchor=north east,color=blue!67] (eq12/2){\textbf{Baker–Hausdorff规则$e^{\epsilon\hat{A}+\epsilon\hat{B}}=e^{\epsilon\hat{A}}e^{\epsilon\hat{B}}e^{\mathcal{O}(\epsilon^{2})}$}};
    \draw [color=blue!57](12eq2.south) |- ([xshift=-0.3ex,color=blue]eq12/2.south west);
\end{tikzpicture}

我们在其中插入一个单位算符的谱分解\sidenote[][]{此时已经开始采取自然单位制了(对单位制可以参考附录).},在下面的式子中,我们使用了归一化假设$\la q|p\ra=L^{-1/2}e^{ipq}\;q\in[0,L]$,$ q $为连续的位置变量,而$p=n\frac{2\pi}L\;n\in \mathbb Z$为离散的动量变量(关于边界$ L $,并且有归一化条件$e^{ipL}=1$),在极限$L\to\infty$,存在$\frac1L\sum_p\to\int_{-\infty}^\infty\frac{\d p}{2\pi}$.\\
于是,式子变为
\begin{equation}
    \vspace{\baselineskip}
    %公式编号: 13
    \begin{aligned}
        U(q_f,q_i;\epsilon)& \begin{aligned}&=\sum_p\langle q_f|e^{-i\epsilon\frac{\hat{p}^2}{2m}}|p\rangle\langle p|e^{-i\epsilon V(\hat{q})}|q_i\rangle\end{aligned} \\
        &=\int\frac{\d p}{2\pi}\exp\biggl\{-i\epsilon\biggl[\frac{p^2}{2m}+V(q_i)\biggr]+ip(q_f-q_i)\biggr\} \\
        &\tikzmarknode{13eq2}{\highlight{blue}{$=$}}\bigg(\frac{m}{2\pi i\epsilon}\bigg)^{1/2}\exp\bigg\{i\epsilon\bigg[\frac{m}{2}\frac{(q_{f}-q_{i})^{2}}{\epsilon^{2}}-V(q_{i})\bigg]\bigg\}.
    \end{aligned}
\end{equation}
\begin{tikzpicture}[overlay,remember picture,>=stealth,nodes={align=left,inner ysep=1pt},<-]
    % 对于 "2" 定位点
    \path (13eq2.south) ++ (0,-1.5em) node[anchor=north west,color=blue!67] (eq13/2){\textbf{此处计算需要利用留数定理,附录中给出了mma计算代码}};
    \draw [color=blue!57](13eq2.south) |- ([xshift=-0.3ex,color=blue]eq13/2.south east);
\end{tikzpicture}

为了使对$ p $的积分收敛,我们假设$\epsilon$包含一个小的负虚部,我们发现,指数上的部分恰好是$ i $乘以无穷小作用量$S(q_f,q_i;\epsilon)$,不难发现这个作用量对应着无穷小时间$\epsilon$内$ q_i $和$ q_f $之间以恒定速度的直线路径,于是,我们把式子写为如下形式:
\begin{equation}\label{4eq14}
    U(q_f,q_i;\epsilon)=\left(\frac{m}{2\pi i\epsilon}\right)^{1/2}e^{iS(q_f,q_i;\epsilon)+\mathcal{O}(\epsilon^2)}
\end{equation}
仅有无穷小时间间隔的传播子显然是远远不够的,现在我们想计算出任意时间间隔$t_f-t_i$的传播子$U(q_f,q_i;t_f-t_i)$.我们考虑将时间段$t_f-t_i$分割为$ N $个长为$\epsilon=(t_f-t_i)/N$的等大小的部分,只要最终我们取极限$N\to\infty\;(\epsilon\to0)$,并对全部时间步积分,就得到了任意时间间隔$t_f-t_i$的传播子.
\begin{equation}
    \begin{aligned}
        \begin{aligned}U(q_f,q_i;t_f-t_i)\end{aligned}& \begin{aligned}=\langle q_{f}|e^{-i\hat{H}\epsilon}\cdots e^{-i\hat{H}\epsilon}|q_{i}\rangle\end{aligned} \\
        &=\int\prod_{k=1}^{N-1}\d q_k\langle q_f|e^{-i\hat{H}\epsilon}|q_{N-1}\rangle\langle q_{N-1}|e^{-i\hat{H}\epsilon}|q_{N-2}\rangle\cdots\langle q_1|e^{-i\hat{H}\epsilon}|q_i\rangle \\
        &=\int\prod_{k=1}^{N-1}\d q_k\prod_{k=1}^NU(q_k,q_{k-1};\epsilon)
    \end{aligned}
\end{equation}
其中$q_0=q_i,q_N=q_f$,在\ref{4eq14}中,我们对每个时间步传播子都忽略了$\epsilon^2$的高阶项,对于整体,其导致了阶为$\epsilon$的总误差. \\
现在我们继续考虑,令$N\to\infty$我们有
\begin{equation}
    U(q_f,q_i;t)=\lim_{N\to\infty}\left(\frac{mN}{2\pi it}\right)^{N/2}\int\prod_{k=1}^{N-1}\d q_k e^{iS[q]}
\end{equation}
其中作用量为
\begin{equation}
    S[q]=\sum_{k=1}^NS(q_k,q_{k-1};\epsilon)=\epsilon\sum_{k=1}^N\left[\frac m2\frac{(q_k-q_{k-1})^2}{\epsilon^2}-V(q_{k-1})\right]
\end{equation}
在极限$N\to\infty$,我们可以把求和写做积分:
\begin{equation}
    \begin{aligned}\epsilon\sum_{k=1}^N\frac{m}{2}\frac{(q_k-q_{k-1})^2}{\epsilon^2}&\to\int_{t_i}^{t_f} \d t\frac{m}{2}\dot{q}^2\\\epsilon\sum_{k=1}^NV(q_{k-1})&\to\int_{t_i}^{t_f} \d t V(q)\end{aligned}
\end{equation}
我们使用$q(t)$表示这个粒子的``路径"(trajectory),对于始末位置$q(t_i)=q_i,q(t_f)=q_f$,但这并不意味着在大$ N $极限下的连续性/可微性.我们定义积分测度为如下形式\sidenote[][]{我们可以认为这个形式不过是把一些成套的东西包装成一个微分算符,依赖这种写法,可以简化我们对于路径积分的表述.}:
\begin{equation}
    \mathcal{D}[q]=\lim_{N\to\infty}\left(\frac{mN}{2\pi i\hbar t}\right)^{N/2}\prod_{k=1}^{N-1}\d q_k
\end{equation}
于是传播子可以简化为
\begin{equation}
    U(q_f,q_i;t_f-t_i)=\int_{q(t_i)=q_i}^{q(t_f)=q_f}\mathcal{D}[q]e^{\frac{i}{\hbar}S[q]}
\end{equation}
同时作用量被定义为

\begin{equation}
    \vspace{\baselineskip}
    %公式编号: 21
    \tikzmarknode{21eq1}{\highlight{red}{$S[q]$}}=\int_{t_i}^{t_f}\d t\tikzmarknode{21eq2}{\highlight{blue}{$L(q,\dot{q})$}}
\end{equation}
\begin{tikzpicture}[overlay,remember picture,>=stealth,nodes={align=left,inner ysep=1pt},<-]
    % 对于 "1" 定位点
    \path (21eq1.north) ++ (0,2em) node[anchor=south east,color=red!67] (eq21/1){\textbf{作用量}};
    \draw [color=red!87](21eq1.north) |- ([xshift=-0.3ex,color=red]eq21/1.south west);
    % 对于 "2" 定位点
    \path (21eq2.south) ++ (0,-1.5em) node[anchor=north west,color=blue!67] (eq21/2){\textbf{拉格朗日量}};
    \draw [color=blue!57](21eq2.south) |- ([xshift=-0.3ex,color=blue]eq21/2.south east);
\end{tikzpicture}

现在我们发现,这个作用量的形式与路径$q(t)$的经典作用量的形式一致.

我们不难注意到:积分测度$\mathcal D$中包含的极限是\textbf{发散的},在处理发散问题之前,我们首先尝试讨论其物理含义:对于传播子,我们从公式角度出发观察传播子的形式,我们不难发现这个积分过程只规定了初值条件(初态位置和时间以及末态位置和时间),我们需要对\textbf{所有可能的}路径进行积分(或者说是求和,这两者并没有太大区别),并且对于求和过程,我们最后得出的答案是依赖于作用量的对每条路径的\textbf{加权和}.而按照物理情景的解释,我们有``当一个物理过程\sidenote[][]{我们并没有区别宏观和微观,这是因为其对宏观仍然适用,但由于对应原理,我们不必对宏观现象如此分析.}可以以多种路径进行时,它的概率幅由每种路径的幅值之和给出\sidenote[][]{英文原文:When a
    process can take place in more than one way, its probability amplitude is given by the
    sum of the amplitudes for each way.}".

但是,我们发现这并没有解决发散问题,于是我们要求路径是足够``好"的.即要求动力学项$[q(t+\epsilon)-q(t)]^2/\epsilon$在极限$\epsilon\to0$时收敛,并认为在不满足这个条件的路径会剧烈震荡,其平均值为$ 0 $,即不那么``好"的路径.事实上,这种方法似乎完全看不出来严格的数学依据,但现实如此(这里可以引用那些经典的物理小笑话了,至少我们目前不用去思考如何把这些东西严谨化.)\sidenote[][]{关于这一大段的文字讨论是必须的,其有助于构建量子场论的物理图景,事实上,在这一章我一直在尝试把更多重点放在公式内部并突出显示它,倘若把重点置于一大堆文字中,读者不加以仔细的阅读的话,便很容易错过去.而且,对于物理这一门学科,过于长段的文字很难真的揭示什么内涵,它们往往起到解释公式的作用,或者说,公式才是文章的主体.同样的,这一大段内容我放在了脚注中,同样为了让人们去注意到它.}.

\subsubsection{经典极限}
或许你们发现在上一部分的结尾中,我们并没有像往常一样去省略约化普朗克常数$\hbar$,这有关于经典极限的讨论.

我们所关注的重点路径为``贴近"经典路径的路径,其作用量是静态的.
\begin{equation}
    \left.\frac{\delta S[q]}{\delta q(t)}\right|_{q=q_c}=0
\end{equation}
对于非静态的路径,其意味着作用量的大幅振荡,其平均值为0,或者说,更准确的,传播子$U(q_f,q_i;t_f-t_i)$由路径$q(t)$所主导,其作用量$S[q]$与经典作用量$S_c=S[q_c]$相差一个$\hbar$阶项:$|S-S_c|\lesssim\hbar$.当$|S_c|\gg\hbar $时,这些路径非常接近经典作用量的路径.而对于相反的极限中,同样满足条件$|S-S_c|\lesssim\hbar$的路径就与经典作用量导致的路径截然不同.于是,形式上的经典极限对于与$\hbar\to0$的极限.

\begin{remark}
    \textit{为了得到极限$\hbar\to0$中的传播子,我们写出$q(t)=q_c(t)+r(t)$(我们假设仅存在唯一一条经典轨道.),并把作用量对于$r(t)$展开到第二阶.}
    \begin{equation}
        \begin{aligned}&U(q_f,q_i;t_f-t_i)\\&\simeq e^{\frac{i}{\hbar}S[q_c]}\int_{r(t_i)=0}^{r(t_f)=0}\mathcal{D}[r]\exp\biggl\{\frac{i}{2\hbar}\int_{t_i}^{t_f}\d t \d t' \frac{\delta^2S[q]}{\delta q(t)\delta q(t')}\biggr|_{q=q_c}r(t)r(t')\biggr\}\end{aligned}
    \end{equation}
    \textit{这类积分被称为\textbf{高斯积分},高斯积分并不需要你去思考如何巧妙积分出来,仅需套用公式就能得出答案,而对应的积分表全部列在下面,即式\ref{GaussIntegral}.}
    \begin{equation}
        U(q_f,q_i;t_f-t_i)\simeq e^{\frac{i}{\hbar}S[q_c]}\det\left(\frac{1}{2\pi i\hbar}\frac{\delta^2S[q]}{\delta q(t)\delta q(t')}\bigg|_{q=q_c}\right)^{-1/2}
    \end{equation}
    \textit{我们最后得到的结果所遵循的过程被称为稳态相位近似.}
\end{remark}
\subsubsection{高斯积分表}
如下表,其中$ K $是对称矩阵.
\begin{equation}\label{GaussIntegral}
    \begin{aligned}
        &\int_{-\infty}^{+\infty} e^{-\frac{1}{2}ax^{2}}\dx=\sqrt{\frac{2\pi}{a}} 
        &&\int_{-\infty}^{+\infty}x^{2n}e^{-\frac12ax^2}\dx=\sqrt{2\pi}a^{-\frac{2n+1}2}(2n-1)!! \\
        &\int_{-\infty}^{+\infty}e^{-\frac12ax^2+Jx}\dx=\sqrt{\frac{2\pi}a}e^{\frac{J^2}{2a}} 
        &&\int_{-\infty}^{+\infty} e^{-\frac{1}{2} ax^{2}+iJx} \dx=\sqrt{\frac{2\pi}{a}}e^{-\frac{J^{2}}{2a}} \\
        &\int_{-\infty}^{+\infty}e^{i(\frac{1}{2}ax^{2}+Jx)}\dx=\sqrt{\frac{2\pi i}{a}}e^{-i\frac{J^{2}}{2a}} 
        &&\int_{-\infty}^{+\infty}e^{-\frac{1}{2}x^{T}Kx}\d^{n}x=\sqrt{\frac{(2\pi)^{n}}{\det K}} \\
        &\int_{-\infty}^{+\infty}e^{-\frac{1}{2}x^{T}Kx+Jx}\d^{n}x=\sqrt{\frac{(2\pi)^{n}}{\det K}}e^{\frac{1}{2}JK^{-1}J} 
        &&\int_{-\infty}^{+\infty}e^{-\frac{1}{2}x^{T}Kx+iJx}\d^{n}x=\sqrt{\frac{(2\pi)^{n}}{\det K}}e^{-\frac{1}{2}JK^{-1}J} \\
        &\int_{-\infty}^{+\infty}e^{i(\frac{1}{2}x^{T}Kx+Jx)}\d^{n}x=\sqrt{\frac{(2\pi i)^{n}}{\det K}}e^{-\frac{i}{2}JK^{-1}J}
    \end{aligned}
\end{equation}
需要强调的是,虽然看似高斯积分较为复杂,但都处于高等数学范畴内的积分,如需要记忆的话也可以仅记忆下面这个,其他都可以通过简单的系数替换得到:
\begin{equation}
    \int_{-\infty}^{+\infty}e^{-\frac12ax^2+Jx}\dx=\sqrt{\frac{2\pi}a}e^{\frac{J^2}{2a}}
\end{equation}
\subsubsection{时序算符以及哈密顿量}
对于海森堡绘景下算符的矩阵元,例如算符$\hat{q}(t)=e^{i\hat{H}t}\hat{q}e^{-i\hat{H}t}$
\begin{equation}
    \langle q_f,t_f|\hat{q}(t)|q_i,t_i\rangle=\langle q_f|e^{-i\hat{H}(t_f-t)}\hat{q}e^{-i\hat{H}(t-t_i)}|q_i\rangle 
\end{equation}
如之前那样,我们将时间间隔无限拆分并再次积分,就可以得到矩阵元的路径积分表示:
\begin{equation}
    \langle q_f,t_f|\hat{q}(t)|q_i,t_i\rangle=\int_{q(t_i)=q_i}^{q(t_f)=q_f}\mathcal{D}[q] q(t)e^{iS[q]}
\end{equation}
而为了使用路径积分表示两个不同时间的算符的乘积,我们引入一个新算符,其被定义为\\

\begin{equation}
    \vspace{\baselineskip}
    %公式编号: 29
    \tikzmarknode{29eq1}{\highlight{red}{$T$}}\left[\prod_{i=1}^n\hat{q}_i(t_i)\right]=\sum_p\left(\prod_{j=1}^{n-1}\tikzmarknode{29eq2}{\highlight{blue}{$\Theta(t_{p_j}-t_{p_{j+1}})$}}\right)\epsilon\prod_{k=1}^{n}\hat{q}_{pk}(t_{pk})
\end{equation}
\begin{tikzpicture}[overlay,remember picture,>=stealth,nodes={align=left,inner ysep=1pt},<-]
    % 对于 "1" 定位点
    \path (29eq1.north) ++ (0,2em) node[anchor=south east,color=red!67] (eq29/1){\textbf{时序算符}};
    \draw [color=red!87](29eq1.north) |- ([xshift=-0.3ex,color=red]eq29/1.south west);
    % 对于 "2" 定位点
    \path (29eq2.south) ++ (0,-1.5em) node[anchor=north west,color=blue!67] (eq29/2){\textbf{阶跃函数,即$x\ge0\to\Theta(x)=1;x<0\to\Theta(x)=0$}};
    \draw [color=blue!57](29eq2.south) |- ([xshift=-0.3ex,color=blue]eq29/2.south east);
\end{tikzpicture}

于是我们不难看出,虽然这个算符形式看起来十分复杂,但并没有对原来的算符进行实质性改变,只不过是按照时间顺序把这些算符的乘积进行排序,这也是这个算符被称为\textbf{时序算符}的原因.并且对于式子中的$\epsilon$,其对于玻色子算符总是取$+1$,对于费米子算符,其依赖于前面的排序的奇偶性:对于偶置换取正,对于奇置换取负.\\
我们以最简单的二阶情景为例:
\begin{equation}
    T\hat{q}(t)\hat{q}(t')=\Theta(t-t')\hat{q}(t)\hat{q}(t')+\Theta(t'-t)\hat{q}(t')\hat{q}(t)
\end{equation}
其路径积分表示为
\begin{equation}
    \langle q_f,t_f|T\hat{q}(t)\hat{q}(t')|q_i,t_i\rangle=\int_{q(t_i)=q_i}^{q(t_f)=q_f}\mathcal{D}[q] q(t)q(t')e^{iS[q]}
\end{equation}
回想这一形式的物理意义,我们发现路径积分自然表示了算符按一个时序的乘积的期望.

\subsubsection{路径积分的哈密顿形式}
我们先前的作用量都是有拉格朗日形式给出,现在我们一并写出同样对于哈密顿形式的传播子,积分测度以及作用量:
\begin{equation}
    \begin{aligned}
        U(q_f,q_i;t_f-t_i)&=\lim_{N\to\infty}\int\prod_{k=1}^{N-1}\d q_{k}\int\prod_{k=1}^{N}\frac{\d p_{k}}{2\pi}e^{\sum_{k=1}^{N}\left[ip_{k}(q_{k}-q_{k-1})-i\epsilon\frac{p_{k}^{2}}{2m}-i\epsilon V(q_{k-1})\right]}\\
        &\equiv\int_{q(t_i)=q_i}^{q(t_f)=q_f}\mathcal{D}[p,q] e^{iS[p,q]} \\
        \mathcal{D}[p,q]&=\lim_{N\to\infty}\prod_{k=1}^{N-1}\d q_k\prod_{k=1}^N\frac{\d p_k}{2\pi}\\
        S[p,q]&=\lim_{N\to\infty}\sum_{k=1}^N\Big[p_k(q_k-q_{k-1})-\epsilon\frac{p_k^2}{2m}-\epsilon V(q_{k-1})\Big] \\
        &\equiv\int_{t_i}^{t_f}\dt [p\dot{q}-H(p,q)]
    \end{aligned}
\end{equation}








\subsection{欧式路径积分}
回忆我们在统计力学中的学习,通过配分函数可以得到几乎全部我们所关心的物理量\sidenote{自然,我们在统计力学所讨论的系统一般都是无相互作用的系统,不过对于同样有相互作用的系统,不妨也采取同样的思路},而同样的,对于凝聚态场论,我们仍需要重点关注配分函数.\\
我们首先写出配分函数的形式
\begin{equation}
    Z=\operatorname{Tr}e^{{-\beta\hat{H}}}=\int \d q\langle q|e^{{-\beta\hat{H}}}|q\rangle
\end{equation}
为了揭示经典与量子统计力学直接的联系,我们从演化算符$e^{-i\hat{H}t}$在一个虚时(时间为虚数)的情况下求解,其中虚时$t=-i\beta$(对于SI单位制,$t=-i\beta\hbar$),自然得出凝聚态场论中的配分函数.我们自然看出来一个重要的事情:经典统计力学与凝聚态场论(也称为量子统计力学)之间仅仅差了一个变换$t\to-i\tau$.我们称这个变换为\textbf{Wick转动(Wick rotation)}.即在复时间平面上旋转了$-\pi/2$角.虚时的概念在凝聚态场论中至关重要,我们依次作Wick转动,传播子变为:
\begin{equation}
    \begin{aligned}
        U(q_f,q_i;-i\tau)& \begin{aligned}&=\lim_{N\to\infty}\left(\frac{mN}{2\pi\tau}\right)^{N/2}\int\prod_{k=1}^{N-1}dq_k e^{-\epsilon\sum_{k=1}^N\left[\frac{m}{2}\frac{\left(q_k-q_{k-1}\right)^2}{\epsilon^2}+V\left(q_{k-1}\right)\right]}\end{aligned} \\
        &=\int_{q(0)=q_i}^{q(\tau)=q_f}\mathcal{D}[q] e^{-S_E[q]}.
    \end{aligned}
\end{equation}
对于虚时路径积分,同样的边界满足$q(0)=q_i;q(\tau)=q_f$,对于不同路径的权重同样有作用量给出:\\

\begin{equation}
    \vspace{\baselineskip}
    %公式编号: 35
    \tikzmarknode{35eq1}{\highlight{red}{$S_E[q]$}}=\int_{0}^{\tau}\d\tau'\tikzmarknode{35eq2}{\highlight{blue}{$\left[\frac m2\dot{q}^2+V[q]\right]$}}
\end{equation}
\begin{tikzpicture}[overlay,remember picture,>=stealth,nodes={align=left,inner ysep=1pt},<-]
    % 对于 "1" 定位点
    \path (35eq1.north) ++ (0,2em) node[anchor=south east,color=red!67] (eq35/1){\textbf{欧几里得作用量(欧式作用量)}};
    \draw [color=red!87](35eq1.north) |- ([xshift=-0.3ex,color=red]eq35/1.south west);
    % 对于 "2" 定位点
    \path (35eq2.south) ++ (0,-1.5em) node[anchor=north west,color=blue!67] (eq35/2){\textbf{欧式拉格朗日量(动能和势能同号)}};
    \draw [color=blue!57](35eq2.south) |- ([xshift=-0.3ex,color=blue]eq35/2.south east);
\end{tikzpicture}

经典极限对于虚时路径积分也是同样的,并且通过Wick旋转,我们可以将欧式作用量变为实时作用量:
\begin{equation}
    S_E[q]\xrightarrow[\tau=it]{\text{\textit{Wick 旋转}}}i\int_0^t\d t'\left[-\frac{m}{2}\dot{q}^2+V(q)\right]=-iS[q].
\end{equation}
自然可以继续写出配分函数为
\begin{equation}
    Z=\int \d qU(q,q,-i\beta)=\int_{q(\beta)=q(0)}\mathcal{D}[q]e^{{-S_{E}[q]}}
\end{equation}
作为所有周期为$\beta$的周期性路径的虚时路径积分,虚时算符$\hat{q}(\tau)=e^{\tau\hat{H}}\hat{q}e^{-\tau\hat{H}}$的期望自然写出
\begin{equation}
    \langle\hat{q}(\tau)\rangle=\frac{1}{Z}\mathrm{Tr}{\left[e^{-\beta\hat{H}}\hat{q}(\tau)\right]}=\frac{1}{Z}\int_{q(\beta)=q(0)}\mathcal{D}[q]q(\tau)e^{{-S_{E}[q]}}
\end{equation}
同样的,对于多个不同时间算符的时序乘积,我们利用时序算符按$\tau$进行排序,并有
\begin{equation}
    \langle T_\tau\hat{q}(\tau)\hat{q}(\tau^{\prime})\rangle=\frac1Z\int_{q(\beta)=q(0)}\mathcal{D}[q] q(\tau)q(\tau^{\prime})e^{-S_E[q]}
\end{equation}
如果$\beta\to0$或$\hbar\to0$,传播子$U(q,q;-i\beta\hbar)$就可以仅利用一个时间步($ N=1 $)计算出来\sidenote[][]{为了直观的和德布罗意关系联系起来,此处再次把$\hbar$显现出来}.其导致
\begin{equation}
    \begin{aligned}
        Z_{\mathrm{cl}}& \begin{aligned}&=\frac{1}{\hbar}\sqrt{\frac{m}{2\pi\beta}}\int \dq e^{-\beta V(q)}\end{aligned} \\
        &\equiv\int_{-\infty}^{\infty}\frac{\d p}{2\pi\hbar}\int_{-\infty}^{\infty}\dq \mathrm{exp}\bigg\{-\beta\bigg[\frac{p^{2}}{2m}+V(q)\bigg]\bigg\}
    \end{aligned}
\end{equation}
我们发现配分函数变为经典配分函数,在有限温度$T<\infty$的情况下,如果势能$V(q)$在德布罗意波长$\xi_{th}\sim\hbar/\sqrt{mT}$的数量级下的长度尺度缓慢变化,那么我们认为量子统计力学的配分函数退化为经典统计力学的配分函数,即\textbf{对应原理}.

方程还显示了经典系统的一个非常重要的特性:热力学和动力学是分离的:粒子的位置和动量是独立变量,可以积分出动量(这会产生对自由能的附加贡献作用),并仅以位置变量的形式写出配分函数.相比之下,在量子系统中,坐标和动量是两个不对易算符,因此静力学和动力学不是独立的.这就是为什么配分函数可以与虚时中的演化算子相关联的原因.

\begin{kaobox}[frametitle=题目:空间内存在一个有质量的一维自由粒子]
\textbf{已知空间内存在一个有质量的一维自由粒子,质量为$ m $.请计算出它的传播子}\\
    对于自由粒子,其拉格朗日量中的势能为$ 0 $,仅有动能项,对于位置$ q $的粒子,我们直接写出其拉格朗日量:
    \begin{equation}
        L(q,\dot q)=\frac12m\dot q^2
    \end{equation}
    由于我们需要使用路径积分解决这个问题,我们把作用量按照极限求和的形式写出:
    \begin{equation}
        S=\int_{t_i}^{t_f} \d t\frac{m}{2}\dot{q}^2\to \epsilon\sum_{k=1}^N\frac{m}{2}\frac{(q_k-q_{k-1})^2}{\epsilon^2}
    \end{equation}
    继续写出路径积分形式的传播子:
    \begin{equation}
        U(q_f,q_i;t)=\lim_{N\to\infty}\left(\frac{mN}{2\pi it}\right)^{N/2}\int\prod_{k=1}^{N-1}\d q_k\exp\left\{i\frac{m}{2\epsilon}\sum_{k=1}^N(q_k-q_{k-1})^2\right\}
    \end{equation}
    对变量逐个积分,得到最终结果,计算过程位于右侧附录
    \marginnote[*-10]{
    我们首先对$q_1$进行积分,式子变形为$$\begin{aligned}
            U(q_f,q_i;t) & =\lim_{N\to\infty}\left(\frac{mN}{2\pi it}\right)^{N/2}\int\prod_{k=1}^{N-1}\d q_k \\
            & \times\exp\left\{i\frac{m}{2\epsilon}(q_1-q_i)^2+i\frac{m}{2\epsilon}\sum_{k=2}^N(q_k-q_{k-1})^2\right\}
        \end{aligned}$$
    我们提取出对于$ q_1 $的部分
    $$\begin{aligned}
            U(q_f,q_i;t) & =\lim_{N\to\infty}\left(\frac{mN}{2\pi it}\right)^{N/2}\\
            &\times\int\d q_1\exp\left\{i\frac{m}{2\epsilon}(q_1^2+2q_1q_0)\right\} \\
            & \times\int\prod_{k=2}^{N-1}\d q_k\\&\times\exp\left\{i\frac{m}{2\epsilon}q_0^2+i\frac{m}{2\epsilon}\sum_{k=2}^N(q_k-q_{k-1})^2\right\}
        \end{aligned}$$
    利用高斯积分,我们有
    $$\begin{aligned}
            U(q_f,q_i;t) & =\lim_{N\to\infty}\left(\frac{mN}{2\pi it}\right)^{N/2}\times\sqrt{\frac{2\pi it}{mN}}\exp\{-i\frac{2mq_0^2}{\epsilon}\} \\
            & \times\int\prod_{k=2}^{N-1}\d q_k\\&\times\exp\left\{i\frac{m}{2\epsilon}q_0^2+i\frac{m}{2\epsilon}\sum_{k=2}^N(q_k-q_{k-1})^2\right\}
        \end{aligned}$$
    不断重复对变量逐一使用高斯积分的这一过程,直至得出结果:
    $$U(q_f,q_i;t)=\left(\frac{m}{2\pi it}\right)^{1/2}\exp\left(i\frac{m(q_f-q_i)^2}{2t}\right)$$
    }
    
    \begin{equation}
        U(q_f,q_i;t)=\left(\frac{m}{2\pi it}\right)^{1/2}\exp\left(i\frac{m(q_f-q_i)^2}{2t}\right)
    \end{equation}
    \textbf{这个累次积分的计算很容易出错,我在对应的mathematica文件\\中给出了相应的数值验证代码供自行验证.}
    
\end{kaobox}

\begin{remark}
    事实上,像示例中那样逐步计算高斯积分虽然可行,但十分麻烦,那么有没有那么一种方法可以简化我们的运算负担呢?答案是肯定的.按照下面的五个步骤可以大大简化这个复杂的过程.
    \begin{enumerate}
        \item 表达传播子:将传播子表示为所有满足边界条件的路径积分
        \begin{equation}
            U(x^{\prime},T;x,0)=\int_{x(0)=x}^{x(T)=x^{\prime}}\mathcal{D}[x(t)]e^{iS[x(t)]}
        \end{equation}
        \item 分解路径:将路径分解为经典路径 $x_{\text{cl}}(t)$ 和量子路径(或者称量子涨落) $y(t)$,即 $x(t) = x_{\text{cl}}(t) + y(t)$,其中 $y(0) = y(T) = 0$.
        \item 计算经典作用量 $S_{\text{cl}}$:求解经典运动方程,代入边界条件得到经典路径,并计算其作用量.\\\sidenote[][]{注意:计算后结果尽量避免使用速度$v$或加速度$a$,统一使用$ x $来表示}
        \item 处理涨落积分:对于二次型作用量,涨落部分的路径积分为高斯型,结果为归一化因子 $N(T)$.
        \item 最终结果:传播子为 $U = N(T) e^{i S_{\text{cl}}}$,通过计算或对比确定 $N(T)$ (实际上$ e $指数的部分是一个wick转动的代换)
    \end{enumerate}
    \sidenote[][]{整个过程最难的部分通常为得出归一化因子,在下面我们将通过谐振子的方式来实际认识如何应用这个方法.}
    \begin{proof}
        谐振子的传播子:
        \begin{enumerate}
            \item 拉格朗日量:$L = \frac{1}{2}m\dot{x}^2 - \frac{1}{2}m\omega^2 x^2$
            \item 经典路径满足 $\ddot{x} + \omega^2 x = 0$,其解为
            \begin{equation}
                x_{\text{cl}}(t) = x \cos(\omega t) + \frac{x' - x\cos(\omega T)}{\sin(\omega T)} \sin(\omega t)
            \end{equation}
            \item 经典作用量
            \begin{equation}
                S_{\text{cl}} = \frac{m\omega}{2\sin(\omega T)} \left[(x'^2 + x^2)\cos(\omega T) - 2xx'\right]
            \end{equation}
            \item 当频率$\omega\to0$时,谐振子退化为自由粒子,由自由粒子的结果对比谐振子的量纲得到归一化因子
            \begin{equation}
                N(T) = \sqrt{\frac{m\omega}{2\pi i \sin(\omega T)}}
            \end{equation}
            \item 最终结果:
            \begin{equation}
                \begin{aligned}
                    &U_{\text{osc}}(x', T; x, 0) =\\
                    &\;\sqrt{\frac{m\omega}{2\pi i \sin(\omega T)}} \exp\left( \frac{i m\omega}{2 \sin(\omega T)} \left[ (x'^2 + x^2)\cos(\omega T) - 2xx' \right] \right)
                \end{aligned}
            \end{equation}
        \end{enumerate}
    \end{proof}
\end{remark}
\subsubsection{矢量势的路径积分}
\marginnote[]{要对经典哈密顿量 $H(p,q)$ 进行量子化,仅仅将经典变量 $ p $ 和 $ q $ 替换为相应的量子算符通常是不够的.当这种对应规则导致产生非对易算符$\hat p$​ 和$\hat q$​ 的乘积时,必须通过附加条件(例如哈密顿量的厄米性)来确定算符的顺序.这些困难在路径积分表述中也表现出来:为了尊重算符的正确顺序,路径积分必须被仔细定义.}
直接对经典哈密顿量是较为困难的,我们以矢量势为例来解释这个事情:

我们同样以一个例子来引入这个问题:考虑一个三维空间内的自由粒子,受到$\mathbf{B}=\mathbf{\nabla}\times\mathbf{A}$的磁场作用,其拉格朗日量可以写为\\

\begin{equation}
    \vspace{\baselineskip}
    %公式编号: 53
    L(\mathbf{q},\dot{\mathbf{q}})=\tikzmarknode{53eq1}{\highlight{red}{$\frac{1}{2}m\dot{\mathbf{q}}^2$}}+\tikzmarknode{53eq2}{\highlight{blue}{$e\dot{\mathbf{q}}\cdot\mathbf{A}(\mathbf{q})$}}
\end{equation}
\begin{tikzpicture}[overlay,remember picture,>=stealth,nodes={align=left,inner ysep=1pt},<-]
    % 对于 "1" 定位点
    \path (53eq1.north) ++ (0,2em) node[anchor=south east,color=red!67] (eq53/1){\textbf{粒子动能}};
    \draw [color=red!87](53eq1.north) |- ([xshift=-0.3ex,color=red]eq53/1.south west);
    % 对于 "2" 定位点
    \path (53eq2.south) ++ (0,-1.5em) node[anchor=north west,color=blue!67] (eq53/2){\textbf{广义势能(附录中给出了详细的讨论)}};
    \draw [color=blue!57](53eq2.south) |- ([xshift=-0.3ex,color=blue]eq53/2.south east);
\end{tikzpicture}

同样的,我们可以以此直接给出哈密顿量
\begin{equation}
    H(\mathbf{p},\mathbf{q})=\frac{[\mathbf{p}-e\mathbf{A}(\mathbf{q})]^2}{2m}
\end{equation}

在确定量子哈密顿量时,必须选定算符乘积$\hat{\mathbf{p}} \cdot \mathbf{A}(\hat{\mathbf{q}})$的排序顺序.通过要求哈密顿量保持厄米性这一条件,可推导出
\begin{equation}
    \hat{H}=\frac{1}{2m}\left[\hat{\mathbf{p}}^{2}-e\hat{\mathbf{p}}\cdot\mathbf{A}(\hat{\mathbf{q}})-e\mathbf{A}(\hat{\mathbf{q}})\cdot\hat{\mathbf{p}}+e^{2}\mathbf{A}(\hat{\mathbf{q}})^{2}\right]=\frac{[\hat{\mathbf{p}}-e\mathbf{A}(\hat{\mathbf{q}})]^{2}}{2m}
\end{equation}
该表达式也可通过规范不变性推导得出, 这意味着哈密顿量只能是规范不变组合$\hat{\mathbf{p}} - e\mathbf{A}(\hat{\mathbf{q}})$的函数. 任何其他量子化方案都会在哈密顿量中引入正比于对易子$e[\hat{\mathbf{p}}, \mathbf{A}(\hat{\mathbf{q}})] = -ie\boldsymbol{\nabla} \cdot \mathbf{A}(\hat{\mathbf{q}})$的项, 这将同时破坏厄米性和规范不变性.

由于拉格朗日量中存在附加项$e\dot{\mathbf{q}} \cdot \mathbf{A}(\mathbf{q})$, 我们预期沿无穷小轨迹\sidenote[][]{这里是trajectory,而不是path,之后的文本中也需要进行区分,通常使用轨迹指代trajectory,路径指代path.}$(\mathbf{q}_k, \mathbf{q}_{k-1})$的作用量$S(\mathbf{q}_k, \mathbf{q}_{k-1}; \epsilon)$可表达为
\begin{equation}\label{56}
    S(\mathbf{q}_k,\mathbf{q}_{k-1};\epsilon)=\frac{m}{2}\frac{(\mathbf{q}_k-\mathbf{q}_{k-1})^2}{\epsilon}+e(\mathbf{q}_k-\mathbf{q}_{k-1})\cdot\mathbf{A}(\mathbf{q})
\end{equation}
由于矢量势$\mathbf{A}(\mathbf{q})$的作用不可忽视, 所以我们面临一个关键问题: 这个矢量势究竟该取路径端点$\mathbf{q}_k$、$\mathbf{q}_{k-1}$, 还是中间某个位置的值? 考虑到典型量子路径满足$|\mathbf{q}_k - \mathbf{q}_{k-1}| \sim \sqrt{\epsilon}$的量级关系, $\dot{\mathbf{q}} \cdot \mathbf{A}(\mathbf{q})$的不同离散化方式会引起作用量$\epsilon$量级的变化——这在路径积分中绝不能忽略.

要解决这个难题, 我们可以从哈密顿量的厄米性要求入手. 该条件直接决定了时间演化算符必须满足对称关系$U(\mathbf{q}_k, \mathbf{q}_{k-1}; \epsilon)^* = U(\mathbf{q}_{k-1}, \mathbf{q}_k; -\epsilon)$, 进而导出作用量的反对称性$S(\mathbf{q}_k, \mathbf{q}_{k-1}; \epsilon) = -S(\mathbf{q}_{k-1}, \mathbf{q}_k; -\epsilon)$. 通过仔细分析这些约束条件, 最终唯一可行的方案是将矢量势$\mathbf{A}$取值于路径中点$(\mathbf{q}_k + \mathbf{q}_{k-1})/2$. 当然, 对称化的处理方式不止一种——例如取端点平均$[\mathbf{A}(\mathbf{q}_k) + \mathbf{A}(\mathbf{q}_{k-1})]/2$, 或是构造其他关于$\mathbf{q}_k$和$\mathbf{q}_{k-1}$对称的表达式——这些方案之所以可行, 关键在于它们都满足量子路径积分的收敛性:
要求差异项为$\mathcal{O}[(\mathbf{q}_k - \mathbf{q}_{k-1})^2] = \mathcal{O}(\epsilon)$,而相应作用量的差异项则为可忽略的$\mathcal{O}(\epsilon^{3/2})$量级.

但是我们发现,由哈密顿量厄米性导出的``中点法则"是其中唯一与规范不变性相容的选择. 在规范变换$\mathbf{A} \to \mathbf{A} + \boldsymbol{\nabla}\Lambda$下, 沿路径$(\mathbf{q}_i, t_i) \to (\mathbf{q}_f, t_f)$的作用量只会改变关于矢量势的那一项
\begin{equation}\label{57}
    e\int_{t_i}^{t_f}\d t\dot{\mathbf{q}}\cdot\boldsymbol{\nabla}\Lambda(\mathbf{q})=e\int_{\mathbf{q}_i}^{\mathbf{q}_f}\d\mathbf{q}\cdot\boldsymbol{\nabla}\Lambda(\mathbf{q})=e[\Lambda(\mathbf{q}_f)-\Lambda(\mathbf{q}_i)]
\end{equation}
传播子相应变换为
\begin{equation}
    \begin{aligned}U(\mathbf{q}_f,\mathbf{q}_i;t_f-t_i)\to e^{ie\Lambda(\mathbf{q}_f)}U(\mathbf{q}_f,\mathbf{q}_i;t_f-t_i)e^{-ie\Lambda(\mathbf{q}_i)}\end{aligned}
\end{equation}
同时反映了波函数相位的变化
\begin{equation}
    \varphi(\mathbf{q})\to e^{ie\Lambda(\mathbf{q})}\varphi(\mathbf{q})
\end{equation}
除一个平庸的相位因子外,传播子与规范函数$\Lambda$无关(规范不变性). 在路径积分表述中,作用量的改变量由
\begin{equation}\label{60}
    e\sum_{k=1}^N(\mathbf{q}_k-\mathbf{q}_{k-1})\cdot\nabla\Lambda(\mathbf{u}_k)
\end{equation}
若选择在点$u_k$处计算式\ref{56}中的矢量势$\mathbf{A}$,为验证式\ref{57}与\ref{60}的一致性(此处假设规范函数$\Lambda$具有良好数学性质),可将式\ref{57}重新表述为:
\begin{equation}
    \begin{aligned}\Lambda(\mathbf{q}_f)-\Lambda(\mathbf{q}_i)&=\sum_{k=1}^N[\Lambda(\mathbf{q}_k)-\Lambda(\mathbf{q}_{k-1})]\\&=\sum_{k=1}^N\Big\{(\mathbf{q}_k-\mathbf{q}_{k-1})\cdot\nabla\Lambda(\mathbf{u}_k)\\&+\frac{1}{2}(1-2\theta)[(\mathbf{q}_k-\mathbf{q}_{k-1})\cdot\boldsymbol{\nabla}]^2\Lambda(\mathbf{u}_k)+\mathcal{O}[(\mathbf{q}_k-\mathbf{q}_{k-1})^3]\bigg\}\end{aligned}
\end{equation}
当选取离散化点$\mathbf{u}_k = \mathbf{q}_{k-1} + \theta(\mathbf{q}_k - \mathbf{q}_{k-1})$时(其中$0 \leq \theta \leq 1$),由于经典路径满足$|\mathbf{q}_k - \mathbf{q}_{k-1}|^2 \sim \epsilon$,若未选择中点法则($\theta \neq 1/2$),则方程右端第二项在$\epsilon \to 0$极限下不会消失.唯一可行的方案是取$\theta = 1/2$,即$\mathbf{u}_k = (\mathbf{q}_k + \mathbf{q}_{k-1})/2$(中点法则).在此选择下,方程\ref{57}与\ref{60}完全一致.

为完整论证中点法则的普适性,我们通过显式计算传播子$U(\mathbf{q}_k, \mathbf{q}_{k-1}; \epsilon)$给出另一种证明.将其重构为:\\

\begin{equation}
    \vspace{\baselineskip}
    %公式编号: 62
    \tikzmarknode{62eq1}{\highlight{red}{$U(\mathbf{q}_k,\mathbf{q}_{k-1};\epsilon)$}}=\frac{1}{(2i\pi)^{3/2}}\int\d^3ue^{\frac{i}{2}\mathbf{u}^2}\langle\mathbf{q}_k|\tikzmarknode{62eq2}{\highlight{blue}{$e^{-i\overline{\epsilon}\mathbf{u}\cdot[\hat{\mathbf{p}}-eA(\hat{\mathbf{q}})]}$}}|\mathbf{q}_{k-1}\rangle
\end{equation}
\begin{tikzpicture}[overlay,remember picture,>=stealth,nodes={align=left,inner ysep=1pt},<-]
    % 对于 "1" 定位点
    \path (62eq1.north) ++ (0,0.5em) node[anchor=south east,color=red!67] (eq62/1){\textbf{传播子}};
    \draw [color=red!87](62eq1.north) |- ([xshift=-0.3ex,color=red]eq62/1.south west);
    % 对于 "2" 定位点
    \path (62eq2.south) ++ (0,-0.5em) node[anchor=north west,color=blue!67] (eq62/2){\textbf{这里$\bar{\epsilon}=\sqrt{\epsilon/m}$}};
    \path (62eq2.south) ++ (0,-1.5em) node[anchor=north west,color=blue!67] (eq62/2){\textbf{直接对指数分解会造成高阶项的误差扩大}};
    \draw [color=blue!57](62eq2.south) |- ([xshift=-0.3ex,color=blue]eq62/2.south east);
\end{tikzpicture}\\

为了避免直接指数分解造成的误差,我们再次应用
\begin{equation}
    e^{\bar{\epsilon}(\hat{A}+\hat{B})}=e^{\frac{\bar{\epsilon}}{2}\hat{B}}e^{\bar{\epsilon}\hat{A}}e^{\frac{\bar{\epsilon}}{2}\hat{B}}+\mathcal{O}(\bar{\epsilon}^3)
\end{equation}
于是得到:
\marginnote[]{
    \begin{kaobox}[frametitle=误差处理]
    处理误差项要注意误差余项!
    \end{kaobox}
}
\begin{equation}
    \begin{aligned}
        \langle\mathbf{q}_k|e^{-i\bar{\epsilon}\mathbf{u}\cdot[\hat{\mathbf{p}}-eA(\hat{\mathbf{q}})]}|\mathbf{q}_{k-1}\rangle&=e^{\frac{i}{2}e\bar{\epsilon}\mathbf{u}\cdot\mathbf{A}(\mathbf{q}_k)}\\
        &\times \langle\mathbf{q}_k|e^{-i\bar{\epsilon}\mathbf{u}\cdot\hat{\mathbf{p}}+\mathcal{O}(\epsilon^{3/2})}|\mathbf{q}_{k-1}\rangle \\
        &\times e^{\frac{i}{2}e\bar{\epsilon}\mathbf{u}\cdot\mathbf{A}(\mathbf{q}_{k-1})}
    \end{aligned}
\end{equation}

现在忽略高阶项误差,利用完备性关系插入两个单位算符,自然得出\marginnote[]{
    \begin{kaobox}[frametitle=计算提示]
        此处利用了$\langle\mathbf{p}|e^{-i\mathbf{u}\cdot\hat{\mathbf{p}}}|\mathbf{p}^{\prime}\rangle=e^{-i\mathbf{u}\cdot\mathbf{p}^{\prime}}\delta(\mathbf{p}-\mathbf{p}^{\prime})$和$\langle \mathbf{q} | \mathbf{p} \rangle = \frac{1}{(2\pi)^{3/2}} e^{i\mathbf{p}\cdot\mathbf{q}}$
        以及关系$ \int\frac{\d^3p}{(2\pi)^3}e^{i\mathbf{p}\cdot(\mathbf{q})}=\delta^{(3)}(\mathbf{q})$
    \end{kaobox}
}
\begin{equation}
    \begin{aligned}
        \langle\mathbf{q}_{k}|e^{-i\bar{\epsilon}\mathbf{u}\cdot\hat{\mathbf{p}}}|\mathbf{q}_{k-1}\rangle
        &=\sum_{\mathbf{p},\mathbf{p}^{\prime}}\langle\mathbf{q}_k|\mathbf{p}\rangle\langle\mathbf{p}|e^{-i\overline{\epsilon}\mathbf{u}\cdot\hat{\mathbf{p}}}|\mathbf{p}^{\prime}\rangle\langle\mathbf{p}^{\prime}|\mathbf{q}_{k-1}\rangle\\
        &=\int\frac{\d^3p}{(2\pi)^3}e^{i\mathbf{p}\cdot(\mathbf{q}_k-\mathbf{q}_{k-1}-\bar{\epsilon}\mathbf{u})}\\
        &=\frac{1}{\bar{\epsilon}^3}\delta\left(\frac{\mathbf{q}_k-\mathbf{q}_{k-1}}{\bar{\epsilon}}-\mathbf{u}\right)
    \end{aligned}
\end{equation}
传播子自然可以得出:
\begin{equation}
    \begin{aligned}U(\mathbf{q}_k,\mathbf{q}_{k-1};\epsilon)&= \left(\frac{m}{2i\pi\epsilon}\right)^{3/2}\exp\biggl\{i\epsilon\biggl[\frac{m}{2}\frac{(\mathbf{q}_k-\mathbf{q}_{k-1})^2}{\epsilon^2}\\&+\frac{e}{2}\frac{\mathbf{q}_k-\mathbf{q}_{k-1}}{\epsilon}\cdot\bigl(\mathbf{A}(\mathbf{q}_k)+\mathbf{A}(\mathbf{q}_{k-1})\bigr)\biggr]\biggr\}\end{aligned}
\end{equation}

我们注意到, 当用表达式 $[A(q_k) + A(q_{k-1})]/2$ 近似表示矢量势$A(q)$ 时(如前所述, 这等同于中点法则), 包含作用量与虚数单位 $i$ 相乘的指数项会呈现特定形式\sidenote[][]{我们也可以直接验证, 由传播子$U(q_k, q_{k-1}; \epsilon)$产生的波函数时间演化结果与哈密顿量所描述的演化一致. 这为路径积分的有效性提供了确凿证明.}.
\subsubsection{多粒子系统}
只需稍加扩展, 我们就能为包含$N$个通过两体势$v(q_i-q_j)$相互作用的粒子系统(为简化考虑一维情形)写出配分函数或传播子的路径积分表达式. 希尔伯特空间被定义为张量积空间$\mathcal{H}\otimes\cdots\otimes\mathcal{H}$($\mathcal{H}$表示单粒子希尔伯特空间)的子空间, 其中包含所有根据粒子的量子统计特性进行适当对称化或反对称化的$N$粒子态. 配分函数可表示为\sidenote[][]{这里给出了一种表述,同样也有使用$|q_1\cdots q_N\ra=|q_1\ra\otimes\cdots\otimes|q_N\ra$的表述方法的,这两种记号均可.}:

\begin{equation}
    \vspace{\baselineskip}
    %公式编号: 67
    Z=\frac{1}{N!}\tikzmarknode{67eq1}{\highlight{red}{$\sum_{P\in S_N}$}}\epsilon_P\int \d q_1\cdots \d q_N\tikzmarknode{67eq2}{\highlight{blue}{$(q_1\cdots q_N|e^{-\beta\hat{H}}|q_{P(1)}\cdots q_{P(N)})$}}
\end{equation}
\begin{tikzpicture}[overlay,remember picture,>=stealth,nodes={align=left,inner ysep=1pt},<-]
    % 对于 "1" 定位点
    \path (67eq1.north) ++ (0,0.5em) node[anchor=south east,color=red!67] (eq67/1){\textbf{对所有可能的排列求和}};
    \draw [color=red!87](67eq1.north) |- ([xshift=-0.3ex,color=red]eq67/1.south west);
    % 对于 "2" 定位点
    \path (67eq2.south) ++ (0,-0.5em) node[anchor=north west,color=blue!67] (eq67/2){\textbf{这里使用$|q_1\cdots q_N)=|q_1\ra\otimes\cdots\otimes|q_N\ra$}};
    \draw [color=blue!57](67eq2.south) |- ([xshift=-0.3ex,color=blue]eq67/2.south east);
\end{tikzpicture}
对于玻色子系统, 置换因子$\epsilon_P$取值为1; 对于费米子系统, $\epsilon_P$则等于排列$P$的置换符号(即奇排列取-1, 偶排列取1).

类似于单粒子情况,我们再次利用完备性关系:
\begin{equation}
    \int \d q_1\cdots \d q_N|q_1\cdots q_N)(q_1\cdots q_N|=1
\end{equation}
对于每个时间步,我们有路径积分
\begin{equation}
    \begin{aligned}Z=\frac{1}{N!}\sum_{P\in S_N}\epsilon_P\int_{q_i(\beta)=q_{P(i)}(0)}\mathcal{D}[q]e^{-S_E[q]}\end{aligned}
\end{equation}
以及对于欧式作用量
\begin{equation}
    S_E[q]=\int_0^\beta\d\tau\bigg[
    \frac{m}{2}\sum_{i=1}^N\dot{q}_i^2+
    \sum_
    {\substack
        {i,j=1\\(i<j)}
    }^Nv(q_i-q_j)
    \bigg]
\end{equation}
%\bigg[\bigg] 大括号,\substack{...}求和号下换行

该表达式与单粒子情形具有明显相似性, 但对于多粒子系统的研究并不适用. 在后续内容中, 我们将再次引入路径积分(更准确地说, 配分函数的泛函积分表示), 这种基于相干态、通过对所有场构型进行加权积分(权重由适当的作用量决定)的表述方式, 最终被证明是更为便捷的数学框架. 
这成为(量子)统计物理与场论的\textbf{核心特征}: 具有无限自由度的系统自然地通过场(而非所有粒子的坐标集合)进行描述. 在深入探究量子多体系统之前, 我们将在下节讨论场论的一个基本范例. 

\subsection{统计物理中的泛函积分}
在这一节中, 我们探讨量子谐振弦\sidenote[][]{在粒子视角下, `谐振'系统(即拉格朗日量在变量中呈二次型的系统)对应无相互作用的粒子体系, 而'非谐振'系统则包含粒子间的相互作用. 这些粒子未必对应裸粒子, 而可能指代元激发(这一概念将在后续章节详细阐述).而弦则区别于之前的单个粒子,注意这里并不是``弦论"当中的弦,应注意区分.}的泛函积分方法. 这个简单模型已经展现出(更复杂的, 即非谐)量子系统场论的诸多核心特征. 
\subsubsection{经典谐振弦}
我们考虑一个具有平衡长度 $L$ 和线质量密度 $\rho$ 的一维谐振弦系统. 该系统的动力学由拉格朗日量描述, 其中 $\phi(x, t)$ 表示位于平衡构型中 $x$ 至 $x+\d x$ 位置之间, 质量为 $\rho \d x$ 的质元相对于平衡位置的位移.  
\begin{equation}
    L[\phi]=\int_0^L\d x\mathcal{L}(\partial_x\phi,\dot{\phi})
\end{equation}
其中$\mathcal{L}$称为拉格朗日密度.

\begin{equation}
    \vspace{\baselineskip}
    %公式编号: 72
    \mathcal{L}(\partial_x\phi,\dot{\phi})=\tikzmarknode{72eq1}{\highlight{red}{$\frac{1}{2}\rho\dot{\phi}^2$}}-\tikzmarknode{72eq2}{\highlight{blue}{$\frac{1}{2}\kappa(\partial_x\phi)^2$}}
\end{equation}
\begin{tikzpicture}[overlay,remember picture,>=stealth,nodes={align=left,inner ysep=1pt},<-]
    % 对于 "1" 定位点
    \path (72eq1.north) ++ (0,0.5em) node[anchor=south east,color=red!67] (eq72/1){\textbf{对应连续介质的动能密度}};
    \draw [color=red!87](72eq1.north) |- ([xshift=-0.3ex,color=red]eq72/1.south west);
    % 对于 "2" 定位点
    \path (72eq2.south) ++ (0,-0.5em) node[anchor=north west,color=blue!67] (eq72/2){\textbf{反映Hooke定律的连续形式的势能}};
    \draw [color=blue!57](72eq2.south) |- ([xshift=-0.3ex,color=blue]eq72/2.south east);
\end{tikzpicture}


该谐振弦可视为一维``晶体"的低能或长波极限——该晶体由质量$m=\rho a$的质点构成(平衡态间距为$a$),质点间通过劲度系数$\kappa/a$的弹簧连接\sidenote[][]{一维晶体系统的拉格朗日量可表述为$L(q_i,\dot{q}_i)=\frac{1}{2}\sum_{i=1}^n\left[m\dot{q}_i^2 - k_s(q_{i+1}-q_i)^2\right]$, 其中$q_i$表示第$i$个原子相对于其平衡位置的位移. 该系统的简正模(声子)以频率$\omega_k = \sqrt{2k_s/m}\left(1 - \cos ka\right)^{1/2}$传播. 在长波极限$|k|a \ll 1$下, 可恢复连续谐振弦的线性频谱关系$\omega = c|k|$, 其中波速$c = \sqrt{\kappa/\rho}$.}.

由欧拉-拉格朗日方程
\begin{equation}
    \frac{\d}{\d t}\frac{\partial\mathcal{L}(\partial_x\phi,\dot{\phi})}{\partial\dot{\phi}}+\frac{\d}{\d x}\frac{\partial\mathcal{L}(\partial_x\phi,\dot{\phi})}{\partial(\partial_x\phi)}=0
\end{equation}
可以求解出波函数(求解时注意把$\partial_x\phi$和$\dot \phi$作为独立变量分别求导):
\begin{equation}
    \rho\ddot{\phi}-\kappa\partial_x^2\phi=0
\end{equation}
解的形式为平面波 $\phi(x,t) \propto \exp\{i(kx - c|k|t)\} + \text{c.c.}$,其中$c.c.$为复共轭项,其传播速度为 $c = \sqrt{\kappa/\rho}$.在周期性边界条件 $\phi(x+L,t) = \phi(x,t)$ 下,波数 $k$ 取分立值 $k = p\frac{2\pi}{L}$ ($p$ 为整数).

\subsubsection{量子谐振弦}
$\phi(x,t)$的共轭动量定义为
\begin{equation}
    \Pi(x,t)=\frac{\partial\mathcal{L}(\partial_x\phi,\dot{\phi})}{\partial\dot{\phi}(x,t)}=\rho\dot{\phi}(x,t)
\end{equation}
相应的哈密顿量则定义为
\begin{equation}
    H=\int_0^L\d x\left[\Pi\dot{\phi}-\mathcal{L}(\partial_x\phi,\dot{\phi})\right]=\int_0^L\d x\left[\frac{\Pi^2}{2\rho}+\frac{1}{2}\kappa(\partial_x\phi)^2\right]
\end{equation}
为了对弦进行量子化,我们将$\phi$和$\Pi$提升为满足对易关系的算符
\begin{equation}
    [\hat{\phi}(x),\hat{\Pi}(x^{\prime})]=i\delta(x-x^{\prime})
\end{equation}
由于满足周期性关系,显然引入傅里叶变换进行处理是极其有效的:
\begin{equation}
    \begin{aligned}
        \hat{\phi}(k)&=\frac{1}{\sqrt{L}}\int_0^L\d xe^{-ikx}\hat{\phi}(x)=\hat{\phi}^\dagger(-k)\\
        \hat{\Pi}(k)&=\frac{1}{\sqrt{L}}\int_0^L\d xe^{-ikx}\hat{\Pi}(x)=\hat{\Pi}^\dagger(-k)
    \end{aligned}
\end{equation}
傅里叶变换后的场算符满足对易关系
\begin{equation}
    [\hat{\phi}(k),\hat{\phi}(k^{\prime})]=[\hat{\Pi}(k),\hat{\Pi}(k^{\prime})]=0\quad\text{以及}\quad[\hat{\phi}(k),\hat{\Pi}^\dagger(k^{\prime})]=i\delta_{k,k^{\prime}}
\end{equation}
\marginnote[]{在这里我们发现,场是一个无穷多自由度的物理量,于是坐标提升为位移场,动量变为共轭动量,但是它们之间满足的对易关系不变.在经典物理学中,我们有百谈不厌的`质点'的概念,但是在场论中,我们描述一个粒子就不能够通过单一的坐标和动量来表述,而是使用位移场算符和共轭动量作为替代,在之后我们会逐步展示这类语言的必要性}
这使我们能够将哈密顿量写成谐振子的求和形式\sidenote{此处使用傅里叶变换把连续化的场变为多个谐振子的叠加}
\begin{equation}
    \hat{H}=\sum_k\left[\frac{1}{2\rho}\hat{\Pi}^\dagger(k)\hat{\Pi}(k)+\frac{1}{2}\rho\omega_k^2\hat{\phi}^\dagger(k)\hat{\phi}(k)\right]
\end{equation}
其中$\omega_k = c|k|$,且求和覆盖所有满足边界条件$e^{ikL}=1$的波矢$k = p\frac{2\pi}{L}$($p\in\mathbb{Z}$).通过引入产生算符和湮灭算符$\hat{a}_k^\dagger$和$\hat{a}_k$来辅助对角化哈密顿量$\hat{H}$
\begin{equation}
    \begin{gathered}\hat{a}(k)\begin{aligned}=\sqrt{\frac{\rho\omega_k}{2}}\left[\hat{\phi}(k)+\frac{i}{\rho\omega_k}\hat{\Pi}(k)\right],\end{aligned}\\\hat{a}^{\dagger}(k)=\sqrt{\frac{\rho\omega_k}{2}}\left[\hat{\phi}^\dagger(k)-\frac{i}{\rho\omega_k}\hat{\Pi}^\dagger(k)\right].\end{gathered}
\end{equation}
产生湮灭算符具有如下对易关系
\begin{equation}
    [\hat{a}(k),\hat{a}(k^{\prime})]=[\hat{a}^\dagger(k),\hat{a}^\dagger(k^{\prime})]=0,\quad[\hat{a}(k),\hat{a}^\dagger(k^{\prime})]=\delta_{k,k^{\prime}}
\end{equation}
显然两个场算符可以通过我们引入的产生算符和湮灭算符重写为如下形式\sidenote[][]{这里通过系数 $\sqrt{\frac{1}{2\rho \omega_k}}$ 和 $i\sqrt{\frac{\rho \omega_k}{2}}$ 的调整,确保算符满足正则对易关系 $[\hat{\phi}(k), \hat{\Pi}(k')] = i\delta(k-k')$}.
\begin{equation}
    \begin{aligned}
        \hat{\phi}(k) &= \sqrt{\frac{1}{2\rho \omega_k}} \left( \hat{a}(k) + \hat{a}^\dagger(-k) \right)\\
        \hat{\Pi}(k) &= i \sqrt{\frac{\rho \omega_k}{2}} \left( \hat{a}^\dagger(-k) - \hat{a}(k) \right)
    \end{aligned}
\end{equation}
将 $\hat{\phi}(k)$ 和 $\hat{\Pi}(k)$ 的表达式代入原哈密顿量,展开后利用算符的对易关系化简\sidenote[][]{这里很多教材都采取`one easily finds'等叙述方法跳过了对角化这一块的计算,实际上对于初学者还是具有很大的门槛的,在相应的mathematica文件中给出了计算代码(3/12,尚未搞明白NCAlgebra包,目前代码还不可用)}.

最终我们得到:
\begin{equation}
    \hat{H}=\sum_k\omega_k\left(\hat{a}_k^\dagger\hat{a}_k+\frac{1}{2}\right)
\end{equation}
定义以$ k $为下标的谐振子本征态为
\begin{equation}
    \begin{aligned}|n_k\rangle&=\frac{1}{\sqrt{n_k!}}\left(\hat{a}_k^\dagger\right)^{n_k}|0\rangle,\\\omega_k\left(\hat{a}_k^\dagger\hat{a}_k+\frac{1}{2}\right)|n_k\rangle&=\omega_k\left(n_k+\frac{1}{2}\right)|n_k\rangle\end{aligned}
\end{equation}
其中 $|0\rangle=|n_k=0\rangle$ 是归一化真空态: $\hat{a}_k|0\rangle=0$ 且 $\langle0|0\rangle=1$. 阶梯算符 $\hat{a}_k$ 和 $\hat{a}_k^\dagger$ 满足
\begin{equation}
    \begin{aligned}&\hat{a}_k|n_k\rangle=\sqrt{n_k}|n_k-1\rangle\\&\hat{a}_k^\dagger|n_k\rangle=\sqrt{n_k+1}|n_k+1\rangle\end{aligned}
\end{equation}
可视为动量$k$的元激发(声子)的湮灭算符和产生算符,其中$n_k=\langle n_k|\hat{a}_k^\dagger\hat{a}_k|n_k\rangle$表示态$|n_k\rangle$中的声子数目.哈密顿量的本征态可通过态$\left|n_k\right\rangle$的张量积构造获得,
\begin{equation}
    \begin{aligned}|n_{k_1}\cdots n_{k_i}\cdots\rangle&=|n_{k_1}\rangle\otimes\cdots\otimes|n_{k_i}\rangle\otimes\cdots=\prod_i\frac{\left(\hat{a}_{k_i}^\dagger\right)^{n_{k_i}}}{\sqrt{n_{k_i}!}}|\mathrm{vac}\rangle\\\hat{H}|n_{k_1}\cdots n_{k_i}\cdots\rangle&=\left[\sum_j\left(n_{k_j}+\frac{1}{2}\right)\omega_{k_j}\right]|n_{k_1}\cdots n_{k_i}\cdots\rangle\end{aligned}
\end{equation}
其中$|vac\rangle$表示真空态, 即满足$\hat{a}_k|vac\rangle=0$(对所有$k$)的归一化态. 本征态$|n_{k_1}\cdots n_{k_i}\cdots\rangle$具有总声子数$n=\sum_i n_{k_i}$. 希尔伯特空间可写为直和$\mathcal{H}=\mathcal{H}_0\oplus\mathcal{H}_1\oplus\cdots\oplus\mathcal{H}_n\oplus\cdots$, 其中$\mathcal{H}_n$表示含$n$个声子的希尔伯特空间. 这类希尔伯特空间通常称为Fock空间, 也即是二次量子化形式的基础.

泛函积分表述可通过遵循前面讨论的路径积分相同步骤获得. 其中$\hat{\phi}(x)$和$\hat{\Pi}(x)$扮演单粒子位置算符$\hat{q}$与动量算符$\hat{p}$的角色(唯一区别在于它们都由$x$标记). 可以引入满足$\hat{\phi}(x)|\phi\rangle=\phi(x)|\phi\rangle$和$\hat{\Pi}(x)|\Pi\rangle=\Pi(x)|\Pi\rangle$的态矢$|\phi\rangle$与$|\Pi\rangle$. 由于这两组态矢$\{|\phi\rangle\}$和$\{|\Pi\rangle\}$构成希尔伯特空间的完备基组, 我们有以下完备性关系:
\begin{equation}
    \begin{aligned}\mathcal{N}\lim_{a\to0}\int\prod_{l=0}^{L/a}\d\phi(la)|\phi\rangle\langle\phi|&=1\\\mathcal{N}^{\prime}\lim_{a\to0}\int\prod_{l=0}^{L/a}\d\Pi(la)|\Pi\rangle\langle\Pi|&=1\end{aligned}
\end{equation}
此处我们对一维弦进行离散化处理, 连续变量$x$变为离散变量$x=la$. 当$a\to0$时恢复为谐振弦\sidenote[][]{此处有多种表述方式,如谐振弦(string),谐振链(chain)等等,此后统一使用弦作为表达(本书中不会涉及弦论)}. 此步骤对于正确定义完备性关系及泛函积分测度是必要的. $\mathcal{N}$和$\mathcal{N}^\prime$是不重要的归一化常数, 它们对配分函数贡献一个因子, 但不影响期望值. 后续推导中将予以忽略.

为将配分函数表示为泛函积分, 我们使用第一个完备性关系得到
\begin{equation}
    Z=\sum_n\langle n|e^{-\beta\hat{H}}|n\rangle=\int \d\phi\sum_n\langle n|e^{-\beta\hat{H}}|\phi\rangle\langle\phi|n\rangle=\int \d\phi\langle\phi|e^{-\beta\hat{H}}|\phi\rangle
\end{equation}
其中$\d\phi\equiv\prod_{l=0}^{L/a}\d\phi(la)$且$\{|n\rangle\}$表示哈密顿量$\hat{H}$的完备基组. 我们现按照开头使用的方法, 将虚时$\beta$分割为$N$个无穷小步长$\epsilon=\beta/N$\sidenote[][]{虚时常用$\tau$或$\beta$表述,并无区别,并且区分这里$k$是离散时间变量,不是之前的动量.}.
\begin{equation}\label{2.90}
    Z=\int\prod_{k=1}^N\d\phi_k\prod_{k=1}^N\langle\phi_k|e^{-\epsilon\hat{H}}|\phi_{k-1}\rangle
\end{equation}
其中$\phi_0=\phi_N$,对于$\epsilon\to0$,我们可以近似得出
\begin{equation}
    \langle\phi_k|e^{-\epsilon\hat{H}}|\phi_{k-1}\rangle=\langle\phi_k|\exp\left\{-\frac{\epsilon}{2\rho}\int \dx\hat{\Pi}^2\right\}\exp\left\{-\frac{\epsilon\kappa}{2}\int \dx(\partial_x\hat{\phi})^2\right\}|\phi_{k-1}\rangle
\end{equation}
同样的,我们再次利用完备性关系插入``$ 1 $''
\begin{equation}
    \begin{aligned}\langle\phi_k|e^{-\epsilon\hat{H}}|\phi_{k-1}\rangle&=\int \d\Pi_k\exp\left\{-\frac{\epsilon}{2\rho}\int \dx\Pi_k^2-\frac{\epsilon\kappa}{2}\int \dx(\partial_x\phi_{k-1})^2\right\}\langle\phi_k|\Pi_k\rangle\langle\Pi_k|\phi_{k-1}\rangle\\&=\int \d\Pi_k\exp\left\{-\frac{\epsilon}{2\rho}\int \dx\Pi_k^2-\frac{\epsilon\kappa}{2}\int \dx(\partial_x\phi_{k-1})^2+i\int \dx\Pi_k(\phi_k-\phi_{k-1})\right\}
    \end{aligned}
\end{equation}
\marginnote[]{此处使用$\langle\phi_k|\Pi_k\rangle=\exp\left\{i\int \dx\Pi_k\phi_k\right\}$结果忽略了一个不相关的乘以公式的系数(即乘性系数)}
将完备性关系代入原本配分函数\ref{2.90}并取连续时间极限后,配分函数变为
\begin{equation}
    Z=\int_{\phi(x,\beta)=\phi(x,0)}\mathcal{D}[\Pi,\phi]\exp\left\{-\int_0^\beta \d\tau\int_0^L\dx\left[\frac{\Pi^2}{2\rho}+\frac{\kappa}{2}(\partial_x\phi)^2-i\Pi\dot{\phi}\right]\right\}
\end{equation}
配分函数可表示为虚时内对实场$\phi(x,\tau)$和$\Pi(x,\tau)$的泛函积分. 对$\Pi$进行高斯积分后, 可得  
\begin{equation}\label{eq:2.91}
    Z=\int_{\phi(x,\beta)=\phi(x,0)}\mathcal{D}[\phi]e^{-S_E[\phi]}
\end{equation}
其中欧式作用量
\begin{equation}
    S_E[\phi]=\frac{1}{2}\int_0^\beta \d\tau\int_0^L\dx\left[\rho\dot{\phi}^2+\kappa(\partial_x\phi)^2\right]
\end{equation}
这里的积分测度被定义为
\begin{equation}
    \mathcal{D}[\phi]=\lim_{N\to\infty}\lim_{a\to0}\prod_{k=1}^N\prod_{l=0}^{L/a}\d\phi(la,k\beta/N)
\end{equation}
(此处同样忽略一个乘性常数). 进行Wick转动$(\tau=it)$后, 可见实时作用量由经典作用量$S[\phi]=\int \d t\int_0^L \dx\mathcal{L}(\partial_x\phi,\dot{\phi})$给出.  

由于场$\phi(x,\tau)$是周期性的,因此可以展开为傅里叶级数
\begin{equation}
    \begin{aligned}
        \phi(x,\tau)&=\frac{1}{\sqrt{\beta}}\sum_{\omega_n}e^{-i\omega_n\tau}\phi(x,i\omega_n)\\
        \phi(x,i\omega_n)&=\frac{1}{\sqrt{\beta}}\int_0^\beta \d\tau e^{i\omega_n\tau}\phi(x,\tau)
    \end{aligned}
\end{equation}
其中离散频率
\begin{equation}
    \omega_n=\frac{2\pi}{\beta}n\quad(n\in\mathbb{Z})
\end{equation}
被称为\textbf{松原频率(Matsubara frequency;虚时频率)}\\
傅里叶变换后的场$\phi(k,i\omega_n)$对角化了作用量
\begin{equation}
    S_E[\phi]=\frac{\rho}{2}\sum_{k,\omega_n}\phi(-k,-i\omega_n)\left(\omega_n^2+\omega_k^2\right)\phi(k,i\omega_n)
\end{equation}
因此配分函数可表示为高斯积分的乘积. 注意由于$\phi(x,\tau)$是实场, 满足$\phi(-k,-i\omega_n)=\phi^*(k,i\omega_n)$.  

从量子谐振弦的泛函积分法可见更复杂体系的量子场论的许多特点:
\begin{enumerate}
    \item 之前的使用泛函积分表示的配分函数\ref{eq:2.91},由于作用量与场呈二次关系,因此可以精确地进行泛函积分,积分得出的结果如下$$Z=\prod_k\frac{e^{-\frac{\beta}{2}\omega_k}}{1-e^{-\beta\omega_k}}$$这是脱耦谐振子组的标准结果\sidenote{显然\textbf{脱耦}这一条件很关键,对于复杂耦合的系统一般不能精确表示.}.更一般地,从哈密顿量得到的所有结果也可以从泛函积分\ref{eq:2.91}导出.当作用量不是二次的时候,例如当它包含一个高阶势能项:$$S_{\mathrm{int}}[\phi]=\int_0^\beta \d\tau\int_0^L\d xV(\phi)$$例如其势能 $V (\phi) $为四次或更高次项,此时泛函积分无法精确计算,必须借助近似方法,诸如微扰展开或鞍点近似等方法.
    \item 
    \item 
\end{enumerate}







\subsubsection{泛函积分形式}
1
\section{二次量子化与格林函数}
1



    %\pagelayout{wide} % No margins
    %
    %\pagelayout{margin} % Restore margins
    %
    %
    %\pagelayout{wide} % No margins
    %\pagelayout{margin} % Restore margins
    
    
    \appendix % From here onwards, chapters are numbered with letters, as is the appendix convention
    
    %\pagelayout{wide} % No margins
    %
    %\pagelayout{margin} % Restore margins
    
    \setchapterstyle{lines}
\labch{appendix}
%\blinddocument


\chapter{补充内容}
\section{带电粒子的拉格朗日量}
我们知道拉格朗日量往往是用来描述保守体系的,面对像洛伦兹力这类非保守力,我们如果仍维持保守体系的拉格朗日方程不变,势必要引入广义势能.

我们根据电动力学的知识写出带电粒子在电磁场中的洛伦兹力:
\begin{equation}
    \boldsymbol F=q(\boldsymbol E+\boldsymbol v\times \boldsymbol  B)
\end{equation}
电磁场又满足麦克斯韦方程组
\begin{equation}
    \left\{\begin{array}{l}\nabla \times \boldsymbol{E}+\dfrac{\partial \boldsymbol{B}}{\partial t}=0 \\ \nabla \cdot \boldsymbol{E}=\rho / \varepsilon_{0} \\ \nabla \times \boldsymbol{B}-\mu_{0} \varepsilon_{0} \dfrac{\partial \boldsymbol{E}}{\partial t}=\mu_{0} \boldsymbol{j} \\ \nabla \cdot \boldsymbol{B}=0\end{array}\right.
\end{equation}
利用
\begin{equation}
    \boldsymbol{B}=\nabla \times \boldsymbol{A}
\end{equation}
代入麦克斯韦方程组的第一个式子,得到$\nabla \times\left(\boldsymbol{E}+\frac{\partial \boldsymbol{A}}{\partial t}\right)=0$,自然定义出标量势

\begin{equation}
    -\nabla \varphi=E+\frac{\partial \boldsymbol{A}}{\partial t}
\end{equation}
于是洛伦兹力可以重写为
\begin{equation}
    \boldsymbol{F}=q\left[-\nabla \varphi-\frac{\partial \boldsymbol{A}}{\partial t}+v \times(\nabla \times \boldsymbol{A})\right]
\end{equation}
现在我们需要将其写为如下形式来得到广义势能
\begin{equation}
    Q_{\alpha}=-\frac{\partial U}{\partial q_{\alpha}}+\frac{\mathrm d}{\mathrm d t} \frac{\partial U}{\partial \dot{q}_{\alpha}}
\end{equation}
首先我们写出其分量形式
\begin{equation}
    \begin{aligned} &(\nabla \varphi)_{x}=\frac{\partial \varphi}{\partial x} \\ &[v \times(\nabla \times \boldsymbol A)]_{x}=v_{y}\left(\frac{\partial A_{y}}{\partial x}-\frac{\partial A_{x}}{\partial y}\right)-v_{z}\left(\frac{\partial A_{x}}{\partial z}-\frac{\partial A_{z}}{\partial x}\right) \\ &\left(\frac{\partial \boldsymbol{A}}{\partial t}\right)_{x}=\frac{\partial A_{x}}{\partial t} \end{aligned}
\end{equation}
于是洛伦兹力的$ x $分量可以写做
\begin{equation}
    F_{x}=q\left[-\frac{\partial \varphi}{\partial x}-\frac{\partial A_{x}}{\partial t}+v_{y}\left(\frac{\partial A_{y}}{\partial x}-\frac{\partial A_{x}}{\partial y}\right)-v_{z}\left(\frac{\partial A_{x}}{\partial z}-\frac{\partial A_{z}}{\partial x}\right)\right]
\end{equation}
由于矢势$ A $是坐标和时间的函数.由于$ A $是粒子所在点的电磁场的矢势,因此$ A $中的坐标变量是粒子在对应时刻的空间位置,它们对时间的微商就是粒子的速度,因此有
\begin{equation}
    \frac{\mathrm{d} A_{x}}{\mathrm{d} t}=\frac{\partial A_{x}}{\partial t}+v_{x} \frac{\partial A_{x}}{\partial x}+v_{y} \frac{\partial A_{x}}{\partial y}+v_{z} \frac{\partial A_{x}}{\partial z} \\
\end{equation}
分量表达式变为
\begin{equation}
    F_{x}=q\left[-\frac{\partial \varphi}{\partial x}-\frac{\mathrm{d} A_{x}}{\mathrm{d} t}+v_{x} \frac{\partial A_{x}}{\partial x}+v_{y} \frac{\partial A_{y}}{\partial x}+v_{z} \frac{\partial A_{z}}{\partial x}\right] \\
\end{equation}
又因为矢量势和标量势都和速度无关,自然有
\begin{equation}
    \begin{aligned} &\frac{\mathrm{d} A_{x}}{\mathrm{d} t}=\frac{\mathrm{d}}{\mathrm{d} t}\left[\frac{\partial}{\partial v_{x}}(\boldsymbol{A} \cdot v)\right]=\frac{\mathrm{d}}{\mathrm{d} t} \frac{\partial}{\partial v_{x}}(-\varphi+\boldsymbol{A} \cdot v)\\ &\left(v_{x} \frac{\partial A_{x}}{\partial x}+v_{y} \frac{\partial A_{y}}{\partial x}+v_{z} \frac{\partial A_{z}}{\partial x}\right)=v \cdot \frac{\partial \boldsymbol{A}}{\partial x}=\frac{\partial}{\partial x}(\boldsymbol{v} \cdot \boldsymbol{A}) \end{aligned}\\
\end{equation}
终于,我们把分量式写为
\begin{equation}
    F_{x}=q\left[-\frac{\partial}{\partial x}(\varphi-\boldsymbol{A} \cdot v) |+\frac{\mathrm{d}}{\mathrm{d} t} \frac{\partial}{\partial v_{x}}(\varphi-\boldsymbol{A} \cdot v)\right]
\end{equation}
对比广义力和广义势能的式子,我们得出了广义势能的表达
\begin{equation}
    \color{red}{\boxed{U=q \varphi-q \boldsymbol{A} \cdot v} }
\end{equation}

相应的拉格朗日量自然写出
\begin{equation}
    \color{blue}{\boxed{L=\frac{1}{2} m v^{2}-q \varphi+q \boldsymbol{A} \cdot v }}
\end{equation}

\section{格林函数}
关于格林函数的发现者:George Green的一篇很有趣的短文:\\
\href{https://chaoli.club/index.php/attachment/59eff93a205af_Green%20of%20Green%20Functions%20-%20revised.pdf/download}{Green函数的那个Green.作者:Lawire Challis; Fred Sheard,翻译:超理汉化组}
\marginnote[]{
    在这里把这本书的前言附注出来:\\
    \textit{1828 年,George Green 发表了一篇数学文章,彼时他仍是Nottingham 的一位磨坊工人,这篇文章一度无人问津.但Green 在文中创造的分析技术并未就此湮没,今天的诸多学科,无论是经典的静电学还是相当现代的量子场论,无一不需要这种技术.}
    坐落在英国中部的Nottingham,是一座充满活力的美丽城市.城墙边坐落着Robin
    Hood1 的雕像,这城因他而闻名.今日的Nottingham 郡长,仍是政府的重要人物,但是不再被善良的市民们所惧怕.一座新起的风车是当下Nottingham 的最新景点.碧空当间,白色扇叶凭风而动,磨出著名的石磨面粉,再由底下的商铺装成小袋售出去.但你可能想不到,他们也出售数学物理的宣传册.面粉与物理间的奇妙纽带是Nottingham 磨坊的独特记号之一,个中缘由已被磨坊上的一块牌子道出:
    \begin{Center}
        生于斯,作于斯的\\
        George Green\\
        一位数学家\\
        1793-1841
    \end{Center}
    这里的Green,正是Green 定理的那位Green ,也是Green 函数的那位Green .全世界物理专业的本科学生都熟悉前者;后者则频频现出现在经典物理和量子物理中.
}

本节将补充格林函数的来源,及为什么要使用格林函数.\\









\chapter{数学内容补充}
\section{楔积,外微分与外代数}
\subsection{楔积}
为了讲清楚楔积\sidenote{楔积,又称外积,但区分向量叉乘,叉乘仅适用于三维情况,而外积对维数没有要求.},我们首先需要回忆一下高数我们已经学过的内容,对于一个二元函数$f(x,y)$,其二重积分
\begin{equation}
    A=\iint_Df(x,y)\dx\dy
\end{equation}
假设为了便于计算积分,需要引入变量代换
\begin{equation}
    \begin{cases}x&=x(x^{\prime},y^{\prime})\\y&=y(x^{\prime},y^{\prime})\end{cases}
\end{equation}
坐标变换后,我们知道可以利用雅可比行列式写为如下形式\marginnote[]{或许这里不能够假定每一个人都学过雅可比行列式,对于这个坐标变换,其雅可比行列式定义为$$\left|\frac{\partial(x,y)}{\partial(x^{\prime},y^{\prime})}\right|=\frac{\partial x}{\partial x^{\prime}}\frac{\partial y}{\partial y^{\prime}}-\frac{\partial x}{\partial y^{\prime}}\frac{\partial y}{\partial x^{\prime}} $$写成行列式形式即$$\mathbf J=\left|\frac{\partial(x,y)}{\partial(x',y')}\right| = \begin{vmatrix} \frac{\partial x}{\partial y^{\prime}} & \frac{\partial x}{\partial x^{\prime}} \\ \frac{\partial y}{\partial y^{\prime}} & \frac{\partial y}{\partial x^{\prime}}  \end{vmatrix}$$其几何意义代表坐标前后体积微元(对于二维则是面积微元)的比值,推导过程并不是这一部分的重点,故省略.}
\begin{equation}
    A=\iint f(x,y)\left|\frac{\partial(x,y)}{\partial(x^{\prime},y^{\prime})}\right|\dx^{\prime}\dy^{\prime}
\end{equation}
我们发现,雅可比行列式的计算往往利用了向量叉乘来求面积,但是我们知道,向量叉乘在高维情况下是非良定义的,这要求我们对高维下的体积微分的运算引入一种新的运算符:楔积.对于上面的二重积分的积分微元$\dx\dy$,我们定义其中的运算关系:$\dx\wedge\dy$.

对于普通矢量的楔积运算,我们根据其楔积``次数"$ k $定义其为k-矢量,如矢量$\vec x$即1-矢量,$\vec x\wedge\vec y$则为2-矢量.一个k-矢量象征一个k维的\textbf{有向}体积,而由k-矢量之间的代数运算关系构成的代数被称为\textbf{外代数},这类代数最显著的特征就是反交换性:
$$\dx\wedge\dy=-\dx\wedge\dy$$
由于这种关系,自然可以发现
$$\dx\wedge \dx=-\dx\wedge \dx=0,\quad \dy\wedge \dy=-\dy\wedge\dy=0$$
利用这种新的代数关系,我们发现微元之间的坐标变换变得更加清晰
\begin{equation}
    \begin{aligned}\dx\wedge \dy&=(\frac{\partial x}{\partial x^{\prime}}\dx^{\prime}+\frac{\partial x}{\partial y^{\prime}}\dy^{\prime})\wedge(\frac{\partial y}{\partial x^{\prime}}\dx^{\prime}+\frac{\partial y}{\partial y^{\prime}}\dy^{\prime})\\&=\frac{\partial x}{\partial x^\prime}\frac{\partial y}{\partial y^\prime}\dx^\prime\wedge \dy^\prime+\frac{\partial x}{\partial y^\prime}\frac{\partial y}{\partial x^\prime}\dy^\prime\wedge \dx^\prime\\&=(\frac{\partial x}{\partial x^\prime}\frac{\partial y}{\partial y^\prime}-\frac{\partial x}{\partial y^\prime}\frac{\partial y}{\partial x^\prime})\dx^\prime\wedge \dy^\prime\\&=\left|\frac{\partial(x,y)}{\partial(x^{\prime},y^{\prime})}\right|\dx^{\prime}\wedge \dy^{\prime}.\end{aligned}
\end{equation}
在运算过程中,雅可比行列式自动出现了,并且,这类关系自然与高维情况下兼容,对于$ n $重微元$\dx^1\dx^2\cdots\dx^n$,其以外代数形式写作$\dx^1\wedge\dx^2\wedge\cdots\wedge\dx^n$,并且满足反交换关系
$$\dx^i\wedge \dx^j=-\dx^j\wedge \dx^i$$
并且,我们将被积函数$f(x^1,x^2,...,x^n)$与$\dx^1\wedge\dx^2\wedge\cdots\wedge\dx^n$乘起来称为一个$ n $重微分形式,即n形式记为$\omega$:
\begin{equation}
    \omega=f(x^1,x^2,...,x^n)\dx^1\wedge\dx^2\wedge\cdots\wedge\dx^n
\end{equation}
于是,我们可以重新写出$ n $元函数的积分表达,其实际上是对n形式$\omega$的积分
\begin{equation}
    A=\int\omega
\end{equation}
对于$ n $个变量的情况,我们对n形式进行推广,定义k-形式(k-form)$\alpha$:
\begin{equation}
    \alpha=\frac{1}{k!}\alpha_{i_1i_2\cdots i_k}\dx^{i_1}\wedge \dx^{i_2}\wedge\cdots\wedge \dx^{i_k}
\end{equation}
这里使用了爱因斯坦求和约定.对于任意的k-形式,我们要求k-形式的分量$\alpha_{i_1i_2\cdots i_k}$对$ k $个指标两两反对称,即全反对称\sidenote{即交换任意两个指标变号}.

对于三维情况,由于只有三个变量,自然仅存在0,1,2,3 四种非零的微分形式\marginnote[]{由于外代数的反交换性质,很显然k-形式中的$k$个指标$i_1,i_2,...,i_k$取值必须两两不同,否则对$\alpha$的贡献将为零.特别的,这意味着,n形式是最高重的非零形式,任何$k>n$的$k$形式都必定为零,因为这时候它的$k$个指标取值必定会出现重复,不可能两两不同.},我们对于这四种微分形式分别讨论:
\begin{enumerate}
    \item 0-形式就是普通的标量函数
    \item 1-形式可以写为$a_1\dx+a_2\dy+a_3\dz$,不难发现其分量$(a_1,a_2,a_3)$恰好构成一个三维空间的矢量场$\mb{a}(\mb x)$,于是三维空间的1-形式可以写为$$a_1\dx+a_2\dy+a_3\dz=\mb{a}(\mb x)\cdot\ddx$$
    \item 2-形式同样可以写为\marginnote[]{\textbf{特别的},对于2-形式,记$a_{12}=b_3,a_{23}=b_1,a_{31}=b_2$,并且定义映射$\dx\wedge \dy\to \dz,\dy\wedge \dz\to \dx,\dz\wedge \dx\to \dy$, 则有$a_{12}\dx\wedge \dy+a_{23}\dy\wedge \dz+a_{31}\dz\wedge \dx\to b_1\dx+b_2\dy+b_3\dz$
    可见,3 维空间 2-形式和 1-形式之间能够建立一对一映射.因此,2-形式的 3 个独立非零分量$(a_{23},a_{31},a_{12})=(b_1,b_2,b_3)=\mathbf{b}$刚好构成一个 3 维空间矢量.
    同时,上面这个对映也告诉我们,$(\dy\wedge \dz,\dz\wedge \dx,\dx\wedge \dy)$也完全类似于一个 3 维空间的矢量微元,通常将之定义为面积元矢量$\d\mathbf{S}$,
        $$\d\mathbf{S}=(\dy\wedge \dz,\dz\wedge \dx,\dx\wedge dy)$$ 我们称这类k-形式与n-k-形式之间的一一映射关系为\textbf{霍奇(Hodge)对偶}}
    $$a=\frac{1}{2}a_{ij}(\mathbf{x})\dx^i\wedge \dx^j=a_{12}\dx\wedge \dy+a_{23}\dy\wedge \dz+a_{31}\dz\wedge \dx$$
    \item 3-形式与0-形式类似,可以写做$f(x,y,z)\dx\wedge\dy\wedge\dz$
\end{enumerate}
\subsection{外微分}
对于微分形式,为了与之前的微分运算进行一定的区分,我们定义一种新的微分运算:外微分.我们首先以一个简单的例子引入:对于二维空间内的1-形式$a=a_x\dx+a_y\dy$,我们定义其外微分$\d a$为
\begin{equation}
    \d a=\d a_x\wedge\dx+\d a_y\wedge \dy
\end{equation}
这样的外微分显然是结合外代数与微分的产物,对其进行计算得到\sidenote{具体运算步骤$$\begin{aligned}\d a&=\d a_x\wedge \dx+\d a_y\wedge\dy\\&=(\partial_xa_x\dx+\partial_ya_x\dy)\wedge \dx\\&\quad+(\partial_xa_y\dx+\partial_ya_y\dy)\wedge \dy\\&=\partial_ya_x\dy\wedge \dx+\partial_xa_y\dx\wedge \dy\\&=(\partial_xa_y-\partial_ya_x)\dx\wedge \dy,\end{aligned}$$}
\begin{equation}
    \d a=(\partial_xa_y-\partial_ya_x)\dx\wedge\dy
\end{equation}
显然,$\d a$只有一个分量$(\partial_xa_y-\partial_ya_x)$,并且其恰好是二维矢量$\mb a$的\textbf{旋度}.

另一方面,在高数中,我们熟知二维空间内有格林公式
\begin{equation}
    \oint_{\partial D}(a_x\dx+a_y\dy)=\int_D(\partial_xa_y-\partial_ya_x)\dx\dy
\end{equation}
其中$\partial D$为一个闭合回路,$D$为这条回路包括的区域,利用微分形式,可以将其重新写为
\begin{equation}\label{B1}
    \int_{\partial D}a=\int_D\d a
\end{equation}
类似于二维空间的1-形式,我们继续分析三维空间的1-形式$a=\mathbf{a}\cdot \d\mathbf{x}=a_x\dx+a_y\dy+a_z\dz$的外微分$\d a$,其定义同样为
\begin{equation}
    \d a=\d a_x\wedge \dx+\d a_y\wedge \d y+\d a_z\wedge \dz
\end{equation}
通过与二维情况完全类似的运算,我们得到结果
\begin{equation}
    \d a=(\partial_xa_y-\partial_ya_x)\dx\wedge \dy+(\partial_ya_z-\partial_za_y)\dy\wedge \dz+(\partial_za_x-\partial_xa_z)\dz\wedge \dx
\end{equation}
这是一个三维空间的2-形式,$\d a$的三个独立分量同样恰好构成三维矢量$\mb a$的\textbf{旋度}$\nabla\times\mb a$,即
\begin{equation}
    \d a=(\nabla\times\mathbf{a})\cdot \d\mathbf{S}
\end{equation}
使用微分形式重写
\begin{equation}\label{B2}
    \int_{\partial D}a=\int_D\d a
\end{equation}
于是我们发现,\ref{B1}与\ref{B2}完全一致.

我们继续向下考虑三维空间的2-形式的外微分$\d a$,其定义同样为
\begin{equation}
    \da=\da_{12}\wedge \dx\wedge \dy+\da_{23}\wedge \dy\wedge \dz+\da_{31}\wedge \dz\wedge \dx
\end{equation}
计算得到
\begin{equation}
    \da=(\partial_3a_{12}+\partial_1a_{23}+\partial_2a_{31})\dx\wedge \dy\wedge \dz
\end{equation}
同样利用$(a_{23},a_{31},a_{12})=(b_1,b_2,b_3)=\mathbf{b}$,我们可以把结果重写为
\marginnote[]{对于外微分,存在两个常用概念.首先,一个微分形式$\alpha$,如果它的外微分等于零,即$\d\alpha=0$ 我们就称它为\textbf{闭形式}.其次,一个微分形式$\alpha$,如果它是另一个微分形式$\beta$的外微分,即有$\alpha=\d\beta$,我们就称这样的$\alpha$为一个\textbf{恰当形式}.很显然,任何恰当形式都必定是闭形式!反过来,闭形式却不一定是恰当形式,闭形式什么时候是恰当形式什么时候不是.这往往和空间的拓扑有关系,是所谓的 de Rahm 上同调研究的内容.}
\begin{equation}
    \da=(\partial_1b_1+\partial_2b_2+\partial_3b_3)\dx\wedge \dy\wedge \dz=(\nabla\cdot\mathbf{b})\dx\wedge \dy\wedge \dz
\end{equation}
很明显结果是一个3-形式,并且恰好给出了三维矢量的\textbf{散度}.

于是,我们发现,对于三维空间的高斯定理
\begin{equation}
    \oint_{\partial V}\mathbf{b}\cdot \d\mathbf{S}=\int_V(\nabla\cdot\mathbf{b})\d V
\end{equation}
利用外微分仍然可以写完类似\ref{B1}与\ref{B2}的形式.我们将其扩展为
\begin{equation}\label{B3}
    \int_{\partial D}\alpha=\int_D\d\alpha
\end{equation}
式中$\alpha$表示三维空间中的一个k-1-形式,$D$表示三维空间中一个以$\partial D$为边界的$k$维曲面 (因此$\partial D$是$k-1$维的,而$d\alpha$则是一个k-形式).$k=2$时,它就是斯托克斯公式, $k=3$时它就是高斯定理.可见,利用外微分运算,我们可以将矢量分析中那些著名的公式和定理统一起来.

我们可以将其扩展到$ n $维形式,k-1-形式的外微分为一个k-形式,\ref{B3}也被称为\textbf{广义斯托克斯公式}.

外微分有一个尤为重要的性质,对任意微分形式连续进行两次外微分,结果恒等于0\marginnote[]{对于外微分,存在类似于普通复合函数的莱布尼茨公式$$\mathrm{d}(\mu\wedge\lambda)=\mathrm{d}\mu\wedge\lambda+(-1)^{\deg\mu}\mu\wedge\mathrm{d}\lambda$$其中$\deg\mu$为微分形式$\mu$的次数.}
\begin{equation}
    \d^2\alpha=\d(\d\alpha)\equiv0
\end{equation}
我们知道了外微分可以令相应的微分形式维度上升一次,同样的,我们引入两个类似的算符来表达对于微分形式的其他操作.\\
在前面,我们了解了k-形式与n-k-形式之间存在一一映射,被称为\textbf{霍奇对偶},于是引入运算符$\star$(有时也使用$*$),即对k-形式取其n-k-形式.我们将在微分几何后面再次讨论.对于
$$\mathrm{d}f=\frac{\partial f}{\partial x}\mathrm{d}x+\frac{\partial f}{\partial y}\mathrm{d}y+\frac{\partial f}{\partial z}\mathrm{d}z$$
取$\star$运算后,即得
\begin{equation}
    \star\mathrm{d}f=\frac{\partial f}{\partial x}\mathrm{d}y\wedge\mathrm{d}z+\frac{\partial f}{\partial y}\mathrm{d}z\wedge\mathrm{d}x+\frac{\partial f}{\partial z}\mathrm{d}x\wedge\mathrm{d}y
\end{equation}
以及与$\d$对应的可以令微分形式降次的运算符$\delta$,定义为
\begin{equation}
    \delta=-(-1)^g(-1)^{n(k+1)}\star\mathrm{d}\star
\end{equation}
其中,$g$为度规.

利用外微分,我们可以把我们熟知的麦克斯韦方程组(四维时空)变得更加简洁
\begin{equation}
    \begin{cases} \mathrm{d}F=0\\ \mathrm{d}\star F=\mu_0\star J \end{cases}
\end{equation}
当然,当我们继续深入学习,我们可以得到最终的数学表述:电磁理论是$U(1)$丛上的\textbf{联络}
\begin{equation}
    \psi(x)\mapsto e^{\mathrm{i}\varphi(x)}\psi(x)
\end{equation}
\subsection{简单实例}
\subsubsection{保守力}
回顾力学中我们对于质点系保守力的定义:
\begin{equation}
    \sum_i\mathbf{F}_i\cdot \d\mathbf{x}_i=-\d V.
\end{equation}
并引入指标$\mu=1,2,3,\cdots,3N$,即对每个质点的3个笛卡尔坐标,于是保守力自然可以重写为
\begin{equation}
    F_\mu \dx^\mu=-\d V(x^1,...,x^{3N})
\end{equation}
显然这是一个1-形式,并将其简记为$ F $,并且我们能够发现其还是一个\textbf{恰当形式},满足
\begin{equation}
    F=-\d V
\end{equation}
势能为0-形式,在前面我们知道一个恰当形式必是\textbf{闭形式},立刻得出
\begin{equation}
    \d F=0
\end{equation}
将$\d F$展开运算,得到
\begin{equation}
    \d F=(\partial_\mu F_\nu)\dx^\mu\wedge \dx^\nu=[\frac{1}{2}(\partial_\mu F_\nu-\partial_\nu F_\mu)+\frac{1}{2}(\partial_\mu F_\nu+\partial_\nu F_\mu)]\dx^\mu\wedge \dx^\nu
\end{equation}
后一项由于反对称关系为$ 0 $,于是有
\begin{equation}
    \d F=\frac{1}{2}(\partial_\mu F_\nu-\partial_\nu F_\mu)\dx^\mu\wedge \dx^\nu
\end{equation}
即等价于
\begin{equation}
    \partial_\mu F_\nu-\partial_\nu F_\mu=0
\end{equation}
对于单个质点,我们得出的结果自然为对力$\mb F$的旋度$\nabla\times\mb F=0$,并且由斯托克斯公式
\begin{equation}
    \int_{\partial D}F=\int_D\d F=0
\end{equation}
这意味着保守力1-形式在坐标空间内的任意闭合回路积分都为01,我们常称这类积分为\textbf{功},于是,这也就是说明保守力在任意闭合回路做的功为$ 0 $.

\subsubsection{热力学}
我们熟知热力学第一定律可以写为
\begin{equation}
    T\d S=P\d V+\d U
\end{equation}
对其应用一次外微分,并注意到$\d^2=0$,有
\begin{equation}
    \d T\wedge\d S=\d P\wedge\d V
\end{equation}
由这个微分形式关系,自然可以推出热力学所有的麦克斯韦关系式,例如,设$ T,P $是关于$S,V$的函数$T=T(S,V),P=P(S,V)$,有关系
\begin{equation}
    (\frac{\partial T}{\partial V})_{S} \d V \wedge \d S = (\frac{\partial P}{\partial S})_{V} \d S \wedge \d V
\end{equation}
即关系
\begin{equation}
    (\frac{\partial T}{\partial V})_{S} = - (\frac{\partial P}{\partial S})_{V}
\end{equation}
而$U,H,F,G$则是勒让德变换的关系,但不是本篇的重点了,可以参考这一本热统教材\href{https://chaoli.club/index.php/attachment/59a781504bce2_CallenThermo.pdf/download}{热力学和热统计物理导论,作者:Herbert B. Callen,翻译:超理汉化组},这一本写的远超国内绝大多数热统教科书.
\section{矢量分析与张量代数}
还没写.
\chapter{mathematica的基本用法}
本篇附录内容大概是基本语法,常用函数介绍,一些匿名函数,绘图.(不一定什么时候会写)
\subsection*{4.1-式子13}
输入代码
\begin{lstlisting}
    Integrate[
    1/(2 \[Pi]) Exp[-I \[Epsilon] (p^2/(2 m) + V) + 
    I p (qf - qi)], {p, -Infinity, Infinity}] // FullSimplify
\end{lstlisting}

\chapter{答案及解析}
\section{第一章}
1
%第一章至少有一道利用代数方法解谐振子的大题.李群李代数若干,乱七八糟的数学象征性来一点.
\section{第二章}
1
\chapter{致谢/参考}
\section{致谢}
感谢在这本讲义完成中提供帮助的人,以及提供模板的\href{https://github.com/fmarotta/kaobook}{https://github.com/fmarotta/kaobook}.

\section{参考}
本文主要参考的书籍和期刊如下:
\begin{enumerate}
    \item Modern Quantum Mechanics 2nd.J.J.Sakurai
    \item Quantum Field Theory in Condensed Matter Physics 2nd.Alexei M.Tsvellk
    \item Entanglement in Many-Body Systems(RMP)
    \item 物理学家用李群李代数
    \item 《李群与李代数》讲义-李世雄
    \item Nicolas Dupuis - Field Theory of Condensed Matter and Ultracold Gases
    \item Conformal Field Theory A.N. Schellekens
    \item 经典力学新讲-陈童(主要参考部分为微分形式部分)
\end{enumerate}
参考了十余本,之后慢慢整理,并会改为bib引用的.

免责声明:本讲义仅限个人学习使用,仅供参考.部分图片为个人重绘,部分图片来源wiki.同时,本讲义的全部内容和代码已公开至github主页,允许二次传播,允许自行更改,但请注意讲义仅为多本书和综述的整理和综合,且仍处于更新状态,不排除未来将前几章重置的可能性.
















%\chapter{单位制}
%我们从小学就逐步接触一些单位,常见的如米(m),千克(kg),秒(s)等是国际统一使用的\textbf{标准度量系统(国际单位制)}.相应的,像是国内经常接触的斤,公里,亩,美国\sidenote[][]{包括美国、开曼群岛、伯利兹等极少数国家和地区}常用的华氏度等,则是生活中使用的独立度量系统,大多数度量系统都和标准度量系统之间存在换算关系.而且生活中使用的度量单位大多比较局限,对于相干度较低的单位往往是不涉及的.\\
%对于初中和高中的物理学习,我们已经熟练使用国际单位制(SI\sidenote[][]{法语 Système International d'Unités,简称SI})来解决一些简单的物理问题.但是,就像生活中使用的单位制一样,人们出于方便的角度对于一些物理场景也构建出一些新的单位制.这些单位制能够简化相关的物理问题.\\物理上使用的单位制与国际单位制的转换往往比较复杂,使用时建议标注使用了哪个单位制.\\
%\begin{remark}
%    在这个附录中,电磁单位制与自然单位制独立分为两节,但是按照较广义的自然单位制的定义\sidenote[][]{区别于粒子物理的``自然单位制"和普朗克单位制},电磁单位制也属于其中的一类,特此说明.
%\end{remark}
%\section{电磁单位制}
%相比于我们常用的国际单位制,也称为MKSA单位制(即米,千克,秒,安培),我们在电磁中常用的高斯单位制被称为CGS单位制(即厘米,克,秒).\\接下来为了避免混乱,列举高斯单位制所常用的单位:电荷$statC$,电势$statV$,力$dyne$\sidenote[][]{中文音译为达因},磁感应强度$gauss$,磁场强度$oersted$,磁通量$mx$,能量$erg$.\\
%
%相比于自然单位制直接将值赋为1,高斯单位制就比较保守,它根据我们熟知的库仑定律,通过定义$1\mathrm{A}=0.1c\cdot\mathrm{\d yne}^{\frac12},1\mathrm{C}=0.1c\cdot\rm{\d yne}^{\frac12}\cdot s$来达到简化的操作.\\
%\begin{table}[htbp]
%    \centering
%    \caption{一些简单对应关系}
%    \begin{tabular}{|c|c|c|c|}
    %        \hline
    %        & SI      & Gaussian        & G/SI                     \\ \hline
    %        E          & $V/m$   & $statV/m$       & $\sqrt{4\pi\epsilon_0}$  \\ \hline
    %        V          & $V$     & $statV$         & $\sqrt{4\pi\epsilon_0}$  \\ \hline
    %        D          & $C/m^2$ & $statC/cm^2$    & $\sqrt{4\pi/\epsilon_0}$ \\ \hline
    %        q          & $C$     & $statC$         & $1/\sqrt{4\pi\epsilon_0}$ \\ \hline
    %        P          & $C/m^2$ & $statC/cm^2$    & $1/\sqrt{4\pi\epsilon_0}$ \\ \hline
    %        I          & $A$     & $statC/s$       & $1/\sqrt{4\pi\epsilon_0}$ \\ \hline
    %        B          & $T$     & $Gauss$         & $\sqrt{4\pi/\mu_0}$      \\ \hline
    %        A          & $Wb/m$  & $Gauss\cdot cm$ & $\sqrt{4\pi/\mu_0}$      \\ \hline
    %        H          & $A/m$   & $oersted$       & $\sqrt{4\pi\mu_0}$       \\ \hline
    %        $\epsilon$ & $F/m$   & 1               & $1/\epsilon_0$           \\ \hline
    %        $\mu$      & $H/m$   & 1               & $1/\mu_0$                \\ \hline
    %    \end{tabular}
%\end{table}
%\section{自然单位制}
%我们熟知,国际单位制的7个基本单位是通过物理常数所定义的,那么,如果我们把其中\textbf{一个或几个}的定义值改为1,那么就又可以构造出来一套度量系统.这其中\textbf{显而易见的优点}是直接导致原本含有大量常数的公式可以被写成更加简洁方便的形式.在物理学里,自然单位制就是一种建立于此类方法的计量单位制度.例如,电荷的自然单位是基本电荷${\displaystyle e}$,速度的自然单位是光速${\displaystyle c}$,角动量的自然单位是约化普朗克常数${\displaystyle \hbar }$,电阻的自然单位是自由空间阻抗${\displaystyle Z_{0}}$,质量的自然单位则有电子质量${\displaystyle m_{e}}$与质子质量${\displaystyle m_{p}}$等.\\
%
%事实上,对于单位的改动,我们至少要求不会导致无量纲常数的值发生改变,如精细结构常数.
%$${\displaystyle \alpha ={\frac {e^{2}k_{e}}{\hbar c}}={\frac {e^{2}}{\hbar c(4\pi \epsilon _{0})}}={\frac {1}{137.035999074}}=7.2973525698\cdot 10^{-3}}$$
%这个常数就要求不能同时把${\displaystyle e},{\displaystyle \hbar },{\displaystyle c},{\displaystyle k_{e}}$同时为1.
%\subsection{普朗克单位制}
%普朗克单位制几乎是最常使用的单位制,它的定义只依赖于最基本的性质.普朗克单位选择将真空光速${\displaystyle c}$,万有引力常数${\displaystyle G}$,约化普朗克常数${\displaystyle \hbar }$,真空电容率${\displaystyle \epsilon _{0}}$,玻尔兹曼常数${\displaystyle k_{B}}$定为1\sidenote[][]{普朗克洛伦兹-亥维赛单位制将${\displaystyle 4\pi G},{\displaystyle \epsilon _{0}}$定为1,普朗克高斯单位制将${\displaystyle G},{\displaystyle 4\pi \epsilon _{0}}$定为1}.\\
%类比国际单位制,普朗克单位制也有一些基本单位(如常常出现在各种科普作品中的普朗克长度,普朗克时间等)和导出单位(普朗克面积,普朗克动量等).具体列表可参考相关wiki\href{https://zh.wikipedia.org/wiki/%E6%99%AE%E6%9C%97%E5%85%8B%E5%96%AE%E4%BD%8D%E5%88%B6}{普朗克单位制},这里不做展开.
%\subsection{``自然单位制"(粒子物理)}
%在粒子物理中,自然单位制特指${\displaystyle \hbar =c=k_{B}=1}$情况下的单位制.通常会根据情况选择使用洛伦兹-亥维赛单位制或高斯单位制来确定电荷定义.
%\subsection{其他单位制}
%\subsubsection*{史东纳单位制}
%第一次出现的单位制,已经不再使用.规定了${\displaystyle c=G=e={\frac {1}{4\pi \epsilon _{0}}}=k_{B}=1}$.
%\subsubsection*{原子单位制}
%这类单位制是特别为了简易表达原子物理学和分子物理学的方程而精心设计,在本篇中仅做介绍.\\
%原子单位制分为两种:哈特里原子单位制和里德伯原子单位制.哈特里原子单位制比里德伯原子单位制常见.两者的主要区别在于质量单位与电荷单位的选取.\\
%哈特里原子单位制的基本单位为${\displaystyle e=m_{e}=\hbar ={\frac {1}{4\pi \epsilon _{0}}}=k_{B}=1}$,${\displaystyle c={\frac {1}{\alpha }}}$.\\
%里德伯原子单位制的基本单位为${\displaystyle {\frac {e}{\sqrt {2}}}=2m_{e}=\hbar ={\frac {1}{4\pi \epsilon _{0}}}=k_{B}=1}$,${\displaystyle c={\frac {2}{\alpha }}}$.
%\chapter{固体物理中的一些概念}
%对于物理研究,把它放在合适的空间下能够简化问题.对于坐标空间(正格子,基矢)和动量空间(倒格子,倒格矢)来讲,相当于从两个角度来描写\textbf{同一}事物.在之后对于晶格的分析中,我们常常要在动量空间上分析这一问题.\\
%\textit{如果对于物理形式较为敏感,应该会容易的想到``两个角度描写同一事物"的表述和傅里叶变换有很大的相似性.实际上,坐标空间和动量空间互为傅里叶变换.如果对于量子力学有一定了解或已经阅读过关于表象变换的内容,对这一部分会有更深的体会.}
%\section*{一些需要了解的概念}
%为了能够便于理解接下来的内容,以下是需要了解的概念.
%\begin{enumerate}
%    \item \textbf{格矢}:联系任两个晶格点的向量
%    \item \textbf{布拉维晶格 Bravais lattices}:由同种原子构成的晶胞,多种原子构成的晶胞可以视为几个布拉维晶格的叠加.
%    \item 待补充
%\end{enumerate}
%
%
%\chapter{$\delta_{ij},\varepsilon_{ijk}$和爱因斯坦求和约定与$\delta$函数}
%\section{克罗内克符号$\delta_{ij}$}
%克罗内克符号是一类二元函数,其通常定义为以下形式
%\begin{equation}
%    \delta _{ij}=\left\{{\begin{matrix}1&(i=j)\\0&(i\neq j)\end{matrix}}\right.\,\!
%\end{equation}
%其具有筛选性(和投影算符类似)
%\begin{equation}
%    \sum _{i=-\infty }^{\infty }\delta _{ij}a_{i}=a_{j}\,\!
%\end{equation}
%其具有和$\delta$函数共同的部分性质,而$\delta$函数也正是源于克罗内克符号.
%\section{列维西维塔符号$\varepsilon_{ijk}$}
%列维-奇维塔符号,对于正整数 $n$ ,它以$1, 2, ..., n $所形成排列的奇偶性来定义.其他名称包括排列符号、反对称符号与交替符号.
%
%而$\varepsilon_{ijk}$的值由下角标$ijk$决定,当存在任意两个角标相同时值取$0$,当全部指标都不相等时,角标的逆序数为偶数取$1$,为奇数则取$0$.
%\subsection*{二维形式}
%\begin{equation}
%    \varepsilon_{ij}={
    %        \begin{cases}+1&\text{当}\left(i,j\right)=\left(1,2\right)\\
        %            -1&\text{当}\left(i,j\right)=\left(2,1\right)\\
        %            0&\text{当}i=j
        %        \end{cases}
    %    }\,
%\end{equation}
%二维较为少见,仅作为了解.
%\subsection*{三维形式}
%我们经常看到的列维西维塔符号常常是三维形式的,即如下
%\begin{equation}
%    \varepsilon _{ijk}={
    %        \begin{cases}
        %            +1&\text{当}(i,j,k)=(1,2,3),(2,3,1),(3,1,2)\\
        %            -1&\text{当}(i,j,k)=(3,2,1),(2,1,3),(1,3,2)\\
        %            0&\text{当}i=j,j=k\text{或}k=i
        %        \end{cases}
    %    }\,
%\end{equation}
%\subsection*{性质}
%两个列维-奇维塔符号的积,可以用一个以克罗内克符号表示的行列式求得
%\begin{equation}
%    \varepsilon _{ijk\dots }\varepsilon _{mnl\dots }={\begin{vmatrix}\delta _{im}&\delta _{in}&\delta _{il}&\dots \\\delta _{jm}&\delta _{jn}&\delta _{jl}&\dots \\\delta _{km}&\delta _{kn}&\delta _{kl}&\dots \\\vdots &\vdots &\vdots \\\end{vmatrix}}
%\end{equation}
%也可以用来表示行列式和向量内积,对于一个$3\times3$的方阵$A$,有表示
%\begin{equation}
%    \det(A)=\sum _{i,j,k=1}^{3}\varepsilon _{ijk}\,a_{1i}\,a_{2j}\,a_{3k}
%\end{equation}
%对于向量内积,有
%\begin{equation}
%    {\boldsymbol {a}}\times {\boldsymbol {b}}={\begin{vmatrix}{\boldsymbol {e}}_{1}&{\boldsymbol {e}}_{2}&{\boldsymbol {e}}_{3}\\a_{1}&a_{2}&a_{3}\\b_{1}&b_{2}&b_{3}\\\end{vmatrix}}=\sum _{1\leq i,j,k\leq 3}\varepsilon _{ijk}\,a_{i}b_{j}\,{\boldsymbol {e}}_{k}
%\end{equation}
%\section{爱因斯坦求和约定}
%爱因斯坦求和约定是一种标记的约定,即重复角标意味着求和,一般未指定的情况下就是由$1$至$3$.
%\section{$\delta$函数}
%我们首先明确,$\delta$函数并不是通常意义的函数,其更准确的称呼是广义函数.其在$x=0$处取值为正无穷,在$x\ne0$处取值为$0$,但是在全定义域上的积分值为$1$,当然,我们所学的黎曼积分并不支持这一操作,我们需要\textbf{勒贝格积分}来处理它,当然,这里并不会讲述测度论的内容,仅仅作为了解即可.
%\subsection{定义}
%我们从其定义开始这部分内容.
%\begin{equation}
%    \delta (x)={\begin{cases}+\infty ,&x=0\\0,&x\neq 0\end{cases}}
%\end{equation}
%且同时满足
%\begin{equation}
%    \int _{-\infty }^{\infty }\delta (x)\,\d x=1
%\end{equation}
%而对于复变函数中,一切在域$D$中闭包的全纯函数,我们可以用柯西积分公式来表示$\delta$函数
%\begin{equation}
%    \delta _{z}[f]=f(z)={\frac {1}{2\pi i}}\oint _{\partial D}{\frac {f(\zeta )\,\d\zeta }{\zeta -z}}.
%\end{equation}
%\subsection{性质}
%首先,回忆克罗内克符号的部分,很容易联想到$\delta$函数也具有筛选的性质:
%\begin{equation}
%    \int _{-\infty }^{\infty }f(x)\delta (x-x_{0})\,\d x=f(x_{0})
%\end{equation}
%以及(高维的情况与之类似,不再重复,相关证明是显然的)
%\begin{equation}
%    \begin{aligned}
    %        \delta (\alpha x)&={\frac {\delta (x)}{|\alpha |}}\\
    %        \delta (-x)&=\delta (x)\\
    %        x\delta (x)&=0\\
    %        \int _{-\infty }^{\infty }\delta (\xi -x)\delta (x-\eta )\,\mathrm {d} x&=\delta (\xi -\eta )
    %    \end{aligned}
%\end{equation}
%狄拉克$\delta$分布是在包含所有平方可积函数的希尔伯特空间$L_2$上所稠密定义的一个无界线性泛函.在许多应用中,可以对$L_2$的某个子空间赋予更强的拓扑,使得$\delta$函数能够定义一个有界线性算子.
    
    %----------------------------------------------------------------------------------------
    
    \backmatter % Denotes the end of the main document content
    \setchapterstyle{plain} % Output plain chapters from this point onwards
    
    %----------------------------------------------------------------------------------------
    %	BIBLIOGRAPHY
    %----------------------------------------------------------------------------------------
    
    % The bibliography needs to be compiled with biber using your LaTeX editor, or on the command line with 'biber main' from the template directory
    
    \defbibnote{bibnote}{Here are the references in citation order.\par\bigskip} % Prepend this text to the bibliography
    %\printbibliography[heading=bibintoc, title=Bibliography, prenote=bibnote] % Add the bibliography heading to the ToC, set the title of the bibliography and output the bibliography note
    \printbibliography[heading=bibintoc, title=参考文献, prenote=bibnote] % Add the bibliography heading to the ToC and set the title of the bibliography
    
    
    
    %----------------------------------------------------------------------------------------
    %	GLOSSARY
    %----------------------------------------------------------------------------------------
    
    % The glossary needs to be compiled on the command line with 'makeglossaries main' from the template directory
    
    \setglossarystyle{listgroup} % Set the style of the glossary (see https://en.wikibooks.org/wiki/LaTeX/Glossary for a reference)
    \printglossary[title=Special Terms, toctitle=List of Terms] % Output the glossary, 'title' is the chapter heading for the glossary, toctitle is the table of contents heading
    
    %----------------------------------------------------------------------------------------
    %	INDEX
    %----------------------------------------------------------------------------------------
    
    % The index needs to be compiled on the command line with 'makeindex main' from the template directory
    
    \printindex % Output the index
    
    %----------------------------------------------------------------------------------------
    %	BACK COVER
    %----------------------------------------------------------------------------------------
    
    % If you have a PDF/image file that you want to use as a back cover, uncomment the following lines
    
    %\clearpage
    %\thispagestyle{empty}
    %\null%
    %\clearpage
    %\includepdf{cover-back.pdf}
    
    %----------------------------------------------------------------------------------------
    
\end{document}
