\chapter*{前言}
\addcontentsline{toc}{chapter}{Preface} % Add the preface to the table of contents as a chapter



\textbf{2025/4/20编: 更改了模板,彻底调整了之前的一些遗留问题}\\


思来想去, 重新写了一版前言, 本书最开始的出发点是作为学习凝聚态场论的\textbf{笔记}, 并且尝试把遇到并解决的问题记录下来, 以一个初学者视角出发, 后续可以慢慢修正表述, 这样或许对其他人有所帮助, 首先需要做的就是尽量写的明白一点, 其次之后我会把我第一次科研的完整经历, 尽可能的记录下来, 之后若是学弟们问起, 也有一个比较好的帮助.

此外, 无论我怎么强调, 依旧有人头铁去硬啃前面的数学部分, 于是我现在把它扔在最后一章了, 同时移除了辛几何部分(此部分并未上传过github), 同时将纤维丛, 同调群, de rham上同调群等等合并为一小节, 因为规范场论相关内容并不是核心内容, 大部分仅作为了解即可.

书内大部分内容Copy自 Nicolas 的 Field Thery of Condensed Matter and Ultracold Gases. 并按自身的理解重新跟着推导了一遍, 添加了自己遇到的一些问题的补充, 尝试为许多内容填上Mathematica代码补充, 目前距离完善还有很长一段距离, 同时侧重点会有所区别.

前言仅供参考,不代表最终前言内容和书中内容,该前言已经是第五版.

TODO:
\begin{enumerate}
    \item 修改一下章节前言, 使其符合现在的状况
    \item 似乎有些人电动力学学的不怎么样, 或是没有接触到规范不变性的概念, 那么A.3附录将是讨论规范变换, 规范不变性, 局域规范变换等等内容的地方
    \item 有些地方表述还是有点模糊了, 之后会修正
\end{enumerate}


\textit{\href{https://www.zhihu.com/question/33553137/answer/3573744730}{知乎上某答主Melancholy的话很有道理,这里摘录过来.}
    事实上,十卷朗道根本没必要通读——对于大多数物理工作者来说,其中比较有用的大概只有力学、场论、热统和凝聚态理论吧,另外我觉得那本力学其实缺陷很多;四大力学远远不像xxx\sidenote{原回答的某人名,这里避免争议故略去,下同}所说的那样困难——四大力学里面的确有十分复杂的技术,但四大力学的原理并不需要巨量的时间,甚至所谓的“天赋”才能够掌握,而作为未来的科研工作者,在学习已与前沿脱节的古典知识时,应当重点关注基本思想原理和理论体系的构建,而不是复杂的技术和特殊情形下的困难应用——更广泛地说,在开始科研之前学习古典知识时,一切“细节”都可以用较简单的例子和直觉替代或者完全忽略,与此同时,应当尽快学习更高级的古典知识、尽快入门前沿,否则会面临困局;然而竞赛却特别强调细节,比如,一个初等而复杂的计算,竞赛考生不能出现任何错误,否则就会丢分、无缘获奖;广义相对论也是一百多年以前提出的老理论了,所依赖的数学基础也不过是把Euclidean space 这个我们习以为常的平凡流形换成了具有pseudo-Riemannian metric 且 oriented 的 smooth manifold,一点也不高端——现代几何学完全建立在流形的基础上,在习惯这套语言之后,使用它们就应该像呼吸一样自然;另外,xxx认为科研前沿距离中学生十分遥远,一个零基础的高中毕业生需要多年的学习才能到达前沿,以及科学的进展会因此而逐渐停滞,而这也是完全错误的;何况,如果学习的唯一目的是入门某一个特定的前沿领域的研究,那么,也不必要按顺序完整学习所有的基础课程,更不需要学习看上去很有趣却完全用不到的知识;此外,跳过前置课程(中的困难部分)学习后续课程也是一种可以选择的方案——所以,到达前沿需要花费的时间可以更短;最后,“膜拜大神”的风气是十分有害的,“天赋崇拜”的风气也是十分有害的,想要在科学研究上取得进步,就必须弃之如敝履,何况那些在高中竞赛中摘金夺银的精良的应试机器们,真的值得它们的同龄人学习、甚至崇拜吗?}

\begin{flushright}
	\textit{xxx}
\end{flushright}
