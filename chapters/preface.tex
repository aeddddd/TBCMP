\chapter*{前言}
\addcontentsline{toc}{chapter}{Preface} % Add the preface to the table of contents as a chapter



\textbf{2025/4/20编: 更改了模板,彻底调整了之前的一些遗留问题}\\

\textbf{2025/3/10编: 完成了部分内容,附录中的两章目前已经被隐藏起来了,之后会挑选出需要的内容进行筛选,正文第二章正在缓慢进行中,后续彻底完成后会注释掉这些无用信息,目前目录仅供参考,还没有完全来得及修正,前言也尚未完成,仅修正了部分明显的问题}\\

\textbf{2025/2/21编: 为了更加贴合书名,后续的计划中会把第一章和第二章放在附录中,供选择性阅读,主体仍放在凝聚态场论中.后续一些细节部分需要逐步修正.一些格式上的错误将由场论初步章节完成后逐一修订.}\\


很多初学者在初次自学较高深的物理的时候往往会犯一个错误:被同等级的高深的数学所迷惑,认为学好这些物理离不开这些数学.但事实上,学好物理确实离不开数学,但仅仅是一小部分的数学.例如,学好量子力学离不开线性代数,但经常有很多物理系的学生拐去学习泛函分析,李代数,辛几何之类的内容,而这些数学内容往往只有一小部分应用在物理上面,打着先学完这些数学再开始学习物理的想法,只会让物理的学习一拖再拖.我们应当意识到,数学不过是物理的工具,切勿舍本逐末.

但这并不意味着数学不重要,离开数学的物理不亚于纸上谈兵,毫无意义.而对于本讲义,前面的部分并不十分紧要,仅作为开启凝聚态场论的先备知识,将重要的部分提取出来,作为先导知识.%第三章是为后续的大量使用做铺垫,第四章开始是文章的主体部分,最后的几章为目前在做的研究方向的总结.

前言仅供参考,不代表最终前言内容和书中内容,该前言已经是第四版.




\begin{flushright}
	\textit{530 Group}
\end{flushright}
