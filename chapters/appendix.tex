\setchapterstyle{lines}
\labch{appendix}
%\blinddocument


\chapter{补充内容}
\section{带电粒子的拉格朗日量}
我们知道拉格朗日量往往是用来描述保守体系的,面对像洛伦兹力这类非保守力,我们如果仍维持保守体系的拉格朗日方程不变,势必要引入广义势能.

我们根据电动力学的知识写出带电粒子在电磁场中的洛伦兹力:
\begin{equation}
    \boldsymbol F=q(\boldsymbol E+\boldsymbol v\times \boldsymbol  B)
\end{equation}
电磁场又满足麦克斯韦方程组
\begin{equation}
    \left\{\begin{array}{l}\nabla \times \boldsymbol{E}+\dfrac{\partial \boldsymbol{B}}{\partial t}=0 \\ \nabla \cdot \boldsymbol{E}=\rho / \varepsilon_{0} \\ \nabla \times \boldsymbol{B}-\mu_{0} \varepsilon_{0} \dfrac{\partial \boldsymbol{E}}{\partial t}=\mu_{0} \boldsymbol{j} \\ \nabla \cdot \boldsymbol{B}=0\end{array}\right.
\end{equation}
利用
\begin{equation}
    \boldsymbol{B}=\nabla \times \boldsymbol{A}
\end{equation}
代入麦克斯韦方程组的第一个式子,得到$\nabla \times\left(\boldsymbol{E}+\frac{\partial \boldsymbol{A}}{\partial t}\right)=0$,自然定义出标量势

\begin{equation}
    -\nabla \varphi=E+\frac{\partial \boldsymbol{A}}{\partial t}
\end{equation}
于是洛伦兹力可以重写为
\begin{equation}
    \boldsymbol{F}=q\left[-\nabla \varphi-\frac{\partial \boldsymbol{A}}{\partial t}+v \times(\nabla \times \boldsymbol{A})\right]
\end{equation}
现在我们需要将其写为如下形式来得到广义势能
\begin{equation}
    Q_{\alpha}=-\frac{\partial U}{\partial q_{\alpha}}+\frac{\mathrm d}{\mathrm d t} \frac{\partial U}{\partial \dot{q}_{\alpha}}
\end{equation}
首先我们写出其分量形式
\begin{equation}
    \begin{aligned} &(\nabla \varphi)_{x}=\frac{\partial \varphi}{\partial x} \\ &[v \times(\nabla \times \boldsymbol A)]_{x}=v_{y}\left(\frac{\partial A_{y}}{\partial x}-\frac{\partial A_{x}}{\partial y}\right)-v_{z}\left(\frac{\partial A_{x}}{\partial z}-\frac{\partial A_{z}}{\partial x}\right) \\ &\left(\frac{\partial \boldsymbol{A}}{\partial t}\right)_{x}=\frac{\partial A_{x}}{\partial t} \end{aligned}
\end{equation}
于是洛伦兹力的$ x $分量可以写做
\begin{equation}
    F_{x}=q\left[-\frac{\partial \varphi}{\partial x}-\frac{\partial A_{x}}{\partial t}+v_{y}\left(\frac{\partial A_{y}}{\partial x}-\frac{\partial A_{x}}{\partial y}\right)-v_{z}\left(\frac{\partial A_{x}}{\partial z}-\frac{\partial A_{z}}{\partial x}\right)\right]
\end{equation}
由于矢势$ A $是坐标和时间的函数.由于$ A $是粒子所在点的电磁场的矢势,因此$ A $中的坐标变量是粒子在对应时刻的空间位置,它们对时间的微商就是粒子的速度,因此有
\begin{equation}
    \frac{\mathrm{d} A_{x}}{\mathrm{d} t}=\frac{\partial A_{x}}{\partial t}+v_{x} \frac{\partial A_{x}}{\partial x}+v_{y} \frac{\partial A_{x}}{\partial y}+v_{z} \frac{\partial A_{x}}{\partial z} \\
\end{equation}
分量表达式变为
\begin{equation}
    F_{x}=q\left[-\frac{\partial \varphi}{\partial x}-\frac{\mathrm{d} A_{x}}{\mathrm{d} t}+v_{x} \frac{\partial A_{x}}{\partial x}+v_{y} \frac{\partial A_{y}}{\partial x}+v_{z} \frac{\partial A_{z}}{\partial x}\right] \\
\end{equation}
又因为矢量势和标量势都和速度无关,自然有
\begin{equation}
    \begin{aligned} &\frac{\mathrm{d} A_{x}}{\mathrm{d} t}=\frac{\mathrm{d}}{\mathrm{d} t}\left[\frac{\partial}{\partial v_{x}}(\boldsymbol{A} \cdot v)\right]=\frac{\mathrm{d}}{\mathrm{d} t} \frac{\partial}{\partial v_{x}}(-\varphi+\boldsymbol{A} \cdot v)\\ &\left(v_{x} \frac{\partial A_{x}}{\partial x}+v_{y} \frac{\partial A_{y}}{\partial x}+v_{z} \frac{\partial A_{z}}{\partial x}\right)=v \cdot \frac{\partial \boldsymbol{A}}{\partial x}=\frac{\partial}{\partial x}(\boldsymbol{v} \cdot \boldsymbol{A}) \end{aligned}\\
\end{equation}
终于,我们把分量式写为
\begin{equation}
    F_{x}=q\left[-\frac{\partial}{\partial x}(\varphi-\boldsymbol{A} \cdot v) |+\frac{\mathrm{d}}{\mathrm{d} t} \frac{\partial}{\partial v_{x}}(\varphi-\boldsymbol{A} \cdot v)\right]
\end{equation}
对比广义力和广义势能的式子,我们得出了广义势能的表达
\begin{equation}
    \color{red}{\boxed{U=q \varphi-q \boldsymbol{A} \cdot v} }
\end{equation}

相应的拉格朗日量自然写出
\begin{equation}
    \color{blue}{\boxed{L=\frac{1}{2} m v^{2}-q \varphi+q \boldsymbol{A} \cdot v }}
\end{equation}

\chapter{数学内容补充}
\section{楔积,外微分与外代数}
\subsection{楔积}
为了讲清楚楔积\sidenote{楔积,又称外积,但区分向量叉乘,叉乘仅适用于三维情况,而外积对维数没有要求.},我们首先需要回忆一下高数我们已经学过的内容,对于一个二元函数$f(x,y)$,其二重积分
\begin{equation}
    A=\iint_Df(x,y)\dx\dy
\end{equation}
假设为了便于计算积分,需要引入变量代换
\begin{equation}
    \begin{cases}x&=x(x^{\prime},y^{\prime})\\y&=y(x^{\prime},y^{\prime})\end{cases}
\end{equation}
坐标变换后,我们知道可以利用雅可比行列式写为如下形式\marginnote[]{或许这里不能够假定每一个人都学过雅可比行列式,对于这个坐标变换,其雅可比行列式定义为$$\left|\frac{\partial(x,y)}{\partial(x^{\prime},y^{\prime})}\right|=\frac{\partial x}{\partial x^{\prime}}\frac{\partial y}{\partial y^{\prime}}-\frac{\partial x}{\partial y^{\prime}}\frac{\partial y}{\partial x^{\prime}} $$写成行列式形式即$$\mathbf J=\left|\frac{\partial(x,y)}{\partial(x',y')}\right| = \begin{vmatrix} \frac{\partial x}{\partial y^{\prime}} & \frac{\partial x}{\partial x^{\prime}} \\ \frac{\partial y}{\partial y^{\prime}} & \frac{\partial y}{\partial x^{\prime}}  \end{vmatrix}$$其几何意义代表坐标前后体积微元(对于二维则是面积微元)的比值,推导过程并不是这一部分的重点,故省略.}
\begin{equation}
    A=\iint f(x,y)\left|\frac{\partial(x,y)}{\partial(x^{\prime},y^{\prime})}\right|\dx^{\prime}\dy^{\prime}
\end{equation}
我们发现,雅可比行列式的计算往往利用了向量叉乘来求面积,但是我们知道,向量叉乘在高维情况下是非良定义的,这要求我们对高维下的体积微分的运算引入一种新的运算符:楔积.对于上面的二重积分的积分微元$\dx\dy$,我们定义其中的运算关系:$\dx\wedge\dy$.

对于普通矢量的楔积运算,我们根据其楔积``次数"$ k $定义其为k-矢量,如矢量$\vec x$即1-矢量,$\vec x\wedge\vec y$则为2-矢量.一个k-矢量象征一个k维的\textbf{有向}体积,而由k-矢量之间的代数运算关系构成的代数被称为\textbf{外代数},这类代数最显著的特征就是反交换性:
$$\dx\wedge\dy=-\dx\wedge\dy$$
由于这种关系,自然可以发现
$$\dx\wedge \dx=-\dx\wedge \dx=0,\quad \dy\wedge \dy=-\dy\wedge\dy=0$$
利用这种新的代数关系,我们发现微元之间的坐标变换变得更加清晰
\begin{equation}
    \begin{aligned}\dx\wedge \dy&=(\frac{\partial x}{\partial x^{\prime}}\dx^{\prime}+\frac{\partial x}{\partial y^{\prime}}\dy^{\prime})\wedge(\frac{\partial y}{\partial x^{\prime}}\dx^{\prime}+\frac{\partial y}{\partial y^{\prime}}\dy^{\prime})\\&=\frac{\partial x}{\partial x^\prime}\frac{\partial y}{\partial y^\prime}\dx^\prime\wedge \dy^\prime+\frac{\partial x}{\partial y^\prime}\frac{\partial y}{\partial x^\prime}\dy^\prime\wedge \dx^\prime\\&=(\frac{\partial x}{\partial x^\prime}\frac{\partial y}{\partial y^\prime}-\frac{\partial x}{\partial y^\prime}\frac{\partial y}{\partial x^\prime})\dx^\prime\wedge \dy^\prime\\&=\left|\frac{\partial(x,y)}{\partial(x^{\prime},y^{\prime})}\right|\dx^{\prime}\wedge \dy^{\prime}.\end{aligned}
\end{equation}
在运算过程中,雅可比行列式自动出现了,并且,这类关系自然与高维情况下兼容,对于$ n $重微元$\dx^1\dx^2\cdots\dx^n$,其以外代数形式写作$\dx^1\wedge\dx^2\wedge\cdots\wedge\dx^n$,并且满足反交换关系
$$\dx^i\wedge \dx^j=-\dx^j\wedge \dx^i$$
并且,我们将被积函数$f(x^1,x^2,...,x^n)$与$\dx^1\wedge\dx^2\wedge\cdots\wedge\dx^n$乘起来称为一个$ n $重微分形式,即n形式记为$\omega$:
\begin{equation}
    \omega=f(x^1,x^2,...,x^n)\dx^1\wedge\dx^2\wedge\cdots\wedge\dx^n
\end{equation}
于是,我们可以重新写出$ n $元函数的积分表达,其实际上是对n形式$\omega$的积分
\begin{equation}
    A=\int\omega
\end{equation}
对于$ n $个变量的情况,我们对n形式进行推广,定义k-形式(k-form)$\alpha$:
\begin{equation}
    \alpha=\frac{1}{k!}\alpha_{i_1i_2\cdots i_k}\dx^{i_1}\wedge \dx^{i_2}\wedge\cdots\wedge \dx^{i_k}
\end{equation}
这里使用了爱因斯坦求和约定.对于任意的k-形式,我们要求k-形式的分量$\alpha_{i_1i_2\cdots i_k}$对$ k $个指标两两反对称,即全反对称\sidenote{即交换任意两个指标变号}.

对于三维情况,由于只有三个变量,自然仅存在0,1,2,3 四种非零的微分形式\marginnote[]{由于外代数的反交换性质,很显然k-形式中的$k$个指标$i_1,i_2,...,i_k$取值必须两两不同,否则对$\alpha$的贡献将为零.特别的,这意味着,n形式是最高重的非零形式,任何$k>n$的$k$形式都必定为零,因为这时候它的$k$个指标取值必定会出现重复,不可能两两不同.},我们对于这四种微分形式分别讨论:
\begin{enumerate}
    \item 0-形式就是普通的标量函数
    \item 1-形式可以写为$a_1\dx+a_2\dy+a_3\dz$,不难发现其分量$(a_1,a_2,a_3)$恰好构成一个三维空间的矢量场$\mb{a}(\mb x)$,于是三维空间的1-形式可以写为$$a_1\dx+a_2\dy+a_3\dz=\mb{a}(\mb x)\cdot\ddx$$
    \item 2-形式同样可以写为\marginnote[]{\textbf{特别的},对于2-形式,记$a_{12}=b_3,a_{23}=b_1,a_{31}=b_2$,并且定义映射$\dx\wedge \dy\to \dz,\dy\wedge \dz\to \dx,\dz\wedge \dx\to \dy$, 则有$a_{12}\dx\wedge \dy+a_{23}\dy\wedge \dz+a_{31}\dz\wedge \dx\to b_1\dx+b_2\dy+b_3\dz$
    可见,3 维空间 2-形式和 1-形式之间能够建立一对一映射.因此,2-形式的 3 个独立非零分量$(a_{23},a_{31},a_{12})=(b_1,b_2,b_3)=\mathbf{b}$刚好构成一个 3 维空间矢量.
    同时,上面这个对映也告诉我们,$(\dy\wedge \dz,\dz\wedge \dx,\dx\wedge \dy)$也完全类似于一个 3 维空间的矢量微元,通常将之定义为面积元矢量$\d\mathbf{S}$,
        $$\d\mathbf{S}=(\dy\wedge \dz,\dz\wedge \dx,\dx\wedge dy)$$ 我们称这类k-形式与n-k-形式之间的一一映射关系为\textbf{霍奇(Hodge)对偶}}
    $$a=\frac{1}{2}a_{ij}(\mathbf{x})\dx^i\wedge \dx^j=a_{12}\dx\wedge \dy+a_{23}\dy\wedge \dz+a_{31}\dz\wedge \dx$$
    \item 3-形式与0-形式类似,可以写做$f(x,y,z)\dx\wedge\dy\wedge\dz$
\end{enumerate}
\subsection{外微分}
对于微分形式,为了与之前的微分运算进行一定的区分,我们定义一种新的微分运算:外微分.我们首先以一个简单的例子引入:对于二维空间内的1-形式$a=a_x\dx+a_y\dy$,我们定义其外微分$\d a$为
\begin{equation}
    \d a=\d a_x\wedge\dx+\d a_y\wedge \dy
\end{equation}
这样的外微分显然是结合外代数与微分的产物,对其进行计算得到\sidenote{具体运算步骤$$\begin{aligned}\d a&=\d a_x\wedge \dx+\d a_y\wedge\dy\\&=(\partial_xa_x\dx+\partial_ya_x\dy)\wedge \dx\\&\quad+(\partial_xa_y\dx+\partial_ya_y\dy)\wedge \dy\\&=\partial_ya_x\dy\wedge \dx+\partial_xa_y\dx\wedge \dy\\&=(\partial_xa_y-\partial_ya_x)\dx\wedge \dy,\end{aligned}$$}
\begin{equation}
    \d a=(\partial_xa_y-\partial_ya_x)\dx\wedge\dy
\end{equation}
显然,$\d a$只有一个分量$(\partial_xa_y-\partial_ya_x)$,并且其恰好是二维矢量$\mb a$的\textbf{旋度}.

另一方面,在高数中,我们熟知二维空间内有格林公式
\begin{equation}
    \oint_{\partial D}(a_x\dx+a_y\dy)=\int_D(\partial_xa_y-\partial_ya_x)\dx\dy
\end{equation}
其中$\partial D$为一个闭合回路,$D$为这条回路包括的区域,利用微分形式,可以将其重新写为
\begin{equation}\label{B1}
    \int_{\partial D}a=\int_D\d a
\end{equation}
类似于二维空间的1-形式,我们继续分析三维空间的1-形式$a=\mathbf{a}\cdot \d\mathbf{x}=a_x\dx+a_y\dy+a_z\dz$的外微分$\d a$,其定义同样为
\begin{equation}
    \d a=\d a_x\wedge \dx+\d a_y\wedge \d y+\d a_z\wedge \dz
\end{equation}
通过与二维情况完全类似的运算,我们得到结果
\begin{equation}
    \d a=(\partial_xa_y-\partial_ya_x)\dx\wedge \dy+(\partial_ya_z-\partial_za_y)\dy\wedge \dz+(\partial_za_x-\partial_xa_z)\dz\wedge \dx
\end{equation}
这是一个三维空间的2-形式,$\d a$的三个独立分量同样恰好构成三维矢量$\mb a$的\textbf{旋度}$\nabla\times\mb a$,即
\begin{equation}
    \d a=(\nabla\times\mathbf{a})\cdot \d\mathbf{S}
\end{equation}
使用微分形式重写
\begin{equation}\label{B2}
    \int_{\partial D}a=\int_D\d a
\end{equation}
于是我们发现,\ref{B1}与\ref{B2}完全一致.

我们继续向下考虑三维空间的2-形式的外微分$\d a$,其定义同样为
\begin{equation}
    \da=\da_{12}\wedge \dx\wedge \dy+\da_{23}\wedge \dy\wedge \dz+\da_{31}\wedge \dz\wedge \dx
\end{equation}
计算得到
\begin{equation}
    \da=(\partial_3a_{12}+\partial_1a_{23}+\partial_2a_{31})\dx\wedge \dy\wedge \dz
\end{equation}
同样利用$(a_{23},a_{31},a_{12})=(b_1,b_2,b_3)=\mathbf{b}$,我们可以把结果重写为
\marginnote[]{对于外微分,存在两个常用概念.首先,一个微分形式$\alpha$,如果它的外微分等于零,即$\d\alpha=0$ 我们就称它为\textbf{闭形式}.其次,一个微分形式$\alpha$,如果它是另一个微分形式$\beta$的外微分,即有$\alpha=\d\beta$,我们就称这样的$\alpha$为一个\textbf{恰当形式}.很显然,任何恰当形式都必定是闭形式!反过来,闭形式却不一定是恰当形式,闭形式什么时候是恰当形式什么时候不是.这往往和空间的拓扑有关系,是所谓的 de Rahm 上同调研究的内容.}
\begin{equation}
    \da=(\partial_1b_1+\partial_2b_2+\partial_3b_3)\dx\wedge \dy\wedge \dz=(\nabla\cdot\mathbf{b})\dx\wedge \dy\wedge \dz
\end{equation}
很明显结果是一个3-形式,并且恰好给出了三维矢量的\textbf{散度}.

于是,我们发现,对于三维空间的高斯定理
\begin{equation}
    \oint_{\partial V}\mathbf{b}\cdot \d\mathbf{S}=\int_V(\nabla\cdot\mathbf{b})\d V
\end{equation}
利用外微分仍然可以写完类似\ref{B1}与\ref{B2}的形式.我们将其扩展为
\begin{equation}\label{B3}
    \int_{\partial D}\alpha=\int_D\d\alpha
\end{equation}
式中$\alpha$表示三维空间中的一个k-1-形式,$D$表示三维空间中一个以$\partial D$为边界的$k$维曲面 (因此$\partial D$是$k-1$维的,而$d\alpha$则是一个k-形式).$k=2$时,它就是斯托克斯公式, $k=3$时它就是高斯定理.可见,利用外微分运算,我们可以将矢量分析中那些著名的公式和定理统一起来.

我们可以将其扩展到$ n $维形式,k-1-形式的外微分为一个k-形式,\ref{B3}也被称为\textbf{广义斯托克斯公式}.

外微分有一个尤为重要的性质,对任意微分形式连续进行两次外微分,结果恒等于0\marginnote[]{对于外微分,存在类似于普通复合函数的莱布尼茨公式$$\mathrm{d}(\mu\wedge\lambda)=\mathrm{d}\mu\wedge\lambda+(-1)^{\deg\mu}\mu\wedge\mathrm{d}\lambda$$其中$\deg\mu$为微分形式$\mu$的次数.}
\begin{equation}
    \d^2\alpha=\d(\d\alpha)\equiv0
\end{equation}
我们知道了外微分可以令相应的微分形式维度上升一次,同样的,我们引入两个类似的算符来表达对于微分形式的其他操作.\\
在前面,我们了解了k-形式与n-k-形式之间存在一一映射,被称为\textbf{霍奇对偶},于是引入运算符$\star$(有时也使用$*$),即对k-形式取其n-k-形式.我们将在微分几何后面再次讨论.对于
$$\mathrm{d}f=\frac{\partial f}{\partial x}\mathrm{d}x+\frac{\partial f}{\partial y}\mathrm{d}y+\frac{\partial f}{\partial z}\mathrm{d}z$$
取$\star$运算后,即得
\begin{equation}
    \star\mathrm{d}f=\frac{\partial f}{\partial x}\mathrm{d}y\wedge\mathrm{d}z+\frac{\partial f}{\partial y}\mathrm{d}z\wedge\mathrm{d}x+\frac{\partial f}{\partial z}\mathrm{d}x\wedge\mathrm{d}y
\end{equation}
以及与$\d$对应的可以令微分形式降次的运算符$\delta$,定义为
\begin{equation}
    \delta=-(-1)^g(-1)^{n(k+1)}\star\mathrm{d}\star
\end{equation}
其中,$g$为度规.

利用外微分,我们可以把我们熟知的麦克斯韦方程组(四维时空)变得更加简洁
\begin{equation}
    \begin{cases} \mathrm{d}F=0\\ \mathrm{d}\star F=\mu_0\star J \end{cases}
\end{equation}
当然,当我们继续深入学习,我们可以得到最终的数学表述:电磁理论是$U(1)$丛上的\textbf{联络}
\begin{equation}
    \psi(x)\mapsto e^{\mathrm{i}\varphi(x)}\psi(x)
\end{equation}
\subsection{简单实例}
\subsubsection{保守力}
回顾力学中我们对于质点系保守力的定义:
\begin{equation}
    \sum_i\mathbf{F}_i\cdot \d\mathbf{x}_i=-\d V.
\end{equation}
并引入指标$\mu=1,2,3,\cdots,3N$,即对每个质点的3个笛卡尔坐标,于是保守力自然可以重写为
\begin{equation}
    F_\mu \dx^\mu=-\d V(x^1,...,x^{3N})
\end{equation}
显然这是一个1-形式,并将其简记为$ F $,并且我们能够发现其还是一个\textbf{恰当形式},满足
\begin{equation}
    F=-\d V
\end{equation}
势能为0-形式,在前面我们知道一个恰当形式必是\textbf{闭形式},立刻得出
\begin{equation}
    \d F=0
\end{equation}
将$\d F$展开运算,得到
\begin{equation}
    \d F=(\partial_\mu F_\nu)\dx^\mu\wedge \dx^\nu=[\frac{1}{2}(\partial_\mu F_\nu-\partial_\nu F_\mu)+\frac{1}{2}(\partial_\mu F_\nu+\partial_\nu F_\mu)]\dx^\mu\wedge \dx^\nu
\end{equation}
后一项由于反对称关系为$ 0 $,于是有
\begin{equation}
    \d F=\frac{1}{2}(\partial_\mu F_\nu-\partial_\nu F_\mu)\dx^\mu\wedge \dx^\nu
\end{equation}
即等价于
\begin{equation}
    \partial_\mu F_\nu-\partial_\nu F_\mu=0
\end{equation}
对于单个质点,我们得出的结果自然为对力$\mb F$的旋度$\nabla\times\mb F=0$,并且由斯托克斯公式
\begin{equation}
    \int_{\partial D}F=\int_D\d F=0
\end{equation}
这意味着保守力1-形式在坐标空间内的任意闭合回路积分都为01,我们常称这类积分为\textbf{功},于是,这也就是说明保守力在任意闭合回路做的功为$ 0 $.

\subsubsection{热力学}
1
\section{微分流形}
1
\section{流形上的外微分}
1
\section{纤维丛}
1
\section{同调群}
1
\section{\textbf{De Rham}上同调群}
1
\section{仿射联络空间与黎曼流形}
1
\section{电磁理论是$U(1)$丛上的联络}
1





\chapter{mathematica的基本用法}
\subsection*{4.1-式子13}
输入代码
\begin{lstlisting}
    Integrate[
    1/(2 \[Pi]) Exp[-I \[Epsilon] (p^2/(2 m) + V) + 
    I p (qf - qi)], {p, -Infinity, Infinity}] // FullSimplify
\end{lstlisting}

\chapter{答案及解析}
\section{第一章}
1
\section{第二章}
1
\chapter{致谢/参考}
\section{致谢}
感谢elegantbook所提供的模板,\href{https://elegantlatex.org/}{https://elegantlatex.org/}.\\
并非elegantbook

\section{参考}
本文主要参考的书籍和期刊如下:
\begin{enumerate}
    \item Modern Quantum Mechanics 2nd.J.J.Sakurai
    \item Quantum Field Theory in Condensed Matter Physics 2nd.Alexei M.Tsvellk
    \item Entanglement in Many-Body Systems(RMP)
    \item 物理学家用李群李代数
    \item 《李群与李代数》讲义-李世雄
    \item Nicolas Dupuis - Field Theory of Condensed Matter and Ultracold Gases
    \item Conformal Field Theory A.N. Schellekens
    \item 经典力学新讲-陈童(主要参考部分为微分形式部分)
\end{enumerate}
参考了十余本,之后慢慢整理,并会改为bib引用的.

免责声明:本讲义仅限个人学习使用,仅供参考.部分图片为个人重绘,部分图片来源wiki.同时,本讲义的全部内容和代码已公开至github主页,允许二次传播,允许自行更改,但请注意讲义仅为多本书和综述的整理和综合,且仍处于更新状态,不排除未来将前几章重置的可能性.
















%\chapter{单位制}
%我们从小学就逐步接触一些单位,常见的如米(m),千克(kg),秒(s)等是国际统一使用的\textbf{标准度量系统(国际单位制)}.相应的,像是国内经常接触的斤,公里,亩,美国\sidenote[][]{包括美国、开曼群岛、伯利兹等极少数国家和地区}常用的华氏度等,则是生活中使用的独立度量系统,大多数度量系统都和标准度量系统之间存在换算关系.而且生活中使用的度量单位大多比较局限,对于相干度较低的单位往往是不涉及的.\\
%对于初中和高中的物理学习,我们已经熟练使用国际单位制(SI\sidenote[][]{法语 Système International d'Unités,简称SI})来解决一些简单的物理问题.但是,就像生活中使用的单位制一样,人们出于方便的角度对于一些物理场景也构建出一些新的单位制.这些单位制能够简化相关的物理问题.\\物理上使用的单位制与国际单位制的转换往往比较复杂,使用时建议标注使用了哪个单位制.\\
%\begin{remark}
%    在这个附录中,电磁单位制与自然单位制独立分为两节,但是按照较广义的自然单位制的定义\sidenote[][]{区别于粒子物理的``自然单位制"和普朗克单位制},电磁单位制也属于其中的一类,特此说明.
%\end{remark}
%\section{电磁单位制}
%相比于我们常用的国际单位制,也称为MKSA单位制(即米,千克,秒,安培),我们在电磁中常用的高斯单位制被称为CGS单位制(即厘米,克,秒).\\接下来为了避免混乱,列举高斯单位制所常用的单位:电荷$statC$,电势$statV$,力$dyne$\sidenote[][]{中文音译为达因},磁感应强度$gauss$,磁场强度$oersted$,磁通量$mx$,能量$erg$.\\
%
%相比于自然单位制直接将值赋为1,高斯单位制就比较保守,它根据我们熟知的库仑定律,通过定义$1\mathrm{A}=0.1c\cdot\mathrm{\d yne}^{\frac12},1\mathrm{C}=0.1c\cdot\rm{\d yne}^{\frac12}\cdot s$来达到简化的操作.\\
%\begin{table}[htbp]
%    \centering
%    \caption{一些简单对应关系}
%    \begin{tabular}{|c|c|c|c|}
    %        \hline
    %        & SI      & Gaussian        & G/SI                     \\ \hline
    %        E          & $V/m$   & $statV/m$       & $\sqrt{4\pi\epsilon_0}$  \\ \hline
    %        V          & $V$     & $statV$         & $\sqrt{4\pi\epsilon_0}$  \\ \hline
    %        D          & $C/m^2$ & $statC/cm^2$    & $\sqrt{4\pi/\epsilon_0}$ \\ \hline
    %        q          & $C$     & $statC$         & $1/\sqrt{4\pi\epsilon_0}$ \\ \hline
    %        P          & $C/m^2$ & $statC/cm^2$    & $1/\sqrt{4\pi\epsilon_0}$ \\ \hline
    %        I          & $A$     & $statC/s$       & $1/\sqrt{4\pi\epsilon_0}$ \\ \hline
    %        B          & $T$     & $Gauss$         & $\sqrt{4\pi/\mu_0}$      \\ \hline
    %        A          & $Wb/m$  & $Gauss\cdot cm$ & $\sqrt{4\pi/\mu_0}$      \\ \hline
    %        H          & $A/m$   & $oersted$       & $\sqrt{4\pi\mu_0}$       \\ \hline
    %        $\epsilon$ & $F/m$   & 1               & $1/\epsilon_0$           \\ \hline
    %        $\mu$      & $H/m$   & 1               & $1/\mu_0$                \\ \hline
    %    \end{tabular}
%\end{table}
%\section{自然单位制}
%我们熟知,国际单位制的7个基本单位是通过物理常数所定义的,那么,如果我们把其中\textbf{一个或几个}的定义值改为1,那么就又可以构造出来一套度量系统.这其中\textbf{显而易见的优点}是直接导致原本含有大量常数的公式可以被写成更加简洁方便的形式.在物理学里,自然单位制就是一种建立于此类方法的计量单位制度.例如,电荷的自然单位是基本电荷${\displaystyle e}$,速度的自然单位是光速${\displaystyle c}$,角动量的自然单位是约化普朗克常数${\displaystyle \hbar }$,电阻的自然单位是自由空间阻抗${\displaystyle Z_{0}}$,质量的自然单位则有电子质量${\displaystyle m_{e}}$与质子质量${\displaystyle m_{p}}$等.\\
%
%事实上,对于单位的改动,我们至少要求不会导致无量纲常数的值发生改变,如精细结构常数.
%$${\displaystyle \alpha ={\frac {e^{2}k_{e}}{\hbar c}}={\frac {e^{2}}{\hbar c(4\pi \epsilon _{0})}}={\frac {1}{137.035999074}}=7.2973525698\cdot 10^{-3}}$$
%这个常数就要求不能同时把${\displaystyle e},{\displaystyle \hbar },{\displaystyle c},{\displaystyle k_{e}}$同时为1.
%\subsection{普朗克单位制}
%普朗克单位制几乎是最常使用的单位制,它的定义只依赖于最基本的性质.普朗克单位选择将真空光速${\displaystyle c}$,万有引力常数${\displaystyle G}$,约化普朗克常数${\displaystyle \hbar }$,真空电容率${\displaystyle \epsilon _{0}}$,玻尔兹曼常数${\displaystyle k_{B}}$定为1\sidenote[][]{普朗克洛伦兹-亥维赛单位制将${\displaystyle 4\pi G},{\displaystyle \epsilon _{0}}$定为1,普朗克高斯单位制将${\displaystyle G},{\displaystyle 4\pi \epsilon _{0}}$定为1}.\\
%类比国际单位制,普朗克单位制也有一些基本单位(如常常出现在各种科普作品中的普朗克长度,普朗克时间等)和导出单位(普朗克面积,普朗克动量等).具体列表可参考相关wiki\href{https://zh.wikipedia.org/wiki/%E6%99%AE%E6%9C%97%E5%85%8B%E5%96%AE%E4%BD%8D%E5%88%B6}{普朗克单位制},这里不做展开.
%\subsection{``自然单位制"(粒子物理)}
%在粒子物理中,自然单位制特指${\displaystyle \hbar =c=k_{B}=1}$情况下的单位制.通常会根据情况选择使用洛伦兹-亥维赛单位制或高斯单位制来确定电荷定义.
%\subsection{其他单位制}
%\subsubsection*{史东纳单位制}
%第一次出现的单位制,已经不再使用.规定了${\displaystyle c=G=e={\frac {1}{4\pi \epsilon _{0}}}=k_{B}=1}$.
%\subsubsection*{原子单位制}
%这类单位制是特别为了简易表达原子物理学和分子物理学的方程而精心设计,在本篇中仅做介绍.\\
%原子单位制分为两种:哈特里原子单位制和里德伯原子单位制.哈特里原子单位制比里德伯原子单位制常见.两者的主要区别在于质量单位与电荷单位的选取.\\
%哈特里原子单位制的基本单位为${\displaystyle e=m_{e}=\hbar ={\frac {1}{4\pi \epsilon _{0}}}=k_{B}=1}$,${\displaystyle c={\frac {1}{\alpha }}}$.\\
%里德伯原子单位制的基本单位为${\displaystyle {\frac {e}{\sqrt {2}}}=2m_{e}=\hbar ={\frac {1}{4\pi \epsilon _{0}}}=k_{B}=1}$,${\displaystyle c={\frac {2}{\alpha }}}$.
%\chapter{固体物理中的一些概念}
%对于物理研究,把它放在合适的空间下能够简化问题.对于坐标空间(正格子,基矢)和动量空间(倒格子,倒格矢)来讲,相当于从两个角度来描写\textbf{同一}事物.在之后对于晶格的分析中,我们常常要在动量空间上分析这一问题.\\
%\textit{如果对于物理形式较为敏感,应该会容易的想到``两个角度描写同一事物"的表述和傅里叶变换有很大的相似性.实际上,坐标空间和动量空间互为傅里叶变换.如果对于量子力学有一定了解或已经阅读过关于表象变换的内容,对这一部分会有更深的体会.}
%\section*{一些需要了解的概念}
%为了能够便于理解接下来的内容,以下是需要了解的概念.
%\begin{enumerate}
%    \item \textbf{格矢}:联系任两个晶格点的向量
%    \item \textbf{布拉维晶格 Bravais lattices}:由同种原子构成的晶胞,多种原子构成的晶胞可以视为几个布拉维晶格的叠加.
%    \item 待补充
%\end{enumerate}
%
%
%\chapter{$\delta_{ij},\varepsilon_{ijk}$和爱因斯坦求和约定与$\delta$函数}
%\section{克罗内克符号$\delta_{ij}$}
%克罗内克符号是一类二元函数,其通常定义为以下形式
%\begin{equation}
%    \delta _{ij}=\left\{{\begin{matrix}1&(i=j)\\0&(i\neq j)\end{matrix}}\right.\,\!
%\end{equation}
%其具有筛选性(和投影算符类似)
%\begin{equation}
%    \sum _{i=-\infty }^{\infty }\delta _{ij}a_{i}=a_{j}\,\!
%\end{equation}
%其具有和$\delta$函数共同的部分性质,而$\delta$函数也正是源于克罗内克符号.
%\section{列维西维塔符号$\varepsilon_{ijk}$}
%列维-奇维塔符号,对于正整数 $n$ ,它以$1, 2, ..., n $所形成排列的奇偶性来定义.其他名称包括排列符号、反对称符号与交替符号.
%
%而$\varepsilon_{ijk}$的值由下角标$ijk$决定,当存在任意两个角标相同时值取$0$,当全部指标都不相等时,角标的逆序数为偶数取$1$,为奇数则取$0$.
%\subsection*{二维形式}
%\begin{equation}
%    \varepsilon_{ij}={
    %        \begin{cases}+1&\text{当}\left(i,j\right)=\left(1,2\right)\\
        %            -1&\text{当}\left(i,j\right)=\left(2,1\right)\\
        %            0&\text{当}i=j
        %        \end{cases}
    %    }\,
%\end{equation}
%二维较为少见,仅作为了解.
%\subsection*{三维形式}
%我们经常看到的列维西维塔符号常常是三维形式的,即如下
%\begin{equation}
%    \varepsilon _{ijk}={
    %        \begin{cases}
        %            +1&\text{当}(i,j,k)=(1,2,3),(2,3,1),(3,1,2)\\
        %            -1&\text{当}(i,j,k)=(3,2,1),(2,1,3),(1,3,2)\\
        %            0&\text{当}i=j,j=k\text{或}k=i
        %        \end{cases}
    %    }\,
%\end{equation}
%\subsection*{性质}
%两个列维-奇维塔符号的积,可以用一个以克罗内克符号表示的行列式求得
%\begin{equation}
%    \varepsilon _{ijk\dots }\varepsilon _{mnl\dots }={\begin{vmatrix}\delta _{im}&\delta _{in}&\delta _{il}&\dots \\\delta _{jm}&\delta _{jn}&\delta _{jl}&\dots \\\delta _{km}&\delta _{kn}&\delta _{kl}&\dots \\\vdots &\vdots &\vdots \\\end{vmatrix}}
%\end{equation}
%也可以用来表示行列式和向量内积,对于一个$3\times3$的方阵$A$,有表示
%\begin{equation}
%    \det(A)=\sum _{i,j,k=1}^{3}\varepsilon _{ijk}\,a_{1i}\,a_{2j}\,a_{3k}
%\end{equation}
%对于向量内积,有
%\begin{equation}
%    {\boldsymbol {a}}\times {\boldsymbol {b}}={\begin{vmatrix}{\boldsymbol {e}}_{1}&{\boldsymbol {e}}_{2}&{\boldsymbol {e}}_{3}\\a_{1}&a_{2}&a_{3}\\b_{1}&b_{2}&b_{3}\\\end{vmatrix}}=\sum _{1\leq i,j,k\leq 3}\varepsilon _{ijk}\,a_{i}b_{j}\,{\boldsymbol {e}}_{k}
%\end{equation}
%\section{爱因斯坦求和约定}
%爱因斯坦求和约定是一种标记的约定,即重复角标意味着求和,一般未指定的情况下就是由$1$至$3$.
%\section{$\delta$函数}
%我们首先明确,$\delta$函数并不是通常意义的函数,其更准确的称呼是广义函数.其在$x=0$处取值为正无穷,在$x\ne0$处取值为$0$,但是在全定义域上的积分值为$1$,当然,我们所学的黎曼积分并不支持这一操作,我们需要\textbf{勒贝格积分}来处理它,当然,这里并不会讲述测度论的内容,仅仅作为了解即可.
%\subsection{定义}
%我们从其定义开始这部分内容.
%\begin{equation}
%    \delta (x)={\begin{cases}+\infty ,&x=0\\0,&x\neq 0\end{cases}}
%\end{equation}
%且同时满足
%\begin{equation}
%    \int _{-\infty }^{\infty }\delta (x)\,\d x=1
%\end{equation}
%而对于复变函数中,一切在域$D$中闭包的全纯函数,我们可以用柯西积分公式来表示$\delta$函数
%\begin{equation}
%    \delta _{z}[f]=f(z)={\frac {1}{2\pi i}}\oint _{\partial D}{\frac {f(\zeta )\,\d\zeta }{\zeta -z}}.
%\end{equation}
%\subsection{性质}
%首先,回忆克罗内克符号的部分,很容易联想到$\delta$函数也具有筛选的性质:
%\begin{equation}
%    \int _{-\infty }^{\infty }f(x)\delta (x-x_{0})\,\d x=f(x_{0})
%\end{equation}
%以及(高维的情况与之类似,不再重复,相关证明是显然的)
%\begin{equation}
%    \begin{aligned}
    %        \delta (\alpha x)&={\frac {\delta (x)}{|\alpha |}}\\
    %        \delta (-x)&=\delta (x)\\
    %        x\delta (x)&=0\\
    %        \int _{-\infty }^{\infty }\delta (\xi -x)\delta (x-\eta )\,\mathrm {d} x&=\delta (\xi -\eta )
    %    \end{aligned}
%\end{equation}
%狄拉克$\delta$分布是在包含所有平方可积函数的希尔伯特空间$L_2$上所稠密定义的一个无界线性泛函.在许多应用中,可以对$L_2$的某个子空间赋予更强的拓扑,使得$\delta$函数能够定义一个有界线性算子.