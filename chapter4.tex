\ifx\allfiles\undefined

% 如果有这一部分另外的package,在这里加上
% 没有的话不需要

\begin{document}
	\else
	\fi
\chapter{场论初步}
\begin{introduction}
	\item 路径积分
	\item 传播子
	\item 配分函数
	\item 系综
\end{introduction}
\begin{comment}
	注释:可爱且漂亮的公式形式.
	\begin{equation}
		\vspace{\baselineskip}
		
		定位点a \tikzmarknode{a}{\highlight{red颜色}{$公式内容$}}
		定位点s \tikzmarknode{s}{\highlight{blue}{$公式内容$}}
		
		定位点编码标准:1eq1 (公式序号eq该公式第几个)
	\end{equation}
	\begin{tikzpicture}[overlay,remember picture,>=stealth,nodes={align=left,inner ysep=1pt},<-]
	% 对于 "1" 定位点
	\path (1eq1.north) ++ (0,2em) node[anchor=south east,color=red!67] (eq1/1){\textbf{动能}};
	\draw [color=red!87](1eq1.north) |- ([xshift=-0.3ex,color=red]eq1/1.south west);
	% 对于 "2" 定位点
	\path (1eq2.south) ++ (0,-1.5em) node[anchor=north west,color=blue!67] (eq1/2){\textbf{势能}};
	\draw [color=blue!57](1eq2.south) |- ([xshift=-0.3ex,color=blue]eq1/2.south east);
	\end{tikzpicture}
\end{comment}
在这一章,我们正式进入主题,同时也是最重要的开始部分,在这一章我们会接触许多新概念,我将尽量的把内容详细的解释出来,在后续的时候可能会对其进行再编辑,请以最新版本为准!
\section{路径积分}
在初等量子力学的学习中,我们在经典量子化的框架内进行表述,在本节,我们将初步探索另一种表述方法:\textbf{路径积分法}.
\subsection{量子系统与经典系统中的路径积分}
我们采取通过一个最基本物理图像的方式来引入路径积分:考虑一维空间内的一个有质量$ m $的粒子,其动力学可以通过拉格朗日量来表述:\footnote{本章开始采取这类更加清晰的公式标注方法.}\\
	
\begin{equation}
	\vspace{\baselineskip}
	L(q,\dot{q})=\tikzmarknode{1eq1}{\highlight{red}{$\frac12m\dot{q}^2$}}-\tikzmarknode{1eq2}{\highlight{blue}{$V(q)$}}
\end{equation}
	\begin{tikzpicture}[overlay,remember picture,>=stealth,nodes={align=left,inner ysep=1pt},<-]
		% 对于 "1" 定位点
		\path (1eq1.north) ++ (0,2em) node[anchor=south east,color=red!67] (eq1/1){\textbf{动能项}};
		\draw [color=red!87](1eq1.north) |- ([xshift=-0.3ex,color=red]eq1/1.south west);
		% 对于 "2" 定位点
		\path (1eq2.south) ++ (0,-1.5em) node[anchor=north west,color=blue!67] (eq1/2){\textbf{势能项}};
		\draw [color=blue!57](1eq2.south) |- ([xshift=-0.3ex,color=blue]eq1/2.south east);
	\end{tikzpicture}
	
并假设粒子位于与时间无关的势$ V(q) $中.我们使用广义坐标$ q $来表述粒子的位置\footnote{广义坐标往往指一组无关联约束的坐标,即对于三维坐标表述,如果$x,y,z$之间没有约束方程,那么$x,y,z$就可以认为是一组广义坐标.不过,目前可以仅把它们当做特殊的坐标/动量(对于广义动量).},对于广义坐标,同样有$\dot{q}=\partial_t q$.于是,根据我们在理论力学中所学到的欧拉-拉格朗日方程
\begin{equation}
	\frac{\d}{\dt}\frac{\partial L(q,\dot{q})}{\partial\dot{q}}=\frac{\partial L(q,\dot{q})}{\partial q}\quad\text{也就是}\quad m\ddot{q}=-\frac{\partial V(q)}{\partial q}.
\end{equation}
同时,我们考虑由下式定义的\textbf{哈密顿量}$H(p,q)$
\begin{equation}
 H(p,q)=p\dot{q}-L(q,\dot{q})=\frac{p^2}{2m}+V(q)\quad\text{其中广义坐标被定义为}\quad p=\frac{\partial L(q,\dot{q})}{\partial\dot{q}}=m\dot{q}
\end{equation}
我们于是通过哈密顿量给出该粒子的经典动力学关系(特别强调的一点是,在对哈密顿量偏导时,我们认为广义坐标和广义动量是无关联的.):
\begin{equation}
	\dot{p}=-\frac{\partial H(p,q)}{\partial q}=-\frac{\partial V(q)}{\partial q},\dot{q}=\frac{\partial H(p,q)}{\partial p}=\frac{p}{m}
\end{equation}

在量子力学中,我们把广义坐标和广义动量的关系上升至对易关系(后续我们把广义坐标和广义动量简称为坐标和动量).
\begin{equation}
	[\hat{q},\hat{p}]=i\hbar
\end{equation}
同时经典变量$A(p,q)$同时上升为算符$\hat{A}\equiv A(\hat{p},\hat{q})$自然经典哈密顿量也变为量子哈密顿量方程.
\begin{equation}
	\hat{H}=\frac{\hat{p}^2}{2m}+V(\hat{q})
\end{equation}


我们知道系统的物理态由希尔伯特空间$ \mathcal H $中的态矢量$|\psi(t)\ra$所描述,我们使用薛定谔方程来描述态矢量的时间演化:\\

\begin{equation}
	\vspace{\baselineskip}
	%公式编号: 7
	\tikzmarknode{7eq1}{\highlight{red}{$i\hbar\partial_t$}}|\psi(t)\rangle=\tikzmarknode{7eq2}{\highlight{blue}{$\hat{H}$}}|\psi(t)\rangle
\end{equation}
\begin{tikzpicture}[overlay,remember picture,>=stealth,nodes={align=left,inner ysep=1pt},<-]
	% 对于 "1" 定位点
	\path (7eq1.north) ++ (0,2em) node[anchor=south east,color=red!67] (eq7/1){\textbf{能量$ E $}};
	\draw [color=red!87](7eq1.north) |- ([xshift=-0.3ex,color=red]eq7/1.south west);
	% 对于 "2" 定位点
	\path (7eq2.south) ++ (0,-1.5em) node[anchor=north west,color=blue!67] (eq7/2){\textbf{哈密顿量算符}};
	\draw [color=blue!57](7eq2.south) |- ([xshift=-0.3ex,color=blue]eq7/2.south east);
\end{tikzpicture}

我们熟知我们可以利用时间演化算符$\hat U$写成方程的解:
\begin{equation}
	|\psi(t)\rangle=\hat{U}(t)|\psi(t=0)\rangle,\quad i\hbar\partial_t\hat{U}(t)=\hat{H}\hat{U}(t).
\end{equation}
由于哈密顿量与时间无关,所以我们可以写成时间演化算符的表示
\begin{equation}
	\hat{U}(t)=e^{-\frac i\hbar\hat{H}t}
\end{equation}
同时我们写出\\

\begin{equation}
	\vspace{\baselineskip}
	%公式编号: 10
	\tikzmarknode{10eq1}{\highlight{red}{$\psi(q_f,t_f)$}}=\langle q_f|\psi(t_f)\rangle=\langle q_f|\hat{U}(t_f-t_i)|\psi(t_i)\rangle=\int \dq_iU(q_f,q_i;t_f-t_i)\tikzmarknode{10eq2}{\highlight{blue}{$\psi(q_i,t_i)$}}
\end{equation}
\begin{tikzpicture}[overlay,remember picture,>=stealth,nodes={align=left,inner ysep=1pt},<-]
	% 对于 "1" 定位点
	\path (10eq1.north) ++ (0,2em) node[anchor=south west,color=red!67] (eq10/1){\textbf{$ t_f $时刻时粒子位于$q_f$的概率}};
	\draw [color=red!87](10eq1.north) |- ([xshift=-0.3ex,color=red]eq10/1.south east);
	% 对于 "2" 定位点
	\path (10eq2.south) ++ (0,-1.5em) node[anchor=north east,color=blue!67] (eq10/2){\textbf{$ t_i $时刻时粒子位于$q_i$的概率}};
	\draw [color=blue!57](10eq2.south) |- ([xshift=-0.3ex,color=blue]eq10/2.south west);
\end{tikzpicture}

其中$U(q_f,q_i;t_f-t_i) = \langle q_f|\hat{U}(t_f-t_i)|q_i\rangle $被称为传播子,其表示了这个粒子在$t_f-t_i$时间内从位置$q_i$传播到位置$q_f$的概率振幅.并且若已知哈密顿量$\hat{H}$本征态$\{|n\ra,\epsilon_n\}$,那么我们以此可以把传播子写作
\begin{equation}
	\begin{aligned}U(q_f,q_i;t_f-t_i)&=\langle q_f|e^{-\frac{i}{\hbar}\hat{H}(t_f-t_i)}|q_i\rangle=\sum_n\langle q_f|n\rangle e^{-\frac{i}{\hbar}\epsilon_n(t_f-t_i)}\langle n|q_i\rangle\\[2ex]&=\sum_ne^{-\frac{i}{\hbar}\epsilon_n(t_f-t_i)}\varphi_n(q_f)\varphi_n^*(q_i)\end{aligned}
\end{equation}
其中$\varphi_n(q)=\la q|n\ra$为坐标表象下的波函数.我们发现传播子给出了关于这个哈密顿量的波函数和能级的全部信息,这也意味着我们可以把求解波函数的问题变为求解这个哈密顿量所对应的传播子的问题.
\subsection*{路径积分}
我们刚刚发现了通过求解传播子可以间接求解波函数,而现在的问题是:如何求出传播子? 这里我们用到费曼的路径积分方法,我们先来首先说明\textbf{传播子可以写为路径积分的形式}.

我们首先考虑一个充分短的时间$\epsilon$

\begin{equation}
	\vspace{\baselineskip}
	%公式编号: 12
	U(q_f,q_i;\epsilon)=\langle q_f|e^{-i\hat{H}\epsilon}|q_i\rangle \tikzmarknode{12eq2}{\highlight{blue}{$\simeq$}}\langle q_f|e^{-i\epsilon\frac{\hat{p}^2}{2m}}e^{-i\epsilon V(\hat{q})}|q_i\rangle
\end{equation}
\begin{tikzpicture}[overlay,remember picture,>=stealth,nodes={align=left,inner ysep=1pt},<-]
	% 对于 "2" 定位点
	\path (12eq2.south) ++ (0,-1.5em) node[anchor=north east,color=blue!67] (eq12/2){\textbf{Baker–Hausdorff规则$e^{\epsilon\hat{A}+\epsilon\hat{B}}=e^{\epsilon\hat{A}}e^{\epsilon\hat{B}}e^{\mathcal{O}(\epsilon^{2})}$}};
	\draw [color=blue!57](12eq2.south) |- ([xshift=-0.3ex,color=blue]eq12/2.south west);
\end{tikzpicture}

我们在其中插入一个单位算符的谱分解\footnote{此时已经开始采取自然单位制了(对单位制可以参考附录).},在下面的式子中,我们使用了归一化假设$\la q|p\ra=L^{-1/2}e^{ipq}\;q\in[0,L]$,$ q $为连续的位置变量,而$p=n\frac{2\pi}L\;n\in \mathbb Z$为离散的动量变量(关于边界$ L $,并且有归一化条件$e^{ipL}=1$),在极限$L\to\infty$,存在$\frac1L\sum_p\to\int_{-\infty}^\infty\frac{\d p}{2\pi}$.\\
于是,式子变为
\begin{equation}
	\vspace{\baselineskip}
	%公式编号: 13
	\begin{aligned}
		U(q_f,q_i;\epsilon)& \begin{aligned}&=\sum_p\langle q_f|e^{-i\epsilon\frac{\hat{p}^2}{2m}}|p\rangle\langle p|e^{-i\epsilon V(\hat{q})}|q_i\rangle\end{aligned} \\
		&=\int\frac{\d p}{2\pi}\exp\biggl\{-i\epsilon\biggl[\frac{p^2}{2m}+V(q_i)\biggr]+ip(q_f-q_i)\biggr\} \\
		&\tikzmarknode{13eq2}{\highlight{blue}{$=$}}\bigg(\frac{m}{2\pi i\epsilon}\bigg)^{1/2}\exp\bigg\{i\epsilon\bigg[\frac{m}{2}\frac{(q_{f}-q_{i})^{2}}{\epsilon^{2}}-V(q_{i})\bigg]\bigg\}.
	\end{aligned}
\end{equation}
\begin{tikzpicture}[overlay,remember picture,>=stealth,nodes={align=left,inner ysep=1pt},<-]
	% 对于 "2" 定位点
	\path (13eq2.south) ++ (0,-1.5em) node[anchor=north west,color=blue!67] (eq13/2){\textbf{此处计算需要利用留数定理,附录中给出了mma计算代码}};
	\draw [color=blue!57](13eq2.south) |- ([xshift=-0.3ex,color=blue]eq13/2.south east);
\end{tikzpicture}

为了使对$ p $的积分收敛,我们假设$\epsilon$包含一个小的负虚部,我们发现,指数上的部分恰好是$ i $乘以无穷小作用量$S(q_f,q_i;\epsilon)$,不难发现这个作用量对应着无穷小时间$\epsilon$内$ q_i $和$ q_f $之间以恒定速度的直线轨迹,于是,我们把式子写为如下形式:
\begin{equation}\label{4eq14}
	U(q_f,q_i;\epsilon)=\left(\frac{m}{2\pi i\epsilon}\right)^{1/2}e^{iS(q_f,q_i;\epsilon)+\mathcal{O}(\epsilon^2)}
\end{equation}
仅有无穷小时间间隔的传播子显然是远远不够的,现在我们想计算出任意时间间隔$t_f-t_i$的传播子$U(q_f,q_i;t_f-t_i)$.我们考虑将时间段$t_f-t_i$分割为$ N $个长为$\epsilon=(t_f-t_i)/N$的等大小的部分,只要最终我们取极限$N\to\infty\;(\epsilon\to0)$,并对全部时间步积分,就得到了任意时间间隔$t_f-t_i$的传播子.
\begin{equation}
	\begin{aligned}
		\begin{aligned}U(q_f,q_i;t_f-t_i)\end{aligned}& \begin{aligned}=\langle q_{f}|e^{-i\hat{H}\epsilon}\cdots e^{-i\hat{H}\epsilon}|q_{i}\rangle\end{aligned} \\
		&=\int\prod_{k=1}^{N-1}\d q_k\langle q_f|e^{-i\hat{H}\epsilon}|q_{N-1}\rangle\langle q_{N-1}|e^{-i\hat{H}\epsilon}|q_{N-2}\rangle\cdots\langle q_1|e^{-i\hat{H}\epsilon}|q_i\rangle \\
		&=\int\prod_{k=1}^{N-1}\d q_k\prod_{k=1}^NU(q_k,q_{k-1};\epsilon)
	\end{aligned}
\end{equation}
其中$q_0=q_i,q_N=q_f$,在\ref{4eq14}中,我们对每个时间步传播子都忽略了$\epsilon^2$的高阶项,对于整体,其导致了阶为$\epsilon$的总误差. \\
现在我们继续考虑,令$N\to\infty$我们有
\begin{equation}
	U(q_f,q_i;t)=\lim_{N\to\infty}\left(\frac{mN}{2\pi it}\right)^{N/2}\int\prod_{k=1}^{N-1}\d q_k e^{iS[q]}
\end{equation}
其中作用量为
\begin{equation}
	S[q]=\sum_{k=1}^NS(q_k,q_{k-1};\epsilon)=\epsilon\sum_{k=1}^N\left[\frac m2\frac{(q_k-q_{k-1})^2}{\epsilon^2}-V(q_{k-1})\right]
\end{equation}
在极限$N\to\infty$,我们可以把求和写做积分:
\begin{equation}
	\begin{aligned}\epsilon\sum_{k=1}^N\frac{m}{2}\frac{(q_k-q_{k-1})^2}{\epsilon^2}&\to\int_{t_i}^{t_f} \d t\frac{m}{2}\dot{q}^2\\\epsilon\sum_{k=1}^NV(q_{k-1})&\to\int_{t_i}^{t_f} \d t V(q)\end{aligned}
\end{equation}
我们使用$q(t)$表示这个粒子的``轨迹"(trajectory),对于始末位置$q(t_i)=q_i,q(t_f)=q_f$,但这并不意味着在大$ N $极限下的连续性/可微性.我们定义积分测度为如下形式\footnote{我们可以认为这个形式不过是把一些成套的东西包装成一个微分算符,依赖这种写法,可以简化我们对于路径积分的表述.}:
\begin{equation}
	\mathcal{D}[q]=\lim_{N\to\infty}\left(\frac{mN}{2\pi i\hbar t}\right)^{N/2}\prod_{k=1}^{N-1}\d q_k
\end{equation}
于是传播子可以简化为
\begin{equation}
	U(q_f,q_i;t_f-t_i)=\int_{q(t_i)=q_i}^{q(t_f)=q_f}\mathcal{D}[q]e^{\frac{i}{\hbar}S[q]}
\end{equation}
同时作用量被定义为

\begin{equation}
	\vspace{\baselineskip}
	%公式编号: 21
	\tikzmarknode{21eq1}{\highlight{red}{$S[q]$}}=\int_{t_i}^{t_f}\d t\tikzmarknode{21eq2}{\highlight{blue}{$L(q,\dot{q})$}}
\end{equation}
\begin{tikzpicture}[overlay,remember picture,>=stealth,nodes={align=left,inner ysep=1pt},<-]
	% 对于 "1" 定位点
	\path (21eq1.north) ++ (0,2em) node[anchor=south east,color=red!67] (eq21/1){\textbf{作用量}};
	\draw [color=red!87](21eq1.north) |- ([xshift=-0.3ex,color=red]eq21/1.south west);
	% 对于 "2" 定位点
	\path (21eq2.south) ++ (0,-1.5em) node[anchor=north west,color=blue!67] (eq21/2){\textbf{拉格朗日量}};
	\draw [color=blue!57](21eq2.south) |- ([xshift=-0.3ex,color=blue]eq21/2.south east);
\end{tikzpicture}

现在我们发现,这个作用量的形式与轨迹$q(t)$的经典作用量的形式一致.

我们不难注意到:积分测度$\mathcal D$中包含的极限是\textbf{发散的},在处理发散问题之前,我们首先尝试讨论其物理含义:对于传播子,我们从公式角度出发观察传播子的形式,我们不难发现这个积分过程只规定了初值条件(初态位置和时间以及末态位置和时间),我们需要对\textbf{所有可能的}路径进行积分(或者说是求和,这两者并没有太大区别),并且对于求和过程,我们最后得出的答案是依赖于作用量的对每条路径的\textbf{加权和}.而按照物理情景的解释,我们有``当一个物理过程\footnote{我们并没有区别宏观和微观,这是因为其对宏观仍然适用,但由于对应原理,我们不必对宏观现象如此分析.}可以以多种路径进行时,它的概率幅由每种路径的幅值之和给出\footnote{英文原文:When a
	process can take place in more than one way, its probability amplitude is given by the
	sum of the amplitudes for each way.}".

但是,我们发现这并没有解决发散问题,于是我们要求轨迹是足够``好"的.即要求动力学项$[q(t+\epsilon)-q(t)]^2/\epsilon$在极限$\epsilon\to0$时收敛,并认为在不满足这个条件的轨迹会剧烈震荡,其平均值为$ 0 $,即不那么``好"的轨迹.事实上,这种方法似乎完全看不出来严格的数学依据,但现实如此(这里可以引用那些经典的物理小笑话了,至少我们目前不用去思考如何把这些东西严谨化.)\footnote{关于这一大段的文字讨论是必须的,其有助于构建量子场论的物理图景,事实上,在这一章我一直在尝试把更多重点放在公式内部并突出显示它,倘若把重点置于一大堆文字中,读者不加以仔细的阅读的话,便很容易错过去.而且,对于物理这一门学科,过于长段的文字很难真的揭示什么内涵,它们往往起到解释公式的作用,或者说,公式才是文章的主体.同样的,这一大段内容我放在了脚注中,同样为了让人们去注意到它.}.

\subsection*{经典极限}
或许你们发现在上一部分的结尾中,我们并没有像往常一样去省略约化普朗克常数$\hbar$,这有关于经典极限的讨论.

我们所关注的重点轨迹为``贴近"经典路径的轨迹,其作用量是静态的.
\begin{equation}
	\left.\frac{\delta S[q]}{\delta q(t)}\right|_{q=q_c}=0
\end{equation}
对于非静态的轨迹,其意味着作用量的大幅振荡,其平均值为0,或者说,更准确的,传播子$U(q_f,q_i;t_f-t_i)$由轨迹$q(t)$所主导
Non-stationary trajectories imply large oscillations of the action and therefore average to zero (the Feynman paths interfere destructively). More precisely, the propagator U(qf , qi; tf −ti) is dominated by the trajectories q(t) whose action S[q] differs from the classical action Sc = S[qc] by a term of order ℏ: |S−Sc| ≲ ℏ. When |Sc| ≫ ℏ, these trajectories are very close to the classical trajectory. In the opposite limit, the condition |S − Sc| ≲ ℏ
= is fulfilled by trajectories very different from the classical one. Formally, the classical limit corresponds to the limit ℏ → 0.

\begin{remark}
		\textit{To obtain the propagator in the limit ℏ → 0, we write q(t) = qc(t) + r(t) (assuming
		there is only one classical trajectory) and expand the action to second order in r(t),}
		\begin{equation}
			\begin{aligned}&U(q_f,q_i;t_f-t_i)\\&\simeq e^{\frac{i}{\hbar}S[q_c]}\int_{r(t_i)=0}^{r(t_f)=0}\mathcal{D}[r]\exp\biggl\{\frac{i}{2\hbar}\int_{t_i}^{t_f}dt dt' \frac{\delta^2S[q]}{\delta q(t)\delta q(t')}\biggr|_{q=q_c}r(t)r(t')\biggr\}\end{aligned}
		\end{equation}
		\textit{The integral is Gaussian and can be done exactly}
		\begin{equation}
			U(q_f,q_i;t_f-t_i)\simeq e^{\frac{i}{\hbar}S[q_c]}\det\left(\frac{1}{2\pi i\hbar}\frac{\delta^2S[q]}{\delta q(t)\delta q(t')}\bigg|_{q=q_c}\right)^{-1/2}
		\end{equation}
		\textit{The procedure we have followed to obtain (1.24) is known as the stationary-phase
		approximation. It gives the leading term of an expansion in powers of ℏ that can be
		systematically carried out to higher orders (see Sec. 1.7).}
\end{remark}

\subsection*{时序算符以及哈密顿量}
1
\subsection{欧式路径积分}
1



















































































































	
\ifx\allfiles\undefined
\end{document}
	\else
	\fi
