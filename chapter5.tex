\ifx\allfiles\undefined

% 如果有这一部分另外的package,在这里加上
% 没有的话不需要

\begin{document}
	\else
	\fi
\chapter{对称关系}
\begin{introduction}
	\item 诺特定理
	\item 简并
	\item 离散对称性
	\item 宇称
\end{introduction}
\section{简并}
\subsection{简并}
现在转向简并概念, 尽管简并可以在经典力学层次讨论一一例如, 在讨论开普勒问题中的闭合(非进动)轨道时(Goldstein 2002)——这个概念在量子力学中起着更重要的作用. 假设对某种对称性算符

$$
\left\lbrack {H,8}\right\rbrack = 0 tag{4.1.13}
$$

而 $|n\rangle$ 为能量本征右矢,本征值为 ${E}_{n}$ . 则 $g \mid n\rangle$ 也是一个具有相同能量的能量本征右矢, 因为

$$
H\left( {\delta |n\rangle }\right) = {\delta H}|n\rangle = {E}_{n}\left( {\delta |n\rangle }\right) . tag{4.1.14}
$$

假定 $\left| {n\rangle \text{和}g}\right| n\rangle$ 代表不同的态. 那么,它们就是具有相同能量的两个态一一这就是说, 它们是简并态. 通常, $8$ 由一个连续参量,比如 $\lambda$ ,表征,在这种情况下,所有的形式为 $g\left( \lambda \right) |n\rangle$ 的态都有相同的能量.

下面特别来考虑转动. 假定哈密顿量是转动不变的, 因此

$$
\left\lbrack {\mathcal{D}\left( R\right), H}\right\rbrack = 0, tag{4.1.15}
$$

它一定意味着

$$
\left\lbrack {\mathbf{J}, H}\right\rbrack = 0,\;\left\lbrack {{\mathbf{J}}^{2}, H}\right\rbrack = 0. tag{4.1.16}
$$

可以构成 $H,{\mathbf{J}}^{2}$ 和 ${J}_{z}$ 的共同本征态,用 $|n;j, m\rangle$ 表示. 刚刚给出的论据意味着所有下列形式的态

$$
\mathcal{D}\left( R\right) |n;j, m\rangle tag{4.1.17}
$$

具有相同的能量. 在第 3 章曾看到转动下不同的 $m$ 值要混起来. 一般而言, $\mathcal{D}\left( R\right) \mid n;j$ , $m\rangle$ 是 ${2j} + 1$ 个独立的态的线性组合. 显然,

$$
\mathcal{D}\left( R\right) \left| {n;j, m\rangle = \mathop{\sum }\limits_{{m}^{\prime }}}\right| n;j,{m}^{\prime }\rangle {\mathcal{D}}_{{m}^{\prime }m}^{\left( j\right) }\left( R\right) , tag{4.1.18}
$$

并且,通过改变表征转动算符 $\mathcal{D}\left( R\right)$ 的连续参量,可以得到 $\left| {n;j,{m}^{\prime }}\right\rangle$ 的不同的线性组合. 如果在任意的 $\mathcal{D}\left( R\right)$ 情况下,所有形为 $\mathcal{D}\left( R\right) \mid n;j, m\rangle$ 的态都有相同的能量,那时十分自然地,有着不同 $m$ 的 $|n;j, m\rangle$ 中,每一个都一定有相同的能量. 所以,在这里简并是 $\left( {{2j} + 1}\right)$ 重的,正好等于可能的 $m$ 值的个数. 从这样的事实来看这一点也是显然的. 即通过与 $H$ 对易的 ${J}_{ \pm }$ 相继作用于 $|n;{jm}\rangle$ 得到的所有的态,具有相同的能量.

作为一个应用,考虑一个原子中的电子,它所受到的势可以写成 $V\left( r\right) + {V}_{LS}\left( r\right) \mathbf{L} \cdot \mathbf{S}$ . 因为 $r$ 和 $\mathbf{L} \cdot \mathbf{S}$ 都是转动不变的,预期每一个原子能级都是 $\left( {{2j} + 1}\right)$ 重简并. 另一方面, 假定存在有一个,比如沿 $z$ 方向的,外电场或磁场. 转动对称性现在明显被破坏了,其结果不再能预期有 $\left( {{2j} + 1}\right)$ 重简并,因而由不同的 $m$ 值表征的态不再有相同的能量. 在第 5 章将考察这种劈裂如何发生.

\subsection{库仑势中的 $\mathrm{{SO}}\left( 4\right)$ 对称性}

量子力学中连续对称性的一个很好例子是由氢原子问题以及库仑势的解提供的. 在 3.7 节完成了这个问题的求解, 在那里能量本征值 (3.7.53) 式表明了 (3.7.56) 式所总结的显著的简并. 假如这一简并只是一种偶然,则会更加引人注目,但事实上,它是 $1/r$ 势束缚态问题特有的一种附加对称性的结果.

在这样的位势中轨道的经典问题, 即开普勒问题, 当然在量子力学之前很早就已经充分研究过了. 它的解导致椭圆轨道都是闭合的, 这一事实意味着应当存在某个保持椭圆主轴取向不变的 (矢量) 运动常数. 即使对于 $1/r$ 只有一个小的偏离也会导致这个轴的进动,所以我们预期,要找的这个运动常数事实上是 $1/r$ 势所特有的.

经典上. 这个新的运动常数是

$$
\mathbf{M} = \frac{\mathbf{p} \times \mathbf{L}}{m} - \frac{Z{e}^{2}}{r}\mathbf{r} tag{4.1.19}
$$

其中引用了 3.7 节用过的符号. 这个量一般称为楞次矢量或者有时候称为龙格-楞次 (Runge-Lenz) 矢量. 与其在这里反复讨论经典处理, 不如转向根据作为这个运动常数起因的对称性的量子力学处理.

这个新的对称性,被称为 $\mathrm{{SO}}\left( 4\right)$ ,完全类似于 3.3 节研究过的 $\mathrm{{SO}}\left( 3\right)$ 对称性. 这就是说, $\mathrm{{SO}}\left( 4\right)$ 是四维空间的转动算符群. 等价地,它是行列式为 1 的 $4 \times 4$ 正交矩阵群. 逐步建立起导致楞次矢量作为一个运动常数的这个对称性的性质, 那时我们将看到这些性质正是由 $\mathrm{{SO}}\left( 4\right)$ 预期的.

所用的方法密切遵循席夫 (1968), pp. 235 ~ 239 给出的方法*. 首先需要修改 (4.1.19) 式,以构造一个厄米算符. 对于两个厄米的矢量算符 $\mathbf{A}$ 和 $\mathbf{B}$ ,易证 ${\left( \mathbf{A} \times \mathbf{B}\right) }^{ + } =$ $- \mathbf{B} \times \mathbf{A}$ . 因此,一个厄米版本的楞次矢量是

$$
\mathbf{M} = \frac{1}{2m}\left( {\mathbf{p} \times \mathbf{L} - \mathbf{L} \times \mathbf{p}}\right) - \frac{Z{e}^{2}}{r}\mathbf{r}. tag{4. 1.20}
$$

可以证明, $\mathbf{M}$ 与哈密顿量

$$
H = \frac{{\mathbf{p}}^{2}}{2m} - \frac{Z{e}^{2}}{r} tag{4. 1.21}
$$

---

* 这种求解氢原子的方法最早是由 Pauli 完成的,发表在 Zeitschrift Phys. , 33 (1925) 879. 其英译文 “Onthehy-drogen spectrum from the stand point of the new quantum mechanics". 发表在 Sources of Quantum Mechanics, B. L. Van der Waerden. Dover(1967) 上. 作者对于告知该参考文献的 Djordje Minic 深表谢意. 至于龙格-楞次矢量的经典处理. 建议读者参考 Goldstein, Poole 和 Safko(2002) 的 3.9 节 (本脚注接勘误表要求译出. - 译者注).

---

对易. 即

$$
\left\lbrack {\mathbf{M}, H}\right\rbrack = 0, tag{4. 1.22}
$$

因此, $\mathbf{M}$ 的确是一个 (量子力学) 运动常数. 可以证明其他一些有用的关系,即

$$
\mathbf{L} \cdot \mathbf{M} = 0 = \mathbf{M} \cdot \mathbf{L} tag{4. 1.23}
$$

$$
\text{和}{\mathbf{M}}^{2} = \frac{2}{m}H\left( {{\mathbf{L}}^{2} + {\hslash }^{2}}\right) + {Z}^{2}{e}^{4}\text{.} tag{4. 1.24}
$$

为了确认作为这个运动常数起源的对称性, 评述一下这个对称性的生成代数是很有意义的. 已知的这个代数的一部分

$$
\left\lbrack {{L}_{i},{L}_{j}}\right\rbrack = i\hslash {\varepsilon }_{ijk}{L}_{k}, tag{4. 1.25}
$$

早些时候曾用这种符号把它写成 (3.6.2) 式,其中重复指标 (在这种情况下是 $k$ ) 表示对于分量的自动求和. 还可以证明

$$
\left\lbrack {{M}_{i},{L}_{j}}\right\rbrack = i\hslash {\varepsilon }_{ijk}{M}_{k}, tag{4. 1.26}
$$

它事实上确立 $\mathbf{M}$ 为一个 (3.11.8) 式意义上的矢量算符. 最后,能够导出

$$
\left\lbrack {{M}_{i},{M}_{j}}\right\rbrack = - i\hslash {\varepsilon }_{ijk}\frac{2}{m}H{L}_{k}, tag{4. 1.27}
$$

无疑, (4.1.25) 式、(4.1.26) 式和 (4.1.27) 式构不成一个封闭代数, 原因在于在 (4.1.27) 式中存在 $H$ ,这使得很难把这些算符看作是一个连续对称性的生成元. 然而, 可以考虑特定的束缚态问题. 在这种情况下,矢量空间被删减为只有 $H$ 的本征态的那部分,其能量 $E < 0$ . 在那种情况下,用 (4.1.27) 式中的 $E$ 取代 $H$ ,于是这个代数就封闭了. 富有启发的做法是,把 $\mathbf{M}$ 代之以重新标度的矢量算符

$$
\mathbf{N} \equiv {\left( -\frac{m}{2E}\right) }^{1/2}\mathbf{M}. tag{4. 1.28}
$$

在这种情况下, 有封闭的代数

$$
\left\lbrack {{L}_{i},{L}_{j}}\right\rbrack = i\hslash {\varepsilon }_{ijk}{L}_{k},
$$

(4. ${1.29a}$ )

$$
\left\lbrack {{N}_{i},{L}_{j}}\right\rbrack = i\hslash {\varepsilon }_{ijk}{N}_{k}, tag{4.1.29b}
$$

$$
\left\lbrack {{N}_{i},{N}_{j}}\right\rbrack = i\hslash {\varepsilon }_{ijk}{L}_{k}.
$$

(4. ${1.29c}$ )

那么,由 (4.1.29) 式中的算符 $\mathbf{L}$ 和 $\mathbf{N}$ 生成的对称性操作是什么呢? 尽管还远非显然, 但答案是 “四维空间的转动”. 第一个线索是生成元的个数, 即六个, 它们中的每一个都应当对应于绕某个轴的转动. 把转动设想为一种将两个正交的轴混合起来的操作. 那么, $n$ 维空间中转动生成元的个数应该是 $n$ 个东西每次取两个的组合数,即 $n\left( {n - 1}\right) /2$ . 结果,二维转动要求一个生成元,即 ${L}_{z}$ . 三维转动要求三个生成元,即 $\mathbf{L}$ ,而四维转动要求六个生成元.

很难看到, (4.1.29) 式是这类转动的合适的代数, 但是按如下步骤操作. 在三维空间中, 轨道角动量算符 (3.6.1) 式生成转动. 在 (3.6.6) 式中清楚地看到了这一点, 在那里 $|\alpha \rangle$ 态上绕 $z$ 轴的一个无穷小转动在 $|x, y, z\rangle$ 基的一个转动后的版本中表示了出来. 这恰是动量算符作为空间平移生成元的一种后果. 事实上,像 ${L}_{z} = x{p}_{y} - y{p}_{x}$ 的这样一种组合的确把 $x$ 轴与 $y$ 轴混合起来,正如人们从绕 $z$ 轴转动的生成元本应期待的那样.

为了把这个做法推广到四维空间,首先把 $\left( {x, y, z}\right)$ 和 $\left( {{p}_{x},{p}_{y},{p}_{z}}\right)$ 视为 $\left( {{x}_{1},{x}_{2},{x}_{3}}\right)$ 和 $\left( {{p}_{1},{p}_{2},{p}_{3}}\right)$ . 它导致把生成元改写成 ${L}_{3} = {\widetilde{L}}_{12} = {x}_{1}{p}_{2} - {x}_{2}{p}_{1},{L}_{1} = {\widetilde{L}}_{23}$ 和 ${L}_{2} = {\widetilde{L}}_{31}$ . 然

后,如果虚构一个空间维度 ${x}_{1}$ 以及它的共轭动量 ${p}_{1}$ (具有通常的对易关系),则可以定义

$$
{\widetilde{L}}_{11} = {x}_{1}{p}_{4} - {x}_{4}{p}_{1} \equiv {N}_{1}, tag{4. 1.30a}
$$

$$
{\widetilde{L}}_{24} = {x}_{2}{p}_{4} - {x}_{4}{p}_{2} \equiv {N}_{2}, tag{4. 1. 30b}
$$

$$
{\widetilde{L}}_{34} = {x}_{3}{p}_{4} - {x}_{4}{p}_{3} \equiv {N}_{3},
$$

(4. ${1.30}\mathrm{c}$ )

易证,这些算符 ${N}_{i}$ 遵从代数 (4.1.29) 式. 例如,

$$
\left\lbrack {{N}_{1},{L}_{2}}\right\rbrack = \left\lbrack {{x}_{1}{p}_{1} - {x}_{1}{p}_{1},{x}_{3}{p}_{1} - {x}_{1}{p}_{3}}\right\rbrack
$$

$$
= {p}_{4}\left\lbrack {{x}_{1},{p}_{1}}\right\rbrack {x}_{3} + {x}_{4}\left\lbrack {{p}_{1},{x}_{1}}\right\rbrack {p}_{3} tag{4. 1.31}
$$

$$
= i\hslash \left( {{x}_{3}{p}_{4} - {x}_{4}{p}_{3}}\right) = i\hslash {N}_{3}.
$$

换句话说, 这是四维空间的代数. 稍后将回到这个符号, 但是现在, 将继续讨论由 (4.1.14) 式所隐含的库伦势中的简并性.

定义算符

$$
\mathbf{I} \equiv \left( {\mathbf{L} + \mathbf{N}}\right) /2, tag{4. 1.32}
$$

$$
\mathbf{K} \equiv \left( {\mathbf{L} - \mathbf{N}}\right) /2, tag{4. 1.33}
$$

可以容易地证明下列代数:

$$
\left\lbrack {{I}_{i},{I}_{j}}\right\rbrack = i\hslash {\varepsilon }_{ijk}{I}_{k},
$$

(4. ${1.34a}$ )

$$
\left\lbrack {{K}_{i},{K}_{j}}\right\rbrack = i\hslash {\varepsilon }_{ijk}{K}_{k}, tag{4. 1.34b}
$$

$$
\left\lbrack {{I}_{i},{K}_{j}}\right\rbrack = 0.
$$

(4. ${1.34}\mathrm{c}$ )

因此,这些算符遵从独立的角动量代数. 而且显然还有 $\left\lbrack {\mathbf{I}, H}\right\rbrack = \left\lbrack {\mathbf{K}, H}\right\rbrack = 0$ . 所以,这些 “角动量” 都是守恒量,于是,算符 ${\mathbf{I}}^{2}$ 和 ${\mathbf{K}}^{2}$ 的本征值分别用 $i\left( {i + 1}\right) {\hslash }^{2}$ 和 $k\left( {k + 1}\right) {\hslash }^{2}$ 代表,其中 $i, k = 0,\frac{1}{2},1,\frac{3}{2},\cdots$ .

因为根据 (4.1.23) 和 (4.1.28), ${\mathbf{I}}^{2} - {\mathbf{K}}^{2} = \mathbf{L} \cdot \mathbf{N} = 0$ ,一定有 $i = k$ . 另一方面, 算符

$$
{\mathbf{I}}^{2} + {\mathbf{K}}^{2} = \frac{1}{2}\left( {{\mathbf{L}}^{2} + {\mathbf{N}}^{2}}\right) = \frac{1}{2}\left( {{\mathbf{L}}^{2} - \frac{m}{2E}{\mathbf{M}}^{2}}\right) tag{4. 1.35}
$$

和 (4.1.24) 一起, 导致数值关系

$$
{2k}\left( {k + 1}\right) {\hslash }^{2} = \frac{1}{2}\left( {-{\hslash }^{2} - \frac{m}{2E}{Z}^{2}{e}^{4}}\right) . tag{4. 1.36}
$$

解出 $E$ ,得知

$$
E = - \frac{m{Z}^{2}{e}^{4}}{2{\hslash }^{2}}\frac{1}{{\left( 2k + 1\right) }^{2}} tag{4. 1.37}
$$

这个结果与 (3.7.53) 相同,只是主量子数 $n$ 用 ${2k} + 1$ 代替. 我们现在看到库仑问题的简并来源于由算符 I 和 $\mathbf{K}$ 表示的两个 “转动” 对称性. 事实上,简并度是 $\left( {{2i} + 1}\right) \left( {{2k} + 1}\right)$ $= {\left( 2k + 1\right) }^{2} = {n}^{2}$ . 除了很显然这个简并不是偶然的之外,它正是我们在 (3.7.56) 式中所得到的结果.

值得注意的是, 刚刚解出了氢原子的本征值, 没有像过去那样诉诸求解薛定谔方程. 相反, 利用了内在的对称性, 得到了同样的答案. 这个解显然是泡利最早得到的.

用 3.3 节开始发展的连续群理论的语言, 可得知代数 (4.1.29) 式对应于 $\mathrm{{SO}}\left( 4\right)$ 群. 此外, 把这个代数改写为 (4.1.34) 式表明, 它还可以视为两个独立的 $\mathrm{{SU}}\left( 2\right)$ 群,即 $\mathrm{{SU}}\left( 2\right) \times \mathrm{{SU}}\left( 2\right)$ . 尽管把群论的介绍包括进来不是本书的目的,但仍将稍微地推进一步, 以表明人们如何从形式上实现 $n$ 维空间的转动,即 $\mathrm{{SO}}\left( n\right)$ 群.

推广 3.3 节中的讨论,考虑实施 $n$ 维转动的 $n \times n$ 正交矩阵 $R$ . 它们可以参量化为

$$
R = \exp \left( {i\mathop{\sum }\limits_{{q = 1}}^{{n\left( {n - 1}\right) /2}}{\phi }^{q}{\tau }^{q}}\right) . tag{4. 1.38}
$$

其中 ${\tau }^{q}$ 是纯虚的反对称 $n \times n$ 矩阵,即 ${\left( {\tau }^{q}\right) }^{T} = - {\tau }^{q}$ ,而 ${\phi }^{q}$ 是广义转角. 反对称条件保证 $R$ 是正交的. 整体的因子 $i$ 意味着虚的矩阵 ${\tau }^{q}$ 也是厄米的.

${\tau }^{q}$ 显然与转动算符的生成元有关. 事实上,它们的对易关系应当由这些生成元的对易关系模仿得到. 像 3.1 节那样接着做下去,把先绕 $q$ 轴做无穷小转动,然后绕 $p$ 轴转动, 同相反次序的转动的作用结果作比较. 那么

$$
\left( {1 + i{\phi }^{p}{\tau }^{p}}\right) \left( {1 + i{\phi }^{q}{\tau }^{q}}\right) - \left( {1 + i{\phi }^{q}{\tau }^{q}}\right) \left( {1 + i{\phi }^{p}{\tau }^{p}}\right)
$$

$$
= - {\phi }^{p}{\phi }^{q}\left\lbrack {{\tau }^{p},{\tau }^{q}}\right\rbrack tag{4. 1.39}
$$

$$
= 1 - \left( {1 + i{\phi }^{p}{\phi }^{q}\mathop{\sum }\limits_{r}{f}_{r}^{pq}{\tau }^{r}}\right) ,
$$

其中 (4.1.39) 式的最后一行确认, 这个结果一定是一个绕这两个轴的二级转动, 且带有生成元的某种线性组合. ${f}_{r}^{pq}$ 称为这个转动群的结构常数. 于是有对易关系

$$
\left\lbrack {{\tau }^{p},{\tau }^{q}}\right\rbrack = i\mathop{\sum }\limits_{r}{f}_{r}^{pq}{\tau }^{r}. tag{4. 1.40}
$$

进一步需要确定这个结构常数 ${f}_{r}^{pq}$ ,这些细节留给专门的群论教科书. 然而不难证明,在三维如所预期的 ${f}_{r}^{pq} = {\varepsilon }_{pqr}$ .

\section{分立对称性,宇称或空间反射}

至此, 连续对称性操作已经考虑了, 即可以通过相继应用无穷小对称性操作得到的那些. 并非所有的在量子力学有用的对称性操作都一定是这种形式的. 在本章将考虑三种与连续性操作相反的、可以认为是分立的对称性操作——宇称、晶格平移和时间反演.


图 4.1 右手 (RH) 和左手 (LH) 坐标系

第一个要考虑的操作是宇称, 或空间反射. 宇称操作应用于坐标系变换时, 把一个右手 (RH) 坐标系变成一个左手 (LH) 坐标系, 如图 4.1 所示. 然而本书中考虑的是对于态右矢的变换,而不是对于坐标系的变换. 给定 $|\alpha \rangle$ ,考虑一个空间反射态,假定它是通过一个所谓的宇称算符的幺正算符 $\pi$ 的作用得到的,如下所示:

$$
\left| {\alpha \rangle \rightarrow \pi }\right| \alpha \rangle tag{4.2.1}
$$

要求 $\mathbf{x}$ 对于空间反射后的态所取的期待值有相反的符号

$$
\left\langle {\alpha \left| {{\pi }^{ \dagger }\mathbf{x}\pi }\right| \alpha }\right\rangle = - \langle \alpha \left| \mathbf{x}\right| \alpha \rangle , tag{4.2.2}
$$

这是一个非常合理的要求. 如果

$$
{\pi }^{ \dagger }\mathbf{x}\pi = - \mathbf{x} tag{4.2.3}
$$

或

$$
\mathbf{x}\pi = - \pi \mathbf{x}, tag{4.2.4}
$$

则该要求可以实现,其中用到了 $\pi$ 是幺正的这一事实. 换句话说, $\mathbf{x}$ 和 $\pi$ 必须反对易.

位置算符的本征右矢在宇称之下如何变换? 可以断言

$$
\pi \left| {\mathbf{x}}^{\prime }\right\rangle = {e}^{i\delta }\left| {-{\mathbf{x}}^{\prime }}\right\rangle , tag{4.2.5}
$$

其中 ${e}^{i\delta }$ 是一个相位因子 ( $\delta$ 为实数). 为了证明这一说法成立,注意

$$
\mathbf{x}\pi \left| {\mathbf{x}}^{\prime }\right\rangle = - \pi \mathbf{x}\left| {\mathbf{x}}^{\prime }\right\rangle = \left( {-{\mathbf{x}}^{\prime }}\right) \pi \left| {\mathbf{x}}^{\prime }\right\rangle . tag{4.2.6}
$$

这个方程是说, $\pi \mid {\mathbf{x}}^{\prime }\rangle$ 是 $\mathbf{x}$ 的一个本征右矢,其本征值为 $- {\mathbf{x}}^{\prime }$ ,因此它一定与位置本征右矢 $\left| {-{\mathbf{x}}^{\prime }}\right\rangle$ 相同,至多差一个相因子.

按惯例,取 ${e}^{i\delta } = 1$ 作为约定. 把它带入到 (4.2.5) 式中,有 ${\pi }^{2}\left| {\mathbf{x}}^{\prime }\right\rangle = \left| {\mathbf{x}}^{\prime }\right\rangle$ . 因此, ${\pi }^{2} = 1$ ,这就是说,用 $\pi$ 作用两次,回到了原来的态. 从 (4.2.5) 式易见, $\pi$ 现在不只是幺正的, 而且也是厄米的:

$$
{\pi }^{-1} = {\pi }^{ \dagger } = \pi tag{4.2.7}
$$

它的本征值只能是 +1 或者 -1 .

动量算符怎么样呢? 动量 $\mathbf{p}$ 与 ${md}\mathbf{x}/{dt}$ 是一样的,所以自然地预期它在宇称作用之下, 像 $\mathbf{x}$ 一样,是奇的. 更满意的论证把动量算符认为是平移生成元,正如可以从图 4.2 可以看出, 平移之后紧接着宇称等价于宇称之后紧接着沿反方向的平移, 因此

$$
\pi \mathcal{J}\left( {d{\mathbf{x}}^{\prime }}\right) = \mathcal{J}\left( {-d{\mathbf{x}}^{\prime }}\right) \pi tag{4.2.8}
$$

$$
\pi \left( {1 - \frac{i\mathbf{p} \cdot d{\mathbf{x}}^{\prime }}{\hslash }}\right) {\pi }^{ + } = 1 + \frac{i\mathbf{p} \cdot d{\mathbf{x}}^{\prime }}{\hslash }, tag{4.2.9}
$$

于是有

$$
\{ \pi ,\mathbf{p}\} = 0\;\text{ 或 }\;{\pi }^{ \dagger }\mathbf{p}\pi = - \mathbf{p}. tag{4.2.10}
$$



图 4.2 平移紧接着宇称或反过来

现在讨论在宇称之下 $\mathbf{J}$ 的行为. 首先,对于轨道角动量显然有

$$
\left\lbrack {\pi ,\mathbf{L}}\right\rbrack = 0 tag{4.2.11}
$$

因为

$$
\mathbf{L} = \mathbf{x} \times \mathbf{p}. tag{4.2.12}
$$

(4. 2.33) 动生成元的事实. 对于 $3 \times 3$ 正交矩阵,有

$$
{R}^{\text{(宇称) }}{R}^{\text{(转动) }} = {R}^{\text{(转动) }}{R}^{\text{(宇称) }} tag{4.2.13}
$$

其中明确地写出来

$$
{R}^{\left( \text{ 宇称 }\right) } = \left( \begin{matrix} - 1 & & 0 \\ & - 1 & \\ 0 & & - 1 \end{matrix}\right) ; tag{4.2.14}
$$

这就是说宇称和转动算符对易. 在量子力学中, 自然地假设对于幺正算符有相应的关系, 因此

$$
\pi \mathcal{D}\left( R\right) = \mathcal{D}\left( R\right) \pi , tag{4.2.15}
$$

其中, $\mathcal{D}\left( R\right) = 1 - i\mathbf{J} \cdot \widehat{\mathbf{n}}\varepsilon /\hslash$ . 从 (4.2.15) 式得出结论

$$
\left\lbrack {\pi ,\mathbf{J}}\right\rbrack = 0\;\text{ 或 }\;{\pi }^{ \dagger }\mathbf{J}\pi = \mathbf{J}. tag{4.2.16}
$$

该式与 (4.2.11) 式一起意味着自旋算符 $\mathbf{S}$ (导致总角动量 $\mathbf{J} = \mathbf{L} + \mathbf{S}$ ) 也以和 $\mathbf{L}$ 一样的相同方式变换.

在转动下, $\mathbf{x}$ 和 $\mathbf{J}$ 以相同方式变换,所以它们都是矢量,或秩为 1 的球张量. 然而 $\mathbf{x}$ (或 p) 在宇称下为奇, [见 (4.2.3) 式和 (4.2.10) 式], 而 $\mathbf{J}$ 在宇称下是偶的 [见 (4.2.16) 式]. 在宇称下为奇的矢量称为极矢量, 而宇称下为偶的矢量称为轴矢量或赝矢量.

现在考虑像 $\mathbf{S} \cdot \mathbf{x}$ 这样的算符. 在转动下,它们像普通标量一样的变换,比如 $\mathbf{S} \cdot \mathbf{L}$ 或 $\mathbf{x} \cdot \mathbf{p}$ 那样. 而在空间反射下有

$$
{\pi }^{-1}\mathbf{S} \cdot \mathbf{x}\pi = - \mathbf{S} \cdot \mathbf{x}, tag{4.2.17}
$$

然而, 对于通常的标量有

$$
{\pi }^{-1}\mathbf{L} \cdot \mathbf{S}\pi = \mathbf{L} \cdot \mathbf{S} tag{4.2.18}
$$

等. 算符 $\mathbf{S} \cdot \mathbf{x}$ 是赝标量的一个例子.

\subsection{宇称下的波函数}

现在来看一看波函数的宇称性质. 首先,设 $\psi$ 是态右矢为 $|\alpha \rangle$ 的一个无自旋粒子的波函数:

$$
\psi \left( {\mathbf{x}}^{\prime }\right) = \left\langle {{\mathbf{x}}^{\prime } \mid \alpha }\right\rangle . tag{4.2.19}
$$

其空间反射态,由态右矢 $\pi \mid \alpha \rangle$ 表示,相应的波函数为

$$
\left\langle {{\mathbf{x}}^{\prime }\left| \pi \right| \alpha }\right\rangle = \left\langle {-{\mathbf{x}}^{\prime } \mid \alpha }\right\rangle = \psi \left( {-{\mathbf{x}}^{\prime }}\right) . tag{4.2.20}
$$

假定 $|\alpha \rangle$ 是一个宇称的本征右矢. 已知宇称的本征值一定是 $\pm 1$ ,所以

$$
\pi \left| {\alpha \rangle = \pm }\right| \alpha \rangle tag{4.2.21}
$$

看一看它的相应的波函数,

$$
\left\langle {{\mathbf{x}}^{\prime }\left| \pi \right| \alpha }\right\rangle = \pm \left\langle {{\mathbf{x}}^{\prime } \mid \alpha }\right\rangle . tag{4.2.22}
$$

但是还有

$$
\left\langle {{\mathbf{x}}^{\prime }\left| \pi \right| \alpha }\right\rangle = \left\langle {-{\mathbf{x}}^{\prime } \mid \alpha }\right\rangle , tag{4.2.23}
$$

所以在宇称下 $|\alpha \rangle$ 是偶的还是奇的依赖于相应的波函数是否满足

$$
\psi \left( {-{\mathbf{x}}^{\prime }}\right) = \pm \psi \left( {\mathbf{x}}^{\prime }\right) \left\{ \begin{array}{l} \text{ 偶宇称,} \\ \text{ 奇宇称. } \end{array}\right. tag{4.2.24}
$$

并非所有物理上感兴趣的波函数都有 (4.2.24) 式意义上的确定宇称. 例如, 考虑动量本征右矢. 动量算符与宇称算符反对易, 所以人们预期动量本征右矢不是一个宇称本征右矢. 的确, 很容易看到, 平面波, 作为一个动量本征右矢的波函数, 不满足 (4.2.24) 式.

预期轨道角动量的本征右矢是一个宇称本征右矢,因为 $\mathbf{L}$ 和 $\pi$ 对易 [见 (4.2.11) 式]. 为了看到 ${\mathbf{L}}^{2}$ 和 ${L}_{z}$ 在宇称之下的行为,考察在空间反射下它的波函数

$$
\left\langle {{\mathbf{x}}^{\prime } \mid \alpha ,{lm}}\right\rangle = {R}_{\alpha }\left( r\right) {Y}_{l}^{m}\left( {\theta ,\phi }\right) tag{4.2.25}
$$

的性质,变换 ${\mathbf{x}}^{\prime } \rightarrow - {\mathbf{x}}^{\prime }$ 可以通过设

$$
r \rightarrow r
$$

$$
\theta \rightarrow \pi - \theta \;\left( {\cos \theta \rightarrow - \cos \theta }\right) tag{4.2.26}
$$

$$
\phi \rightarrow \phi + \pi \;\left( {{e}^{im\phi } \rightarrow {\left( -1\right) }^{m}{e}^{im\phi }}\right)
$$

来实现. 对于正的 $m$ 以及 (3.6.38) 式,利用显式表示式

$$
{Y}_{l}^{m} = {\left( -1\right) }^{m}\sqrt{\frac{\left( {{2l} + 1}\right) \left( {l - m}\right) !}{{4\pi }\left( {l + m}\right) !}}{P}_{l}^{m}\left( {\cos \theta }\right) {e}^{im\phi } tag{4.2.27}
$$

其中

$$
{P}_{l}^{\left| m\right| }\left( {\cos \theta }\right) = \frac{{\left( -1\right) }^{m + l}}{{2}^{l}l!}\frac{\left( {l + \left| m\right| }\right) !}{\left( {l - \left| m\right| }\right) !}{\sin }^{-\left| m\right| }\theta {\left( \frac{d}{d\left( {\cos \theta }\right) }\right) }^{l - \left| m\right| }{\sin }^{2l}\theta , tag{4.2.28}
$$

当 $\theta$ 和 $\phi$ 按照 (4.2.26) 式改变时,很容易证明

$$
{Y}_{l}^{m} \rightarrow {\left( -1\right) }^{l}{Y}_{l}^{m}. tag{4.2.29}
$$

因此, 得出

$$
\pi \mid \alpha ,{lm}\rangle = {\left( -1\right) }^{l}|\alpha ,{lm}\rangle . tag{4. 2.30}
$$

实际上,不必去看 ${Y}_{l}^{m}$ ,得到同样结果的更容易的方法是取 $m = 0$ ,且注意到 ${L}_{ \pm }^{r}|l, m = 0\rangle$ $\left( {r = 0,1,\cdots, l}\right)$ 一定具有同样的宇称,因为 $\pi$ 与 ${\left( {L}_{ \pm }\right) }^{r}$ 对易.

现在来看一看能量本征态的宇称性质, 从一个非常重要的定理开始.

定理 4.1 假定

$$
\left\lbrack {H,\pi }\right\rbrack = 0 tag{4. 2.31}
$$

而 $|n\rangle$ 是 $H$ 的一个非简并的本征右矢,本征值为 ${E}_{n}$ :

$$
H\left| {n\rangle = {E}_{n}}\right| n\rangle ; tag{4. 2.32}
$$

则 $|n\rangle$ 也是一个宇称的本征右矢.

证明 首先注意到

$$
\frac{1}{2}\left( {1 \pm \pi }\right) |n\rangle tag{4. 2.33}
$$

是一个本征值为 $\pm 1$ (只要利用 ${\pi }^{2} = 1$ ) 的宇称本征右矢,来证明这个定理. 但是,它也是一个本征值为 ${E}_{n}$ 的能量本征右矢. 此外, $|n\rangle$ 和 (4.2.33) 式必须代表相同的态; 否则,就会存在两个态有相同的能量——这与非简并假设相矛盾. 因此可以得出结论, $|n\rangle$ 和 (4.2.33) 式是一样的, 至多差一个相乘常数因子, 一定是一个宇称为 $\pm 1$ 的宇称本征右矢.

作为一个例子, 看一看简谐振子. 基态 $|0\rangle$ 有偶宇称,因为它的波函数是高斯函数, 在 ${\mathbf{x}}^{\prime } \rightarrow - {\mathbf{x}}^{\prime }$ 时是偶的. 第一激发态

$$
\left| {1\rangle = {a}^{ \dagger }}\right| 0\rangle , tag{4. 2.34}
$$

一定是奇宇称,因为 ${a}^{ \dagger }$ 对 $x$ 和 $p$ 是线性的,它们都是奇的 [见 (2.3.2) 式]. 一般而论, 简谐振子的第 $n$ 个激发态的宇称由 ${\left( -1\right) }^{n}$ 给出.

重要的是要注意, 这里的非简并假设是至关重要的. 例如, 考虑非相对论量子力学中的氢原子. 众所周知,其能量本征值仅依赖于主量子数 $n$ (例如, ${2p}$ 和 ${2s}$ 态是简并的)——库伦势在宇称之下显然是不变的——然而一个能量本征右矢

$$
{c}_{p}\left| {{2p}\rangle + {c}_{s}}\right| {2s}\rangle tag{4. 2.35}
$$

显然不是一个宇称本征右矢.

作为另一个例子, 考虑一个动量本征右矢. 动量与宇称反对易, 所以一一尽管自由粒子哈密顿量 $H$ 是宇称下不变量——动量本征右矢(尽管显然是一个能量本征右矢)不是一个宇称本征右矢. 我们的定理依然正确无误,因为在这里有 $\left| {\mathbf{p}}^{\prime }\right\rangle$ 和 $\left| {-{\mathbf{p}}^{\prime }}\right\rangle$ 之间的简并,二者有相同的能量. 事实上,可以很容易地构造线性组合 $\left( {1/\sqrt{2}}\right) \left( {\left| {\mathbf{p}}^{\prime }\right\rangle \pm \left| {-{\mathbf{p}}^{\prime }}\right\rangle }\right)$ , 它们是宇称的本征右矢,本征值为 $\pm 1$ . 利用波函数的语言, ${e}^{i{p}^{\prime } \cdot {x}^{\prime }/\hslash }$ 没有确定宇称,但是, ${\operatorname{cosp}}^{\prime } \cdot {\mathbf{x}}^{\prime }/\hslash$ 和 ${\operatorname{sinp}}^{\prime } \cdot {\mathbf{x}}^{\prime }/\hslash$ 都有确定宇称.

\subsection{对称的双阱势}

图 4.3 有两个最低态 $|S\rangle$ (对称的)

和 $|A\rangle$ (反对称的) 的对称双阱势

作为一个基本的但具有启发意义的例子, 考虑一个对称的双阱势, 见图 4.3. 哈密顿量显然在宇称下不变. 事实上, 两个最低态如图 4.3 所示, 通过在经典允许区求出明显包含正弦和余弦, 而在经典禁戒区包含双曲正弦 (sinh) 和双曲余弦 (cosh) 的解, 可以看到这一点. 在势能不连续处把这些解匹配起来, 把它们称为对称态 $|S\rangle$ 和反对称态 $|A\rangle$ . 当然,它们是

$H$ 和 $\pi$ 的共同本征右矢. 计算还表明,

$$
{E}_{A} > {E}_{S}, tag{4.2.36}
$$

通过注意到反对称态的波函数具有更大的曲率, 可以从图 4.3 推出这个结果. 如果中间的势垒很高, 则能量差非常小, 稍后将讨论这一点.

构造

$$
|R\rangle = \frac{1}{\sqrt{2}}\left( {\left| {S\rangle + }\right| A\rangle }\right)
$$

(4. ${2.37a}$ )

和

$$
|L\rangle = \frac{1}{\sqrt{2}}\left( {\left| {S\rangle - }\right| A\rangle }\right) . tag{4. 2.37b}
$$

(4.2.37a) 式和 (4.2.37b) 式的波函数的大部分分别集中在右边和左边. 它们显然不是宇称的本征态; 事实上,在宇称下, $|R\rangle$ 和 $|L\rangle$ 互换. 注意,它们也不是能量本征态. 的确,它们是非定态的典型例子. 确切地讲,假定在 $t = 0$ 时,系统由 $|R\rangle$ 表示. 在稍后时刻, 则有

$$
\left| {R,{t}_{0} = 0;t}\right\rangle = \frac{1}{\sqrt{2}}\left( {{e}^{-i{E}_{S}t/\hslash }\left| {S\rangle + {e}^{i{E}_{\Lambda }t/\hslash }}\right| A\rangle }\right) tag{4. 2.38}
$$

$$
= \frac{1}{\sqrt{2}}{e}^{-i{E}_{{S}^{\prime \prime \hslash }}}\left( {\left| {S\rangle + {e}^{i\left( {{E}_{A} - {E}_{S}}\right) t/\hslash }}\right| A\rangle }\right) .
$$

在 $t = T/2 \equiv {2\pi }\hslash /2\left( {{E}_{A} - {E}_{S}}\right)$ 时刻,发现系统处在纯 $|L\rangle$ 态. 在 $t = T$ 时刻,回到纯 $|R\rangle$ , 依此类推. 于是,一般而言,我们有一个 $\left| {R\rangle \text{与}}\right| L\rangle$ 之间的振荡,其角频率为

$$
\omega = \frac{\left( {E}_{\Lambda } - {E}_{S}\right) }{\hslash }. tag{4. 2.39}
$$

这种振荡行为也可以从量子力学穿透观点考虑. 原来被限制在右边的粒子可以通过经典禁戒区 (中间的势垒) 穿透到左边, 然后返回到右边, 依此类推. 但是现在设中间的势垒变成无穷高,如图 4.4 所示. 则 $\left| {S\rangle \text{和}}\right| A\rangle$ 态现在是简并的,所以 (4.2.37a) 式和 (4.2.37b) 式也都是能量本征右矢, 尽管它们都不是宇称本征右矢. 一旦系统被发现处在 $|R\rangle$ 态,将会永远保持在那里 ( $|S\rangle$ 和 $|A\rangle$ 之间的振荡时间现在是 $\infty$ ). 因为这个中间势垒无穷高, 没有任何穿透的可能性. 于是, 当存在简并时, 物理上可实现的能量本征右矢不需要是宇称本征右矢. 尽管哈密顿量本身在空间反射下是对称的, 但因为有一个反对称的基态,所以在简并的情况下, $H$ 的对称性不一定被能量本征态 $|S\rangle$ 和 $|A\rangle$ 所遵从.

这是破缺对称性和简并的一个非常简单的例子. 自然界充满了与此相类似的情况. 考虑一块铁磁体. 对于铁原子的基本哈密顿量是转动不变的, 但是这块铁磁体明显地在空间有一个确定的方向; 因此 (无穷多个) 基态都不是转动不变的, 因为自旋都沿着某个确定的 (但是任意的) 方向排列起来.



图 4.4 有无穷高中间势垒的对称双阱

教科书给出的证明对称双阱势实际重要性的系统的范例是一个氨分子 ${\mathrm{{NH}}}_{3}$ ,如图 4.5 所示. 想象三个氢 (H) 原子形成一个等边三角形的三个角. 氮 (N) 原子可以在上方或在下方, 其中上和下的方向都是确定的, 因为正如图 4.5 所示的这个分子正在绕这个轴转动. 氮原子的上和下的两个位置类似于双阱势中的 $R$ 和 $L$ . 宇称和能量本征态分别为 (4.2.37a) 式和 (4.2.37b) 式意义上的图 4.5a 和图 4.5b 的叠加, 而在能量和宇称的共同本征态之间的能量差对应于一个 ${24},{000}\mathrm{{MHz}}$ (兆赫) 的振荡频率——大约 $1\mathrm{\;{cm}}$ 的波长,它是在微波范围. 事实上, ${\mathrm{{NH}}}_{3}$ 在微波激射器物理学中有着基本的重要性.

存在一些天然有机分子,诸如氨基酸和糖,它们都只是 $R$ 类 (或 $L$ 类) 的. 有确定手性的这样的一些分子称为旋光异构体. 在许多情况下, 振荡时间实际上是无穷大——在 ${10}^{1}$ 到 ${10}^{6}$ 年的量级——因此, $R$ 类分子对于所有的实用目的都保持为右手. 有趣的是, 如果试图在实验室中合成这样的一些有机分子,会发现为 $R$ 与 $L$ 的均等的混合物. 为什么会有某一类占了优势, 是自然界最深的奥秘. 这是由于一种像蜗牛的螺旋壳那样的遗传事故, 还是由于事实上我们的心脏就是在左边呢? *

---

* 有人建议在生命形成期间活跃的核过程中的宇称破坏对于这种手性有贡献. 见 W. A. Bonner. Parity Violation and the Evolution of Biomolecular Homochirality, Chirality, 12(2000)114.

---


图 4.5 氨分子 ${\mathrm{{NH}}}_{3}$ ,其中三个氢原子形成一个等边三角形的三个角顶

\subsection{宇称-选择定则}

假定 $|\alpha \rangle$ 和 $|\beta \rangle$ 都是宇称本征态:

$$
\pi \left| {\alpha \rangle = {\varepsilon }_{\alpha }}\right| \alpha \rangle tag{4.2.40a}
$$

和

$$
\pi \left| {\beta \rangle = {\varepsilon }_{\beta }}\right| \beta \rangle , tag{4.2.40b}
$$

其中 ${\varepsilon }_{\alpha },{\varepsilon }_{\beta }$ 是宇称本征值 (±1). 可以证明,除非 ${\varepsilon }_{\alpha } = - {\varepsilon }_{\beta }$ ,否则

$$
\langle \beta \left| \mathbf{x}\right| \alpha \rangle = 0. tag{4.2.41}
$$

换句话说,宇称为奇的算符 $\mathbf{x}$ 连接相反宇称的态. 该式证明如下:

$$
\langle \beta \left| \mathbf{x}\right| \alpha \rangle = \left\langle {\beta \left| {{\pi }^{-1}\pi \mathbf{x}{\pi }^{-1}\pi }\right| \alpha }\right\rangle = {\varepsilon }_{\alpha }{\varepsilon }_{\beta }\left( {-\langle \beta \left| \mathbf{x}\right| \alpha \rangle }\right) , tag{4.2.42}
$$

除非 ${\varepsilon }_{\alpha }$ 和 ${\varepsilon }_{\beta }$ 符号相反,否则得到一个有限的、非零的 $\left\langle {\beta \left| \mathbf{x}\right| \alpha }\right\rangle$ 是不可能的. 如果 ${\psi }_{\beta }$ 和 ${\psi }_{\alpha }$ 具有相同的宇称, 则由下式

$$
\int {\psi }_{\beta }^{ * }\mathbf{x}{\psi }_{a}{d\tau } = 0 tag{4.2.43}
$$

读者或许会熟悉这一论证. 最早由维格纳提出的这个选择定则对于讨论原子之间的辐射跃 (4) 为多级展开形式的一个后果. 在量子力学诞生之前, 这个规则是从谱线分析唯象学上知道的, 称作拉波特 (Laporte) 规则. 正是维格纳证明了拉波特规则是宇称选择定则的结果.

如果基本哈密顿量 $H$ 是宇称下不变的,则非简并的能量本征态 [作为 (4.2.43) 式的一个推论] 不可能具有一个永久的电偶极矩:

$$
\langle n\left| \mathbf{x}\right| n\rangle = 0. tag{4.2.44}
$$

该式可以从 (4.2.43) 式平凡导出, 因为在非简并的假设下, 能量本征态也是宇称本征态 [见 (4.2.32) 式和 (4.2.33) 式]. 对于一个简并态, 具有一个电偶极矩是完全没有问题的. 在第 5 章讨论线性斯塔克 (Stark) 效应时, 将看到一个这样的例子.

可以考虑推广: 宇称下为奇的算符,比如 $\mathbf{p}$ 或 $\mathbf{S} \cdot \mathbf{x}$ ,只在相反宇称的态之间有非零矩阵元. 相反, 宇称下为偶的算符把相同宇称的态连接起来.

\subsection{宇称不守恒}

导致基本粒子的所谓弱相互作用的基本哈密顿量, 在宇称下不是不变的. 衰变过程中能有相反宇称态叠加的终态. 像衰变产物的角分布那样的一些可观测量, 可能依赖于诸如 $\langle \mathbf{S}\rangle \cdot \mathbf{p}$ 那样的赝标量. 值得一提的是,以前人们一直都相信宇称守恒是一个不可动摇的神圣原理, 直到 1956 年, 李政道和杨振宁推测在弱相互作用中宇称不守恒, 并且提出了一个检验宇称守恒的有效性的关键实验. 随后进行的一些实验的确表明, 一些可观测效应确实依赖于诸如 $\langle \mathbf{S}\rangle$ 与 $\mathbf{p}$ 之间的关联那样的赝标量.

至今, 宇称不守恒的最清晰的证明之一是最早的这个实验. 其结果 [见 Wu, Ambler, Phys. Rev. 105 (1957) 1413] 表明了一个依赖于 $\langle \mathbf{S}\rangle \cdot \mathbf{p}$ 的衰变率. 观测的衰变是 ${}^{60}\mathrm{{Co}} \rightarrow {}^{60}\mathrm{{Ni}}$ $+ {e}^{ - } + {\bar{\nu }}_{e}$ ,其中 $\mathbf{S}$ 是 ${}^{60}\mathrm{{Co}}$ 原子核的自旋,发射的 ${e}^{ - }$ 动量是 $\mathbf{p}$ . 自旋极化的放射性 ${}^{60}\mathrm{{Co}}$ 原子核的样品在低温下制备,衰变得到的 ${e}^{ - }$ 在与自旋平行或反平行的方向探测,它依赖于极化磁场的符号. 样品的极化通过观测激发的子核 ${}^{60}\mathrm{{Ni}}$ 衰变中 $\gamma$ 射线的各向异性监控,这是一个宇称守恒的效应. 结果如图 4.6 所示. 在几分钟的一段时间,样品升温, $\beta$ 衰变不对称性消失, 消失的速率与 $\gamma$ 射线的各向异性消失的速率相同.


图 4.6 宇称不守恒的实验证明. 左图显示的关键观测是按照其核自旋方向取向的放射性钴原子核优先沿反方向发射 “ $\beta$ 射线”. 右图显示的实验数据表明 $\beta$ 衰变的上/下不对称性 (底部的图框) 如何完美地与标明核极化度 (上部的图框) 的信号相关联. 随着时间的推移,样品发热而且钴原子核退极化. [右边的数据来自 Wu, Phys. Rev. 105 (1957) 1413.]

因为在弱作用中宇称不守恒, 以前认为是 “纯” 的核与原子的态实际上是宇称混合态. 这些难以捉摸的效应也在实验上发现了.

\section{晶格平移作为一种分立对称性}

现在考虑另一类分立对称性操作, 即晶格平移. 这一主题在固体物理中有极为重要的应用.

考虑一个一维周期势,其中 $V\left( {x \pm a}\right) = V\left( x\right)$ ,如图 4.7 所示. 但实际上,可以考虑在一个空间位置等间隔排列的正离子链中一个电子的运动. 一般而言,这个哈密顿量在 $l$ 取任意值时由 $\tau \left( l\right)$ 所表示的一个平移之下不是不变的,其中 $\tau \left( l\right)$ 有如下性质 (见 1.6 节)

$$
{\tau }^{ \dagger }\left( l\right) {x\tau }\left( l\right) = x + l,\;\tau \left( l\right) \left| {x}^{\prime }\right\rangle = \left| {{x}^{\prime } + l}\right\rangle . tag{4.3.1}
$$

然而,当 $l$ 与晶格间距 $a$ 相符时,会有

$$
{\tau }^{ \dagger }\left( a\right) V\left( x\right) \tau \left( a\right) = V\left( {x + a}\right) = V\left( x\right) . tag{4.3.2}
$$

因为哈密顿量的动能部分在任何位移的平移下都是不变的, 所以整个哈密顿量满足

$$
{\tau }^{ \dagger }\left( a\right) {H\tau }\left( a\right) = H. tag{4.3.3}
$$

因为 $\tau \left( a\right)$ 是幺正的,从 (4.3.3) 式有

$$
\left\lbrack {H,\tau \left( a\right) }\right\rbrack = 0, tag{4.3.4}
$$

因此该哈密顿量与 $\tau \left( a\right)$ 可以同时对角化. 尽管 $\tau \left( a\right)$ 是幺正的,但不是厄米的,所以它的本征值预期是一个模为 1 的复数.


图 4.7 (a) 周期为 $a$ 的一维周期势. (b) 当相邻两个格点之间势垒高度变成无穷大时的周期势

在确定 $\tau \left( a\right)$ 的本征右矢和本征值并考察它们的物理意义之前,富有启发意义的是, 看一看如图 ${4.7}\mathrm{\;b}$ 所示的、当相邻两个格点之间势垒高度趋向无穷大时的周期势的特殊情况. ${4.7}\mathrm{\;b}$ 位势的基态是什么呢? 显然,粒子完全定位于晶格中的一个格点的态可能是基态的一个候选者. 为确定起见,假设粒子定位于第 $n$ 个格点,并且用 $|n\rangle$ 代表相应的右矢. 这是一个能量本征右矢,其能量本征值为 ${E}_{0}$ ,即 $H \mid n\rangle = {E}_{0} \mid n\rangle$ . 它的波函数 $\left\langle {{x}^{\prime } \mid n}\right\rangle$ 仅在第 $n$ 个格点处有限. 然而,注意到,定位于某另一格点的一个类似的态也有同样的能量 ${E}_{0}$ ,因此实际上有可数的无穷多个基态 $n$ ,其中的 $n$ 从 $- \infty$ 到 $\infty$ .

显然, $|n\rangle$ 不是一个晶格平移算符的本征右矢,因为当晶格平移算符作用于它时,得到 $|n + 1\rangle$ :

$$
\tau \left( a\right) \left| {n\rangle = }\right| n + 1\rangle . tag{4.3.5}
$$

所以,尽管事实上 $\tau \left( a\right)$ 与 $H$ 对易, $|n\rangle$ 一一是 $H$ 的一个本征右矢一一却不是 $\tau \left( a\right)$ 的一个本征右矢. 这一点与早些时候的对称性定理完全自洽, 因为有一个无穷维简并. 当存在这样的简并时,这个世界的对称性不需要是能量本征态的对称性. 现在的任务是找到 $H$ 与 $\tau \left( a\right)$ 的共同本征右矢.

在这里, 可以回忆一下如何处理前一节有点类似的对称双阱势情况. 注意到, 尽管 $\left| {R\rangle \text{和}}\right| L\rangle$ 都不是 $\pi$ 的本征右矢,却能够容易地构成 $\left| {R\rangle \text{和}}\right| L\rangle$ 的一个对称组合与一个反对称组合, 它们都是宇称的本征右矢. 这种情况与这里类似. 下面具体地构成一个线性组合

$$
|\theta \rangle \equiv \mathop{\sum }\limits_{{n = - \infty }}^{\infty }{e}^{in\theta }|n\rangle , tag{4.3.6}
$$

其中的 $\theta$ 是一个实参量,取值为 $- \pi \leq \theta \leq \pi$ . 可以断定 $|\theta \rangle$ 是 $H$ 和 $\tau \left( a\right)$ 的一个共同本征右矢. 它是 $H$ 的一个本征右矢是显然的,因为 $|n\rangle$ 是能量的一个本征右矢,本征值为 ${E}_{0}$ , 它不依赖于 $n$ . 为了证明它也是晶格平移算符的一个本征右矢,用 $\tau \left( a\right)$ 如下作用

$$
\tau \left( a\right) \left| {\theta \rangle = \mathop{\sum }\limits_{{n = - \infty }}^{\infty }{e}^{in\theta }}\right| n + 1\rangle = \mathop{\sum }\limits_{{n = - \infty }}^{\infty }{e}^{i\left( {n - 1}\right) \theta }|n\rangle tag{4.3.7}
$$

$$
= {e}^{-{i\theta }}|\theta \rangle
$$

注意, $H$ 和 $\tau \left( a\right)$ 的这个共同本征右矢用一个连续参量 $\theta$ 参量化. 此外,能量本征值 ${E}_{0}$ 不依赖于 $\theta$ .

现在回到图 4.7a 所示的更为现实的情况, 在那里两个相邻格点之间的势垒不是无穷高. 可以构造一个局域的右矢 $|n\rangle$ ,它恰和前面一样,具有性质 $\tau \left( a\right) \left| {n\rangle = }\right| n + 1\rangle$ . 然而, 这一次可以预期, 由于量子力学穿透的结果, 存在一些可能进入到相邻格点的泄漏. 换言之,波函数 $\left\langle {{x}^{\prime } \mid n}\right\rangle$ 有一个尾巴延伸到第 $n$ 个格点之外的一些格点中. 因为平移不变性, 所以在基 $\{ |n\rangle \}$ 中 $H$ 的对角元都相等. 即

$$
\langle n\left| H\right| n\rangle = {E}_{0}, tag{4.3.8}
$$

如前一样,它不依赖于 $n$ . 然而,我们怀疑,作为泄漏的一个后果,在基 $\{ |n\rangle \}$ 中 $H$ 不能完全对角化. 现在, 假定相邻格点之间的势垒很高 (但不是无限高). 那么可以预期, 在距离远的格点之间 $H$ 的矩阵元完全可以忽略. 假设唯一重要的一些非对角元连接最近邻. 这就是说

$$
\left\langle {{n}^{\prime }\left| H\right| n}\right\rangle \neq 0\text{ 反当 }{n}^{\prime } = n\text{ 或 }{n}^{\prime } = n \pm 1, tag{4.3.9}
$$

在固体物理中, 这种假设被称为紧束缚近似. 定义

$$
\langle n \pm 1\left| H\right| n\rangle = - \Delta tag{4.3.10}
$$

显然,再一次由于哈密顿量的平移不变性, $\Delta$ 不依赖于 $n$ . 在 $n \neq {n}^{\prime }$ 时 $|n\rangle$ 和 $\left| {n}^{\prime }\right\rangle$ 正交的范围内, 得到

$$
H\left| {n\rangle = {E}_{0}}\right| n\rangle - \Delta \left| {n + 1\rangle - \Delta }\right| n - 1\rangle . tag{4.3.11}
$$

注意, $|n\rangle$ 不再是一个能量本征右矢.

正如图 4.7b 所示的位势情况下所做的, 构成一个线性组合

$$
|\theta \rangle = \mathop{\sum }\limits_{{n = - \infty }}^{\infty }{e}^{in\theta }|n\rangle tag{4.3.12}
$$

显然, $|\theta \rangle$ 是平移算符 $\tau \left( a\right)$ 的本征右矢,因为 (4.3.7) 式中的那些步骤仍然成立. 一个自然的问题是, $|\theta \rangle$ 是一个能量本征右矢吗? 为了回答这个问题,用 $H$ 作用:

$$
H\sum {e}^{in\theta }\left| {n\rangle = {E}_{0}\sum {e}^{in\theta }}\right| n\rangle - \Delta \sum {e}^{in\theta }\left| {n + 1\rangle - \Delta \sum {e}^{in\theta }}\right| n - 1\rangle
$$

$$
= {E}_{0}\sum {e}^{in\theta }\left| {n\rangle - \Delta \sum \left( {{e}^{{in\theta } - {i\theta }} + {e}^{{in\theta } + {i\theta }}}\right) }\right| n\rangle tag{4.3.13}
$$

$$
= \left( {{E}_{0} - {2\Delta }\cos \theta }\right) \sum {e}^{in\theta }|n\rangle ,
$$

该式与以前的情况之间的最大差别是,能量本征值现在依赖于连续的实参量 $\theta$ . 当 $\Delta$ 变成有限时,简并解除了,而且在 ${E}_{0} - {2\Delta }$ 和 ${E}_{0} + {2\Delta }$ 之间有了一个能量本征值的连续分布. 图 4.8 形象化地显示了随着 $\Delta$ 从零增大,能级如何开始形成一个连续的能带.

为了理解参量 $\theta$ 的物理意义,研究波函数 $\left\langle {{x}^{\prime } \mid \theta }\right\rangle$ . 对于晶格平移后的态 $\tau \left( a\right) |\theta \rangle$ 的波函数,通过让 $\tau \left( a\right)$ 作用在 $\left\langle {x}^{\prime }\right|$ 上,有

$$
\left\langle {{x}^{\prime }\left| {\tau \left( a\right) }\right| \theta }\right\rangle = \left\langle {{x}^{\prime } - a \mid \theta }\right\rangle . tag{4.3.14}
$$

也可以让 $\tau \left( a\right)$ 作用在 $|\theta \rangle$ 上,然后利用 (4.3.7) 式. 于是

$$
\left\langle {{x}^{\prime }\left| {\tau \left( a\right) }\right| \theta }\right\rangle = {e}^{-{i\theta }}\left\langle {{x}^{\prime } \mid \theta }\right\rangle , tag{4.3.15}
$$

所以,

$$
\left\langle {{x}^{\prime } - a \mid \theta }\right\rangle = \left\langle {{x}^{\prime } \mid \theta }\right\rangle {e}^{-{i\theta }} tag{4.3.16}
$$

通过设

$$
\left\langle {{x}^{\prime } \mid \theta }\right\rangle = {e}^{{ik}{x}^{\prime }}{u}_{k}\left( {x}^{\prime }\right) , tag{4.3.17}
$$

且取 $\theta = {ka}$ ,求解上述方程,其中 ${u}_{k}\left( {x}^{\prime }\right)$ 是一个周期函数,周期为 $a$ ,通过明显地代换容易证明这一点, 即

$$
{e}^{{ik}\left( {{x}^{\prime } - a}\right) }{u}_{k}\left( {{x}^{\prime } - a}\right) = {e}^{{ik}{x}^{\prime }}{u}_{k}\left( {x}^{\prime }\right) {e}^{-{ika}}. tag{4. 3.18}
$$



图 4.8 随着 $\Delta$ 从零增大,能级形成一个连续的能带

这样,得到布洛赫 (Bloch) 定理的重要条件: 作为 $\tau \left( a\right)$ 的一个本征右矢, $|\theta \rangle$ 的波函数可以写成一个平面波 ${e}^{{ik}{x}^{\prime }}$ 乘以一个周期为 $a$ 的周期函数. 注意,利用的唯一的事实是 $|\theta \rangle$ 是 $\tau \left( a\right)$ 的一个本征右矢,本征值为 ${e}^{-{i\theta }}\left\lbrack \text{见 (4.3.7) 式}\right\rbrack$ . 特别是,即使紧束缚近似 (4.3.9) 式被破坏了, 这个定理仍然成立.


图 4.9 在布里渊区 $\left| k\right| \leq \pi /a$ 中 $E\left( k\right)$ 对 $k$ 的色散曲线

现在解释对于 $|\theta \rangle$ 给出的较早一些的结果 (4.3.13) 式. 我们知道,这个波函数是一个平面波,由受到了一个周期函数 ${u}_{k}\left( {x}^{\prime }\right)$ 调制的、传播的波矢量 $k$ 所表征 [见 (4.3.17)]. 当 $\theta$ 从一 $\pi$ 变到 $\pi$ 时,波矢量 $k$ 从 $- \pi /a$ 变到 $\pi /a$ . 而能量本征值 $E$ 现在对 $k$ 的依赖关系如下:

$$
E\left( k\right) = {E}_{0} - {2\Delta }\cos {ka}. tag{4.3.19}
$$

注意, 只要紧束缚近似适用, 这个能量本征值方程不依赖于位势的具体形状. 还要注意, 布洛赫波函数 (4.3.17) 式的波矢量 $k$ 中存在一个截断,由 $\left| k\right| = \pi /a$ 给出. 方程 (4.3.19) 式定义了一条色散曲线, 如图 4.9 所示. 作为穿透的一个结果, 可数的无穷多重简并现在完全解除了,而可允许的能量值在 ${E}_{0} - {2\Delta }$ 和 ${E}_{0} + {2\Delta }$ 之间形成,一个连续的带, 所谓的布里渊区.

到此为止, 我们仅仅考虑了一个粒子在周期势中运动. 在更为现实的情况下, 必须考虑多个电子在这样的位势中运动. 实际上, 电子满足泡利不相容原理, 正如将在第 7 章更系统地讨论的那样, 它们开始填充能带. 以这种方式, 金属、半导体以及类似的材料的定性特点可以被理解为平移不变性加上不相容原理的一个推论.

读者可能已经注意到了 4.2 节的对称双阱问题与本节的周期势之间的相似性. 比较图 4.3 与图 4.7, 注意到, 它们可以看成是具有有限数目谷底的位势的两个相反的极端 (2 与无限大).

\section{时间反演分立对称性}

在这一节将研究另一种分立对称算符, 所谓的时间反演. 对于初学者, 这是一个困难的问题, 部分原因是术语时间反演是一个误称, 令人想起科幻小说. 在这一节所要做的, 可以更合适地用术语运动反演来表征. 的确, 这是由维格纳使用的短语, 1932 年, 他写的一篇非常基本的文章中明确表述了时间反演.

为使目的明确, 来看一看经典力学. 假定有一条遭受到某一力场作用的粒子的运动轨道,见图 4.10. $t = 0$ 时,让粒子停止,然后反演它的运动: ${\left. \mathbf{p}\right| }_{t = 0} \rightarrow - {\left. \mathbf{p}\right| }_{t = 0}$ . 则该粒子反向遍历同样的轨道. 假如像 (b) 中那样向后倒着播放这个轨道 (a) 的运动图片, 可能很难讲清楚这是否是正确的顺序.

更正式地讲,如果 $\mathbf{x}\left( t\right)$ 是下列方程的一个解:

$$
m\ddot{\mathbf{x}} = - \nabla V\left( \mathbf{x}\right) , tag{4. 4.1}
$$

则 $\mathbf{x}\left( {-t}\right)$ 也是在由 $V$ 导出的同样的力场中一个可能的解. 当然,特别要注意,在这里我们没有耗散力. 桌面上的一个滑块 (由于摩擦力) 逐渐减速并最终停了下来. 但是, 你见过桌面上的滑块会自动开始运动并逐渐加速吗?


图 4.10 (a) $t = 0$ 时停止的经典轨道,和 (b) 反演了它的运动, ${\left. \mathrm{p}\right| }_{t = 0} \rightarrow - {\left. \mathrm{p}\right| }_{t = 0}$

在有磁场存在的情况下, 或许能够讲出这种差别. 设想正在拍摄在磁场中的一个做螺旋运动电子轨迹的运动图片. 通过比较旋转相对于标记为 $\mathrm{N},\mathrm{S}$ 磁极的感觉,或许能够辨别出运动图片是向前放还是向后倒着放. 然而, 从微观观点看, $\mathbf{B}$ 是由移动的电荷通过电流产生的; 假如能把引起 $\mathbf{B}$ 的电流也反演的话,则情况就会是完全对称的了. 依据图 4.11 所示的图像,你或许会已经猜测出 $\mathrm{N}$ 和 $\mathrm{S}$ 标错了! 另外一种叙述这一切的更正式的方式是, 麦克斯韦方程, 例如

$$
\nabla \cdot \mathbf{E} = {4\pi \rho },\;\nabla \times \mathbf{B} - \frac{1}{c}\frac{\partial \mathbf{E}}{\partial t} = \frac{{4\pi }\mathbf{j}}{c},\;\nabla \times \mathbf{E} = - \frac{1}{c}\frac{\partial \mathbf{B}}{\partial t}, tag{4.2}
$$

和洛伦兹力方程 $\mathbf{F} = e\left\lbrack {\mathbf{E} + \left( {1/c}\right) \left( {\mathbf{v} \times \mathbf{B}}\right) }\right\rbrack$ 在 $t \rightarrow - t$ 之下是不变的, 只要令

$$
\mathbf{E} \rightarrow \mathbf{E},\;\mathbf{B} \rightarrow - \mathbf{B},\;\rho \rightarrow \rho ,\;\mathbf{j} \rightarrow - \mathbf{j},\;\mathbf{v} \rightarrow - \mathbf{v}.({4.4} tag{4. 4.3}
$$

现在看一看波动力学的基本方程, 即薛定谔方程为

假定 $\psi \left( {\mathbf{x}, t}\right)$ 是一个解. 容易证明 $\psi \left( {\mathbf{x}, - t}\right)$ 不是一个解,因为出

$$
i\hslash \frac{\partial \psi }{\partial t} = \left( {-\frac{{\hslash }^{2}}{2m}{\nabla }^{2} + V}\right) \psi . tag{4.4.4}
$$

能量本征态说服自己相信这一点, 即通过把

$$
\psi \left( {\mathbf{x}, t}\right) = {u}_{n}\left( \mathbf{x}\right) {e}^{-i{E}_{n}t/\hslash },\;{\psi }^{ * }\left( {\mathbf{x}, - t}\right) = {u}_{n}^{ * }\left( \mathbf{x}\right) {e}^{-i{E}_{n}t/\hslash } tag{4. 4.5}
$$



图 4.11 在一块磁体的南北极之间电子的轨迹

代入到薛定谔方程 (4.4.4) 式中. 因此, 可以猜测时间反演一定与复共轭有某种关系. 如果 $t = 0$ 时,波函数由下式给出:

$$
\psi = \langle \mathbf{x} \mid \alpha \rangle , tag{4. 4.6}
$$

则相应的时间反演态波函数由 $\langle \mathbf{x} \mid \alpha {\rangle }^{ * }$ 给出. 稍后将证明,对于一个无自旋系统波函数, 情况确是如此. 作为一个例子, 可以容易现了一阶时间的微商. 然而, ${\psi }^{ * }\left( {\mathbf{x}, - t}\right)$ 是一个解,这一点你可以通过取 (4.4.4) 式的复共轭证实. 有启发意义的是, 对于一个地对于一个平面波的波函数检验这一点, 见本章的习题 4.8 .

\subsection{关于对称性操作的题外话}

在开始系统处理时间反演算符之前, 关于对称性操作给出一些一般性的评论是很合适的. 考虑一种对称性操作

$$
\left| {\alpha \rangle \rightarrow }\right| \bar{\alpha }\rangle ,\;\left| {\beta \rangle \rightarrow }\right| \bar{\beta }\rangle tag{4. 4.7}
$$

要求内积 $\langle \beta \mid \alpha \rangle$ 保持不变是很自然的,这就是说

$$
\langle \widetilde{\beta } \mid \widetilde{\alpha }\rangle = \langle \beta \mid \alpha \rangle . tag{4. 4.8}
$$

的确,对于诸如转动、平移、甚至是宇称等这类对称操作,情况确是如此. 如果 $|\alpha \rangle$ 被转动了,而 $|\beta \rangle$ 也按相同方式被转动了,则 $\langle \beta \mid \alpha \rangle$ 是不变的. 从形式上讲,这源自这样一件事实, 即对于前几节中考虑的对称操作, 相应的对称算符是幺正的, 因此,

$$
\langle \beta \mid \alpha \rangle \rightarrow \left\langle {\beta \left| {{U}^{ \dagger }U}\right| \alpha }\right\rangle = \langle \beta \mid \alpha \rangle . tag{4. 4.9}
$$

然而, 在讨论时间反演时, 可以看到 (4.4.8) 式的要求被证明是限制太严了. 的确, 只要强加更弱的要求

$$
\left| {\langle \widetilde{\beta }}\right| \widetilde{\alpha }\rangle \left| = \right| \langle \beta \mid \alpha \rangle \mid . tag{4. 4.10}
$$

则 (4.4.8) 式的要求显然满足 (4.4.10) 式. 但是, 这不是唯一的方式;

$$
\langle \bar{\beta } \mid \bar{\alpha }\rangle = \langle \beta \mid \alpha {\rangle }^{ * } = \langle \alpha \mid \beta \rangle tag{4. 4.11}
$$

同样也可用. 本节中采用后一种可能性, 因为, 从早些时候基于对薛定谔方程的讨论, 推断出时间反演与复共轭有某种联系.

定义 变换

$$
\left| {\alpha \rangle \rightarrow }\right| \widetilde{\alpha }\rangle = \theta \left| {\alpha \rangle ,\;}\right| \beta \rangle \rightarrow \left| {\widetilde{\beta }\rangle = \theta }\right| \beta \rangle tag{4. 4.12}
$$

被称为反幺正的, 如果

$$
\langle \widetilde{\beta } \mid \widetilde{\alpha }\rangle = \langle \beta \mid \alpha {\rangle }^{ * },
$$

(4. ${4.13a}$ )

$$
\theta \left( {{c}_{1}\left| {\alpha \rangle + {c}_{2}}\right| \beta \rangle }\right) = {c}_{1}^{ * }\theta \left| {\alpha \rangle + {c}_{2}^{ * }\theta }\right| \beta \rangle . tag{4. 4.13b}
$$

在这样的情况下,算符 $\theta$ 是一个反幺正算符. 关系式 (4.4.13b) 独自定义一个反线性算符.

现在要求一个反幺正算符可以写成

$$
\theta = {UK}, tag{4. 4.14}
$$

其中 $U$ 是一个幺正算符,而 $K$ 是复共轭算符,它使乘在一个右矢上的任何系数(位于 $K$ 的右边) 变成复共轭. 在检验 (4.4.13) 式之前,先考察 $K$ 算符的性质. 假定有一个乘上一复数 $c$ 的右矢. 则有

$$
{Kc}\left| {\alpha \rangle = {c}^{ * }K}\right| \alpha \rangle . tag{4. 4.15}
$$

人们可能进一步问,如果 $|\alpha \rangle$ 用基右矢 $\left\{ \left| {a}^{\prime }\right\rangle \right\}$ 展开会发生什么? 在 $K$ 的作用下,有

$$
\left| {\alpha \rangle = \mathop{\sum }\limits_{{a}^{\prime }}\left| {a}^{\prime }\right\rangle \left\langle {{a}^{\prime } \mid \alpha }\right\rangle \overset{K}{ \rightarrow }}\right| \bar{\alpha }\rangle = \mathop{\sum }\limits_{{a}^{\prime }}\left\langle {{a}^{\prime } \mid \alpha }\right\rangle * K\left| {a}^{\prime }\right\rangle tag{4. 4.16}
$$

$$
= \mathop{\sum }\limits_{{a}^{\prime }}{\left\langle {a}^{\prime }\left| \alpha {\rangle }^{ * }\right| {a}^{\prime }\right\rangle }^{\prime }
$$

注意, $K$ 作用于基右矢上不改变这个基右矢. $\left| {a}^{\prime }\right\rangle$ 的显式表示式是

$$
\left| {a}^{\prime }\right\rangle = \left( \begin{matrix} 0 \\ 0 \\ \vdots \\ 0 \\ 1 \\ 0 \\ \vdots \\ 0 \end{matrix}\right) tag{4. 4.17}
$$

因而,没有任何会被 $K$ 改变的. 读者可能疑惑,例如,对于一个自旋 $\frac{1}{2}$ 系统, ${S}_{y}$ 本征右矢在 $K$ 下是否改变. 答案是,如果 ${S}_{z}$ 本征右矢被用作基右矢时,则必须把 ${S}_{y}$ 本征右矢改变,因为 ${S}_{y}$ 本征右矢 (1.1.14) 式在 $K$ 作用下经受

$$
K\left( {\frac{1}{\sqrt{2}}\left| {+\rangle \pm \frac{i}{\sqrt{2}}}\right| - \rangle }\right) \rightarrow \frac{1}{\sqrt{2}}\left| {+\rangle \mp \frac{i}{\sqrt{2}}}\right| - \rangle . tag{4. 4.18}
$$

另一方面,如果 ${S}_{y}$ 本征右矢自身被用作基右矢时,则在 $K$ 作用下,不改变 ${S}_{y}$ 本征右矢. 因此, $K$ 的效果随基改变. 作为结果,在 (4.4.14) 式中的 $U$ 的形式也依赖于所用的特殊的表象 (即基右矢的选择).

回到 $\theta = {UK}$ 和 (4.4.13) 式,首先检验性质 (4.4.13b) 式. 有

$$
\theta \left( {{c}_{1}\left| {\alpha \rangle + {c}_{2}}\right| \beta \rangle }\right) = {UK}\left( {{c}_{1}\left| {\alpha \rangle + {c}_{2}}\right| \beta \rangle }\right)
$$

$$
= {c}_{1}^{ * }{UK}\left| {\alpha \rangle + {c}_{2}^{ * }{UK}}\right| \beta \rangle tag{4. 4.19}
$$

$$
= {c}_{1}^{ * }\theta \left| {\alpha \rangle + {c}_{2}^{ * }\theta }\right| \beta \rangle
$$

所以,(4.4.13b) 式的确成立. 在检验 (4.4.13a) 式之前,可以断言在 $\theta$ 只作用于右矢时, 使用该式总是安全的. 恰是通过观察相应的右矢, 可以想到左矢会如何改变. 特别是,不必考虑 $\theta$ 从右边作用在左矢上,也不必定义 ${\theta }^{ \dagger }$ . 有

$$
\left| {\alpha \rangle \overset{\theta }{ \rightarrow }}\right| \bar{\alpha }\rangle = \mathop{\sum }\limits_{{a}^{\prime }}{\left\langle {a}^{\prime } \mid \alpha \right\rangle }^{ * }{UK}\left| {a}^{\prime }\right\rangle
$$

$$
= \mathop{\sum }\limits_{{a}^{\prime }}\left\langle {{a}^{\prime } \mid \alpha }\right\rangle * U\left| {a}^{\prime }\right\rangle tag{4. 4.20}
$$

$$
= \mathop{\sum }\limits_{{a}^{\prime }}\left\langle {\alpha \left| {a}^{\prime }\right\rangle {}^{ * }U \mid {a}^{\prime }}\right\rangle .
$$

至于 $|\beta \rangle$ ,有

$$
\left| {\widetilde{\beta }\rangle = \mathop{\sum }\limits_{{a}^{\prime }}{\left\langle {a}^{\prime } \mid \beta \right\rangle }^{ * }U \mid {a}^{\prime }\rangle \overset{DC}{ \leftrightarrow }\langle \widetilde{\beta }}\right| = \mathop{\sum }\limits_{{a}^{\prime }}\left\langle {{a}^{\prime } \mid \beta }\right\rangle \left\langle {a}^{\prime }\right| {U}^{ \dagger }
$$

$$
\langle \widetilde{\beta } \mid \widetilde{\alpha }\rangle = \mathop{\sum }\limits_{{a}^{\prime \prime }}\mathop{\sum }\limits_{{a}^{\prime }}\left\langle {{a}^{\prime \prime } \mid \beta }\right\rangle \left\langle {{a}^{\prime \prime }\left| {{U}^{ \dagger }U}\right| {a}^{\prime }}\right\rangle \left\langle {\alpha \mid {a}^{\prime }}\right\rangle tag{4. 4.21}
$$

$$
= \mathop{\sum }\limits_{{a}^{\prime }}\left\langle {\alpha \left| {a}^{\prime }\right\rangle \left\langle {{a}^{\prime } \mid \beta }\right\rangle = \langle \alpha \mid \beta }\right\rangle
$$

$$
= \langle \beta \mid \alpha {\rangle }^{ * },
$$

于是该式得到了检验. (回忆一下 1.2 节的 “对偶对应”, 或 DC 符号.)

为使 (4.4.10) 式得到满足, 物理上感兴趣的是只考虑两类变换一幺正的和反幺正的. 其他的可能性都与前述的这两种中任何一种通过平凡的相位改变联系起来. 证明这一断言实际非常困难, 在这里不再进一步讨论. 然而, 可以参见 Gottfried 和 Yan (2003),7.1 节.

\subsection{时间反演算符}

现在终于能够给出时间反演的一种形式理论了. 时间反演算符用 $\Theta$ 代表,以便与一个一般的反幺正算符 $\theta$ 区分开. 考虑

$$
\left| {\alpha \rangle \rightarrow \Theta }\right| \alpha \rangle , tag{4. 2 2}
$$

其中, $\Theta \mid \alpha \rangle$ 是时间反演态. 更恰当地, $\Theta \mid \alpha \rangle$ 应当称为运动反演态. 如果 $|\alpha \rangle$ 是一个动量本征态 $\left| {\mathbf{p}}^{\prime }\right\rangle$ ,则预期 $\Theta \mid \alpha \rangle$ 是 $\left| {-{\mathbf{p}}^{\prime }}\right\rangle$ ,至多差一个可能的相位. 同样地,在时间反演下 $\mathbf{J}$ 也被反转.



图 4.12 在 $t = 0$ 和 $t = \pm {\delta t}$ 时刻,时间反演之前与之后的动量

现在, 通过观察时间反演态的时间演化, 推导时间反演算符的基本性质. 考虑一个物理系统,在 $t = 0$ 时,由右矢 $|\alpha \rangle$ 表示. 那么,在稍后时刻, $t = {\delta t}$ ,发现系统处在

$$
\left| {\alpha ,{t}_{0} = 0;t = {\delta t}}\right\rangle = \left( {1 - \frac{iH}{\hslash }{\delta t}}\right) |\alpha \rangle , tag{4. 4.23}
$$

其中, $H$ 是表征时间演化的哈密顿量. 代替上面的这个方程,假定 $t = 0$ 时,先作用 $\Theta$ . 而然后让系统在哈密顿量 $H$ 的影响下演化. 那么,在 ${\delta t}$ 时,有

$$
\left( {1 - \frac{iH\delta t}{\hslash }}\right) \Theta |\alpha \rangle
$$

(4. ${4.24a}$

如果运动遵从时间反演下的对称性, 预期上述的态右矢与下式相同

$$
\Theta \left| {\alpha ,{t}_{0} = 0;t = - {\delta t}}\right\rangle . tag{4. 4.24b}
$$

这就是说,先考虑在早些时刻 $t = - {\delta t}$ 的一个态右矢,然后反转 $\mathbf{p}$ 和 $\mathbf{J}$ ,见图 4.12. 数学上有

$$
\left( {1 - \frac{iH}{\hslash }{\delta t}}\right) \Theta \left| {\alpha \rangle = \Theta \left( {1 - \frac{iH}{\hslash }\left( {-{\delta t}}\right) }\right) }\right| \alpha \rangle . tag{4. 2.5}
$$

如果上述关系对任何右矢都对, 必须有

$$
- {iH\Theta }\left| {\rangle = {\Theta iH}}\right| \rangle , tag{4. 4.26}
$$

其中, 空的 $|\rangle$ 强调, (4.2.26) 式对任何右矢都成立.

现在论证. 如果时间反演的运动有意义, $\Theta$ 不可能是幺正的. 假定 $\Theta$ 真是幺正的,那

么消掉 (4.4.26) 式中的 $i$ 就会是合理的了,于是就会有算符方程:

$$
- {H\Theta } = {\Theta H}. tag{4. 4.27}
$$

考虑一个能量本征态 $|n\rangle$ ,其能量本征值为 ${E}_{n}$ . 相应的时间反演态就会是 $\Theta |n\rangle$ ,并且由于 (4.4.27) 式, 就会有

$$
{H\Theta }\left| {n\rangle = - {\Theta H}}\right| n\rangle = \left( {-{E}_{n}}\right) \Theta |n\rangle . tag{4. 4.28}
$$

这个方程表明, $\Theta \mid n\rangle$ 是哈密顿量的一个本征右矢,能量本征值为 $- {E}_{n}$ ,但是,甚至在自由粒子这种非常基本的情况下, 这也是荒谬的. 众所周知, 自由粒子的能量谱是半正定的 ——从 0 到 $+ \infty$ . 不存在任何比静止粒子 (动量本征值为零的动量本征态) 还要低的态; 能谱从 $- \infty$ 到 0 的范围是完全不可接受的. 通过观察自由粒子哈密顿量的结构也可以看到这一点. 预期 $\mathbf{p}$ 改变符号,但 ${\mathbf{p}}^{2}$ 不会; 然而 (4.4.27) 式会意味着

$$
{\Theta }^{-1}\frac{{\mathbf{p}}^{2}}{2m}\Theta = \frac{-{\mathbf{p}}^{2}}{2m}. tag{4. 4.29}
$$

所有这些论证都强烈地暗示, 如果时间反演果真是一个有用的对称性, 就不会允许消去 (4.4.26) 式中的 $i$ ; 因此, $\Theta$ 更为恰当的是反幺正的. 在这种情况下,根据反线性 (4.4.13b) 式, (4.4.26) 式的右边变成

$$
{\Theta iH}\left| {\rangle = - {i\Theta H}}\right| \rangle tag{4. 4.30}
$$

现在,终于可以消去 (4.4.26) 式中的 $i$ 了. 通过 (4.4.30) 式,这最终导致

$$
{\Theta H} = {H\Theta }. tag{4. 4.31}
$$

方程 (4.4.31) 式表示了哈密顿量在时间反演下的基本性质. 用了这个方程, 早些时候提到的困难 [见 (4.4.27) 式到 (4.4.29) 式] 不再存在, 因而得到了物理上合理的结果. 从现在起, $\Theta$ 总是取为反幺正的.

早些时候曾提到过, 最好避免一个反幺正算符从右边作用于左矢上. 不过, 可以用

$$
\langle \beta \left| \Theta \right| \alpha \rangle \text{.} tag{4. 42}
$$

它总是理解为

$$
\left( {\langle \beta \mid }\right) \cdot \left( {\Theta \mid \alpha \rangle }\right) tag{4. 3 3}
$$

而绝不是

$$
\left( {\langle \beta \left| {\Theta \rangle \cdot }\right| \alpha \rangle }\right) \text{.} tag{4. 4.34}
$$

事实上不打算定义 $\langle \beta \mid \Theta$ . 这是一处狄拉克的左矢-右矢符号有点混淆的地方. 毕竟,这个符号被发明用来处理线性算符, 而不是反线性算符.

伴随这种警示评注, 有必要讨论算符在时间反演下的行为. 继续采用这样的观点, 即 $\Theta$ 算符作用于右矢上

$$
\left| {\widetilde{\alpha }\rangle = \Theta }\right| \alpha \rangle ,\;\left| {\widetilde{\beta }\rangle = \Theta }\right| \beta \rangle , tag{4. 4.35}
$$

谈论算符一特别是可观测量一在时间反演下是奇的或偶的, 通常也是很方便的. 从一个重要的恒等式开始:

$$
\langle \beta \left| \otimes \right| \alpha \rangle = \left\langle {\alpha \left| {\Theta { \otimes }^{ \dagger }{\Theta }^{-1}}\right| \bar{\beta }}\right\rangle , tag{4. 4.36}
$$

其中 $\otimes$ 是个线性算符. 这个恒等式单由 $\Theta$ 的反幺正即可得到. 要证明这一点,定义

$$
\left| {\gamma \rangle \equiv { \otimes }^{ \dagger }}\right| \beta \rangle tag{4. 4.37}
$$

通过对偶对应, 有

$$
\left| {\gamma \rangle \leftrightarrow \langle \beta }\right| \otimes = \langle \gamma | tag{4. 3.38}
$$

因此,

$$
\langle \beta \left| \otimes \right| \alpha \rangle = \langle \gamma \mid \alpha \rangle = \langle \widetilde{\alpha } \mid \widetilde{\gamma }\rangle
$$

$$
= \left\langle {\alpha \left| {\Theta { \otimes }^{ \dagger }}\right| \beta }\right\rangle = \left\langle {\alpha \left| {\Theta { \otimes }^{ \dagger }{\Theta }^{-1}\Theta }\right| \beta }\right\rangle tag{4. 4.39}
$$

$$
= \left\langle {\alpha \left| {\Theta { \otimes }^{ \dagger }{\Theta }^{-1}}\right| \widetilde{\beta }}\right\rangle ,
$$

它证明了这个恒等式. 特别是,对于厄米可观测量 $A$ ,得到

$$
\langle \beta \left| A\right| \alpha \rangle = \left\langle {\alpha \left| {{\Theta A}{\Theta }^{-1}}\right| \widetilde{\beta }}\right\rangle . tag{4.4.40}
$$

可观测量在时间反演下是偶的还是奇的, 按照在下式有上面的符号还是下面的符号:

$$
{\Theta A}{\Theta }^{-1} = \pm A. tag{4.4.41}
$$

注意,这个方程,与 (4.4.40) 式一起,对于 $A$ 对时间反演态取的矩阵元给了一个相位限制如下:

$$
\langle \beta \left| A\right| \alpha \rangle = \pm \langle \widetilde{\beta }\left| A\right| \widetilde{\alpha }{\rangle }^{ * }, tag{4.4.42}
$$

如果 $|\beta \rangle$ 恒等于 $|\alpha \rangle$ ,正在谈的就是关于期待值的,有

$$
\langle \alpha \left| A\right| \alpha \rangle = \pm \langle \widetilde{\alpha }\left| A\right| \widetilde{\alpha }\rangle . tag{4. 4.43}
$$

其中 $\langle \bar{\alpha }\left| A\right| \bar{\alpha }\rangle$ 是对时间反演态取的期待值.

作为一个例子,看一看 $\mathbf{p}$ 的期待值. 假定 $\mathbf{p}$ 对于时间反演态取的期待值将有相反的符号是合理的. 这样

$$
\langle \alpha \left| \mathbf{p}\right| \alpha \rangle = - \langle \bar{\alpha }\left| \mathbf{p}\right| \bar{\alpha }\rangle . tag{4.4.44}
$$

因此,取 $\mathbf{p}$ 是个奇算符,即

$$
\Theta \mathbf{p}{\Theta }^{-1} = - \mathbf{p}. tag{4.4.45}
$$

这意味着

$$
\mathbf{p}\Theta \left| {\mathbf{p}}^{\prime }\right\rangle = - \Theta \mathbf{p}{\Theta }^{-1}\Theta \left| {\mathbf{p}}^{\prime }\right\rangle tag{4.4.46}
$$

$$
= \left( {-{\mathbf{p}}^{\prime }}\right) \Theta \left| {\mathbf{p}}^{\prime }\right\rangle \text{.}
$$

方程 (4.4.46) 式符合早些时候的断言: $\Theta \mid {\mathbf{p}}^{\prime }$ ) 是一个动量的本征态,本征值为 $- {\mathbf{p}}^{\prime }$ . 在适当选择的相位下,它可以认为就是 $\left| {-{\mathbf{p}}^{\prime }}\right\rangle$ 本身. 同样地,得到

$$
\Theta \mathbf{x}{\Theta }^{-1} = \mathbf{x} tag{4.4.47}
$$

$$
\Theta \left| {\mathbf{x}}^{\prime }\right\rangle = \left| {\mathbf{x}}^{\prime }\right\rangle \;\text{(至多差一个相因子)}
$$

它来自下列 (绝对合理的) 要求:

$$
\langle \alpha \left| \mathbf{x}\right| \alpha \rangle = \langle \widetilde{\alpha }\left| \mathbf{x}\right| \widetilde{\alpha }\rangle . tag{4. 4.48}
$$

现在来核对基本对易关系

$$
\left. {\left. \left\lbrack {{x}_{i},{p}_{j}}\right\rbrack \right\rangle = i\hslash {\delta }_{ij} \mid }\right\rangle tag{4.4.49}
$$

的不变性,其中的空右矢 $|\rangle$ 代表任何右矢. 把 $\Theta$ 作用于 (4.4.49) 式的两边,有

$$
\Theta \left\lbrack {{x}_{i},{p}_{j}}\right\rbrack {\Theta }^{-1}\Theta \left| {\rangle = {\Theta i}\hslash {\delta }_{ij}}\right| \rangle , tag{4. 4.50}
$$

在 $\Theta$ 跨过 $i\hslash$ 之后,该式导致

$$
\left. \left. {\left. {\left. \left. {\left. \left. {\left. \left. {x}_{i}\right\rbrack \right\rbrack ,\left( {-{p}_{j}}\right) - {p}_{j}\Theta }\right\rbrack \right\rbrack = - i\hslash {\delta }_{ij}\Theta }\right\rbrack \right\rbrack .}\right\rbrack .}\right\rangle \right\rangle tag{4. 4.51}
$$

注意,基本对易关系 $\left\lbrack {{x}_{i},{p}_{j}}\right\rbrack = i\hslash {\delta }_{ij}$ 凭借 $\Theta$ 是反幺正的而得以保持不变. 这也可以作为取 $\Theta$ 为反幺正的另一个理由; 否则的话,不得不放弃或者 (4.4.45) 式、或者 (4.4.47) 式! 类似地, 要保持

$$
\left\lbrack {{J}_{i},{J}_{j}}\right\rbrack = i\hslash {\varepsilon }_{ijk}{J}_{k}, tag{4. 4.52}
$$

角动量算符在时间反演下必须是奇的, 即

$$
\Theta \mathbf{J}{\Theta }^{-1} = - \mathbf{J}. tag{4. 4.53}
$$

对于一个无自旋系统该式是事实自洽的,因为 $\mathbf{J}$ 恰为 $\mathbf{x} \times \mathbf{p}$ . 作为一种选择,通过注意到转动算符和时间反演算符对易(注意额外的 $i!$ )也能够推导出这个关系.

\subsection{波函数}

假定在某一给定时刻,比如在 $t = 0$ 时,一个无自旋单粒子系统被发现处在由 $|\alpha \rangle$ 表示的一个态中. 它的波函数 $\left\langle {{\mathbf{x}}^{\prime } \mid \alpha }\right\rangle$ 表现为位置表象中的展开系数

$$
|\alpha \rangle = \int {d}^{3}{x}^{\prime }\left| {\mathbf{x}}^{\prime }\right\rangle \left\langle {{\mathbf{x}}^{\prime } \mid \alpha }\right\rangle . tag{4. 4.54}
$$

作用上时间反演算符, 得到

$$
\Theta \left| {\alpha \rangle = \int {d}^{3}{x}^{\prime }\Theta \left| {\mathbf{x}}^{\prime }\right\rangle \left\langle {{\mathbf{x}}^{\prime } \mid \alpha }\right\rangle .}\right| tag{4. 4.55}
$$

$$
= \int {d}^{3}{x}^{\prime }\left| {\mathbf{x}}^{\prime }\right\rangle \left\langle {{\mathbf{x}}^{\prime } \mid \alpha }\right\rangle {}^{ * },
$$

其中选择相位规则,以使 $\Theta \left| {\mathbf{x}}^{\prime }\right\rangle$ 就是 $\left| {\mathbf{x}}^{\prime }\right\rangle$ 自己. 那么重新获得如下规则

$$
\psi \left( {\mathbf{x}}^{\prime }\right) \rightarrow {\psi }^{ * }\left( {\mathbf{x}}^{\prime }\right) tag{4. 4.56}
$$

它在早些时候通过观察薛定谔波动方程 [见 (4.4.5) 式] 而推论得到. 波函数的角度部分由球谐函数 ${Y}_{l}^{m}$ 给出. 在通常的相位约定下,有

$$
{Y}_{l}^{m}\left( {\theta ,\phi }\right) \rightarrow {Y}_{l}^{m * }\left( {\theta ,\phi }\right) = {\left( -1\right) }^{m}{Y}_{l}^{-m}\left( {\theta ,\phi }\right) . tag{4. 4.57}
$$

既然, ${Y}_{l}^{m}\left( {\theta ,\phi }\right)$ 是 $|l, m\rangle$ 的波函数 [见 (3.6.23) 式]; 因此,从 (4.4.56) 式,导出

$$
\Theta \left| {l, m\rangle = {\left( -1\right) }^{m}}\right| l, - m\rangle . tag{4. 4.58}
$$

如果对于一个 (3.6.22) 式类型的、样子像 $R\left( r\right) {Y}_{l}^{\prime \prime }$ 的波函数,研究概率流密度 (2.4.16) 式,将得到结论,对于 $m > 0$ ,从 $z$ 轴正方向看,流密度沿逆时针方向流动. 而对于相应的时间反演态的波函数,其概率流密度沿相反方向流动,因为 $m$ 的符号反过来了. 所有的这些都是非常合理的.

作为时间反演不变性的一个非平凡结果, 我们陈述一个关于无自旋粒子能量本征函数的实性的重要定理.

定理 4.2 假定哈密顿量在时间反演下不变,而且能量本征右矢 $|n\rangle$ 是非简并的; 则相应的能量本征函数是实的 (或者,更一般地,一个实函数乘上一个不依赖于 $\mathbf{x}$ 的相因子.)

证明 要证明这个定理, 首先注意到

$$
{H\Theta }\left| {n\rangle = {\Theta H}}\right| n\rangle = {E}_{n}\Theta |n\rangle . tag{4. 4.59}
$$

所以 $|n\rangle$ 和 $\Theta \mid n\rangle$ 有相同的能量. 非简并假设促使我们得到结论: $|n\rangle$ 和 $\Theta \mid n\rangle$ 一定代表同一个态; 否则的话,就会有两个不同的态有同样的能量 ${E}_{n}$ ,显然矛盾! 回忆一下,对于 $|n\rangle$ 和 $\Theta \mid n\rangle$ 的波函数,分别为 $\left\langle {{\mathbf{x}}^{\prime } \mid n}\right\rangle$ 和 $\left\langle {{\mathbf{x}}^{\prime } \mid n}\right\rangle {}^{ * }$ . 它们一定是相同的一一即实际上,

$$
\left\langle {{\mathbf{x}}^{\prime } \mid n}\right\rangle = {\left\langle {\mathbf{x}}^{\prime } \mid n\right\rangle }^{ * } tag{4. 4.60}
$$

——或者更精确地说,它们至多可以差一个与 $\mathbf{x}$ 无关的相因子.

这样,如果有一个非简并的束缚态,它的波函数总是实的. 另一方面,对于 $l \neq 0, m \neq 0$ 的氢原子,用确定的量子数 $\left( {n, l, m}\right)$ 表征的能量本征函数是复的,因为 ${Y}_{l}^{m}$ 是复的; 这与该定理并不矛盾,因为 $|n, l, m\rangle$ 与 $|n, l, - m\rangle$ 是简并的. 类似地,平面波的波函数 ${e}^{i\mathbf{p} \cdot \mathbf{x}/\hslash }$ 是复的,但是它是和 ${e}^{-i\mathbf{p} \cdot \mathbf{x}/\hslash }$ 简并的.

可以看到,对于一个无自旋系统,时间反演态的波函数,比如说在 $t = 0$ 时刻,通过求复共轭就可以直接得到. 按照写成 (4.4.16) 式或 (4.4.54) 式形式的右矢 $|\alpha \rangle$ ,算符 $\Theta$ 就是复共轭算符 $\mathrm{K}$ 本身,因为 $\mathrm{K}$ 和 $\Theta$ 当作用于基右矢 $\left| {\alpha }^{\prime }\right\rangle$ (或 $\left| {\mathbf{x}}^{\prime }\right\rangle$ ) 时,有同样的效果. 然而,注意到,当右矢 $|\alpha \rangle$ 用动量本征右矢展开时,情况完全不同,因为 $\Theta$ 一定会如下把 $\left| {\mathbf{p}}^{\prime }\right\rangle$ 变成 $\left| {-{\mathbf{p}}^{\prime }}\right\rangle$ :

$$
\Theta \left| {\alpha \rangle = \int {d}^{3}{p}^{\prime }}\right| - {\mathbf{p}}^{\prime }\rangle \left\langle {{\mathbf{p}}^{\prime } \mid \alpha }\right\rangle \cdot = \int {d}^{3}{p}^{\prime }\left| {\mathbf{p}}^{\prime }\right\rangle \left\langle {-{\mathbf{p}}^{\prime } \mid \alpha }\right\rangle \cdot . tag{4. 4.61}
$$

显然, 时间反演态的动量空间波函数不只是原来动量空间波函数的复共轭; 更恰当地是, 必须把 ${\phi }^{\prime }\left( {-{\mathbf{p}}^{\prime }}\right)$ 认同为时间反演态的动量空间波函数. 这种情况再一次说明关键之点在于, $\Theta$ 的特殊形式依赖于所用的特殊表象.

\subsection{对于自旋 $\frac{1}{2}$ 系统的时间反演}

对于一具有自旋一一特别是自旋 $\frac{1}{2}$ ——的粒子情况甚至更有趣. 由 3.2 节回忆起, $\mathbf{S}$ - $\widehat{\mathbf{n}}$ 的本征值为 $\hslash /2$ 的本征右矢可以写成

$$
\left| {\widehat{\mathbf{n}}; + \rangle = {e}^{-i{S}_{z}\alpha /\hslash }{e}^{-i{S}_{y}\beta /\hslash }}\right| + \rangle , tag{4. 4.62}
$$

其中 $\widehat{\mathbf{n}}$ 由分别为 $\beta$ 和 $\alpha$ 的极角和方位角表征. 注意 (4.4.53) 式,有

$$
\Theta \left| {\widehat{\mathbf{n}}; + \rangle = {e}^{-i{S}_{z}\alpha /\hslash }{e}^{-i{S}_{y}\beta /\hslash }\Theta }\right| + \rangle = \eta |\widehat{\mathbf{n}}; - \rangle . tag{4. 4.63}
$$

另一方面, 可以很容易地证明

$$
\left| {\widehat{\mathbf{n}}; - \rangle = {e}^{-{ia}{S}_{z}/h}{e}^{-i\left( {\pi + \beta }\right) {S}_{y}/h}}\right| + \rangle . tag{4. 4.64}
$$

一般地,早些时候曾注意到,乘积 ${UK}$ 是一个反幺正算符. 把 (4.4.63) 式与 (4.4.64) 式相比较,并设 $\Theta$ 等于 ${UK}$ ,而且注意到 $K$ 作用于基右矢 $| + \rangle$ 仍只给出 $| + \rangle$ ,得到

$$
\Theta = \eta {e}^{-{i\pi }{S}_{y}/\hslash }K = - {i\eta }\left( \frac{2{S}_{y}}{\hslash }\right) K tag{4. 4.65}
$$

其中 $\eta$ 代表一个任意的相位 (一个模为 1 的复数). 另一种使我们相信 (4.4.65) 式的方法是证明,如果 $\chi \left( {\widehat{\mathbf{n}}; + }\right)$ 是相应于 $|\widehat{\mathbf{n}}; + \rangle$ 的二分量本征旋量 [在 $\sigma \cdot \widehat{\mathbf{n}}\chi \left( {\widehat{\mathbf{n}}; + }\right) = \chi \left( {\widehat{\mathbf{n}}; + }\right)$ 的意义上], 则

$$
- i{\sigma }_{y}\chi \left( {\widehat{\mathbf{n}}; + }\right) tag{4. 4.66}
$$

(注意复共轭!) 是相应于 $|\widehat{\mathbf{n}}; - \rangle$ 的本征旋量,再一次最多差一个任意的相位,见本章的习题 4.7. ${S}_{y}$ 或 ${\sigma }_{y}$ 的出现可以追溯到这样的一件事实,即正在使用的表象是其中 ${S}_{z}$ 为对角, 而 ${S}_{y}$ 的非零矩阵元都是纯虚的.

现在注意

$$
{e}^{-{i\pi }{S}_{y}/h}\left| {+\rangle = + }\right| - \rangle ,\;{e}^{-{i\pi }{S}_{y}/h}\left| {-\rangle = - }\right| + \rangle . tag{4. 4.67}
$$

利用 (4.4.67) 式,能够求出形如 (4.4.65) 式的 $\Theta$ 作用在最普遍的自旋 $\frac{1}{2}$ 右矢的效果:

$$
\Theta \left( {{c}_{ + }\left| {+\rangle + {c}_{ - }}\right| - \rangle }\right) = + \eta {c}_{ + }^{ * }\left| {-\rangle - \eta {c}_{ - }^{ * }}\right| + \rangle . tag{4. 4.68}
$$

再用 $\Theta$ 作用一次:

$$
{\Theta }^{2}\left( {{c}_{ + }\left| {+\rangle + {c}_{ - }}\right| - \rangle }\right) = - {\left| \eta \right| }^{2}{c}_{ + }\left| {+\rangle - }\right| \eta \left| {{}^{2}{c}_{ - }}\right| - \rangle tag{4. 4.69}
$$

$$
= - \left( {{c}_{ + }\left| {+\rangle + {c}_{ - }}\right| - \rangle }\right)
$$

或

$$
{\Theta }^{2} = - 1 tag{4. 4. 70}
$$

(其中 -1 被理解为 -1 乘上恒等算符) 对任何自旋取向都成立. 这是一个非同寻常的结果. 在这里,至关重要的是要注意结论完全与位相的选择无关; 不管 $\eta$ 取什么样的相位约定, (4.4.70) 式都成立. 与此相反, 我们可能注意到了, 对于一个无自旋的态连续用 $\Theta$ 作用两次给出

$$
{\Theta }^{2} = + 1, tag{4. 4.71}
$$

这一点, 从 (4.4.58) 式是显然的.

更一般地, 现在证明

$$
{\Theta }^{2}\left| {j\text{ 半奇数 }\rangle = - }\right| j\text{ 半奇数 }\rangle
$$

(4. ${42}\mathrm{a}$ )

${\Theta }^{2} \mid j$ 整数 $\rangle = + \mid j$ 整数 $\rangle$ .(4. 4.72b)

于是, ${\Theta }^{2}$ 的本征值由 ${\left( -1\right) }^{2j}$ 给出. 首先注意到,(4.4.65) 式对于任意自旋推广为

$$
\Theta = \eta {e}^{-{i\pi }{J}_{y}/h}K. tag{4. 4.73}
$$

对于用基右矢 $|j, m\rangle$ 展开的一个右矢 $|\alpha \rangle$ ,有

$$
\Theta \left( {\Theta \sum \left| {{jm}\rangle \left\langle {{jm} \mid \alpha }\right\rangle }\right| = \Theta \left( {\eta \sum {e}^{-{i\pi }{J}_{y}/h} \mid {jm}\rangle \langle {jm} \mid \alpha {\rangle }^{ * }}\right) }\right) tag{4. 4.74}
$$

$$
= {\left| \eta \right| }^{2}{e}^{-{2i\pi }{J}_{y}/h}\sum |{jm}\rangle \langle {jm} \mid \alpha \rangle .
$$

但是

$$
{e}^{-{2n\pi }{J}_{{y}^{\prime h}}}\left| {{jm}\rangle = {\left( -1\right) }^{2j}}\right| {jm}\rangle , tag{4. 4.75}
$$

从角动量本征态在 ${2\pi }$ 转动之下的性质知道,该式是显然的*.

在 (4.4.72b) 式中, $\mid j$ 整数) 可以代表为双电子系统的自旋态

$$
\frac{1}{\sqrt{2}}\left( {\left| {+ - \rangle \pm }\right| - + \rangle }\right) tag{4. 4.76}
$$

或一个无自旋粒子的轨道态 $|l, m\rangle$ . 唯一重要的是, $j$ 是一个整数. 同样地, $\mid j$ 半奇数 $\rangle$ 代表在任何组态下的一个三电子系统. 实际上, 对于一个专门由电子组成的系统而言, 任何有奇(偶)数个电子的系统——不管它们的空间取向(例如,相对的轨道角动量)—— 在 ${\Theta }^{2}$ 之下为奇 (偶); 它们甚至不需要是 ${\mathbf{J}}^{2}$ 的本征态!

插入一些关于相位约定的评注. 在前面基于位置表象的讨论中, 我们看到, 在对于球谐函数通常的约定的情况下,自然的做法是选择对于 $|l, m\rangle$ 在时间反演下的任意相位,使得

$$
\Theta \left| {l, m\rangle = {\left( -1\right) }^{m}}\right| l, - m\rangle . tag{4. 4.77}
$$

有些作者发现, 非常吸引人的是把上式推广为

$$
\Theta \left| {j, m\rangle = {\left( -1\right) }^{m}}\right| j, - m\rangle \;\left( {j\text{ 为一个整数 }}\right) , tag{4. 4.78}
$$

而不管 $j$ 指的是 $l$ 还是 $s$ (对于一个整数自旋系统). 自然要问,当 $|j, m\rangle$ 被视为由 “原始的” 自旋 $\frac{1}{2}$ 客体按照维格纳和施温格做法构成时,对于自旋 $\frac{1}{2}$ 系统该式与 (4.4.72a) 式相容吗? 易见,只要选择 (4.4.73) 式中的 $\eta$ 是 $+ i$ ,则 (4.4.72a) 式的确是自洽的. 事实上,一般而言,对于任何的 $j$ ——或为一个半奇数 $j$ ,或为一个整数 $j$ ,均可以取

$$
\Theta \left| {j, m\rangle = {i}^{2m}}\right| j, - m\rangle tag{4. 4.79}
$$

见本章习题 4.10. 然而, 应当提醒读者, 这并不是文献中能见到的唯一的约定. 例如, Henley 和 Garcia (2007). 对于某些物理应用,更方便的是利用其他选择; 例如,使 ${\mathbf{J}}_{ \pm }$ 算符的矩阵元简单的相位约定, 而不是使时间反演算符性质简单的相位约定. 再一次强调,

* 这一说法过于草率. 对于所有的 $j$ ,因子 ${\left( -1\right) }^{2j}$ 并非显然,然而,可以从定义式 (3.5.51) 证明其等于 ${d}_{mm}^{\left( j\right) }\left( {2\pi }\right)$ . 然后用 (3.9.33) 式求出其值. (本脚注按勘误表要求译出. 一译者注) (4.4.72) 式完全不依赖相位约定.

解决了角动量本征态在时间反演下的行为后, 就能再一次研究一个厄米算符的期待值. 回忆一下 (4.4.43) 式,在时间反演下 (消去 ${i}^{2m}$ 因子) 得到

$$
\langle \alpha, j, m\left| A\right| \alpha, j, m\rangle = \pm \langle \alpha, j, - m\left| A\right| \alpha, j, - m\rangle . tag{4. 4.80}
$$

现在,假定 $A$ 是球张量 ${T}_{q}^{\left( k\right) }$ 的一个分量. 由于有维格纳-埃卡特定理,只要考察 $q = 0$ 分量的矩阵元就足够了. 一般而言,称 ${T}^{\left( k\right) }$ (假定是厄米的) 为时间反演下为偶还是奇的,依赖于它的 $q = 0$ 分量如何满足下式的靠上边还是靠下边的符号:

$$
\Theta {T}_{q = 0}^{\left( k\right) }{\Theta }^{-1} = \pm {T}_{q = 0}^{\left( k\right) }. tag{4. 4.81}
$$

对于 $A = {T}_{0}^{\left( k\right) }$ ,方程 (4.4.80) 式变成

$$
\left\langle {\alpha, j, m\left| {T}_{0}^{\left( k\right) }\right| \alpha, j, m}\right\rangle = \pm \left\langle {\alpha, j, - m\left| {T}_{0}^{\left( k\right) }\right| \alpha, j, - m}\right\rangle . tag{4. 4.82}
$$

依靠 $\left( {3.6.46}\right) \sim \left( {3.6.49}\right)$ 式,预期 $\left| {\alpha, j, - m\rangle = D\left( {0,\pi ,0}\right) }\right| \alpha, j, m\rangle$ ,至多差一个位相. 其次,对于 ${T}_{0}^{\left( k\right) }$ 利用 (3.11.22) 式,导致

$$
{\mathcal{D}}^{ + }\left( {0,\pi ,0}\right) {T}_{0}^{\left( k\right) }\mathcal{D}\left( {0,\pi ,0}\right) = {\left( -1\right) }^{k}{T}_{0}^{\left( k\right) } + \left( {q \neq 0\text{ 分量 }}\right) . tag{4. 4.83}
$$

其中,用到了 ${\mathcal{D}}_{00}^{\left( k\right) }\left( {0,\pi ,0}\right) = {P}_{k}\left( {\cos \pi }\right) = {\left( -1\right) }^{k}$ ,而且当夹在 $\langle \alpha, j, m \mid$ 和 $\mid \alpha, j, m\rangle$ 之间时, $q \neq 0$ 分量贡献为零. 净结果为

$$
\left\langle {\alpha, j, m\left| {T}_{0}^{\left( k\right) }\right| \alpha, j, m}\right\rangle = \pm {\left( -1\right) }^{k}\left\langle {\alpha, j, m\left| {T}_{0}^{\left( k\right) }\right| \alpha, j, m}\right\rangle . tag{4. 4.84}
$$

作为一个例子,当取 $k = 1$ 时,对于 $j, m$ 本征态求得的期待值 $\langle \mathbf{x}\rangle$ 为零. 可以论证,如果对于宇称本征态求期待值,则从宇称反演已经知道 $\langle \mathbf{x}\rangle = 0$ [见 (4.2.41) 式]. 但是, 这里要注意 $\left| {\alpha, j, m}\right\rangle$ 不需要是宇称本征态! 例如,对于自旋 $\frac{1}{2}$ 粒子, $|j, m\rangle$ 可能是 ${c}_{s}$ $\left. {s}_{1/2}\right\rangle + {c}_{p}\left| {p}_{1/2}\right\rangle$ .

\subsection{与电场和磁场的相互作用及克拉默斯简并}

考虑在一个外部电场或磁场中的带电粒子. 如果只有一个静电场与该电荷相互作用, 则哈密顿量的相互作用部分仅为

$$
V\left( \mathbf{x}\right) = {e\phi }\left( \mathbf{x}\right) , tag{4. 4.85}
$$

其中 $\phi \left( \mathbf{x}\right)$ 是静电势. 因为 $\phi \left( \mathbf{x}\right)$ 是时间反演偶算符 $\mathbf{x}$ 的一个实函数,有

$$
\left\lbrack {\Theta, H}\right\rbrack = 0. tag{4. 4.86}
$$

和宇称的情况不同, (4.4.86) 式并不导致有趣的守恒定律. 理由是即使上式成立,

$$
{\Theta U}\left( {t,{t}_{0}}\right) \neq U\left( {t,{t}_{0}}\right) \Theta , tag{4. 4.87}
$$

因此, 在紧接着 4.1 节的 (4.1.9) 式的讨论不再有效. 结果, 不存在像 “时间反演量子数守恒” 这样的东西. 然而, 正如已经提到过的, (4.4.86) 式的确导致一种非平庸的相位限制: 对于一个无自旋系统非简并波函数的实数性 [见 (4.4.59) 式和 (4.4.60) 式].

时间反演不变性的另一个影响深远的结果是克拉默斯简并. 假定 $H$ 和 $\Theta$ 对易,且设

$\left| {n\rangle \text{和}\Theta }\right| n\rangle$ 分别为能量本征态和它的时间反演态. 由 (4.4.86) 式显然可见, $|n\rangle$ 和 $\Theta$

$|n\rangle$ 属于同样的能量本征值 ${E}_{n}\left( {{H\Theta }\left| {n\rangle = {\Theta H}}\right| n\rangle = {E}_{n}\Theta |n\rangle }\right)$ . 问题是, $|n\rangle$ 和 $\Theta |n\rangle$ 代表同一个态吗? 如果是的话, $\left| {n\rangle \text{和}\Theta }\right| n\rangle$ 至多可以靠一个相因子来区别. 因此,

$$
\Theta \mid n\rangle = {e}^{i\delta } \mid n\rangle , tag{4. 8.88}
$$

再一次用 $\Theta$ 作用于 (4.4.88) 式,有 ${\Theta }^{2}\left| {n\rangle = \Theta {e}^{i\delta }}\right| n\rangle = {e}^{-{i\delta }}\Theta \left| {n\rangle = {e}^{-{i\delta }}{e}^{+{i\delta }}}\right| n\rangle$ ; 因此,

$$
{\Theta }^{2}\left| {n\rangle = + }\right| n\rangle tag{4. 4.89}
$$

但是,对于半奇数 $j$ 的系统,这个关系式是不可能的,对于它, ${\Theta }^{2}$ 总是 -1,因此它导致: $\left| {n\rangle \text{和}\Theta }\right| n\rangle$ ,尽管有相同的能量,一定对应于不同的态——这就是说,一定是简并的态. 这意味着,例如,对于在一个外电场 $\mathbf{E}$ 中由奇数个电子组成的系统,无论 $\mathbf{E}$ 可能会如何复杂, 每一个能级至少必须是二重简并的. 沿着这样的线索考虑, 对于晶体中的电子具有有趣的应用, 在那里, 奇数电子和偶数电子系统显示出非常不同的行为. 历史上, 克拉默斯通过查看薛定谔方程的解, 推论出这类简并; 随后, 维格纳指出克拉默斯简并是时间反演不变性的结果.

现在转向与外磁场的相互作用. 那么,哈密顿量 $H$ 包含有如下的这类项:

$$
\mathbf{S} \cdot \mathbf{B},\;\mathbf{p} \cdot \mathbf{A} + \mathbf{A} \cdot \mathbf{p},\;\left( {\mathbf{B} = \nabla \times \mathbf{A}}\right) , tag{4. 4.90}
$$

其中磁场被看作是外部的. 算符 $\mathrm{S}$ 和 $\mathrm{p}$ 在时间反演下都是奇的,这些相互作用项会导致

$$
{\Theta H} \neq {H\Theta }\text{.} tag{4. 4.91}
$$

作为一个平庸的例子,对于一个自旋 $\frac{1}{2}$ 系统,自旋向上的态 $| + \rangle$ 和它的时间反演态 $| - \rangle$ , 在存在一个外磁场的情况下, 不再有相同的能量. 一般而言, 在一个包含奇数个电子的系统中通过作用上一个外磁场, 克拉默斯简并能够被解除.

注意,当把 $\mathbf{B}$ 当作外部的处理时,在时间反演下不改变 $\mathbf{B}$ ; 这是因为原子中的电子被看作是一个我们作用上了时间反演算符的封闭量子力学系统. 这不应与早些时候涉及的麦克斯韦方程 (4.4.2) 式以及洛伦兹力方程在 $t \rightarrow - t$ 以及 (4.4.3) 式之下的不变性所做的一些评注混淆. 在那里时间反演作用于整个世界,例如,甚至作用于导线中产生磁场 $\mathbf{B}$ 的电流!

习题

4.1 (假定等质量的可区分粒子) 计算出下列的系统的三个最低能级以及它们的简并度.

(a) 在一个边长为 $L$ 的 (三维) 盒子中的三个无相互作用的自旋 $\frac{1}{2}$ 粒子.

(b) 在一个边长为 $L$ 的 (三维) 盒子中的四个无相互作用的自旋 $\frac{1}{2}$ 粒子.

4.2 设 ${g}_{d}$ 代表平移算符 (位移矢量为 $\mathbf{d}$ ); 设 $\mathcal{D}\left( {\widehat{\mathbf{n}},\phi }\right)$ 代表转动算符 ( $\widehat{\mathbf{n}}$ 和 $\phi$ 分别为转轴和转角); 而设 $\pi$ 代表宇称算符. 下列的各对算符中, 如果有的话, 哪几对是对易的? 为什么?

(a) ${g}_{d}$ 和 ${g}_{d}$ (d 和 ${d}^{\prime }$ 沿不同方向),

(b) $\mathcal{D}\left( {\widehat{\mathbf{n}},\phi }\right)$ 和 $\mathcal{D}\left( {{\widehat{\mathbf{n}}}^{\prime },{\phi }^{\prime }}\right)$ ( $\widehat{\mathbf{n}}$ 和 ${\widehat{\mathbf{n}}}^{\prime }$ 沿不同方向).

(c) ${g}_{d}$ 和 $\pi$ .

(d) $\mathcal{D}\left( {\widehat{\mathbf{n}},\phi }\right)$ 和 $\pi$ .

4.3 已知一个量子力学态 $\Psi$ 是两个厄米算符 $A$ 和 $B$ 的一个共同本征态,且 $A$ 和 $B$ 反对易:

$$
{AB} + {BA} = 0.
$$

关于 $|\Psi \rangle$ 态上 $A$ 和 $B$ 的本征值能说些什么? 用宇称算符 (可以选择它满足 $\pi = {\pi }^{-1} = {\pi }^{ + }$ ) 和动量算符为例来说明你的观点.

4.4 一个自旋 $\frac{1}{2}$ 的粒子被一个球对称势束缚于一个固定的中心

(a) 写出自旋角度函数 ${\forall }_{l = 0}^{j = 1/2, m = 1/2}$ .

(b) 把 $\left( {\sigma \cdot \mathbf{x}}\right) {\forall }_{l = 0}^{j = 1,2, m = 1,2}$ 用一些另外的 ${\forall }_{l}^{j, m}$ 表示出来.

(c) 证明若考虑到算符 $\left( {\mathbf{S} \cdot \mathbf{x}}\right)$ 在转动下及空间反射 (宇称) 下的变换性质,在 (b) 中你的结果可以理解.

4.5 由于弱 (中性流) 相互作用, 在原子中的电子和原子核之间存在一种宇称破坏的位势如下

$$
V = \lambda \left\lbrack {{\delta }^{\left( 3\right) }\left( \mathbf{x}\right) \mathbf{S} \cdot \mathbf{p} + \mathbf{S} \cdot \mathbf{p}{\delta }^{\left( 3\right) }\left( \mathbf{x}\right) }\right\rbrack ,
$$

其中 $\mathbf{S}$ 和 $\mathbf{p}$ 是电子的自旋和动量算符,且假定原子核位于原点. 结果,通常用 $|n, l, j, m\rangle$ 表征的碱金属原子的基态, 实际上包含有来自其他本征态的非常小的贡献如下:

$$
\left| {n, l, j, m\rangle \rightarrow }\right| n, l, j, m\rangle + \mathop{\sum }\limits_{{{n}^{\prime }{l}^{\prime },{m}^{\prime }}}{C}_{{n}^{\prime },{l}^{\prime },{j}^{\prime },{m}^{\prime }}\left| {{n}^{\prime },{l}^{\prime },{j}^{\prime },{m}^{\prime }}\right\rangle .
$$

单单在考虑对称性的基础上,关于给出非零贡献的 $\left( {{n}^{\prime },{l}^{\prime },{j}^{\prime },{m}^{\prime }}\right)$ 能说些什么? 假定径向波函数和能级都是已知的. 指出怎样可以计算 ${C}_{{n}^{\prime }{j}^{\prime }{m}^{\prime }}$ . 能得到关于 $\left( {{n}^{\prime },{l}^{\prime },{j}^{\prime },{m}^{\prime }}\right)$ 的进一步限制吗?

4.6 考虑一个对称的矩形双阱势:

$$
V = \left\{ \begin{array}{ll} \infty & \text{ 对于 }\left| x\right| > a + b; \\ 0 & \text{ 对于 }a < \left| x\right| < a + b; \\ {V}_{0} > 0 & \text{ 对于 }\left| x\right| < a. \end{array}\right.
$$

假定 ${V}_{0}$ 与低能级能态的量子化能量相比非常高,求在两个低能级的能态之间能级劈裂的近似表示式.

4.7 (a) 设 $\psi \left( {\mathbf{x}, t}\right)$ 是一个无自旋粒子的波函数,相应于一个三维平面波. 证明 ${\psi }^{\prime }\left( {\mathbf{x}, - t}\right)$ 是动量方向反转的平面波波函数.

(b) 设 $\chi \left( \widehat{\mathbf{n}}\right)$ 是 $\mathbf{\sigma } \cdot \widehat{\mathbf{n}}$ 的二分量本征旋量,本征值为 +1 . 利用 $\chi \left( \widehat{\mathbf{n}}\right)$ (借助于表征 $\widehat{\mathbf{n}}$ 的极角和方位角 $\beta$ 和 $\gamma$ ) 的显式形式,证明 $- i{\sigma }_{2}{\chi }^{ * }$ ( $\widehat{\mathbf{n}}$ ) 是自旋方向反转的二分量本征旋量.

4.8 (a) 假定哈密顿量在时间反演下不变, 证明对于一个无自旋非简并系统在任意给定时刻的波函数总可以选择为实的.

(b) 对于 $t = 0$ 时刻的一个平面波,其波函数由一个复函数 ${e}^{i\mathbf{p} \cdot \mathbf{x}/\hslash }$ 给出. 为什么这不破坏时间反演不变性.

4.9 设 $\phi \left( {\mathbf{p}}^{\prime }\right)$ 是 $|\alpha \rangle$ 态的动量空间波函数,即 $\phi \left( {\mathbf{p}}^{\prime }\right) = \left\langle {{\mathbf{p}}^{\prime } \mid \alpha }\right\rangle$ . 对于时间反演态 $\Theta \mid \alpha \rangle$ . 其动量空间波函数是由 $\phi \left( {\mathbf{p}}^{\prime }\right) ,\phi \left( {-{\mathbf{p}}^{\prime }}\right) ,{\phi }^{\prime }\left( {\mathbf{p}}^{\prime }\right)$ ,还是 ${\phi }^{\prime }\left( {-{\mathbf{p}}^{\prime }}\right)$ 给出? 证明你的答案是正确的.

4.10 (译者注: 按勘误表的要求, 该题被重新改写了, 修改后的形式如下所示.)

(a) 用 (4.4.53) 式证明 $\Theta \mid j, m\rangle$ 等于 $|j, - m\rangle$ ,至多差一个包含有因子 ${\left( -1\right) }^{m}$ 的相因子. 这就是说,证明 $\Theta \mid j, m\rangle = {e}^{i\delta }{\left( -1\right) }^{m} \mid j, - m\rangle$ ,其中的 $\delta$ 不依赖于 $m$ .

(b) 利用同样的相位约定求相应于 $\mathcal{D}\left( R\right) \mid j, m\rangle$ 的时间反演态. 先用无穷小形式 $\mathcal{D}\left( {\widehat{\mathbf{n}},{d\phi }}\right)$ 处理, 然后推广到有限转动.

(c) 从这些结果出发证明,不依赖于 $\delta$ ,有

$$
{\mathcal{D}}_{{m}^{\prime }m}^{\left( j\right) } \cdot \left( R\right) = {\left( -1\right) }^{m - {m}^{\prime }}{\mathcal{D}}_{-{m}^{\prime }, - m}^{\left( j\right) }\left( R\right) .
$$

(d) 可以得出如下结论: 可以自由地选取 $\delta = 0$ ,以及 $\Theta \mid j, m\rangle = {\left( -1\right) }^{m}\left| {j, - m\rangle = {i}^{2m}}\right| j, - m\rangle$ .

4.11 假定一个无自旋粒子被一个位势 $V\left( \mathbf{x}\right)$ 束缚在一个固定的中心,该位势是如此之反对称,以至于没有任何能级是简并的. 利用时间反演不变性证明, 对于任何能量本征态有

$$
\langle \mathbf{L}\rangle = 0.
$$

(这称为轨道角动量的 “弱化”, 即 “quenching”) 如果这样的一个非简并态的波函数展开为

$$
\mathop{\sum }\limits_{l}\mathop{\sum }\limits_{m}{F}_{lm}\left( r\right) {Y}_{l}^{m}\left( {\theta ,\phi }\right) .
$$

则对于 ${F}_{lm}\left( r\right)$ 得到什么类型的相位限制呢?

4.12 对于一个自旋为 1 的系统, 其哈密顿量由下式给定:

$$
H = A{S}_{z}^{2} + B\left( {{S}_{x}^{2} - {S}_{y}^{2}}\right) .
$$

准确地求解这个问题, 找到归一化的能量本征态和本征值. (一个这类的自旋相关的哈密顿量实际出现在晶体物理学中. ) 这个哈密顿量是时间反演下不变的吗? 求得的归一化本征态在时间反演下如何变换?



	
	
	
	
	
\ifx\allfiles\undefined
\end{document}
	\else
	\fi
