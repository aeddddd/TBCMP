\ifx\allfiles\undefined

% 如果有这一部分另外的package,在这里加上
% 没有的话不需要

\begin{document}
	\else
	\fi
\chapter{角动量理论}
\begin{introduction}
	\item 角动量对易关系
	\item $\frac12$自旋
	\item SO(3)和SU(2)
	\item 系综
\end{introduction}
这一章关注角动量理论和相关论题的系统处理. 在现代物理中角动量理论的重要性怎么强调也不过分. 在分子、原子及核谱学中, 彻底理解角动量是极为重要的; 角动量的考虑不仅在散射和碰撞问题中, 也在束缚态问题中起着重要的作用. 此外, 角动量概念还有着重要的推广——核物理中的同位旋,粒子物理中的 $\mathrm{{SU}}\left( 3\right) ,\mathrm{{SU}}\left( 2\right) \otimes \mathrm{U}\left( 1\right)$ 等等.
\section{转动与角动量对易关系}
\subsection{有限转动与无穷小转动的对比}

我们回想起在基础物理绕同一个轴的转动是对易的, 而绕不同轴的转动不对易. 例如,绕 $z$ 轴转 ${30}^{ \circ }$ 紧接着绕同一个 $z$ 轴再转 ${60}^{ \circ }$ 显然等价于绕同样的轴先转一个 ${60}^{ \circ }$ 再转一个 ${30}^{ \circ }$ . 然而,让我们考虑一个绕 $z$ 轴的 ${90}^{ \circ }$ 的转动,用 ${R}_{z}\left( {\pi /2}\right)$ 表示,紧接着绕 $x$ 轴转 ${90}^{ \circ }$ ,用 ${R}_{x}\left( {\pi /2}\right)$ 表示; 把它和一个绕 $x$ 轴转 ${90}^{ \circ }$ ,紧跟着绕 $z$ 轴转 ${90}^{ \circ }$ 对比. 最终的结果是不同的, 这一点我们可以从图 3.1 看到.

我们的第一个基本任务是找出一个定量的、绕不同轴的转动不对易的方式. 为此, 我们先回忆一下三维的转动怎样用 $3 \times 3$ 的实正交矩阵来表示. 考虑一个矢量 $\mathbf{V}$ ,它的三个分量为 ${V}_{x},{V}_{y}$ 和 ${V}_{z}$ . 当我们转动时,这三个分量变成某个另外的数组 ${V}_{x}^{\prime },{V}_{y}^{\prime }$ 和 ${V}_{z}^{\prime }$ . 老的分量和新的分量通过一个 $3 \times 3$ 的正交矩阵 $R$ 联系起来:

$$
\left( \begin{array}{l} {V}_{, r}^{\prime } \\ {V}_{, y}^{\prime } \\ {V}_{, z}^{\prime } \end{array}\right) = \left( \begin{array}{l} \\ R \\ \\ \end{array}\right) \left( \begin{array}{l} {V}_{, r} \\ {V}_{, y} \\ {V}_{, z} \end{array}\right) , tag{3.1.1a}
$$

$$
R{R}^{T} = {R}^{T}R = 1, tag{3.1.1b}
$$

其中的上标代表矩阵的转置. 正交矩阵有一个性质:

$$
\sqrt{{V}_{x}^{2} + {V}_{y}^{2} + {V}_{z}^{2}} = \sqrt{{V}_{x}^{\prime 2} + {V}_{y}^{\prime 2} + {V}_{z}^{\prime 2}} tag{3.1.2}
$$

是自动满足的.

为明确起见,我们考虑一个绕 $z$ 轴转 $\phi$ 角的转动. 本书中自始至终沿用的约定是一个转动算符影响物理系统本身,如图 3.1 所示,而坐标轴保持不变. 当所讨论的转动在 ${xy}$ 平面上从正 $z$ 轴看是逆时针的,我们就取 $\phi$ 角为正. 如果把一个右手螺旋与这样一个转动联系起来,一个绕 $z$ 轴的正 $\phi$ 角的转动意味着这个螺旋朝正 $z$ 方向前进. 在这样的约定下, 很容易证明

$$
{R}_{z}\left( \phi \right) = \left( \begin{matrix} \cos \phi & - \sin \phi & 0 \\ \sin \phi & \cos \phi & 0 \\ 0 & 0 & 1 \end{matrix}\right) . tag{3.1.3}
$$

假如我们采用不同的约定,让物理系统保持固定不动,而坐标轴被转动,这个具有正 $\phi$ 角的同一矩阵,就会表示从正 $z$ 方向看的 $x$ 轴和 $y$ 轴的顺时针转动. 显然,重要的是不要把这两种约定弄混! 有一些作者通过使用术语 “主动转动” 表示物理系统转动而 “被动转动” 表示坐标轴转动来区分这两种方法.



图 3.1 演示有限转动不对易的例子

我们特别感兴趣的是 ${R}_{z}$ 的无穷小形式:

$$
{R}_{z}\left( \varepsilon \right) = \left( \begin{matrix} 1 - \frac{{\varepsilon }^{2}}{2} & - \varepsilon & 0 \\ \varepsilon & 1 - \frac{{\varepsilon }^{2}}{2} & 0 \\ 0 & 0 & 1 \end{matrix}\right) , tag{3.1.4}
$$

其中略去了 ${\varepsilon }^{3}$ 以及更高阶的项. 同样我们有

$$
{R}_{x}\left( \varepsilon \right) = \left( \begin{matrix} 1 & 0 & 0 \\ 0 & 1 - \frac{{\varepsilon }^{2}}{2} & - \varepsilon \\ 0 & \varepsilon & 1 - \frac{{\varepsilon }^{2}}{2} \end{matrix}\right)
$$

(3.1. ${5a}$ )

和

$$
{R}_{y}\left( \varepsilon \right) = \left( \begin{matrix} 1 - \frac{{\varepsilon }^{2}}{2} & 0 & \varepsilon \\ 0 & 1 & 0 \\ - \varepsilon & 0 & 1 - \frac{{\varepsilon }^{2}}{2} \end{matrix}\right) ,
$$

(3. 1. $5\mathrm{\;b}$ )

它们可以由 (3.1.4) 式通过 $x, y, z$ 的循环置换一一即, $x \rightarrow y, y \rightarrow z, z \rightarrow x$ -得到. 现在,把一个 $y$ 轴转动紧跟着一个 $x$ 轴转动与一个 $x$ 轴转动紧跟着一个 $y$ 轴转动的效果加以比较. 由矩阵乘法得

$$
{R}_{x}\left( \varepsilon \right) {R}_{y}\left( \varepsilon \right) = \left( \begin{matrix} 1 - \frac{{\varepsilon }^{2}}{2} & 0 & \varepsilon \\ {\varepsilon }^{2} & 1 - \frac{{\varepsilon }^{2}}{2} & - \varepsilon \\ - \varepsilon & \varepsilon & 1 - {\varepsilon }^{2} \end{matrix}\right)
$$

(3.1. ${6a}$ )

和

$$
{R}_{y}\left( \varepsilon \right) {R}_{x}\left( \varepsilon \right) = \left( \begin{matrix} 1 - \frac{{\varepsilon }^{2}}{2} & {\varepsilon }^{2} & \varepsilon \\ 0 & 1 - \frac{{\varepsilon }^{2}}{2} & - \varepsilon \\ - \varepsilon & \varepsilon & 1 - {\varepsilon }^{2} \end{matrix}\right) tag{3.1.6b}
$$

从 (3.1.6a) 式与 (3.1.6b) 式得到第一个重要结果: 如果忽略 ${\varepsilon }^{2}$ 量级以及更高量级的项, 则绕不同轴的无穷小转动对易*. 第二个, 也是更重要的结果, 涉及这样的一种方式: 当 ${\varepsilon }^{2}$ 量级的项被保留下来时,绕不同轴的转动不再对易

$$
{R}_{x}\left( \varepsilon \right) {R}_{y}\left( \varepsilon \right) - {R}_{y}\left( \varepsilon \right) {R}_{x}\left( \varepsilon \right) = \left( \begin{matrix} 0 & - {\varepsilon }^{2} & 0 \\ {\varepsilon }^{2} & 0 & 0 \\ 0 & 0 & 0 \end{matrix}\right) tag{3. 1.7}
$$

$$
= {R}_{z}\left( {\varepsilon }^{2}\right) - 1,
$$

在这里,整个推导过程中所有比 ${\varepsilon }^{2}$ 量级更高的项都被忽略了. 我们还有

$$
1 = {R}_{\text{任何 }}\left( 0\right) , tag{3.1.8}
$$

其中的 “任何” 代表对任意转动轴. 于是最终的结果可以写成

$$
{R}_{x}\left( \varepsilon \right) {R}_{y}\left( \varepsilon \right) - {R}_{y}\left( \varepsilon \right) {R}_{x}\left( \varepsilon \right) = {R}_{z}\left( {\varepsilon }^{2}\right) - {R}_{\text{任何 }}\left( 0\right) , tag{3.1.9}
$$

这是绕不同转动轴的转动操作之间的对易关系的一个例子, 稍后我们将用它推导量子力学中的角动量对易关系.

\subsection{量子力学中的无穷小转动} 

到此为止,我们还没有用到量子力学概念. 矩阵 $R$ 只是一个 $3 \times 3$ 的正交矩阵,它作用于写成列矩阵形式的矢量 $\mathbf{V}$ 上. 现在我们必须弄懂如何表征量子力学中的转动.

因为转动影响物理系统, 预期转动后系统的态右矢与原来未经转动系统的态右矢有所不同. 给定由一个 $3 \times 3$ 正交矩阵 $R$ 表征的一个转动算符 $R$ ,我们将其与适当的右矢空间里的一个算符 $\mathcal{D}\left( R\right)$ 这样联系起来,使得:

$$
|\alpha {\rangle }_{R} = \mathcal{D}\left( R\right) |\alpha \rangle , tag{3.1.10}
$$

其中 ${\left| \alpha \right\rangle }_{R}$ 和 $|\alpha \rangle$ 分别代表转动后系统和原始系统的右矢**. 注意, $3 \times 3$ 正交矩阵 $R$ 作用在一个由经典矢量的三个分量构成的一个列矩阵上,然而,算符 $\mathcal{D}\left( R\right)$ 作用在右矢空间的态矢量上. $\mathcal{D}\left( R\right)$ 的矩阵表示一将在随后几节非常详细地研究——依赖所涉及的特定的右矢空间的维数 $N$ . 对于 $N = 2$ ,适用于描写没有其他自由度的自旋 $\frac{1}{2}$ 系统, $\mathcal{D}\left( R\right)$ 用一个 2 $\times 2$ 矩阵表示; 而对于自旋为 1 的系统,适当的表示是一个 $3 \times 3$ 的幺正矩阵,等等.

---

* 在基础力学中有一个熟悉的例子. 角速度矢量 $\omega$ ,它表征转动角在一个无穷小时间间隔内的一个无穷小改变. 遵从矢量加法的通常规则, 包括矢量加法的交换性. 然而, 我们不能认为一个有限的角变化具有这一矢量性质.

** 符号 $\mathcal{D}$ 来自德文的 Drehung,意思是 “转动”.

---

要构建转动算符 $\mathcal{D}\left( R\right)$ ,最有成效的方法仍然是先考查它在一个无穷小转动下的性质. 我们几乎可以猜到必须怎样通过类比来进行. 在我们于 1.6 节和 2.1 节分别研究的平移与时间演化这两种情况下,适用的无穷小算符都可以用一个厄米算符 $G$ 写成

$$
{U}_{\varepsilon } = 1 - {iG\varepsilon } tag{3.1.11}
$$

具体说来,对于沿 $x$ 方向位移 $d{x}^{\prime }$ 的无穷小平移取

$$
G \rightarrow \frac{{p}_{x}}{\hslash },\;\varepsilon \rightarrow d{x}^{\prime } tag{3.1.12}
$$

而对于时间平移 ${dt}$ 的无穷小时间演化取

$$
G \rightarrow \frac{H}{h},\;\varepsilon \rightarrow {dt} tag{3.1.13}
$$

由经典力学我们知道角动量是转动的生成元, 就像动量和哈密顿量分别为平移与时间演化的生成元一样. 因此,我们以这样的方式来定义角动量算符 ${J}_{k}$ ,通过在 (3.1.11) 式中让

$$
G \rightarrow \frac{{J}_{k}}{\hslash },\;\varepsilon \rightarrow {d\phi } tag{3.1.14}
$$

得到一个绕第 $k$ 个轴转 ${d\phi }$ 角的无穷小转动算符. 若 ${J}_{k}$ 是厄米的,无穷小转动算符保证是幺正的,而且在 ${d\phi } \rightarrow 0$ 的极限下约化为单位算符. 更普遍地,对一个绕着由单位矢量 $\widehat{\mathbf{n}}$ 表征的方向转无穷小 ${d\phi }$ 角的转动,我们有

$$
\mathcal{D}\left( {\widehat{\mathbf{n}},{d\phi }}\right) = 1 - i\left( \frac{\mathbf{J} \cdot \widehat{\mathbf{n}}}{\hslash }\right) {d\phi } tag{3.1.15}
$$

在本书中,我们强调并没有把角动量算符定义为 $\mathbf{x} \times \mathbf{p}$ . 这一点很重要,因为自旋角动量一一我们的普遍形式也适用于它一一与 ${x}_{i}$ 和 ${p}_{j}$ 毫无关系. 换个方式讲,在经典力学中,可以证明定义为 $\mathbf{x} \times \mathbf{p}$ 的角动量是转动的生成元; 相比之下,在量子力学中我们定义 $\mathbf{J}$ 使得一个无穷小转动算符取 (3.1.15) 式形式.

一个有限转动可以通过组合相继地绕同一个轴的无穷小转动得到. 例如, 如果我们对于一个绕 $z$ 轴转 $\phi$ 角的有限转动感兴趣,我们考虑

$$
{\mathcal{D}}_{z}\left( \phi \right) = \mathop{\lim }\limits_{{N \rightarrow \infty }}{\left\lbrack 1 - i\left( \frac{{J}_{z}}{\hslash }\right) \left( \frac{\phi }{N}\right) \right\rbrack }^{N}
$$

$$
= \exp \left( \frac{-i{J}_{z}\phi }{\hslash }\right) tag{3. 1.16}
$$

$$
= 1 - \frac{i{J}_{z}\phi }{\hslash } - \frac{{J}_{z}^{2}{\phi }^{2}}{2{\hslash }^{2}} + \cdots .
$$

为了求得角动量对易关系, 我们还需要一个概念. 正如我们前面评注的, 对每个用 3 $\times 3$ 正交矩阵 $R$ 表示的转动 $R$ ,在恰当的右矢空间中都存在一个转动算符 $\mathcal{D}\left( R\right)$ . 我们进一步假设 $\mathcal{D}\left( R\right)$ 与 $R$ 有相同的群性质

$$
\text{单位元:}R \cdot 1 = R \Rightarrow \mathcal{D}\left( R\right) \cdot 1 = \mathcal{D}\left( R\right) tag{3.1.17a}
$$

封闭性: ${R}_{1}{R}_{2} = {R}_{3} \Rightarrow \mathcal{D}\left( {R}_{1}\right) \mathcal{D}\left( {R}_{2}\right) = \mathcal{D}\left( {R}_{3}\right)$(3.1.17b)

逆: $R{R}^{-1} = 1 \Rightarrow \mathcal{D}\left( R\right) {\mathcal{D}}^{-1}\left( R\right) = 1$

(3.1. ${17}\mathrm{c}$ )

$$
{R}^{-1}R = 1 \Rightarrow {\mathcal{D}}^{-1}\left( R\right) \mathcal{D}\left( R\right) = 1
$$

$$
\text{结合律:}{R}_{1}\left( {{R}_{2}{R}_{3}}\right) = \left( {{R}_{1}{R}_{2}}\right) {R}_{3} = {R}_{1}{R}_{2}{R}_{3}
$$

$$
\Rightarrow \mathcal{D}\left( {R}_{1}\right) \left\lbrack {\mathcal{D}\left( {R}_{2}\right) \mathcal{D}\left( {R}_{3}\right) }\right\rbrack tag{3.1.17d}
$$

$$
= \left\lbrack {\mathcal{D}\left( {R}_{1}\right) \mathcal{D}\left( {R}_{2}\right) }\right\rbrack \mathcal{D}\left( {R}_{3}\right)
$$

$$
= \mathcal{D}\left( {R}_{1}\right) \mathcal{D}\left( {R}_{2}\right) \mathcal{D}\left( {R}_{3}\right) .
$$

现在让我们返回到基于 $R$ 矩阵写成的转动操作 (3.1.9) 式的基本对易关系. 它的转动算符类似公式为

$$
\left( {1 - \frac{i{J}_{x}\varepsilon }{\hslash } - \frac{{J}_{x}^{2}{\varepsilon }^{2}}{2{\hslash }^{2}}}\right) \left( {1 - \frac{i{J}_{y}\varepsilon }{\hslash } - \frac{{J}_{y}^{2}{\varepsilon }^{2}}{2{\hslash }^{2}}}\right) tag{3. 1.18}
$$

$$
- \left( {1 - \frac{i{J}_{y}\varepsilon }{\hslash } - \frac{{J}_{y}^{2}{\varepsilon }^{2}}{2{\hslash }^{2}}}\right) \left( {1 - \frac{i{J}_{x}\varepsilon }{\hslash } - \frac{{J}_{x}^{2}{\varepsilon }^{2}}{2{\hslash }^{2}}}\right) = 1 - \frac{i{J}_{z}{\varepsilon }^{2}}{\hslash } - 1.
$$

$\varepsilon$ 量级的项已自动地消掉了. 让 (3.1.18) 式两边的 ${\varepsilon }^{2}$ 量级的项相等,我们得到

$$
\left\lbrack {{J}_{x},{J}_{y}}\right\rbrack = i\hslash {J}_{z}, tag{3.1.19}
$$

将这种做法重复用于绕其他轴的转动可得到

$$
\left\lbrack {{J}_{i},{J}_{j}}\right\rbrack = i\hslash {\varepsilon }_{ijk}{J}_{k}, tag{3.1.20}
$$

该式称为角动量的基本对易关系.

一般说来, 当无穷小变换的生成元不对易时, 相应操作的群称为非阿贝尔群. 基于 (3.1.20)式, 三维转动群是非阿贝尔的. 相比之下, 三维平移群是阿贝尔的, 因为即使 $i \neq j,{p}_{i}$ 和 ${p}_{j}$ 也是对易的.

注意在获得对易关系 (3.1.20) 时, 使用了以下两个概念

1. ${J}_{k}$ 是绕第 $k$ 轴转动的生成元.

2. 绕不同轴的转动不对易.

毫不夸张地说, 对易关系 (3.1.20) 式以一种紧凑的方式归纳了三维转动的一切基本性质.

\section{自旋1/2系统和有限转动}
\subsection{自旋$\frac{1}{2}$的转动算符}

角动量对易关系 (3.1.20) 可以实现的最低维数 $N = 2$ . 在第 1 章的习题 1.8 中读者已经核对过. 由

$$
{S}_{x} = \left( \frac{\hslash }{2}\right) \{ \left( {\left| {+\rangle \langle - }\right| ) + \left( \left| {-\rangle \langle + }\right| \right) \} }\right) ,
$$

$$
{S}_{y} = \left( \frac{i\hslash }{2}\right) \{ - \left( {\left| {+\rangle \langle - }\right| ) + \left( \left| {-\rangle \langle + }\right| \right) \} }\right) , tag{3.2.1}
$$

$$
{S}_{z} = \left( \frac{\hslash }{2}\right) \{ \left( {\left| {+\rangle \langle + }\right| ) - \left( \left| {-\rangle \langle - }\right| \right) \} }\right)
$$

定义的算符满足把 ${J}_{k}$ 换成 ${S}_{k}$ 的对易关系 (3.1.20) 式. 自然界利用 (3.1.20) 式的最低维实现不是先验显然的, 但许多实验——从原子光谱到核磁共振——足以让我们相信这就是实际的情况.

考虑一个绕 $z$ 轴转有限角度 $\phi$ 的转动. 如果一个自旋 $\frac{1}{2}$ 系统的右矢在转动前由 $|\alpha \rangle$ 给定, 则转动后的右矢为

$$
|\alpha {\rangle }_{R} = {\left. {D}_{z}\left( \phi \right) \right| }_{\alpha }\rangle tag{3.2.2}
$$

其中

$$
{\mathcal{D}}_{z}\left( \phi \right) = \exp \left( \frac{-i{S}_{z}\phi }{\hslash }\right) . tag{3.2.3}
$$

为了看到这个算符真的转动了这个物理系统,让我们看一看它对于 $\left\langle {S}_{x}\right\rangle$ 的影响. 转动之下这个期待值变化如下

$$
\left\langle {S}_{x}\right\rangle { \rightarrow }_{R}\left\langle {\alpha \left| {S}_{x}\right| \alpha }\right\rangle {}_{R} = \left\langle {\alpha \left| {{\mathcal{D}}_{z}^{ \dagger }\left( \phi \right) {S}_{x}{\mathcal{D}}_{z}\left( \phi \right) }\right| \alpha }\right\rangle , tag{3.2.4}
$$

因此我们必须计算

$$
\exp \left( \frac{i{S}_{z}\phi }{\hslash }\right) {S}_{x}\exp \left( \frac{i{S}_{z}\phi }{\hslash }\right) . tag{3. 2.5}
$$

为了教学原因, 我们用两种不同的方法计算它.

推导 1: 这里我们使用由 (3.2.1) 式给出的 ${S}_{x}$ 的具体形式. 于是对 (3.2.5) 式,我们得到

$$
\left( \frac{\hslash }{2}\right) \exp \left( \frac{i{S}_{z}\phi }{\hslash }\right) \{ \left( {\left| {+\rangle \langle - }\right| ) + \left( \left| {-\rangle \langle + }\right| \right) }\right) \} \exp \left( \frac{i{S}_{z}\phi }{\hslash }\right)
$$

$$
= \left( \frac{\hslash }{2}\right) \left( {{e}^{{i\phi }/2}\left| {+\rangle \left\langle {-\left| {{e}^{{i\phi }/2} + {e}^{-{i\phi }/2}}\right| - }\right\rangle \langle + }\right| {e}^{-{i\phi }/2}}\right) tag{3.2.6}
$$

$$
= \frac{\hslash }{2}\left\lbrack {\{ \left( {\left| {+\rangle \langle - }\right| ) + \left( {\left| {-\rangle \langle + }\right| )}\right) \cos \phi + i\{ \left( {\left| {+\rangle \langle - }\right| ) - \left( {\left| {-\rangle \langle + }\right| )}\right) \sin \phi }\right) }\right) }\right\rbrack
$$

$$
= {S}_{x}\cos \phi - {S}_{y}\sin \phi .
$$

推导 2: 换一种做法, 我们可以使用 (2.3.47) 式计算 (3.2.5) 式:

$$
\exp \left( \frac{i{S}_{z}\phi }{\hslash }\right) {S}_{x}\exp \left( \frac{-i{S}_{z}\phi }{\hslash }\right) = {S}_{x} + \left( \frac{i\phi }{\hslash }\right) \underset{{ih}{S}_{y}}{\underbrace{\left\lbrack {S}_{z},{S}_{x}\right\rbrack }}
$$

$$
+ \left( \frac{1}{2!}\right) {\left( \frac{i\phi }{\hslash }\right) }^{2}\underset{{h}^{2}{S}_{x}}{\underbrace{\left\lbrack {S}_{z},\underset{i\hslash {S}_{y}}{\underbrace{\left\lbrack {S}_{z},{S}_{x}\right\rbrack }}\right\rbrack }} + \left( \frac{1}{3!}\right) \underset{{h}^{3}{S}_{x}}{\underbrace{\left( \frac{i\phi }{\hslash }\right. }}\underset{i{h}^{3}{S}_{y}}{\underbrace{\left\lbrack {S}_{z},\underset{{h}^{2}{S}_{x}}{\underbrace{\left\lbrack {S}_{z},\left\lbrack {S}_{x},\left\lbrack {S}_{x}\right\rbrack \right\rbrack \right\rbrack }}\right\rbrack }} + \cdots
$$

$$
= {S}_{x}\left\lbrack {1 - \frac{{\phi }^{2}}{2!} + \cdots }\right\rbrack - {S}_{y}\left\lbrack {\phi - \frac{{\phi }^{3}}{3!} + \cdots }\right\rbrack
$$

$$
= {S}_{x}\cos \phi - {S}_{y}\sin \phi . tag{3.2.7}
$$

注意,在推导 2 中我们只用到了 ${S}_{i}$ 的对易关系,所以这个方法可被推广到角动量高于 $\frac{1}{2}$ 的系统的转动.

对于自旋 $\frac{1}{2}$ ,这两种方法都给出

$$
\left\langle {S}_{x}\right\rangle { \rightarrow }_{R}\left\langle {\alpha \left| {S}_{x}\right| \alpha }\right\rangle {}_{R} = \left\langle {S}_{x}\right\rangle \cos \phi - \left\langle {S}_{y}\right\rangle \sin \phi , tag{3.2.8}
$$

其中, 无下标的期待值被理解为是对 (老的) 未转动的系统取的. 类似地,

$$
\left\langle {S}_{y}\right\rangle \rightarrow \left\langle {S}_{y}\right\rangle \cos \phi + \left\langle {S}_{x}\right\rangle \sin \phi . tag{3.2.9}
$$

至于 ${S}_{z}$ 的期待值,由于 ${S}_{z}$ 与 ${\mathcal{D}}_{z}\left( \phi \right)$ 对易,因而无变化

$$
\left\langle {S}_{z}\right\rangle \rightarrow \left\langle {S}_{z}\right\rangle . tag{3. 2.10}
$$

关系式 (3.2.8)、(3.2.9) 和 (3.2.10) 是十分合理的. 它们表明当转动算符 (3.2.3)

作用于态右矢时,它的确把 $\mathbf{S}$ 的期待值绕 $z$ 轴转动了 $\phi$ 角. 换句话说,自旋算符期待值的行为仿佛是在旋转

$$
\left\langle {S}_{k}\right\rangle \rightarrow \mathop{\sum }\limits_{l}{R}_{kl}\left\langle {S}_{l}\right\rangle , tag{3.2.11}
$$

之下的经典矢量,其中 ${R}_{kl}$ 是所涉及问题中确定转动的 $3 \times 3$ 正交矩阵 $R$ 的矩阵元. 从我们的推导 2 应该很清楚,这个性质不只限于自旋 $\frac{1}{2}$ 系统的自旋算符. 一般地说,在转动下, 我们有

$$
\left\langle {J}_{k}\right\rangle \rightarrow \mathop{\sum }\limits_{l}{R}_{kl}\left\langle {J}_{l}\right\rangle tag{3.2.12}
$$

其中 ${J}_{k}$ 是满足角动量对易关系 (3.1.20) 式的转动生成元. 稍后,将证明这种关系可以进一步推广到任何矢量算符.

至此每件事情都在预料之中. 但现在, 准备给你一个惊喜! 稍微仔细地考查一下转动算符 (3.2.3) 式在一般的态右矢

$$
\left| {\alpha \rangle = }\right| + \rangle \langle + \mid \alpha \rangle + \mid - \rangle \langle - \mid \alpha \rangle , tag{3.2.13}
$$

上的效应. 我们看到

$$
\exp \left( \frac{-i{S}_{z}\phi }{\hslash }\right) \left| {\alpha \rangle = {e}^{-{i\phi }/2}}\right| + \rangle \left\langle {+\left| {\alpha \rangle + {e}^{{i\phi }/2}}\right| - }\right\rangle \langle - \mid \alpha \rangle . tag{3.2.14}
$$

在这里,半角 $\phi /2$ 的出现具有一个很意思的后果.

让我们考虑一个转 ${2\pi }$ 角的转动. 那时我们有

$$
\left| {\alpha {\rangle }_{{R}_{{z}^{\left( 2\pi \right) }}} \rightarrow - }\right| \alpha \rangle \text{.} tag{3. 2.15}
$$

因此,转了 ${360}^{ \circ }$ 以后的态右矢与原来的右矢差了一个负号. 我们需要一个 ${720}^{ \circ }\left( {\phi = {4\pi }}\right)$ 的转动,才能回到具有正号的同样的右矢. 注意,对于 $\mathbf{S}$ 的期待值,这个负号消失了,因为 $\mathbf{S}$ 被 $|\alpha \rangle$ 和 $\langle \alpha \mid$ 夹在中间,而这两个右矢和左矢都改变了符号. 这个负号能被观测到吗? 在讨论自旋进动之后, 我们将给出这个有趣问题的答案.

\subsection{再谈自旋进动} 

现在用一种新的观点来处理在 2.1 节已经讨论过的自旋进动问题. 回想一下这个问题的基本哈密顿量为

$$
H = - \left( \frac{e}{{m}_{e}c}\right) \mathbf{S} \cdot \mathbf{B} = \omega {S}_{z}, tag{3.2.16}
$$

其中

$$
\omega \equiv \frac{\left| e\right| B}{{m}_{e}c}. tag{3.2.17}
$$

基于该哈密顿量的时间演化算符由

$$
u\left( {t,0}\right) = \exp \left( \frac{-{iHt}}{\hslash }\right) = \exp \left( \frac{-i{S}_{z}{\omega t}}{\hslash }\right) . tag{3. 2.18}
$$

给出. 把这个方程与 (3.2.3) 式比较,我们看到在 (3.2.3) 式中令 $\phi$ 等于 ${\omega t}$ ,这个时间演化算符就与 (3.2.3) 式中的转动算符精确地相同. 这样, 我们就可立即看到为什么这个哈密顿量引起自旋进动. 重新改写 (3.2.8) 式、 (3.2.9) 式和 (3.2.10) 式, 我们得到

$$
{\left\langle {S}_{x}\right\rangle }_{t} = {\left\langle {S}_{x}\right\rangle }_{t = 0}\cos {\omega t} - {\left\langle {S}_{y}\right\rangle }_{t = 0}\sin {\omega t},
$$

(3. ${2.19a}$ )

$$
{\left\langle {S}_{y}\right\rangle }_{t} = {\left\langle {S}_{y}\right\rangle }_{t = 0}\cos {\omega t} + {\left\langle {S}_{x}\right\rangle }_{t = 0}\sin {\omega t},
$$

(3. ${2.19}\mathrm{\;b}$ )

$$
{\left\langle {S}_{z}\right\rangle }_{t} = {\left\langle {S}_{z}\right\rangle }_{t = 0}.
$$

(3. ${2.19c}$ )

在 $t = {2\pi }/\omega$ 之后,自旋回到原始方向.

这组方程可以用于讨论一个 $\mathbf{\mu }$ 子的自旋进动,该粒子是一个类电子的粒子,其重量是电子的 210 倍. $\mu$ 子的磁矩可以从其他一些实验——例如, $\mu$ 子偶素,一个正的 $\mu$ 子与一个电子的束缚态的超精细分裂——确定,结果正如自旋 $\frac{1}{2}$ 粒子的狄拉克相对论理论所预期的那样为 $e\hslash /2{m}_{\mu }c$ . (在这里我们将忽略非常小的、来自量子场论效应的修正.) 知道了磁矩我们就能预言进动的角频率. 因此, (3.2.19) 式就可以, 事实上已经, 被实验检验 (见图 2.1). 实际上,当外磁场引起自旋进动时,可利用来自 $\mu$ 衰变的电子倾向于优先沿 $\mu$ 子自旋反方向发射来分析自旋的方向.

现在让我们来看一下态右矢自身的时间演化. 假定初始时 $\left( {t = 0}\right)$ 右矢由 (3.2.13) 式给定,在 $t$ 时刻之后我们得到

$$
\left| {\alpha ,{t}_{0} = 0;t}\right\rangle = {e}^{-{i\omega t}/2}\left| {+\rangle \left\langle {+ \mid \alpha }\right\rangle + {e}^{+{i\omega t}/2}}\right| - \rangle \langle - \mid \alpha \rangle . tag{3.2.20}
$$

表示式 (3.2.20) 在 $t = {2\pi }/\omega$ 得到了一个负号,我们必须等到 $t = {4\pi }/\omega$ 时才能回到有同样符号的原始的态右矢. 总之, 态右矢的周期是自旋进动周期的两倍长:

$$
{\tau }_{\text{进动 }} = \frac{2\pi }{\omega },
$$

(3. ${2.21a}$ )

$$
{\tau }_{\text{态右矢 }} = \frac{4\pi }{\omega }, tag{3.2.21b}
$$

\subsection{研究$2\pi$ 转动的中子干涉仪实验}

我们现在描述用于探测 (3.2.15) 式中负号的实验. 非常清楚, 假如宇宙中每个态右矢都乘上一个负号, 就不会有任何可观测效应. 探测所预言负号的唯一方法是将一个未转动的态和一个转动后的态做比较. 正如在 2.7 节讨论的引力引起的量子干涉一样, 我们依靠中子干涉仪的技巧证明量子力学的这种反常预言.



图 3.2 研究在 ${2\pi }$ 转动下所预言的负号的实验

一束几乎是单能的热中子束流被劈裂为两部分——路径 $A$ 和路径 $B$ ,见图 3.2. 路径 $A$ 始终通过无磁场区; 反之,路径 $B$ 进入一个有静磁场存在的小区域. 结果是: 经过路径 $B$ 的中子态右矢遭遇到了一个相位的改变 ${e}^{+{i\omega T}/2}$ ,其中 $T$ 是穿过 $\mathbf{B} \neq 0$ 的区域花费的时间, 而 $\omega$ 是一个磁矩为 ${g}_{n}e\hslash /2{m}_{p}c$ 的中子的自旋进动频率

$$
\omega = \frac{{g}_{n}{eB}}{{m}_{p}c},\;\left( {{g}_{n} \simeq - {1.91}}\right) tag{3.2.22}
$$

如果将其与磁矩为 $e\hslash /2{m}_{r}c$ 的电子适用的 (3.2.17) 式比较,就可以看到. 当路径 $A$ 和路径 $B$ 再一次在图 3.2 的干涉区相遇时,穿过路径 $B$ 到达的中子的振幅为

$$
{c}_{2} = {c}_{2}\left( {B = 0}\right) {e}^{\mp {i\omega T}/2}, tag{3.2.23}
$$

而通过路径 $A$ 到达的中子的振幅是 ${c}_{1}$ ,与 $\mathbf{B}$ 无关. 因此,在干涉区观测到的强度一定显示出一种正弦的变化

$$
\cos \left( {\frac{\mp {\omega T}}{2} + \delta }\right) , tag{3.2.24}
$$

其中 $\delta$ 是 ${c}_{1}$ 和 ${c}_{2}\left( {B = 0}\right)$ 之间的相位差. 实际上,花费在 $B \neq 0$ 区的时间 $T$ 是固定的,但进动频率 $\omega$ 是通过磁场强度的改变而变化的. 故预言: 在干涉区的强度作为 $B$ 的函数按正弦变化. 如果把产生相邻极大值所需的 $B$ 值的差称为 ${\Delta B}$ ,很容易证明

$$
{\Delta B} = \frac{{4\pi }\hslash c}{e{g}_{n}{\lambda l}}, tag{3.2.25}
$$

其中 $l$ 为路径的长度.

在推导该式时, 我们用到了这样的事实: 正如我们的理论形式所要求的那样, 为使态右矢回到有相同符号的原始右矢,需要转动 ${4\pi }$ . 另一方面,假如我们对于自旋 $\frac{1}{2}$ 系统的描写是不正确的,在 ${2\pi }$ 转动之下右矢就会回到有相同符号的原始右矢,那么预言的 ${\Delta B}$ 就会只有 (3.2.25) 式的一半.

两个不同的组用实验令人信服地证明 (3.2.25) 式的预言在小于百分之一的精度内是正确的*. 这是量子力学的又一次胜利. 以直接的方式用实验确认了 (3.2.15) 式的非平庸预言.

\subsection{泡利二分量形式} 

利用泡利在 1926 年引入的二分量旋量形式,可以很方便地处理自旋 $\frac{1}{2}$ 系统的态右矢. 在 1.3 节我们学会了怎样使用一个列 (行) 矩阵表示一个右矢 (左矢); 我们所要做的是把基于某个指定基右矢的展开系数安排到一个列(行)矩阵中. 在自旋 $\frac{1}{2}$ 的情况下,对于基右矢和基左矢, 我们有

$$
\left| {+\rangle \doteq \left( \begin{array}{l} 1 \\ 0 \end{array}\right) \equiv {\chi }_{ + }\;}\right| - \rangle \doteq \left( \begin{array}{l} 0 \\ 1 \end{array}\right) \equiv {\chi }_{ - } tag{3.2.26}
$$

$$
\langle + \mid \doteq \left( {1,0}\right) = {\chi }_{ + }^{ \dagger }\;\langle - \mid \doteq \left( {0,1}\right) = {\chi }_{ - }^{ \dagger }
$$

和对一个任意的态右矢和相应的态左矢有

$$
\left| {\alpha \rangle = }\right| + \rangle \langle + \left| {\alpha \rangle + }\right| - \rangle \langle - \mid \alpha \rangle \doteq \left( \begin{array}{l} \langle + \mid \alpha \rangle \\ \langle - \mid \alpha \rangle \end{array}\right) tag{3.2.27a}
$$

和

$$
\langle \alpha \left| { = \langle \alpha }\right| + \rangle \langle + \left| {+\langle \alpha }\right| - \rangle \langle - | \doteq \left( {\langle \alpha \mid + \rangle ,\langle \alpha \mid - \rangle }\right) tag{3.2.27b}
$$

列矩阵 (3.2.27a) 式称为二分量旋量, 记为

$$
\chi = \left( \begin{array}{l} \langle + \mid \alpha \rangle \\ \langle - \mid \alpha \rangle \end{array}\right) \equiv \left( \begin{array}{l} {c}_{ + } \\ {c}_{ - } \end{array}\right)
$$

$$
= {c}_{ + }{\chi }_{ + } + {c}_{ - }{\chi }_{ - }, tag{3.2.28}
$$

---

* H. Rauch et al. Phys. Lett. 54A(1975) 425; S. A. Werner et al. Phys. Rev. Lett. 35(1975) 1053.

---

其中 ${c}_{ + }$ 和 ${c}_{ - }$ 一般都是复数. 对于 ${\chi }^{ \dagger }$ 我们有

$$
{\chi }^{ \dagger } = \left( {\langle \alpha \mid + \rangle ,\langle \alpha \mid - \rangle }\right) = \left( {{c}_{ + }^{ * },{c}_{ - }^{ * }}\right) , tag{3.2.29}
$$

矩阵元 $\left\langle {\pm \left| {S}_{k}\right| + }\right\rangle$ 和 $\left\langle {\pm \left| {S}_{k}\right| - }\right\rangle$ ,除去 $\hslash /2$ 之外,被设定等于那些 $2 \times 2$ 的、以泡利矩阵著称的矩阵 ${\sigma }_{k}$ . 我们确定

$$
\left\langle {\pm \left| {S}_{k}\right| + }\right\rangle \equiv \left( \frac{\hslash }{2}\right) {\left( {\sigma }_{k}\right) }_{\pm , + },\;\left\langle {\pm \left| {S}_{k}\right| - }\right\rangle \equiv \left( \frac{\hslash }{2}\right) {\left( {\sigma }_{k}\right) }_{\pm , - }. tag{3. 2.30}
$$

现在我们可以用 $\chi$ 和 ${\sigma }_{k}$ 写出期待值 $\left\langle {S}_{k}\right\rangle$

$$
\left\langle {S}_{k}\right\rangle = \left\langle {\alpha \left| {S}_{k}\right| \alpha }\right\rangle = \mathop{\sum }\limits_{{{a}^{\prime } = + , - }}\mathop{\sum }\limits_{{{a}^{\prime \prime } = + , - }}\left\langle {\alpha \left| {a}^{\prime }\right\rangle \left\langle {{a}^{\prime }\left| {S}_{k}\right| {a}^{\prime \prime }}\right\rangle \left\langle {a}^{\prime \prime }\right| \alpha }\right\rangle tag{3. 2.31}
$$

$$
= \left( \frac{\hslash }{2}\right) {\chi }^{ \dagger }{\sigma }_{k}\chi
$$

其中在最后一行用到了矩阵乘法的常用规则. 显然我们可从 (3.2.1) 和 (3.2.30) 式看到

$$
{\sigma }_{1} = \left( \begin{array}{ll} 0 & 1 \\ 1 & 0 \end{array}\right) ,\;{\sigma }_{2} = \left( \begin{matrix} 0 & - i \\ i & 0 \end{matrix}\right) ,\;{\sigma }_{3} = \left( \begin{matrix} 1 & 0 \\ 0 & - 1 \end{matrix}\right) , tag{3. 2.32}
$$

其中下标 1,2 和 3 分别代表 $x, y$ 和 $z$ .

下面列出泡利矩阵的一些性质. 首先,

$$
{\sigma }_{i}^{2} = 1
$$

(3. ${2.33a}$ )

$$
{\sigma }_{i}{\sigma }_{j} + {\sigma }_{j}{\sigma }_{i} = 0,\;\text{ 对于 }i \neq j,
$$

(3. ${2.33}\mathrm{\;b}$ )

其中 (3.2.33a) 式的右边被理解为 $2 \times 2$ 单位矩阵. 当然,这两个关系式等价于反对易关系

$$
\left\{ {{\sigma }_{i},{\sigma }_{j}}\right\} = 2{\delta }_{ij}, tag{3. 2.34}
$$

还有对易关系

$$
\left\lbrack {{\sigma }_{i},{\sigma }_{j}}\right\rbrack = {2i}{\varepsilon }_{ijk}{\sigma }_{k}, tag{3. 2.35}
$$

可以看出,它显然就是角动量对易关系 (3.1.20) 的 $2 \times 2$ 矩阵实现. 比较 (3.2.34) 式和 (3.2.35) 式, 我们可以得到

$$
{\sigma }_{1}{\sigma }_{2} = - {\sigma }_{2}{\sigma }_{1} = i{\sigma }_{3}\cdots tag{3. 2.36}
$$

还要注意

$$
{\sigma }_{i}^{ \dagger } = {\sigma }_{i},
$$

(3. ${2.37a}$ )

$$
\det \left( {\sigma }_{i}\right) = - 1, tag{3. 2.37b}
$$

$$
\operatorname{Tr}\left( {\sigma }_{i}\right) = 0.
$$

(3. ${2.37}\mathrm{c}$ )

现在考虑 $\mathbf{\sigma } \cdot \mathbf{a}$ ,其中 $\mathbf{a}$ 是一个三维矢量. 要把这个量理解为实际上是一个 $2 \times 2$ 矩阵. 于是

$$
\mathbf{\sigma } \cdot \mathbf{a} \equiv \mathop{\sum }\limits_{k}{a}_{k}{\sigma }_{k}
$$

$$
= \left( \begin{matrix} + {a}_{3} & {a}_{1} - i{a}_{2} \\ {a}_{1} + i{a}_{2} & - {a}_{3} \end{matrix}\right) tag{3. 2.38}
$$

还有一个非常重要的恒等式

$$
\left( {\mathbf{\sigma } \cdot \mathbf{a}}\right) \left( {\mathbf{\sigma } \cdot \mathbf{b}}\right) = \mathbf{a} \cdot \mathbf{b} + i\mathbf{\sigma } \cdot \left( {\mathbf{a} \times \mathbf{b}}\right) . tag{3. 2.39}
$$

为证明该式, 所需的是反对易关系和对易关系, 它们分别为 (3.2.34) 式与 (3.2.35) 式:

$$
\mathop{\sum }\limits_{j}{\sigma }_{j}a\mathop{\sum }\limits_{k}{\sigma }_{k}{b}_{k} = \mathop{\sum }\limits_{j}\mathop{\sum }\limits_{k}\left( {\frac{1}{2}\left\{ {{\sigma }_{j},{\sigma }_{k}}\right\} + \frac{1}{2}\left\lbrack {{\sigma }_{j},{\sigma }_{k}}\right\rbrack }\right) {a}_{j}{b}_{k}
$$

$$
= \mathop{\sum }\limits_{j}\mathop{\sum }\limits_{k}\left( {{\delta }_{jk} + i{\varepsilon }_{jkl}{\sigma }_{l}}\right) {a}_{j}{b}_{k} tag{3. 2.40}
$$

$$
= \mathbf{a} \cdot \mathbf{b} + i\mathbf{\sigma } \cdot \left( {\mathbf{a} \times \mathbf{b}}\right) .
$$

如果 $\mathbf{a}$ 的分量都是实的,我们有

$$
{\left( \mathbf{\sigma } \cdot \mathbf{a}\right) }^{2} = {\left| \mathbf{a}\right| }^{2}, tag{3. 2.41}
$$

其中 $\left| \mathbf{a}\right|$ 是矢量 $\mathbf{a}$ 的长度.

\subsection{二分量形式中的转动}

现在研究转动算符 $\mathcal{D}\left( {\widehat{\mathbf{n}},\phi }\right)$ 的 $2 \times 2$ 矩阵表示. 如下

$$
\exp \left( \frac{-i\mathbf{S} \cdot \widehat{\mathbf{n}}\phi }{\hslash }\right) \doteq \exp \left( \frac{-i\mathbf{\sigma } \cdot \widehat{\mathbf{n}}\phi }{2}\right) . tag{3. 2.42}
$$

利用从 (3.2.41) 式得到的

$$
{\left( \mathbf{\sigma } \cdot \widehat{\mathbf{n}}\right) }^{n} = \left\{ \begin{array}{ll} 1 & \text{ 对于偶数的 }n, \\ \mathbf{\sigma } \cdot \widehat{\mathbf{n}} & \text{ 对于奇数的 }n, \end{array}\right. tag{3.2.43}
$$

我们可以写出

$$
\exp \left( \frac{-i\mathbf{\sigma } \cdot \widehat{\mathbf{n}}\phi }{2}\right) = \left\lbrack {1 - \frac{{\left( \mathbf{\sigma } \cdot \widehat{\mathbf{n}}\right) }^{2}}{2!}{\left( \frac{\phi }{2}\right) }^{2} + \frac{{\left( \mathbf{\sigma } \cdot \widehat{\mathbf{n}}\right) }^{4}}{4!}{\left( \frac{\phi }{2}\right) }^{4} - \cdots }\right\rbrack
$$

$$
- i\left\lbrack {\left( {\mathbf{\sigma } \cdot \widehat{\mathbf{n}}}\right) \frac{\phi }{2} - \frac{{\left( \mathbf{\sigma } \cdot \widehat{\mathbf{n}}\right) }^{3}}{3!}{\left( \frac{\phi }{2}\right) }^{3} + \cdots }\right\rbrack tag{3. 2.44}
$$

$$
= 1\cos \left( \frac{\phi }{2}\right) - i\mathbf{\sigma } \cdot \widehat{\mathbf{n}}\sin \left( \frac{\phi }{2}\right) .
$$

显然,以 $2 \times 2$ 形式我们有

$$
\exp \left( \frac{-i\mathbf{\sigma } \cdot \widehat{\mathbf{n}}\phi }{2}\right) = \left( \begin{array}{ll} \cos \left( \frac{\phi }{2}\right) - i{n}_{z}\sin \left( \frac{\phi }{2}\right) & \left( {-i{n}_{x} - {n}_{y}}\right) \sin \left( \frac{\phi }{2}\right) \\ \left( {-i{n}_{x} + {n}_{y}}\right) \sin \left( \frac{\phi }{2}\right) & \cos \left( \frac{\phi }{2}\right) + i{n}_{z}\sin \left( \frac{\phi }{2}\right) \end{array}\right) , tag{3. 2.45}
$$

正像算符 $\exp \left( {-i\mathbf{S} \cdot \widehat{\mathbf{n}}\phi /\hslash }\right)$ 作用在态右矢 $|\alpha \rangle$ 上一样, $2 \times 2$ 矩阵 $\exp \left( {-i\mathbf{\sigma } \cdot \widehat{\mathbf{n}}\phi /2}\right)$ 作用于一个二分量旋量 $\chi$ 上. 在转动之下,我们使 $\chi$ 发生如下改变:

$$
\chi \rightarrow \exp \left( \frac{-i\mathbf{\sigma } \cdot \widehat{\mathbf{n}}\phi }{2}\right) \chi tag{3.2.46}
$$

另一方面, ${\sigma }_{k}$ 自身在转动下保持不变. 因此,严格地说,尽管 $\mathbf{\sigma }$ 的外貌像矢量,人们却不把它看成一个矢量; 而是把遵从矢量的变换性质的 ${\chi }^{ \dagger }\mathbf{\sigma }\chi$ 看成一个矢量:

$$
{\chi }^{ \dagger }{\sigma }_{k}\chi \rightarrow \mathop{\sum }\limits_{l}{R}_{kl}{\chi }^{ \dagger }{\sigma }_{l}\chi tag{3.2.47}
$$

它的明确证明可以利用

$$
\exp \left( \frac{i{\sigma }_{3}\phi }{2}\right) {\sigma }_{1}\exp \left( \frac{-i{\sigma }_{3}\phi }{2}\right) = {\sigma }_{1}\cos \phi - {\sigma }_{2}\sin \phi tag{3. 2.48}
$$

等给出,它是 (3.2.6) 式的 $2 \times 2$ 矩阵类比.

在利用右矢形式讨论 ${2\pi }$ 转动时,我们曾经看到一个自旋 $\frac{1}{2}$ 的右矢 $|\alpha \rangle$ 变成了 $- |\alpha \rangle$ . 这个说法的 $2 \times 2$ 类比是

$$
{\left. \exp \left( \frac{-i\mathbf{\sigma } \cdot \widehat{\mathbf{n}}\phi }{2}\right) \right| }_{\phi = {2\pi }} = - 1,\;\text{ 对任何 }\widehat{\mathbf{n}}, tag{3.2.49}
$$

它从 (3.2.44) 式显然可得.

作为转动矩阵 (3.2.45) 式的一个有益的应用, 让我们看一下怎样构建本征值为 +1 的 $\mathbf{\sigma } \cdot \widehat{\mathbf{n}}$ 的本征旋量,其中 $\widehat{\mathbf{n}}$ 是某特定方向的一个单位矢量. 我们的目的是构建一个满足

$$
\mathbf{\sigma } \cdot \widehat{\mathbf{n}}\chi = \chi tag{3. 2.50}
$$

的 $\chi$ . 换句话说,我们要寻找由

$$
\mathbf{S} \cdot \widehat{\mathbf{n}}\left| {\mathbf{S} \cdot \widehat{\mathbf{n}}; + \rangle = \left( \frac{\hslash }{2}\right) }\right| \mathbf{S} \cdot \widehat{\mathbf{n}}; + \rangle . tag{3. 2.51}
$$

定义的 $\left| {\mathbf{S} \cdot \widehat{\mathbf{n}}; + }\right\rangle$ 的二分量列矩阵表示. 实际上,该式可作为一个直接的本征值问题来求解 (见第 1 章的习题 1.9), 但是这里, 我们给出另一种可选的基于转动矩阵 (3.2.45) 的方法.


图 3.3 构建 $\sigma \cdot \widehat{\mathrm{n}}$ 的本征旋量

设表征 $\widehat{\mathbf{n}}$ 的极角与方位角分别为 $\beta$ 和 $\alpha$ . 我们从表示自旋向上态的二分量旋量 $\left( \begin{array}{l} 1 \\ 0 \end{array}\right)$ 开始. 给定这个旋量之后,我们先绕 $y$ 轴转 $\beta$ 角; 随后绕 $z$ 轴转 $\alpha$ 角. 然后我们看到,所期待的自旋态就得到了,见图 3.3. 用泡利旋量语言,这一系列操作等价于把 $\exp \left( {-i{\sigma }_{2}\beta /2}\right)$ 作用于 $\left( \begin{array}{l} 1 \\ 0 \end{array}\right)$ ,随后再用 $\exp \left( {-i{\sigma }_{3}\alpha /2}\right)$ 作用. 净结果是

$$
\chi = \left\lbrack {\cos \left( \frac{\alpha }{2}\right) - i{\sigma }_{3}\sin \left( \frac{\alpha }{2}\right) }\right\rbrack \left\lbrack {\cos \left( \frac{\beta }{2}\right) - i{\sigma }_{2}\sin \left( \frac{\beta }{2}\right) }\right\rbrack \left( \begin{array}{l} 1 \\ 0 \end{array}\right)
$$

$$
= \left( \begin{matrix} \cos \left( \frac{\alpha }{2}\right) - i\sin \left( \frac{\alpha }{2}\right) & 0 \\ 0 & \cos \left( \frac{\alpha }{2}\right) + i\sin \left( \frac{\alpha }{2}\right) \end{matrix}\right) \left( \begin{matrix} \cos \left( \frac{\beta }{2}\right) & - \sin \left( \frac{\beta }{2}\right) \\ \sin \left( \frac{\beta }{2}\right) & \cos \left( \frac{\beta }{2}\right) \end{matrix}\right) \left( \begin{array}{l} 1 \\ 0 \end{array}\right)
$$

$$
= \left( \begin{array}{l} \cos \left( \frac{\beta }{2}\right) {e}^{-{i\alpha }/2} \\ \sin \left( \frac{\beta }{2}\right) {e}^{{i\alpha }/2} \end{array}\right) . tag{3. 2.52}
$$

如果我们意识到上分量和下分量的共同相位是没有什么物理意义的话, 则上式与第 1 章的习题 1.9 是完全一致的.

\section{SO(3)SU(2)和欧拉转动}
\subsection{正交群}

现在, 我们更系统地研究在前两节曾经涉及的一些操作的群的性质.

研究转动的最基本方法是建立在规定了转动轴和转动角的基础之上的. 显然, 我们需要用三个实数来表征一个一般的转动: 沿转轴方向单位矢量 $\widehat{\mathbf{n}}$ 的极角和方位角以及转角 $\phi$ 本身. 等价地,同样一个转动可以用矢量 $\widehat{\mathbf{n}}\phi$ 的三个笛卡尔分量确定. 然而,表征转动的这些方法从研究转动的群性质观点来看并不太方便. 首先,除非 $\phi$ 是无穷小,或者 $\widehat{\mathbf{n}}$ 永远沿着同一个方向,否则我们不能添加一些 $\widehat{\mathbf{n}}\phi$ 形式的矢量去表征一系列相继的转动. 用一个 $3 \times 3$ 的正交矩阵 $R$ 来工作比较容易,因为相继转动的效果只要通过适当的正交矩阵相乘即可求得.

一个 $3 \times 3$ 正交矩阵有多少个独立参量呢? 一个实的 $3 \times 3$ 正交矩阵有 9 个矩阵元,但我们有正交性约束

$$
R{R}^{T} = 1\text{.} tag{3. 3.1}
$$

它对应于 6 个独立的方程,因为乘积 $R{R}^{T}$ 和 ${R}^{T}R$ 相同都是有 6 个独立元素的对称矩阵. 作为结果, $R$ 中只有 3 (即 9-6) 个独立的数,同样的数字我们以前用更基本的方法得到过.

正交矩阵的所有乘法运算的集合构成一个群. 此时, 满足下列四个要求:

1. 任何两个正交矩阵之积是另一个正交矩阵, 它之所以被满足是因为

$$
\left( {{R}_{1}{R}_{2}}\right) {\left( {R}_{1}{R}_{2}\right) }^{T} = {R}_{1}{R}_{2}{R}_{2}^{T}{R}_{1}^{T} = 1. tag{3. 3.2}
$$

2. 结合律成立:

$$
{R}_{1}\left( {{R}_{2}{R}_{3}}\right) = \left( {{R}_{1}{R}_{2}}\right) {R}_{3}. tag{3.3.3}
$$

3. 恒等矩阵 1-一物理上对应没有任何转动——由下式定义

$$
{R1} = {1R} = R tag{3.3.4}
$$

它是所有正交矩阵类中的一个成员.

4. 逆矩阵 ${R}^{-1}$ ——物理上对应于相反意义上的转动——由

$$
R{R}^{-1} = {R}^{-1}R = 1 tag{3.3.5}
$$

定义, 也是一个群元.

这个群名为 $\mathrm{{SO}}\left( 3\right)$ ,其中的 $\mathrm{S}$ 代表 “特殊”, $\mathrm{O}$ 代表 “正交”,3 代表三维. 注意, 这里只考虑转动操作, 所以有 $\mathrm{{SO}}\left( 3\right)$ 而不是 $\mathrm{O}\left( 3\right)$ (它可以包括将在第 4 章讨论的反演操作).

\subsection{幺正么模群}

在前一节我们还学会了另外一种表征一个任意转动的方法, 即观察作用于二分量旋量

$\chi$ 上的 $2 \times 2$ 矩阵 (3.2.45) 式. 显然,(3.2.45) 是么正的. 作为一个结果,对于 (3.2.28) 式中定义的 ${c}_{ + }$ 和 ${c}_{ - }$ ,

$$
{\left| {c}_{ + }\right| }^{2} + {\left| {c}_{ - }\right| }^{2} = 1 tag{3. 3.6}
$$

是保持不变的. 此外, 矩阵 (3.2.45) 式是幺模的, 即它的行列式为 1 , 下面将会明确地

证明这一点.

最一般的幺正么模矩阵可以写成

$$
U\left( {a, b}\right) = \left( \begin{matrix} a & b \\ - {b}^{ * } & {a}^{ * } \end{matrix}\right) , tag{3.3.7}
$$

其中 $a$ 和 $b$ 都是复数,满足幺模条件

$$
{\left| a\right| }^{2} + \left| {b}^{2}\right| = 1. tag{3.3.8}
$$

可以很容易地建立 (3.3.7) 式的么正性质:

$$
U{\left( a, b\right) }^{ \dagger }U\left( {a, b}\right) = \left( \begin{matrix} {a}^{ * } & - b \\ {b}^{ * } & a \end{matrix}\right) \left( \begin{matrix} a & b \\ - {b}^{ * } & {a}^{ * } \end{matrix}\right) = 1, tag{3.3.9}
$$

很容易看到,表征一个自旋 $\frac{1}{2}$ 系统转动的 $2 \times 2$ 矩阵 (3.2.45) 可以写成 $U\left( {a, b}\right)$ . 比

较 (3.2.45) 式与 (3.3.7) 式, 我们确认

$$
\operatorname{Re}\left( a\right) = \cos \left( \frac{\phi }{2}\right) ,\;\operatorname{Im}\left( a\right) = - {n}_{z}\sin \left( \frac{\phi }{2}\right) , tag{3. 3.10}
$$

$$
\operatorname{Re}\left( b\right) = - {n}_{y}\sin \left( \frac{\phi }{2}\right) ,\;\operatorname{Im}\left( b\right) = - {n}_{x}\sin \left( \frac{\phi }{2}\right) ,
$$

由此, (3.3.8) 式的幺模性质立即可得. 反过来说, 形为 (3.3.7) 式的最普遍的幺正么模矩阵可以解释为表示一个转动.

$a$ 和 $b$ 这两个复数称为凯莱-克莱因 (Cayley-Klein) 参量. 历史上,在远早于量子力学诞生之前人们就知道了幺正么模矩阵和转动之间的联系. 事实上, 凯莱-克莱因参量在刚体运动学中曾用于表征陀螺仪的复杂运动.

无需借助转动来解释幺正么模矩阵, 就可以直接检验幺正么模矩阵乘法运算的群性质. 特别注意到

$$
U\left( {{a}_{1},{b}_{1}}\right) U\left( {{a}_{2},{b}_{2}}\right) = U\left( {{a}_{1}{a}_{2} - {b}_{1}{b}_{2}^{ * },{a}_{1}{b}_{2} + {a}_{2}^{ * }{b}_{1}}\right) , tag{3. 3.11}
$$

其中乘积矩阵的幺模条件是

$$
{\left| {a}_{1}{a}_{2} - {b}_{1}{b}_{2}^{ * }\right| }^{2} + {\left| {a}_{1}{b}_{2} + {a}_{2}^{ * }{b}_{1}\right| }^{2} = 1. tag{3. 3.12}
$$

对于 $U$ 的逆,我们有

$$
{U}^{-1}\left( {a, b}\right) = U\left( {{a}^{ * }, - b}\right) . tag{3. 3.13}
$$

这个群常称 $\mathrm{{SU}}\left( 2\right)$ 群,其中的 $\mathrm{S}$ 代表 “特殊”, $\mathrm{U}$ 代表 “幺正”,2 代表维数为 2 . 相比之下,由一般的 $2 \times 2$ 么正矩阵 (不必受幺模的限制) 的乘法定义的群称为 $\mathrm{U}\left( 2\right)$ 群. 最普遍的 2 维幺正矩阵有 4 个独立参量,可以写成 ${e}^{i\gamma }\left( {\gamma \text{为实数}}\right)$ 乘以一个幺正么模矩阵:

$$
U = {e}^{i\gamma }\left( \begin{matrix} a & b \\ - {b}^{ * } & {a}^{ * } \end{matrix}\right) ,\;{\left| a\right| }^{2} + {\left| b\right| }^{2} = 1,\;{\gamma }^{ * } = \gamma . tag{3. 3.14}
$$

$\mathrm{{SU}}\left( 2\right)$ 称为 $\mathrm{U}\left( 2\right)$ 的一个子群.

因为既能用 $\mathrm{{SO}}\left( 3\right)$ 语言也可以用 $\mathrm{{SU}}\left( 2\right)$ 语言表征转动,可能会想得出结论: $\mathrm{{SO}}\left( 3\right)$ 和 $\mathrm{{SU}}\left( 2\right)$ 群是同构的,这就是说, $\mathrm{{SO}}\left( 3\right)$ 群的一个元素和 $\mathrm{{SU}}\left( 2\right)$ 群的一个元素之间存在着一一对应关系. 这个推论并不正确. 考虑一个 ${2\pi }$ 转动和另一个 ${4\pi }$ 转动. 用 $\mathrm{{SO}}\left( 3\right)$ 语言,表示一个 ${2\pi }$ 转动的矩阵和表示一个 ${4\pi }$ 转动的矩阵都是 $3 \times 3$ 单位矩阵; 然而,用 $\mathrm{{SU}}\left( 2\right)$ 语言, 相应的矩阵分别为 -1 乘以 $2 \times 2$ 单位矩阵和该单位矩阵自身. 更一般地说, $U\left( {a, b}\right)$ 和 $U\left( {-a, - b}\right)$ 二者都对应于 $\mathrm{{SO}}\left( 3\right)$ 语言中的一个单独的 $3 \times 3$ 矩阵. 因此这种对应是二对一的,对于一个给定的 $R$ ,对应的 $U$ 是双值的. 然而,人们可以说,这两个群是局域同构的.

\subsection{欧拉转动}

由经典力学, 读者可能熟悉这样的事实, 即: 一个刚体的任意的转动可以分成三步完成, 它们被称为欧拉转动. 这种由三个欧拉角确定的欧拉转动语言, 提供了另外一种方式来表征最一般的三维转动.

欧拉转动的三步如下所示. 第一步,将刚体反时针绕 $z$ 轴 (从正 $z$ 一边看) 转 $\alpha$ 角. 现在设想,比如说,有一个如此嵌入刚体的本体 $y$ 轴,使得绕 $z$ 轴转动之前,这个本体 $y$ 轴与通常的、称为空间固定的 $y$ 轴相重合. 显然,在绕 $z$ 轴转动之后,这个本体 $y$ 轴不再与空间固定 $y$ 轴相重合了,让我们把前者称为 ${y}^{\prime }$ 轴. 为了看到对一个薄盘子这些现象怎样出现,可参考图 3.4a. 现在我们做一个第二次转动,这一次绕 ${y}^{\prime }$ 轴转 $\beta$ 角. 结果,本体 $z$ 轴不再指向空间固定的 $z$ 轴方向. 我们称第二次转动之后的这个本体固定的 $z$ 轴为 ${z}^{\prime }$ 轴, 见图 3.4b. 第三次最终的转动是绕 ${z}^{\prime }$ 轴转 $\gamma$ 角. 现在本体 $y$ 轴变成图 3.4c 的 ${y}^{\prime \prime }$ 轴. 借助 $3 \times 3$ 正交矩阵,这三次操作的乘积可以写成

$$
R\left( {\alpha ,\beta ,\gamma }\right) \equiv {R}_{{z}^{\prime }}\left( \gamma \right) {R}_{{y}^{\prime }}\left( \beta \right) {R}_{z}\left( \alpha \right) . tag{3. 3.15}
$$


图 3.4 欧拉转动

在这里有必要给出一个警示性的评注. 在许多经典力学的教科书中,更喜欢绕本体 $x$ 轴而不是本体 $y$ 轴做第二次转动 [中间的那次转动,例如, Goldstein (2002)]. 这个约定由于一种原因没有被量子力学采用, 该原因过一会就会明了.

在 (3.3.15) 式中,出现了 ${R}_{y}$ 和 ${R}_{z}$ ,它们都是绕本体轴的转动矩阵. 欧拉转动的方法在量子力学中是相当不方便的,因为早先我们得到的是 $\mathbf{S}$ 算符在各空间固定轴 (不带撇的) 上的分量的简单表示式, 而不是在各本体轴上的分量. 因此, 所需要的做法是借助绕空间固定轴的转动来表示所考虑的绕本体轴的转动. 幸运的是, 存在一个非常简单的关系:

$$
{R}_{{y}^{\prime }}\left( \beta \right) = {R}_{z}\left( \alpha \right) {R}_{y}\left( \beta \right) {R}_{z}^{-1}\left( \alpha \right) . tag{3.3.16}
$$

该式右边的意义如下. 首先,通过绕 $z$ 轴顺时针(从正 $z$ 一边看)转 $\alpha$ 角,把图 3.4a 中的本体 $y$ 轴 (即 ${y}^{\prime }$ 轴) 转回到原来的固定空间 $y$ 方向; 然后绕着这个 $y$ 轴转 $\beta$ 角. 最后,通过绕固定空间 $z$ 轴 (不是绕 ${z}^{\prime }$ 轴!) 转 $\alpha$ 角,使本体 $y$ 轴回到 ${y}^{\prime }$ 轴方向. 方程 (3.3.16) 式告诉我们,这些转动的净效果是一个单独的绕 ${y}^{\prime }$ 轴转 $\beta$ 角的转动.

为了证明这个论点, 让我们仔细观察 (3.3.16) 式两边在图 3.4a 的圆盘上的效果. 显然,在这两种情况下,本体 $y$ 轴的取向是不变的,即沿 ${y}^{\prime }$ 方向. 此外,无论我们使用 ${R}_{y}\left( \beta \right)$ 还是 ${R}_{z}\left( \alpha \right) {R}_{y}\left( \beta \right) {R}_{z}^{-1}\left( \alpha \right)$ ,最终本体 $z$ 轴的取向是相同的. 在这两种情况下,最终的本体 $z$ 轴都相对固定 $z$ 轴 (与初始的 $z$ 轴相同) 转了一个极角 $\beta$ ,而其方位角正是如同在固定坐标系中测量到的 $\alpha$ . 换句话说,最后的本体 $z$ 轴与图 3.4b 的 ${z}^{\prime }$ 轴是相同的. 类似地可以证明

$$
{R}_{{s}^{\prime }}\left( \gamma \right) = {R}_{{y}^{\prime }}\left( \beta \right) {R}_{z}\left( \gamma \right) {R}_{{y}^{\prime }}^{-1}\left( \beta \right) . tag{3. 3.17}
$$

现在我们利用 (3.3.16) 式和 (3.3.17) 式, 可以改写 (3.3.15) 式. 我们得到

$$
{R}_{{z}^{\prime }}\left( \gamma \right) {R}_{{y}^{\prime }}\left( \beta \right) {R}_{z}\left( \alpha \right) = {R}_{{y}^{\prime }}\left( \beta \right) {R}_{z}\left( \gamma \right) {R}_{{y}^{\prime }}^{-1}\left( \beta \right) {R}_{{y}^{\prime }}\left( \beta \right) {R}_{z}\left( \alpha \right)
$$

$$
= {R}_{z}\left( \alpha \right) {R}_{y}\left( \beta \right) {R}_{z}^{-1}\left( \alpha \right) {R}_{z}\left( \gamma \right) {R}_{z}\left( \alpha \right) tag{3. 3.18}
$$

$$
= {R}_{z}\left( \alpha \right) {R}_{y}\left( \beta \right) {R}_{z}\left( \gamma \right) ,
$$

其中在最后一步中我们用到了 ${R}_{z}\left( \gamma \right)$ 和 ${R}_{z}\left( \alpha \right)$ 对易的事实. 总之,

$$
R\left( {\alpha ,\beta ,\gamma }\right) = {R}_{z}\left( \alpha \right) {R}_{y}\left( \beta \right) {R}_{z}\left( \gamma \right) , tag{3. 3.19}
$$

其中右边所有的三个矩阵都只涉及固定轴的转动.

现在,把这一组运算应用到量子力学中自旋 $\frac{1}{2}$ 的系统. 对应于 (3.3.19) 式中的正交矩阵之积,在所考虑的自旋 $\frac{1}{2}$ 系统的右矢空间中,存在着一个转动算符的乘积:

$$
\mathcal{D}\left( {\alpha ,\beta ,\gamma }\right) = {\mathcal{D}}_{z}\left( \alpha \right) {\mathcal{D}}_{y}\left( \beta \right) {R}_{z}\left( \gamma \right) , tag{3. 3.20}
$$

这个乘积的 $2 \times 2$ 矩阵表示是

$$
\exp \left( \frac{-i{\sigma }_{3}\alpha }{2}\right) \exp \left( \frac{-i{\sigma }_{2}\beta }{2}\right) \exp \left( \frac{-i{\sigma }_{3}\gamma }{2}\right)
$$

$$
= \left( \begin{matrix} {e}^{-{i\alpha }/2} & 0 \\ 0 & {e}^{{i\alpha }/2} \end{matrix}\right) \left( \begin{matrix} \cos \left( {\beta /2}\right) & - \sin \left( {\beta /2}\right) \\ \sin \left( {\beta /2}\right) & \cos \left( {\beta /2}\right) \end{matrix}\right) \left( \begin{matrix} {e}^{-{i\gamma }/2} & 0 \\ 0 & {e}^{{i\gamma }/2} \end{matrix}\right) tag{3. 3.21}
$$

$$
= \left( \begin{matrix} {e}^{-i\left( {\alpha + \gamma }\right) /2}\cos \left( {\beta /2}\right) & - {e}^{-i\left( {\alpha - \gamma }\right) /2}\sin \left( {\beta /2}\right) \\ {e}^{i\left( {\alpha - \gamma }\right) /2}\sin \left( {\beta /2}\right) & {e}^{i\left( {\alpha + \gamma }\right) /2}\cos \left( {\beta /2}\right) \end{matrix}\right) ,
$$

其中用到了 (3.2.44) 式. 这个矩阵显然是幺正幺模形式的. 反过来,最普遍的 $2 \times 2$ 幺正幺模矩阵均可以写成这种欧拉角形式.

注意,第二个 (中间的) 转动 $\exp \left( {-i{\sigma }_{y}\phi /2}\right)$ 是纯实的. 假如像许多经典力学教科书中所做的那样,我们选择绕 $x$ 轴而不是 $y$ 轴转动,则情况就不会是这样了. 在量子力学中坚持我们的约定是值得的, 因为我们更喜欢这种第二个转动的矩阵元, 它是唯一的包含非对角元、且为纯实数的转动矩阵 * .

(3.3.21) 式中的 $2 \times 2$ 矩阵称为转动算符 $\mathcal{D}\left( {\alpha ,\beta ,\gamma }\right)$ 的 $j = \frac{1}{2}$ 不可约表示. 它的矩阵元用 ${\mathcal{D}}_{{m}^{\prime }m}^{\left( 1/2\right) }\left( {\alpha ,\beta ,\gamma }\right)$ 表示. 借助角动量算符,有

$$
{\mathcal{D}}_{m, m}^{\left( 1/2\right) }\left( {\alpha ,\beta ,\gamma }\right) = \left\langle {j = \frac{1}{2},{m}^{\prime }}\right| \exp \left( \frac{-i{J}_{z}\alpha }{\hslash }\right)
$$

$$
\times \exp \left( \frac{-i{J}_{y}\beta }{\hslash }\right) \exp \left( \frac{-i{J}_{z}\gamma }{\hslash }\right) \left| {j = \frac{1}{2}, m}\right\rangle tag{3. 3.22}
$$

在 3.5 节我们将广泛地研究 (3.3.21) 的更高 $j$ 值的类似公式.

\section{密度算符和纯系综与混合系综}
\subsection{极化束与非极化束}

至此, 已发展的量子力学形式关于一个全同制备的物理系统的系综, 即一个集合作出了统计的预言. 更精确地说,在这样的一个系综内假定所有的成员都由同样的态右矢 $|\alpha \rangle$ 所表征. 通过 $\mathrm{{SG}}$ 过滤器的一束银原子就是这种系综的一个好例子. 束流中的每个原子其自旋指向相同的方向, 即由该过滤器的非均匀磁场确定的方向. 我们迄今还没有讨论过如何用量子力学描述这样的一个系综,其中的某些物理系统,比如 ${60}\%$ ,由 $|\alpha \rangle$ 表征,而其余的 ${40}\%$ 由某个其他的右矢 $|\beta \rangle$ 表征.

为了生动地说明到此为止发展的形式体系的不完备性, 让我们考虑直接由热炉子飞出的银原子, 它们将受到斯特恩-盖拉赫类型的过滤器的作用. 基于对称性, 我们预期这样的原子具有随机的自旋取向; 换句话说, 没有任何优势的方向与这样的一个原子系综相联系. 按照到此为止发展的形式体系,一个自旋 $\frac{1}{2}$ 系统的最普遍的态右矢由下式给出

$$
\left| {\alpha \rangle = {c}_{ + }}\right| + \rangle + {c}_{ - }| - \rangle tag{3. 4.1}
$$

这个方程能够描写有着随机自旋取向的集团吗? 显然答案是不能. (3.4.1) 式表征一个这样的态右矢,它的自旋指向某个确定的方向,即沿 $\widehat{\mathbf{n}}$ 方向,它的极角和方位角分别为 $\beta$ 和 $\alpha$ ,它们可以通过求解

$$
\frac{{c}_{ + }}{{c}_{ - }} = \frac{\cos \left( {\beta /2}\right) }{{e}^{i\alpha }\sin \left( {\beta /2}\right) }, tag{3. 4.2}
$$

得到, 见 (3.2.52) 式.

为了应对这类情况, 我们引入分数布居或概率权重的概念. 一个有着完全随机自旋取向的银原子系综可以看作是这样的一团银原子,其中 ${50}\%$ 的系综成员由 $| + \rangle$ 表征,而其余的 ${50}\%$ 由 $| - \rangle$ 表征. 我们通过指定

$$
{w}_{ + } = {0.5},\;{w}_{ - } = {0.5}, tag{3.4.3}
$$

其中 ${w}_{ + }$ 和 ${w}_{ - }$ 分别为自旋向上和向下的分数布居,来确定这样的一个系综. 因为这样的一束束流不存在任何优势方向,可合理地预期这同一系综也可以被看作 $\left| {{S}_{x}; + }\right\rangle$ 和 ${S}_{x}$ ; 一) 一半对一半的混合. 不久就会见到实现这一点所需要的数学形式.

---

* 这当然依赖于我们的约定: ${S}_{y}$ (或更普遍的 ${J}_{y}$ ) 的矩阵元取为纯虚的.

---

非常重要的是要注意,我们只不过引进了两个实数 ${w}_{ + }$ 和 ${w}_{ - }$ . 没有任何有关自旋向上和自旋向下的右矢间的相对相位信息. 通常, 我们称这样的情况为自旋向上和自旋向下状态的非相干混合. 在这里我们正在做的是要把我们在相干线性叠加时所做的清楚地区分开——例如,

$$
\left( \frac{1}{\sqrt{2}}\right) + \rangle + \left( \frac{1}{\sqrt{2}}\right) - \rangle , tag{3. 4.4}
$$

在那里, $\left| {+\rangle \text{和}}\right| - \rangle$ 之间的相位关系包含了有关在 ${xy}$ 平面中自旋取向的至关重要的信息, 在这种情况下,自旋沿正 $x$ 方向. 一般说来,我们不应把 ${w}_{ + }$ 和 ${w}_{ - }$ 与 ${\left| {c}_{ + }\right| }^{2}$ 和 ${\left| {c}_{ - }\right| }^{2}$ 混淆. 与 ${w}_{ + }$ 和 ${w}_{ - }$ 联系起来的概率概念更靠近经典概率论中遇到的那种. 在处理直接来自热炉子的银原子所遇到的情况可以与一个毕业班中 ${50}\%$ 的毕业生是男生,其余的 ${50}\%$ 是女生的情况相比. 当我们随机地挑选出一个学生时, 这位特定的学生是男生 (或女生) 的概率是 0.5 . 究竟谁听说过一个学生可视为男生与女生以一种特定的相位关系线性相关叠加吗?

直接从炉子飞出来的银原子束流是一个完全随机系综的例子; 该束流被称为非极化的, 因为其自旋取向没有优势方向. 相比之下, 通过了一个有选择的斯特恩-盖拉赫类测量的束流是一个纯系综的例子; 该束流称为极化束, 因为这个系综的所有成员由一个单一的共同右矢所表征, 它描述自旋指向某确定方向的一个态. 为了领会完全随机系综和纯系综之间的区别, 让我们考虑一个可以转动的 SG 仪器, 在那里通过转动该仪器就可以改变非均匀磁场 $\mathbf{B}$ 的方向. 当直接从炉子飞出来的完全非极化的束流遇到这样一个仪器时,不管仪器可能是什么样的取向, 我们总是得到两个强度相等的出射束流. 相比之下, 如果一个极化束流遭遇到这样一个仪器时, 两个出射束流的相对强度随着仪器的转动而变化. 对于某个特定的取向, 实际上的强度之比变为 1 到 0 间的值. 事实上, 第 1 章所发展的形式体系告诉我们,这个相对强度只不过是 ${\cos }^{2}\left( {\beta /2}\right)$ 和 ${\sin }^{2}\left( {\beta /2}\right)$ ,其中 $\beta$ 是原子的自旋方向和 SG 仪器中的非均匀磁场方向之间的夹角.

一个完全随机系综和一个纯系综可以看成所谓混合系综的两个极端. 在一个混合系综中,其成员的某个部分,例如 ${70}\%$ 由一个态右矢 $|\alpha \rangle$ 表征,而其余的 ${30}\%$ 用 $|\beta \rangle$ 表征. 在这样的情况下,该束流称为部分极化. 这里的 $|\alpha \rangle$ 和 $|\beta \rangle$ 甚至不需要是正交的,例如, 有 ${70}\%$ 的自旋沿正 $x$ 方向,而 ${30}\%$ 的自旋沿负 $z$ 方向 *.

\subsection{系综平均和密度算符}

现在介绍冯$\cdot$诺依曼 (J. von Neumann) 在 1927 年首先提出来的密度算符形式, 它可定量地描述纯系综以及混合系综的物理情况. 在这里, 我们的一般性讨论不只限于自旋 $\frac{1}{2}$ 系统,但是为了举例说明的目的,我们会反复回到自旋 $\frac{1}{2}$ 的系统.

一个纯系综被定义为一个这样的物理系统的集合,它的每个成员都由同样的右矢 $|\alpha \rangle$ 表征. 相比之下,在一个混合系综内,相对布居数为 ${w}_{1}$ 的那部分成员由 $\left| {\alpha }^{\left( 1\right) }\right\rangle$ 表征,而相对布居数为 ${w}_{2}$ 的其他部分成员由 $\left| {\alpha }^{\left( 2\right) }\right\rangle$ 表征,等等. 粗略地讲,一个混合系综可以看作纯系综的混合, 正如它的名字所示. 布居数受到满足归一条件的约束, 即

$$
\mathop{\sum }\limits_{i}{w}_{i} = 1 tag{3. 4.5}
$$

---

* 在文献中, 称之为纯的或混合的系综经常指的是纯态和混态. 然而, 在本书中我们使用态意味着用一个确定的态右矢 $|\alpha \rangle$ 描写的物理系统.

---

正如我们在前面提到的, $\left| {\alpha }^{\left( 1\right) }\right\rangle$ 和 $\left| {\alpha }^{\left( 2\right) }\right\rangle$ 是不需要正交的. 此外在 (3.4.5) 式中,对 $i$ 求和中的项数不需要与右矢空间的维数 $N$ 一致,它可以很容易超过 $N$ . 例如,对 $N = 2$ 的自旋 $\frac{1}{2}$ 系统,可以考虑有 ${40}\%$ 的态具有沿正 $z$ 方向的自旋, ${30}\%$ 的态具有沿正 $x$ 方向的自旋,而其余 ${30}\%$ 的态具有沿负 $y$ 方向的自旋.

假定在一个混合系综上测量某可观测量 $A$ . 我们可以问: 当测量的次数很多时, $A$ 的平均测量值是多少. 答案由 $A$ 的系综平均值确定,它被定义为

$$
\left\lbrack A\right\rbrack \equiv \mathop{\sum }\limits_{i}{w}_{i}\left\langle {{\alpha }^{\left( i\right) }\left| A\right| {\alpha }^{\left( i\right) }}\right\rangle tag{3. 4.6}
$$

$$
= \mathop{\sum }\limits_{i}\mathop{\sum }\limits_{{a}^{\prime }}{w}_{i}{\left| \left\langle {a}^{\prime } \mid {\alpha }^{\left( i\right) }\right\rangle \right| }^{2}{a}^{\prime },
$$

其中 $\left| {\alpha }^{\prime }\right\rangle$ 是 $A$ 的一个本征右矢. 回想 $\left\langle {{\alpha }^{\left( i\right) }\left| A\right| {\alpha }^{\left( i\right) }}\right\rangle$ 是通常量子力学中 $A$ 在 $\left| {\alpha }^{\left( i\right) }\right\rangle$ 上的期待值. (3.4.6) 式告诉我们,这些期待值必须再用相应的分数布居数 ${w}_{i}$ 加权. 注意,概率概念是如何进来两次的: 第一次是在 $A$ 的一个本征态 $\left| {\alpha }^{\prime }\right\rangle$ 中找到态 $\left| {\alpha }^{\left( i\right) }\right\rangle$ 的量子力学概率 ${\left| \left\langle {a}^{\prime }\right\rangle {\alpha }^{\left( i\right) }\right| }^{2}$ ,而第二次是在系综中找到一个由 $\left| {\alpha }^{\left( i\right) }\right\rangle$ 表征的量子力学态的概率因子 *.

现在可以利用一个更一般的基 $\left\{ \left| {b}^{\prime }\right\rangle \right\}$ 改写系综平均值 (3.4.6) 式:

$$
\left\lbrack A\right\rbrack = \mathop{\sum }\limits_{i}{w}_{i}\mathop{\sum }\limits_{{b}^{\prime }}\mathop{\sum }\limits_{{b}^{\prime \prime }}\left\langle {{\alpha }^{\left( i\right) }\left| {{b}^{\prime }\rangle \left\langle {{b}^{\prime }\left| A\right| {b}^{\prime \prime }}\right\rangle \left\langle {b}^{\prime \prime }\right| {\alpha }^{\left( i\right) }}\right| }\right\rangle tag{3.4.7}
$$

$$
= \mathop{\sum }\limits_{{b}^{\prime }}\mathop{\sum }\limits_{{b}^{\prime \prime }}\left( {\mathop{\sum }\limits_{i}{w}_{i}\left\langle {{b}^{\prime \prime }\left| {\alpha }^{\left( i\right) }\right\rangle \left\langle {{\alpha }^{\left( i\right) } \mid {b}^{\prime }}\right\rangle }\right\rangle \left\langle {{b}^{\prime }\left| A\right| {b}^{\prime \prime }}\right\rangle }\right) .
$$

在对 ${b}^{\prime }\left( {b}^{\prime \prime }\right)$ 求和中的项数就是右矢空间的维数,而对 $i$ 求和中的项数依赖于这个混合系综是怎么被看作纯系综的混合的. 注意,在这种形式中,不依赖于特定可观测量 $A$ 的系综的基本性质被分离了出来. 这诱使我们把密度算符 $\rho$ 定义如下:

$$
\rho \equiv \mathop{\sum }\limits_{i}{w}_{i}\left| {\alpha }^{\left( i\right) }\right\rangle \left\langle {\alpha }^{\left( i\right) }\right| . tag{3.4.8}
$$

相应的密度矩阵的矩阵元有下列形式:

$$
\left\langle {{b}^{\prime \prime }\left| \rho \right| {b}^{\prime }}\right\rangle = \mathop{\sum }\limits_{i}{w}_{i}\left\langle {{b}^{\prime \prime }\left| {\alpha }^{\left( i\right) }\right\rangle \left\langle {\alpha }^{\left( i\right) }\right| {b}^{\prime }}\right\rangle . tag{3.4.9}
$$

密度算符包含了我们可能得到的, 关于所论及系综的所有有物理意义的信息. 回到 (3.4.7) 式, 系综平均可以写成

$$
\left\lbrack A\right\rbrack = \mathop{\sum }\limits_{{b}^{\prime }}\mathop{\sum }\limits_{{b}^{\prime \prime }}\left\langle {{b}^{\prime \prime }\left| \rho \right| {b}^{\prime }}\right\rangle \left\langle {{b}^{\prime }\left| A\right| {b}^{\prime \prime }}\right\rangle tag{3. 4. 10}
$$

$$
= \operatorname{tr}\left( {\rho A}\right) \text{.}
$$

因为矩阵的迹不依赖于表象, $\operatorname{tr}\left( {\rho A}\right)$ 可以利用任何方便的基计算. 因此,(3.4.10) 式是一个极为有用的关系.

密度算符有两个性质值得记住. 首先, 密度算符是厄米的, 这从 (3.4.8) 式显然可见. 第二, 密度算符满足归一化条件

- 在文献中相当常见, 系综平均也称为期待值. 然而, 在本书中, 术语期待值是指在一个纯系综中进行测量时得到的平均测量值.

$$
\operatorname{tr}\left( \rho \right) = \mathop{\sum }\limits_{i}\mathop{\sum }\limits_{{b}^{\prime }}{w}_{i}\left\langle {{b}^{\prime }\left| {{\alpha }^{\left( i\right) }\rangle \left\langle {{\alpha }^{\left( i\right) } \mid {b}^{\prime }}\right\rangle }\right| }\right\rangle
$$

$$
= \mathop{\sum }\limits_{i}{w}_{i}\left\langle {{\alpha }^{\left( i\right) } \mid {\alpha }^{\left( i\right) }}\right\rangle tag{3. 4.11}
$$

$$
= 1\text{.}
$$

由于厄米性和归一条件,对于维数为 2 的自旋 $\frac{1}{2}$ 的系统,密度算符或相应的密度矩阵由三个实参量表征. 4 个实数可以表征一个 $2 \times 2$ 厄米矩阵. 然而,由于归一化条件,只有 3 个是独立的. 所需要的 3 个数是 $\left\lbrack {S}_{x}\right\rbrack ,\left\lbrack {S}_{y}\right\rbrack$ 和 $\left\lbrack {S}_{z}\right\rbrack$ ,读者可以证明知道了这三个系综平均值, 足以重建密度算符. 一个混合系综形成的方式可以是相当复杂的. 我们可以把有所有类型的 $\left| {\alpha }^{\left( i\right) }\right\rangle$ 所表征的纯系综用适当的 ${w}_{i}$ 混合起来; 而对自旋 $\frac{1}{2}$ 的系统,三个实数完全可以表征所涉及的系综. 这强烈地暗示, 一个混合系综能以许多种不同的方式分解为纯系综. 在本章末尾有一个习题说明了这一点.

一个纯系综可以通过对某个 $\left| {\alpha }^{\left( i\right) }\right\rangle$ 一一例如, $i = n$ 时, ${w}_{i} = 1$ ,而对所有的其他可能的态右矢 ${w}_{i} = 0$ 来确定,因此相应的密度算符可写成不求和的

$$
\rho = \left| {\alpha }^{\left( n\right) }\right\rangle \left\langle {\alpha }^{\left( n\right) }\right| tag{3. 4.12}
$$

显然, 一个纯系综的密度算符是幂等的. 即

$$
{\rho }^{2} = \rho tag{3. 4.13}
$$

或, 等价地

$$
\rho \left( {\rho - 1}\right) = 0. tag{3. 4.14}
$$

于是, 除了 (3.4.11) 式以外, 只对纯系综有

$$
\operatorname{tr}\left( {\rho }^{2}\right) = 1\text{.} tag{3. 4.15}
$$

纯系综密度算符的本征值是 0 或 1,这一点可以通过在 (3.4.14) 式的 $\rho$ 和 $\rho - 1$ 之间插入一个基右矢完全集合看到,该集合能对角化厄米算符 $\rho$ . 因此,当对角化时,一个纯系综的密度矩阵看起来一定像

$$
\rho \doteq \left( \begin{matrix} 0 & & & & & & & & 0 \\ & 0 & & & & & & & \\ & & \ddots & & & & & & \\ & & & 0 & & & & & \\ & & & & 1 & & & & \\ & & & & & 0 & & & \\ & & & & & & 0 & & \\ & & & & & & & 0 & \\ & & & & & & & & \ddots \\ 0 & & & & & & & & 0 \end{matrix}\right) \text{ (对角形式) } tag{3. 4.16}
$$

可以证明,当系综是纯的时候, $\operatorname{tr}\left( {\rho }^{2}\right)$ 取最大值; 对于一个混合系综, $\operatorname{tr}\left( {\rho }^{2}\right)$ 是一个小于 1 的正数.

给定一个密度算符, 让我们看一下在一些具体的基中, 怎样构建相应的密度矩阵. 为此, 首先回顾

$$
\left| {\alpha \rangle \langle \alpha }\right| = \mathop{\sum }\limits_{{b}^{\prime }}\mathop{\sum }\limits_{{b}^{\prime \prime }}\left| {b}^{\prime }\right\rangle \left\langle {{b}^{\prime } \mid \alpha }\right\rangle \left\langle {\alpha \mid {b}^{\prime \prime }}\right\rangle \left\langle {b}^{\prime \prime }\right| . tag{3. 4.17}
$$

这表明,在外积的意义上,可以通过把 $\left\langle {{b}^{\prime } \mid {\alpha }^{\left( i\right) }}\right\rangle$ 形成的列矩阵与 $\left\langle {{\alpha }^{\left( i\right) } \mid {b}^{\prime \prime }}\right\rangle$ (当然,它等于 $\left\langle {{b}^{\prime \prime } \mid {\alpha }^{\left( i\right) }{\rangle }^{ * }\text{. }}\right\rangle$ 形成的行矩阵组合在一起,形成一个对应于 $\left| {{\alpha }^{\left( i\right) }\rangle \left\langle {\alpha }^{\left( i\right) }\right. }\right|$ 的方阵. 最后一步,就像 (3.4.8) 式指出的,用权重因子 ${w}_{i}$ 对这样的方阵求和. 正如所期待的,最后的形式与 (3.4.9) 式一致.

研究几个例子是很有益的,它们都涉及自旋 $\frac{1}{2}$ 的系统.

例 3.1 一束 ${S}_{z} +$ 的完全极化束流

$$
\rho = \left| {+\rangle \langle + }\right| \pm \left( \begin{array}{l} 1 \\ 0 \end{array}\right) \left( {1,0}\right) tag{3. 4.18}
$$

$$
= \left( \begin{array}{ll} 1 & 0 \\ 0 & 0 \end{array}\right)
$$

例 3.2 一束 ${S}_{x} \pm$ 的完全极化束流

$$
\rho = \left| {{S}_{x}; \pm }\right\rangle \left\langle {{S}_{x}; \pm }\right| = \left( \frac{1}{\sqrt{2}}\right) \left( {\left| {+\rangle \pm }\right| - \rangle }\right) \left( \frac{1}{\sqrt{2}}\right) \left( {\langle + \left| {\pm \langle - }\right| }\right)
$$

$$
\doteq \left( \begin{matrix} \frac{1}{2} & \pm \frac{1}{2} \\ \pm \frac{1}{2} & \frac{1}{2} \end{matrix}\right) tag{3. 4.19}
$$

例 3.1 和 3.2 的系综都是纯的.

例 3.3 一束非极化束流. 它可以看作一个自旋向上的系综和一个自旋向下的系综以相等权重 (各为 ${50}\%$ ) 非相干混合:

$$
\rho = \left( \frac{1}{2}\right) \left| {+\rangle \langle + }\right| + \left( \frac{1}{2}\right) \left| {-\rangle \langle - }\right|
$$

$$
\doteq \left( \begin{array}{ll} \frac{1}{2} & 0 \\ 0 & \frac{1}{2} \end{array}\right) , tag{3. 4.20}
$$

它正好是一个单位矩阵除以 2 . 正如早些时候我们曾指出的, 同样的这个系综还可以看作是一种具有相等权重的 ${S}_{x}$ 十系综和 ${S}_{x}$ 一系综的非相干混合. 令人欣慰的是,我们的形式自动地满足预期

$$
\left( \begin{matrix} \frac{1}{2} & 0 \\ 0 & \frac{1}{2} \end{matrix}\right) = \frac{1}{2}\left( \begin{matrix} \frac{1}{2} & \frac{1}{2} \\ \frac{1}{2} & \frac{1}{2} \end{matrix}\right) + \frac{1}{2}\left( \begin{matrix} \frac{1}{2} & - \frac{1}{2} \\ - \frac{1}{2} & \frac{1}{2} \end{matrix}\right) , tag{3. 4.21}
$$

由例 3.2 可以看到,右手边的两项就是 ${S}_{x}$ 十和 ${S}_{x}$ 一两个纯系综的密度矩阵. 因为这种情况下的 $\rho$ 正是单位算符除以 2 (维数),我们有

$$
\operatorname{tr}\left( {\rho {S}_{x}}\right) = \operatorname{tr}\left( {\rho {S}_{y}}\right) = \operatorname{tr}\left( {\rho {S}_{z}}\right) = 0, tag{3. 4.22}
$$

其中用到了 ${S}_{k}$ 是无迹的. 于是,对于 $\mathbf{S}$ 的系综平均值,我们有

$$
\left\lbrack \mathbf{S}\right\rbrack = 0. tag{3. 4.23}
$$

这是合理的,因为在一个自旋 $\frac{1}{2}$ 系统的完全随机系综中不应当存在自旋的优势方向.

例 3.4 作为部分极化束的例子,考虑两个纯系综的 75-25 混合,一个是 ${S}_{z} +$ 而另一

个是 ${S}_{x} +$ :

$$
w\left( {{S}_{z} + }\right) = {0.75},\;w\left( {{S}_{x} + }\right) = {0.25}. tag{3. 4.24}
$$

相应的 $\rho$ 可以表示为

$$
\rho \doteq \frac{3}{4}\left( \begin{array}{ll} 1 & 0 \\ 0 & 0 \end{array}\right) + \frac{1}{4}\left( \begin{array}{ll} \frac{1}{2} & \frac{1}{2} \\ \frac{1}{2} & \frac{1}{2} \end{array}\right) tag{3. 4.25}
$$

$$
= \left( \begin{array}{ll} \frac{7}{8} & \frac{1}{8} \\ \frac{1}{8} & \frac{1}{8} \end{array}\right) ,
$$

从上式得到

$$
\left\lbrack {S}_{x}\right\rbrack = \frac{\hslash }{8},\;\left\lbrack {S}_{y}\right\rbrack = 0,\;\left\lbrack {S}_{z}\right\rbrack = \frac{3\hslash }{8}. tag{3. 4.26}
$$

把证明这个系综能以异于 (3.4.24) 式的方式分解作为一个练习留给读者.

\subsection{系综的时间演化}

作为一个时间函数,密度算符 $\rho$ 如何变化? 让我们假定在某个 ${t}_{0}$ 时刻密度算符由下式给定

$$
\rho \left( {t}_{0}\right) = \mathop{\sum }\limits_{i}{w}_{i}\left| {\alpha }^{\left( i\right) }\right\rangle \left\langle {\alpha }^{\left( i\right) }\right| . tag{3. 4.27}
$$

如果系综保持不受扰动,我们就不可能改变分数布居数 ${w}_{i}$ . 所以 $\rho$ 的变化只是由态右矢 $\left| {\alpha }^{\left( i\right) }\right\rangle$ 的时间演化所控制:

$$
{t}_{0}\text{时}\left. {\left. {\alpha }^{\left( i\right) }\right\rangle \rightarrow \mid {\alpha }^{\left( i\right) },{t}_{0};t}\right\rangle \text{.} tag{3. 4.28}
$$

由于 $\left| {{\alpha }^{\left( i\right) },{t}_{0};t}\right\rangle$ 满足薛定谔方程,我们得到

$$
i\hslash \frac{\partial \rho }{\partial t} = \mathop{\sum }\limits_{i}{w}_{i}\left( {H\left| {{\alpha }^{\left( i\right) },{t}_{0};t}\right\rangle \left\langle {{\alpha }^{\left( i\right) },{t}_{0};t}\right\rangle - \left| {{\alpha }^{\left( i\right) },{t}_{0};t}\right\rangle \left\langle {{\alpha }^{\left( i\right) },{t}_{0};t}\right| H}\right) tag{3. 4.29}
$$

$$
= - \left\lbrack {\rho, H}\right\rbrack \text{.}
$$

除了符号相反,该公式的样子很像海森伯运动方程! 这一点不会带来烦恼,因为 $\rho$ 不是一个海森伯绘景的动力学可观测量. 相反, $\rho$ 是由薛定谔绘景的态右矢和态左矢构成的,它们的时间演化都是遵从薛定谔方程的.

有意思的是, (3.4.29) 式可以看作是经典统计力学中刘维尔定理

$$
\frac{\partial {\rho }_{\text{经典 }}}{\partial t} = - {\left\lbrack {\rho }_{\text{经典 }}, H\right\rbrack }_{\text{经典 }} tag{3.4.30}
$$

的量子力学类似定理,其中的 ${\rho }_{\text{经典 }}$ 是相空间中代表点的密度 *. 因此,对出现在 (3.4.29) 式中的 $\rho$ ,密度算符的名字的确是合适的. 对于某个可观测量 $A$ 的系综平均值,(3.4.10) 式的经典类似公式由下式给定:

$$
{A}_{\text{平均 }} = \frac{\int {\rho }_{\text{经典 }}A\left( {q, p}\right) d{\Gamma }_{q, p}}{\int {\rho }_{\text{经典 }}d{\Gamma }_{q, p}}, tag{3. 4.31}
$$

---

* 记住,一个经典的纯态是用相空间 $\left( {{q}_{1},\cdots ,{q}_{f},{p}_{1},\cdots ,{p}_{f}}\right)$ 中每个时刻的一个单独运动的点表示. 另一方面,一个经典的统计态用我们的非负的密度函数 ${\rho }_{\text{轻典 }}\left( {{q}_{1},\cdots ,{q}_{f},{p}_{1},\cdots ,{p}_{f}, t}\right)$ 描写,它使得在 $t$ 时刻在间隔 $d{q}_{1},\cdots, d{p}_{f}$ 内找到一个系统的概率是 ${\rho }_{\text{经典 }}d{q}_{1},\cdots, d{p}_{f}$ .

---

其中的 $d{\Gamma }_{q.p}$ 表示相空间中的体积元.

\subsection{连续的推广} 

至此, 已经考虑了右矢空间中的密度算符, 在那里基右矢是用某个可观测量的分立的本征值标记的. 密度矩阵的概念可以推广到使用的基右矢用连续的本征值标记. 特别是, 让我们考虑由位置本征右矢 $\left| {\mathbf{x}}^{\prime }\right\rangle$ 所张的右矢空间. (3.4.10) 式的类似表达式由下式给定

$$
\left\lbrack A\right\rbrack = \int {d}^{3}{x}^{\prime }\int {d}^{3}{x}^{\prime \prime }\left\langle {{\mathbf{x}}^{\prime \prime }\left| \rho \right| {\mathbf{x}}^{\prime }}\right\rangle \left\langle {{\mathbf{x}}^{\prime }\left| A\right| {\mathbf{x}}^{\prime \prime }}\right\rangle . tag{3. 4.32}
$$

这里的密度矩阵实际上是 ${\mathbf{x}}^{\prime }$ 和 ${\mathbf{x}}^{\prime \prime }$ 的一个函数,即

$$
\left\langle {{\mathbf{x}}^{\prime \prime }\left| \rho \right| {\mathbf{x}}^{\prime }}\right\rangle = \left\langle {{\mathbf{x}}^{\prime \prime }\left| \left( {\mathop{\sum }\limits_{i}{w}_{i}\left| {\alpha }^{\left( i\right) }\right\rangle \left\langle {\alpha }^{\left( i\right) }\right| }\right) \right| {\mathbf{x}}^{\prime }}\right\rangle tag{3. 4.33}
$$

$$
= \mathop{\sum }\limits_{i}{w}_{i}{\psi }_{i}\left( {\mathbf{x}}^{\prime \prime }\right) {\psi }_{i}^{ * }\left( {\mathbf{x}}^{\prime }\right) ,
$$

其中 ${\psi }_{i}$ 是相应于态右矢 $\left| {\alpha }^{\left( i\right) }\right\rangle$ 的波函数. 注意,该矩阵的对角元 (即 ${\mathbf{x}}^{\prime } = {\mathbf{x}}^{\prime \prime }$ ) 正是概率密度加权后的求和. 再次印证了术语密度矩阵的确是合适的.

在连续态情况下也一样, 重要的是要记住同一个混合系综可以按不同方式分解为纯系综. 例如, 一个 “真实的” 粒子束流既可以看作平面波态 (单能量的自由粒子态) 的混合也可以看作波包态的混合.

\subsection{量子统计力学}

我们以简略讨论密度算符公式形式与统计力学之间的联系结束这一节. 让我们先记下完全随机的系综和纯系综的一些性质. 一个完全随机系综的密度矩阵在任何表象看起来都像

$$
\rho \doteq \frac{1}{N}\left( \begin{array}{llllll} 1 & & & & 0 & \\ & 1 & & & & \\ & & 1 & & & \\ & & & \ddots & & \\ & & & & 1 & \\ & & & & & 1 \\ 0 & & & & & 1 \end{array}\right) tag{3. 4.34}
$$

[比较例 3.3 与 (3.4.20) 式]. 这从对应于写出密度矩阵所用的基右矢的所有的态是同等布居的就可以看到. 相比之下,在 $\rho$ 是对角化的基中,对一个纯系综密度算符的矩阵表示我们有 (3.4.16) 式. 而 (3.4.34) 和 (3.4.16) 这两个对角矩阵, 都满足归一化要求 (3.4.11), 它们看上去不可能有太大差别. 假如我们能够以某种方式构建一个量, 表征这种巨大差异的话, 将会是我们所希望的.

因此, 用

$$
\sigma = - \operatorname{tr}\left( {\rho \ln \rho }\right) . tag{3. 4.35}
$$

定义一个称为 $\sigma$ 的量. 算符 $\rho$ 的对数似乎十分令人生畏,但是如果采用使 $\rho$ 对角化的基, 则 (3.4.35) 式的意义就十分清楚了:

$$
\sigma = - \mathop{\sum }\limits_{k}{\rho }_{kk}^{\left( \text{ 对角 }\right) }\ln {\rho }_{kk}^{\left( \text{ 对角 }\right) }. tag{3. 4.36}
$$

由于每个元素 ${\rho }_{kk}^{\left( \text{ 对 }\right) }$ 都是 0 和 1 之间的实数, $\sigma$ 必然是半正定的. 对于一个完全随机的系综 (3.4.34), 有

$$
\sigma = - \mathop{\sum }\limits_{{k = 1}}^{N}\frac{1}{N}\ln \left( \frac{1}{N}\right) = \ln N. tag{3. 4.37}
$$

与之相比, 对于纯系综 (3.4.16), 有

$$
\sigma = 0 tag{3. 4.38}
$$

其中对 (3.4.36) 式的每一项, 都使用了

$$
{\rho }_{kk}^{\left( \text{ 对角 }\right) } = 0\;\text{ 或 }\;\ln {\rho }_{kk}^{\left( \text{ 对角 }\right) } = 0 tag{3. 4.39}
$$

现在从物理上论证, $\sigma$ 可以被看作一个无序的定量测度. 一个纯系综是一个最大程度的有序系综, 因为它的所有的成员都用同样的量子力学的态右矢表征, 它很像在一支训练有素部队中齐步走的士兵. 按照 (3.4.38) 式,对这样的一个系综, $\sigma$ 为零. 在另一个极端, 一个完全随机系综, 其中所有的量子力学态是同等可能的, 它可以比作一群喝醉了酒的士兵四面八方地游荡. 根据 (3.4.37) 式, $\sigma$ 很大. 的确,稍后将证明,在归一条件

$$
\mathop{\sum }\limits_{k}{\rho }_{kk} = 1 tag{3.4.40}
$$

的约束下, $\sigma$ 最大的可能值是 $\ln N$ . 在热力学中我们知道,一个称为熵的量是量度无序的. 结果是, $\sigma$ 与表示系综每个成员的熵 $S$ 通过

$$
S = {k\sigma } tag{3. 4.41}
$$

相关联,其中 $k$ 是一个可确定为玻尔兹曼 (Boltzman) 常数的普适常数. 事实上, (3.4.41) 式可以作为量子统计力学中熵的定义.

现在证明,对于一个处于热平衡的系综,怎样可以求得密度算符 $\rho$ . 所做的基本假定是,在哈密顿量的系综平均值有某一指定值的约束下,自然界趋向于把 $\sigma$ 极大化. 证明这个假定的正确性, 将使我们卷入一场关于平衡是如何作为与环境相互作用的结果而建立起来的微妙的讨论, 它已超出了本书的范围. 不管怎么样, 一旦热平衡得以确立, 我们预期

$$
\frac{\partial \rho }{\partial t} = 0 tag{3. 4.42}
$$

由于 (3.4.29) 式,上式意味着 $\rho$ 与 $H$ 可以同时对角化. 所以,在写出 (3.4.36) 式时使用的右矢可以被取作能量本征右矢. 采用这一选择, ${\rho }_{kk}$ 则表示一个能量本征值为 ${E}_{k}$ 的能量本征态的分数布居数.

通过要求

$$
{\delta \sigma } = 0 tag{3. 4.43}
$$

极大化 $\sigma$ . 然而,必须考虑 $H$ 的系综平均值具有某一指定值的约束. 用统计力学的说法, $\left\lbrack H\right\rbrack$ 被确定为每个组分的内能,用 $U$ 表示

$$
\left\lbrack H\right\rbrack = \operatorname{tr}\left( {\rho H}\right) = U. tag{3. 4.44}
$$

此外, 不应该忘记归一化约束 (3.4.40) 式. 因此, 基本任务是要求 (3.4.43) 式受到如下两个约束:

$$
\delta \left\lbrack H\right\rbrack = \mathop{\sum }\limits_{k}\delta {\rho }_{kk}{E}_{k} = 0
$$

(3.4. ${45a}$ )

和

$$
\delta \left( {\operatorname{tr}\rho }\right) = \mathop{\sum }\limits_{k}\delta {\rho }_{kk} = 0.
$$

(3. 4. ${45}\mathrm{\;b}$ )

可使用拉格朗日乘子法快捷地完成这个任务. 得到

$$
\mathop{\sum }\limits_{k}\delta {\rho }_{kk}\left\lbrack {\left( {\ln {\rho }_{kk} + 1}\right) + \beta {E}_{k} + \gamma }\right\rbrack = 0, tag{3.4.46}
$$

且仅当

$$
{\rho }_{kk} = \exp \left( {-\beta {E}_{k} - \gamma - 1}\right) . tag{3. 4.47}
$$

时,它才对任意变分都是可能的. 常数 $\gamma$ 可以用归一条件 (3.4.40) 式消掉,最终结果为

$$
{\rho }_{kk} = \frac{\exp \left( {-\beta {E}_{k}}\right) }{\mathop{\sum }\limits_{l}^{N}\exp \left( {-\beta {E}_{l}}\right) }. tag{3. 4.48}
$$

它直接给出一个能量本征值为 ${E}_{k}$ 的能量本征态的分数布居数. 要始终理解求和是对分立的能量本征态进行的; 如果有简并存在, 还必须对具有相同能量本征值的态求和.

密度矩阵元 (3.4.48) 式对于统计力学中所谓的正则系综是适用的. 假如试图在没有内能约束 (3.4.45a) 的条件下极大化 $\sigma$ ,就会相反地得到

$$
{\rho }_{kk} = \frac{1}{N},\;\left( {\text{ 不依赖于 }k}\right) , tag{3.4.49}
$$

它是适用于一个完全随机系综的密度矩阵元. 把 (3.4.48) 式与 (3.4.49) 式比较, 我们推断,一个完全随机的系综可以看作一个正则系综在 $\beta \rightarrow 0$ 时的极限 (即物理上高温极限). (3.4.48) 式的分母为统计力学中的配分函数

$$
Z = \mathop{\sum }\limits_{k}^{N}\exp \left( {-\beta {E}_{k}}\right) tag{3. 4.50}
$$

它还可以写成

$$
Z = \operatorname{tr}\left( {e}^{-{\beta H}}\right) . tag{3. 4.51}
$$

知道了在能量基中给出的 ${\rho }_{kk}$ ,可以把密度算符写成

$$
\rho = \frac{{e}^{-{\beta H}}}{Z}. tag{3. 4.52}
$$

这是最基本的方程,所需的一切都可以从它得出来. 可以立即计算任何一个可观测量 $A$ 的系综平均值:

$$
\left\lbrack A\right\rbrack = \frac{\operatorname{tr}\left( {{e}^{-{\beta H}}A}\right) }{Z}
$$

$$
= \frac{\left\lbrack \mathop{\sum }\limits_{k}^{N}\langle A{\rangle }_{k}\exp \left( -\beta {E}_{k}\right) \right\rbrack }{\mathop{\sum }\limits_{k}^{N}\exp \left( {-\beta {E}_{k}}\right) }. tag{3. 4.53}
$$

特别是, 求得每组分的内能为

$$
U = \frac{\left\lbrack \mathop{\sum }\limits_{k}^{N}{E}_{k}\exp \left( -\beta {E}_{k}\right) \right\rbrack }{\mathop{\sum }\limits_{k}^{N}\exp \left( {-\beta {E}_{k}}\right) } tag{3. 4.54}
$$

$$
= - \frac{\partial }{\partial \beta }\left( {\ln Z}\right) ,
$$

它是每个统计力学的学生都熟知的一个公式.

参量 $\beta$ 与温度 $T$ 的关系如下

$$
\beta = \frac{1}{kT}, tag{3. 4. 55}
$$

其中 $k$ 是玻尔兹曼常数. 通过把简谐振子系综平均 $\left\lbrack H\right\rbrack$ 与在经典极限下内能的预期值 ${kT}$ 相比较,确信这个关系是有益的,这被留作一个练习. 已经阐明在高温极限下,一个正则系综变成一个所有的能量本征态被同等布居的完全随机的系综. 在相反的低温极限下 $\left( {\beta \rightarrow \infty }\right) ,\left( {3.4.48}\right)$ 式告诉我们,一个正则系综变成一个只有基态被布居的纯系综.

作为一个例证考虑一个自旋 $\frac{1}{2}$ 的系统组成的正则系综. 它的每一个成员都具有 ${eh}/$ $2{m}_{e}c$ 的磁矩,受到一个沿 $z$ 方向均匀磁场的作用. 与该问题相关的哈密顿量已经给出 [见 (3.2.16) 式]. 因为 $H$ 和 ${S}_{z}$ 对易,这个正则系综的密度矩阵在 ${S}_{z}$ 基中是对角的. 于是

$$
\rho \doteq \frac{\left( \begin{matrix} {e}^{-\beta \hslash \omega /2} & 0 \\ 0 & {e}^{\beta \hslash \omega /2} \end{matrix}\right) }{Z}, tag{3. 4.56}
$$

其中的配分函数恰为

$$
Z = {e}^{-\beta \hslash \omega /2} + {e}^{\beta \hslash \omega /2}. tag{3. 4.57}
$$

由此得出

$$
\left\lbrack {S}_{x}\right\rbrack = \left\lbrack {S}_{y}\right\rbrack = 0,\;\left\lbrack {S}_{z}\right\rbrack = - \left( \frac{\hslash }{2}\right) \tanh \left( \frac{\beta \hslash \omega }{2}\right) . tag{3. 4.58}
$$

磁矩分量的系综平均值正是 $e/{m}_{e}c$ 乘以 $\left\lbrack {S}_{z}\right\rbrack$ . 顺磁磁化率 $\chi$ 可由下式计算

$$
\left( \frac{e}{{m}_{e}c}\right) \left\lbrack {S}_{z}\right\rbrack = {\chi B} tag{3. 4.59}
$$

用这种方法得到 $\chi$ 的布里渊公式:

$$
\chi = \left( \frac{\left| e\right| \hslash }{2{m}_{e}{cB}}\right) \tanh \left( \frac{\beta \hslash \omega }{2}\right) . tag{3. 4.60}
$$

\section{角动量的本征值和本征态}

到现在为止. 关于角动量的讨论仅限于维数 $N = 2$ 的自旋 $\frac{1}{2}$ 系统. 在这一节以及其后的几节,研究更一般的角动量态. 为此,首先求出 ${\mathbf{J}}^{2}$ 和 ${J}_{z}$ 的本征值和本征右矢,然后推导角动量算符矩阵元的表示式, 它们最早由玻恩, 海森伯和约当在 1926 年的文章中给出.

\subsection{对易关系和阶梯算符}

将要做的每一件事都是由角动量对易关系 (3.1.20) 式推出来的, 在那里我们可回顾 ${J}_{i}$ 被定义为无穷小转动的生成元. 从这些基本对易关系导出的第一个重要性质是存在一个新的算符 ${\mathbf{J}}^{2}$ ,它的定义为

$$
{\mathbf{J}}^{2} \equiv {J}_{x}{J}_{x} + {J}_{y}{J}_{y} + {J}_{z}{J}_{z}, tag{3.5.1}
$$

它与每一个 ${J}_{k}$ 都对易:

$$
\left\lbrack {{\mathbf{J}}^{2},{J}_{k}}\right\rbrack = 0,\;\left( {k = 1,2,3}\right) . tag{3.5.2}
$$

为了证明上式,看一下 $k = 3$ 的情况:

$$
\left\lbrack {{J}_{x}{J}_{x} + {J}_{y}{J}_{y} + {J}_{z}{J}_{z},{J}_{z}}\right\rbrack = {J}_{x}\left\lbrack {{J}_{x},{J}_{z}}\right\rbrack + \left\lbrack {{J}_{x},{J}_{z}}\right\rbrack {J}_{x} + {J}_{y}\left\lbrack {{J}_{y},{J}_{z}}\right\rbrack + \left\lbrack {{J}_{y},{J}_{z}}\right\rbrack {J}_{y}
$$

$$
= {J}_{x}\left( {-i\hslash {J}_{y}}\right) + \left( {-i\hslash {J}_{y}}\right) {J}_{x} + {J}_{y}\left( {i\hslash {J}_{x}}\right) + \left( {i\hslash {J}_{x}}\right) {J}_{y}
$$

$$
= 0\text{.} tag{3.5.3}
$$

对 $k = 1$ 或 2 情况的证明可以从指标的循环置换 $\left( {1 \rightarrow 2 \rightarrow 3 \rightarrow 1}\right)$ 得到. 因为 ${J}_{x},{J}_{y}$ 和 ${J}_{z}$ 彼此不对易,所以只能选其中之一作为与 ${\mathbf{J}}^{2}$ 同时对角化的可观测量. 按照惯例,我们选取 $J$ ...

接下来求 ${\mathbf{J}}^{2}$ 和 $J$ 。的共同本征右矢. 用 $a$ 和 $b$ 分别代表 ${\mathbf{J}}^{2}$ 和 $J$ 。的本征值:

$$
{\mathbf{J}}^{2}\left| {a, b\rangle = a}\right| a, b\rangle
$$

(3. 5. ${4a}$ )

$$
{J}_{z}\left| {a, b\rangle = b}\right| a, b\rangle
$$

(3. 5. $4\mathrm{\;b}$ )

要确定 $a$ 和 $b$ 的允许值,最方便的是使用被称为阶梯算符的非厄米算符

$$
{J}_{ \pm } \equiv {J}_{x} \pm i{J}_{y} tag{3.5.5}
$$

而不用 ${J}_{x}$ 和 ${J}_{y}$ . 它们满足对易关系

$$
\left\lbrack {{J}_{ + },{J}_{ - }}\right\rbrack = {2h}{J}_{z}
$$

(3.5. ${6a}$ )

和

$$
\left\lbrack {{J}_{ : },{J}_{ \pm }}\right\rbrack = \pm \hslash {J}_{ \pm },
$$

(3. 5. $6\mathrm{\;b}$ )

上述二式可以很容易由 (3.1.20) 式得到. 还要注意

$$
\left\lbrack {{\mathbf{J}}^{2},{J}_{ \pm }}\right\rbrack = 0, tag{3.5.7}
$$

它是 (3.5.2) 式的明显结果.

${J}_{ \pm }$ 的物理意义是什么呢? 要回答这个问题,先考查 ${J}_{z}$ 如何作用于 ${J}_{ \pm }|a, b\rangle$ :

$$
{J}_{z}\left( {{J}_{ \pm }|a, b\rangle }\right) = \left( {\left\lbrack {{J}_{z},{J}_{ \pm }}\right\rbrack + {J}_{ \pm },{J}_{z}}\right) |a, b\rangle tag{3.5.8}
$$

$$
= \left( {b \pm \hslash }\right) \left( {{J}_{ \pm }|a, b\rangle }\right) ,
$$

其中用到了 (3.5.6b) 式. 换句话说,如果把 ${J}_{ + }\left( {J}_{ - }\right)$ 作用于 ${J}_{z}$ 的一个本征右矢上,作为结果的右矢仍然是 ${J}_{z}$ 的一个本征右矢,除了其本征值增加 (减少) 了一个 $h$ 单位. 所以现在明白了为什么 ${J}_{ \pm }$ 一它在 ${J}_{ \mp }$ 本征值的 “阶梯”上向上(向下)迈了一步——以阶梯算符著称.

现在先离题回忆一下, (3.5.6b) 式中的对易关系让人想起在前几章遇到的一些对易关系. 在讨论平移算符 $\mathcal{T}\left( 1\right)$ 时,我们有

$$
\left\lbrack {{x}_{i},\mathcal{T}\left( \mathbf{I}\right) }\right\rbrack = {l}_{i}\mathcal{T}\left( \mathbf{I}\right) , tag{3.5.9}
$$

而在讨论简谐振子时, 有

$$
\left\lbrack {N,{a}^{ \dagger }}\right\rbrack = {a}^{ \dagger },\;\left\lbrack {N, a}\right\rbrack = - a. tag{3. 5.10}
$$

可以看到 (3.5.9) 式和 (3.5.10) 式都有着类似于 (3.5.6b) 式的结构. 平移算符的物理解释是,它把位置算符 $\mathbf{x}$ 的本征值改变了 1,这种作用方式与阶梯算符 ${J}_{ + }$ 使 ${J}_{z}$ 本征值的改变了一个单位 $h$ 的方式差不多相同. 同样地,简谐振子的产生算符 ${a}^{ \dagger }$ 使粒子数算符 $N$ 的本征值增加了一个单位.

尽管 ${J}_{ \pm }$ 使 ${J}_{ \pm }$ 的本征值改变了一个 $h$ 的单位,它并不改变 ${\mathbf{J}}^{2}$ 的本征值:

$$
{\mathbf{J}}^{2}\left( {{J}_{ \pm }|a, b\rangle }\right) = {J}_{ \pm }{\mathbf{J}}^{2}|a, b\rangle tag{3. 5.11}
$$

$$
= a\left( {{J}_{ \pm }|a, b\rangle }\right) ,
$$

其中用到了 (3.5.7) 式. 总而言之, ${J}_{ \pm }|a, b\rangle$ 是 ${\mathbf{J}}^{2}$ 和 ${J}_{z}$ 的共同本征右矢,其本征值为 $a$ 和 $b \pm h$ . 可以写成

$$
{J}_{ \pm }\left| {a, b\rangle = {c}_{ \pm }}\right| a, b \pm \hslash \rangle , tag{3. 5.12}
$$

其中比例常数 ${c}_{ \pm }$ 将稍后由角动量本征右矢的归一化要求确定.

\subsection{${\mathbf{J}}^{2}$ 和 ${\mathbf{J}}_{z}$ 的本征值}

现在有了构造角动量本征右矢并研究它们的本征值谱所需要的工具. 假定我们把 ${J}_{ + }$ 连续地,比如 $n$ 次,作用于 ${\mathbf{J}}^{2}$ 和 ${J}_{z}$ 的共同本征右矢上. 则得到 ${\mathbf{J}}^{2}$ 和 ${J}_{z}$ 的另一个本征右矢,其 ${J}_{z}$ 本征值增加了 $n\hslash$ ,同时 ${\mathbf{J}}^{2}$ 的本征值不变. 然而这个过程不可能无限地继续下去. 结果是,对于一个给定的 $\left( {\mathrm{J}}^{2}\right.$ 本征值) $a$ ,存在一个 $b\left( {J}_{z}\right.$ 本征值) 的上限:

$$
a \geq {b}^{2}\text{.} tag{3.5.13}
$$

为证明这一说法, 首先注意

$$
{\mathbf{J}}^{2} - {J}_{z}^{2} = \frac{1}{2}\left( {{J}_{ + }{J}_{ - } + {J}_{ - }{J}_{ + }}\right) tag{3. 5.14}
$$

$$
= \frac{1}{2}\left( {{J}_{ + }{J}_{ + }^{ + } + {J}_{ + }^{ + }{J}_{ + }}\right) .
$$

既然因为

$$
{J}_{ + }^{ + }\left| {a, b\rangle \overset{DC}{ \leftrightarrow }\left\langle {a, b}\right| {J}_{ + },\;{J}_{ + }}\right| a, b\rangle \overset{DC}{ \leftrightarrow }\langle a, b \mid {J}_{ + }^{ + } tag{3. 5.15}
$$

${J}_{ + }{J}_{ + }^{ + }$ 和 ${J}_{ + }^{ + }{J}_{ + }$ 必须有非负的期待值,于是

$$
\left\langle {a, b\left| \left( {{\mathbf{J}}^{2} - {J}_{z}^{2}}\right) \right| a, b}\right\rangle \geq 0, tag{3. 5.16}
$$

反过来,它暗含着 (3.5.13) 式. 因此,一定存在一个 ${b}_{最大}$ ,使

$$
{J}_{ + }\left| {a,{b}_{\text{最大 }}}\right\rangle = 0. tag{3.5.17}
$$

换句话说, $b$ 的本征值不可能增加到超过 ${b}_{\mathbf{R} \star }$ 的值. 现在,(3.5.17) 还意味着

$$
{J}_{ - }{J}_{ + }\left| {a,{b}_{\text{最大 }}}\right\rangle = 0. tag{3. 5.18}
$$

但是

$$
{J}_{ - }{J}_{ + } = {J}_{x}^{2} + {J}_{y}^{2} - i\left( {{J}_{y}{J}_{x} - {J}_{x}{J}_{y}}\right) tag{3.5.19}
$$

$$
= {\mathbf{J}}^{2} - {J}_{z}^{2} - \hslash {J}_{z},
$$

因此

$$
\left( {{\mathbf{J}}^{2} - {J}_{z}^{2} - \hslash {J}_{z}}\right) \left| {a,{b}_{\text{最大 }}}\right\rangle = 0. tag{3. 5.20}
$$

因为 $\left| {a,{b}_{最大}}\right\rangle$ 本身不是一个零矢量,这个关系仅当下式被满足时才是可能的:

$$
a - {b}_{\text{最大 }}^{2} - {b}_{\text{最大 }}\hslash = 0 tag{3. 5.21}
$$

或

$$
a = {b}_{\text{最大 }}\left( {{b}_{\text{最大 }} + \hslash }\right) . tag{3. 5.22}
$$

以类似的方式,从 (3.5.13) 式推断,一定还存在一个 ${b}_{\text{最小 }}$ ,使得

$$
{J}_{ - }\left| {a,{b}_{\text{最小 }}}\right\rangle = 0. tag{3. 5.23}
$$

与 (3.5.19) 式类似,通过把 ${J}_{ + }{J}_{ - }$ 写作

$$
{J}_{ + }{J}_{ - } = {\mathbf{J}}^{2} - {J}_{z}^{2} + \hslash {J}_{z} tag{3. 5.24}
$$

得到:

$$
a = {b}_{\text{最小 }}\left( {{b}_{\text{最小 }} - \hslash }\right) , tag{3. 5.25}
$$

比较 (3.5.22) 式与 (3.5.25) 式, 得出

$$
{b}_{\text{最大 }} = - {b}_{\text{最小 }}, tag{3.5.26}
$$

由于 ${b}_{\text{最大 }k}$ 为正值,于是 $b$ 的允许值处与在以下范围之内:

$$
- {b}_{最大} \leq b \leq {b}_{最大}. tag{3. 5.27}
$$

显然通过把 ${J}_{ + }$ 连续作用于 $\left| {a,{b}_{\text{最小 }}}\right\rangle$ 有限次,一定能够达到 $\left| {a,{b}_{\text{最大 }}}\right\rangle$ . 因此有

$$
{b}_{\text{最大 }} = {b}_{\text{最小 }} + n\hslash , tag{3. 5.28}
$$

其中 $n$ 是某个整数. 作为结果,我们得到

$$
{b}_{\text{最大 }} = \frac{n\hslash }{2}. tag{3. 5.29}
$$

更为传统的做法是代替 ${b}_{\text{最大 }}$ 使用定义为 ${b}_{\text{最大 }}/\hslash$ 的 $j$ ,所以

$$
j = \frac{n}{2}. tag{3. 5.30}
$$

${J}_{z}$ 的最大本征值为 ${jh}$ ,其中 $j$ 既可以是一个整数,也可以是一个半奇数. (译者注: 原书为半整数. 显然不够确切.) 方程 (3.5.22) 式意味着 ${\mathbf{J}}^{2}$ 的本征值由

$$
a = {\hslash }^{2}j\left( {j + 1}\right) . tag{3. 5.31}
$$

给定. 还定义一个 $m$ ,使得

$$
b \equiv m\hslash . tag{3. 5.32}
$$

若 $j$ 是一个整数,则所有的 $m$ 值都是整数; 若 $j$ 是一个半奇数,则所有的 $m$ 值都是半奇数. 对于一个给定的 $j$ ,允许的 $m$ 值为

$$
m = \underset{{2j} + 1\text{ 个态 }}{\underbrace{-j, j + 1,\cdots, j - 1, j}}. tag{3. 5.33}
$$

代替 $|a, b\rangle$ ,更方便的是用 $|j, m\rangle$ 表示 ${\mathbf{J}}^{2}$ 和 ${J}_{z}$ 的共同本征右矢. 基本的本征值方程现在为

$$
{\mathbf{J}}^{2}\left| {j, m\rangle = j\left( {j + 1}\right) {\hslash }^{2}}\right| j, m\rangle
$$

(3. ${5.34a}$ )

和

$$
{J}_{z}\left| {j, m\rangle = m\hslash }\right| j, m\rangle , tag{3. 5.34b}
$$

其中 $j$ 或者为一个整数,或者为一个半奇数,而 $m$ 由 (3.5.33) 给定. 非常重要的是回忆起为了获得这些结果, 仅用到对易关系 (3.1.20) 式. 在 (3.5.34) 式中显示的角动量量子化, 是角动量对易关系的一个直接后果, 该对易关系又是由转动的性质以及作为转动生成元 ${J}_{k}$ 的定义一起推导出来的.

\subsection{角动量算符的矩阵元}
下面来求各种角动量算符的矩阵元. 假定 $|j, m\rangle$ 已被归一化,由 (3.5.34) 式显然有

$$
\left\langle {{j}^{\prime },{m}^{\prime }\left| {\mathbf{J}}^{2}\right| j, m}\right\rangle = j\left( {j + 1}\right) {\hslash }^{2}{\delta }_{{j}^{\prime }j}{\delta }_{{m}^{\prime }m}
$$

(3. ${5.35a}$ )

和

$$
\left\langle {{j}^{\prime },{m}^{\prime }\left| {J}_{z}\right| j, m}\right\rangle = m\hslash {\delta }_{{j}^{\prime }j}{\delta }_{{m}^{\prime }m}. tag{3. 5.35b}
$$

为了获得 ${J}_{ \pm }$ 的矩阵元,首先考虑

$$
\left\langle {j, m\left| {{J}_{ + }^{ + }{J}_{ + }}\right| j, m}\right\rangle = \left\langle {j, m\left| \left( {{\mathbf{J}}^{2} - {J}_{z}^{2} - \hslash {J}_{z}}\right) \right| j, m}\right\rangle tag{3. 5.36}
$$

$$
= {\hslash }^{2}\left\lbrack {j\left( {j + 1}\right) - {m}^{2} - m}\right\rbrack .
$$

由于 ${J}_{ + } \mid j, m\rangle$ 必须和 $|j, m + 1\rangle$ (已归一) 在最多差一个常数因子的情况下相同 [见 (3.5.12) 式], 因此

$$
{J}_{ + }\left| {j, m\rangle = {c}_{jm}^{ + }}\right| j, m + 1\rangle . tag{3. 5.37}
$$

与 (3.5.36) 式的比较将导致

$$
{\left| {c}_{jm}^{ + }\right| }^{2} = {\hslash }^{2}\left\lbrack {j\left( {j + 1}\right) - m\left( {m + 1}\right) }\right\rbrack tag{3. 5.38}
$$

$$
= {\hslash }^{2}\left( {j - m}\right) \left( {j + m + 1}\right) \text{.}
$$

这样,在最多差一个任意相因子的情况下确定了 ${c}_{jm}^{ + }$ . 通常的惯例是选 ${c}_{jm}^{ + }$ 为正实数,于是

$$
{J}_{ + }\left| {j, m\rangle = \sqrt{\left( {j - m}\right) \left( {j + m + 1}\right) }\hslash }\right| j, m + 1\rangle . tag{3.5.39}
$$

可以类似地导出

$$
{J}_{ - }\left| {j, m\rangle = \sqrt{\left( {j + m}\right) \left( {j - m + 1}\right) }\hslash }\right| j, m - 1\rangle . tag{3. 5.40}
$$

最后, ${J}_{ \pm }$ 的矩阵元确定为

$$
\left\langle {{j}^{\prime },{m}^{\prime }\left| {J}_{ \pm }\right| j, m}\right\rangle = \sqrt{\left( {j \mp m}\right) \left( {j \pm m + 1}\right) }\hslash {\delta }_{{j}^{\prime }j}{\delta }_{{m}^{\prime }m \pm 1}. tag{3. 5.41}
$$

\subsection{转动算符的表示}

得到了 ${J}_{z}$ 和 ${J}_{ \pm }$ 的矩阵元之后,就能够研究转动算符 $\mathcal{D}\left( R\right)$ . 如果一个转动 $R$ 由 $\widehat{\mathbf{n}}$ 和 $\phi$ 确定, 就可以用

$$
{\mathcal{D}}_{{m}^{\prime }m}^{\left( j\right) }\left( R\right) = \left\langle {j,{m}^{\prime }\left| {\exp \left( \frac{-i\mathbf{J} \cdot \widehat{\mathbf{n}}\phi }{\hslash }\right) }\right| j, m}\right\rangle . tag{3. 5.42}
$$

定义它的矩阵元. 这些矩阵元有时被称为维格纳函数, 它以对量子力学中转动的群论性质作出了开拓性贡献的维格纳的名字命名. 这里要注意,在 (3.5.42) 式中相同的 $j$ 值出现在右矢和左矢中,不需要考虑不同 $j$ 值的态之间的 $\mathcal{D}\left( R\right)$ 矩阵元,因为它们显而易见都是零. 这是因为 ${\left. \mathcal{D}\left( R\right) \mid j, m\right| }^{2}$ 仍是 ${\mathbf{J}}^{2}$ 的具有相同本征值 $j\left( {j + 1}\right) {\hslash }^{2}$ 的本征态:

$$
{\mathbf{J}}^{2}\mathcal{D}\left( R\right) \left| {j, m\rangle = \mathcal{D}\left( R\right) {\mathbf{J}}^{2}}\right| j, m\rangle tag{3. 5.43}
$$

$$
= j\left( {j + 1}\right) {\hslash }^{2}\left\lbrack {\mathcal{D}\left( R\right) \mid j, m\rangle }\right\rbrack ,
$$

该式可以直接由 ${\mathbf{J}}^{2}$ 与 ${J}_{k}$ (因此与 ${J}_{k}$ 的任何函数) 对易得到. 简单地说,转动不改变 $j$ 值, 这是一个非常有用的结果.

在文献中经常把 ${\mathcal{D}}_{m, m}^{\left( j\right) }\left( R\right)$ 构成的 $\left( {{2j} + 1}\right) \times \left( {{2j} + 1}\right)$ 矩阵称为转动算符 $\mathcal{D}\left( R\right)$ 的 ${2j} + 1$ 维不可约表示. 这意味着,使用一组适当选择的基,在不一定由一个单一的 $j$ 值表征的右矢空间上, 对应一个任意转动算符的矩阵可以具有分块对角形式:(3. 5.44)


其中有阴影的方块是由 ${\mathcal{D}}_{m \mid m}^{\left( j\right) }\left( R\right)$ 构成的、具有某个确定 $j$ 值的 $\left( {{2j} + 1}\right) \times \left( {{2j} + 1}\right)$ 方阵. 而且, 每个方阵本身不可能通过任何基的选择, 破缺成更小的块.(3.5.45)



由确定的 $j$ 表征的转动矩阵构成一个群. 首先,单位矩阵是一个群元,因为对应于没有任何转动 $\left( {\phi = 0}\right)$ 的转动矩阵是 $\left( {{2j} + 1}\right) \times \left( {{2j} + 1}\right)$ 的单位矩阵. 其次,逆也是一个群元,只不过在不改变转动轴 $\widehat{\mathbf{n}}$ 的情况下,反转了转角 $\phi \rightarrow - \phi$ . 第三,任何两个元的乘积也是一个群元, 明确地有

$$
\mathop{\sum }\limits_{{m}^{\prime }}{\mathcal{D}}_{{m}^{\prime }{m}^{\prime }}^{\left( j\right) }\left( {R}_{1}\right) {\mathcal{D}}_{{m}^{\prime }m}^{\left( j\right) }\left( {R}_{2}\right) = {\mathcal{D}}_{{m}^{\prime }m}^{\left( j\right) }\left( {{R}_{1}{R}_{2}}\right) , tag{3. 5.46}
$$

其中,乘积 ${R}_{1}{R}_{2}$ 代表一个单一的转动. 注意,因为相应的转动算符是幺正的,所以转动矩阵是幺正的, 显然有

$$
{\mathcal{D}}_{{m}^{\prime }m}\left( {R}^{-1}\right) = {\mathcal{D}}_{m{m}^{\prime }}^{ * }\left( R\right) . tag{3.5.47}
$$

为了领会转动矩阵的物理意义,从 $|j, m\rangle$ 所代表的一个态出发. 现在转动它:

$$
\left| {j, m\rangle \rightarrow \mathcal{D}\left( R\right) }\right| j, m\rangle . tag{3. 5.48}
$$

尽管这个转动操作并不改变 $j$ ,但一般将得到一些不同于原来 $m$ 的 $m$ 值态. 为了得到发现处于 $\left| {j,{m}^{\prime }}\right\rangle$ 态的振幅,只要把转动后的态展开如下

$$
\mathcal{D}\left( R\right) \left| {j, m\rangle = \mathop{\sum }\limits_{{m}^{\prime }}}\right| j,{m}^{\prime }\rangle \left\langle {j,{m}^{\prime }\left| {\mathcal{D}\left( R\right) }\right| j, m}\right\rangle tag{3. 5.49}
$$

$$
= \mathop{\sum }\limits_{{m}^{\prime }}\left| {j,{m}^{\prime }}\right\rangle {\mathcal{D}}_{{m}^{\prime }m}^{\left( j\right) }\left( R\right) ,
$$

其中,在使用完备性关系时,利用了 $\dot{D}\left( R\right)$ 只联系具有相同 $j$ 态的特性. 所以,当原始未转动态由 $|j, m\rangle$ 给出时,矩阵元 ${\mathcal{D}}_{{m}^{\prime }m}^{\left( j\right) }\left( R\right)$ 只不过是找到转动后态处于 $\left| {j,{m}^{\prime }}\right\rangle$ 的振幅.

在 3.3 节曾看到怎样利用欧拉角表征最一般的转动. 现在考虑一个任意 $j$ (不一定是 $\left. \frac{1}{2}\right)$ 的 (3.3.20) 式的矩阵实现:

$$
{\mathcal{D}}_{{m}^{\prime }m}^{\left( j\right) }\left( {\alpha ,\beta ,\gamma }\right) = \left\langle {j,{m}^{\prime }\left| {\;\exp \left( \frac{-i{J}_{z}\alpha }{\hslash }\right) \exp \left( \frac{-i{J}_{y}\beta }{\hslash }\right) \exp \left( \frac{-i{J}_{z}\gamma }{\hslash }\right) }\right. j, m}\right\rangle tag{3. 5.50}
$$

$$
= {e}^{-i\left( {{m}^{\prime }\alpha + {m\gamma }}\right) }\left\langle {j,{m}^{\prime }\left| {\exp \left( \frac{-i{J}_{y}\beta }{\hslash }\right) }\right| j, m}\right\rangle .
$$

注意,唯一非平庸的部分是中间绕 $y$ 轴的转动,它混合了不同 $m$ 值的态. 方便的做法是把一个新矩阵 ${d}^{\left( j\right) }\left( \beta \right)$ 定义为

$$
{d}_{{m}^{\prime }m}^{\left( j\right) }\left( \beta \right) \equiv \left\langle {j,{m}^{\prime }\left| {\exp \left( \frac{-i{J}_{y}\beta }{\hslash }\right) }\right| j, m}\right\rangle . tag{3. 5.51}
$$

最后,转向一些实例. $j = \frac{1}{2}$ 的情况已经在 3.3 节求解出. 请参见 (3.3.21) 式的中间矩阵

$$
{d}^{1/2} = \left( \begin{matrix} \cos \left( \frac{\beta }{2}\right) & - \sin \left( \frac{\beta }{2}\right) \\ \sin \left( \frac{\beta }{2}\right) & \cos \left( \frac{\beta }{2}\right) \end{matrix}\right) . tag{3. 5.52}
$$

下一个最简单的是 $j = 1$ 的情况,将较为详细考虑. 显然,首先必须求出 ${J}_{y}$ 的 $3 \times 3$ 矩阵表示. 因为由 ${J}_{ \pm }$ 定义方程 (3.5.5) 有

$$
{J}_{y} = \frac{\left( {J}_{ + } - {J}_{ - }\right) }{2i}, tag{3. 5.53}
$$

可以利用 (3.5.41) 式得到

$$
m = 1\;m = 0\;m = - 1
$$

$$
{J}_{y}^{\left( j = 1\right) } = \left( \frac{\hslash }{2}\right) \left( \begin{matrix} 0 & - \sqrt{2}i & 0 \\ \sqrt{2}i & 0 & - \sqrt{2}i \\ 0 & \sqrt{2}i & 0 \end{matrix}\right) \;\begin{array}{l} {m}^{\prime } = 1 \\ {m}^{\prime } = 0 \\ {m}^{\prime } = - 1 \end{array} tag{3. 5.54}
$$

下一个任务是求 $\exp \left( {-i{J}_{y}\beta /\hslash }\right)$ 的泰勒 (Taylor) 展开. 与 $j = \frac{1}{2}$ 的情况不同, ${\left\lbrack {J}_{y}^{\left( j = 1\right) }\right\rbrack }^{2}$ 不依赖于 1 和 ${J}_{y}^{\left( j = 1\right) }$ . 然而,很容易求出:

$$
{\left( \frac{{J}_{y}^{\left( j = 1\right) }}{\hslash }\right) }^{3} = \frac{{J}_{y}^{\left( j = 1\right) }}{\hslash }. tag{3. 5.55}
$$

于是,仅对 $j = 1$ ,可合理地做如下替换

$$
\exp \left( \frac{-i{J}_{y}\beta }{\hslash }\right) \rightarrow 1 - {\left( \frac{{J}_{y}}{\hslash }\right) }^{2}\left( {1 - \cos \beta }\right) - i\left( \frac{{J}_{y}}{\hslash }\right) \sin \beta , tag{3. 5.56}
$$

读者可以详细地证明它. 具体有

$$
{d}^{\left( 1\right) }\left( \beta \right) = \left( \begin{matrix} \left( \frac{1}{2}\right) \left( {1 + \cos \beta }\right) & - \left( \frac{1}{\sqrt{2}}\right) \sin \beta & \left( \frac{1}{2}\right) \left( {1 - \cos \beta }\right) \\ \left( \frac{1}{\sqrt{2}}\right) \sin \beta & \cos \beta & - \left( \frac{1}{\sqrt{2}}\right) \sin \beta \\ \left( \frac{1}{2}\right) \left( {1 - \cos \beta }\right) & \left( \frac{1}{\sqrt{2}}\right) \sin \beta & \left( \frac{1}{2}\right) \left( {1 + \cos \beta }\right) \end{matrix}\right) tag{3. 5.57}
$$

显然,这种方法对于大的 $j$ 值会很耗时. 别的一些更容易的方法是有可能的,但是在本书中将不继续讨论它们.

\section{轨道角动量}
通过定义角动量为一个无穷小转动的生成元引入了角动量的概念. 当自旋角动量为零或者可以忽略时,还有另一种处理角动量问题的方法. 那时,一个单粒子的角动量 $\mathbf{J}$ 与定义为
$$
\mathbf{L} = \mathbf{x} \times \mathbf{p} tag{3.6.1}
$$
的轨道角动量是一样的. 本节来探讨这两种方法之间的联系.
\subsection{作为转动生成元的轨道角动量}

首先注意到,由于 $\mathbf{x}$ 和 $\mathbf{p}$ 各分量之间的对易关系,定义为 (3.6.1) 式的轨道角动量算符满足角动对易关系

$$
\left\lbrack {{L}_{i},{L}_{j}}\right\rbrack = i{\varepsilon }_{ijk}\hslash {L}_{k} tag{3.6.2}
$$

这一点很容易证明如下:

$$
\left\lbrack {{L}_{x},{L}_{y}}\right\rbrack = \left\lbrack {y{p}_{z} - z{p}_{y}, z{p}_{x} - x{p}_{z}}\right\rbrack
$$

$$
= \left\lbrack {y{p}_{z}, z{p}_{x}}\right\rbrack + \left\lbrack {z{p}_{y}, x{p}_{z}}\right\rbrack
$$

$$
= y{p}_{x}\left\lbrack {{p}_{z}, z}\right\rbrack + {p}_{y}x\left\lbrack {z,{p}_{z}}\right\rbrack tag{3.6.3}
$$

$$
= i\hslash \left( {x{p}_{y} - y{p}_{x}}\right)
$$

$$
= i\hslash {L}_{z}
$$

$$
\vdots
$$

其次令

$$
1 - i\left( \frac{\delta \phi }{\hslash }\right) {L}_{z} = 1 - i\left( \frac{\delta \phi }{\hslash }\right) \left( {x{p}_{y} - y{p}_{x}}\right) tag{3.6.4}
$$

作用于一个任意的位置本征右矢 $\left| {{x}^{\prime },{y}^{\prime },{z}^{\prime }}\right\rangle$ ,以考察它是否可以解释为绕 $z$ 轴旋转 ${\delta \phi }$ 角的无穷小转动算符. 利用动量是平移生成元, 得到 [见 (1.6.32) 式]

$$
\left. {\left\lbrack {1 - i\left( \frac{\delta \phi }{\hslash }\right) {L}_{z}}\right\rbrack \left| {{x}^{\prime },{y}^{\prime },{z}^{\prime }}\right\rangle = \left\lbrack {1 - i\left( \frac{{p}_{y}}{\hslash }\right) \left( {{\delta \phi }{x}^{\prime }}\right) + i\left( \frac{{p}_{x}}{\hslash }\right) \left( {{\delta \phi }{y}^{\prime }}\right) }\right\rbrack \mid {x}^{\prime },{y}^{\prime },{z}^{\prime }}\right\}
$$

$$
= \left| {{x}^{\prime } - {y}^{\prime }{\delta \phi },{y}^{\prime } + {x}^{\prime }{\delta \phi },{z}^{\prime }}\right\rangle . tag{3.6.5}
$$

如果 ${L}_{z}$ 生成一个绕 $z$ 轴的无穷小转动,则上式正是所预期的结果. 所以,证明了如果 $\mathbf{p}$ 生成平移,则 $\mathbf{L}$ 生成转动.

假定一个无自旋粒子的任意物理态的波函数由 $\left\langle {{x}^{\prime },{y}^{\prime },{z}^{\prime } \mid \alpha }\right\rangle$ 给定. 在绕 $z$ 轴做一个无穷小转动之后, 转动后态的波函数为

$$
\left\langle {{x}^{\prime },{y}^{\prime },{z}^{\prime }}\right\rangle \left\lbrack {1 - i\left( \frac{\delta \phi }{\hslash }\right) {L}_{z}}\right\rbrack |\alpha \rangle = \left\langle {{x}^{\prime } + {y}^{\prime }{\delta \phi },{y}^{\prime } - {x}^{\prime }{\delta \phi },{z}^{\prime } \mid \alpha }\right\rangle . tag{3.6.6}
$$

实际上改变坐标基

$$
\left\langle {{x}^{\prime },{y}^{\prime },{z}^{\prime } \mid \alpha }\right\rangle \rightarrow \langle r,\theta ,\phi \mid \alpha \rangle . tag{3.6.7}
$$

会更为清楚. 按照 (3.6.6) 式, 转动后的态为

$$
\left\langle {r,\theta ,\phi \left| \left\lbrack {1 - i\left( \frac{\delta \phi }{\hslash }\right) {L}_{z}}\right\rbrack \right| \alpha }\right\rangle = \langle r,\theta ,\phi - {\delta \phi } \mid \alpha \rangle tag{3.6.8}
$$

$$
= \langle r,\theta ,\phi \mid \alpha \rangle - {\delta \phi }\frac{\partial }{\partial \phi }\langle r,\theta ,\phi \mid \alpha \rangle .
$$

因为 $\langle r,\theta ,\phi \mid$ 是一个任意的位置本征左矢,可以确认

$$
\left\langle {{\mathbf{x}}^{\prime }\left| {L}_{z}\right| \alpha }\right\rangle = - i\hslash \frac{\partial }{\partial \phi }\left\langle {{\mathbf{x}}^{\prime } \mid \alpha }\right\rangle , tag{3.6.9}
$$

它是一个众所周知的波动力学结果. 尽管这个关系也可轻易地利用动量算符的位置表象求得,这里给出的推导强调了 ${L}_{z}$ 作为转动生成元的作用.

其次考虑一个绕 $x$ 轴旋转 $\delta {\phi }_{x}$ 角的转动. 与 (3.6.6) 式类似,有

$$
\left\langle {{x}^{\prime },{y}^{\prime },{z}^{\prime }\left| \left\lbrack {1 - i\left( \frac{\delta {\phi }_{x}}{\hslash }\right) {L}_{x}}\right\rbrack \right| \alpha }\right\rangle = \left\langle {{x}^{\prime },{y}^{\prime } + {z}^{\prime }\delta {\phi }_{x},{z}^{\prime } - {y}^{\prime }\delta {\phi }_{x} \mid \alpha }\right\rangle . tag{3. 6.10}
$$

通过把 ${x}^{\prime },{y}^{\prime }$ 和 ${z}^{\prime }$ 用球坐标表示,可以证明

$$
\left\langle {{\mathbf{x}}^{\prime }\left| {L}_{x}\right| \alpha }\right\rangle = - i\hslash \left( {-\sin \phi \frac{\partial }{\partial \theta } - \cot \theta \cos \phi \frac{\partial }{\partial \phi }}\right) \left\langle {{\mathbf{x}}^{\prime } \mid \alpha }\right\rangle . tag{3. 6.11}
$$

同样地,

$$
\left\langle {{\mathbf{x}}^{\prime }\left| {L}_{y}\right| \alpha }\right\rangle = - i\hslash \left( {\cos \phi \frac{\partial }{\partial \theta } - \cot \theta \sin \phi \frac{\partial }{\partial \phi }}\right) \left\langle {{\mathbf{x}}^{\prime } \mid \alpha }\right\rangle . tag{3. 6.12}
$$

利用 (3.6.11) 式和 (3.6.12) 式,对于由 (3.5.5) 式定义的阶梯算符 ${L}_{ \pm }$ ,有

$$
\left\langle {{\mathbf{x}}^{\prime }\left| {L}_{ \pm }\right| \alpha }\right\rangle = - i\hslash {e}^{\pm {i\phi }}\left( {\pm i\frac{\partial }{\partial \theta } - \cot \theta \frac{\partial }{\partial \phi }}\right) \left\langle {{\mathbf{x}}^{\prime } \mid \alpha }\right\rangle . tag{3. 6.13}
$$

最后, 利用
%
%$$
%{\mathbf{L}}^{2} = {L}_{z}^{2} + \left( \frac{1}{2}\right) \left( {{L}_{ + }{L}_{ - } + {L}_{ - }{L}_{ + }}\right) , tag{3. 6.14}
%$$
%
%以及 (3.6.9) 式和 (3.6.13) 式,可以把 $\left\langle {{\mathbf{x}}^{\prime }{\left| {\mathbf{L}}^{2}\right| }_{\alpha }}\right\rangle$ 写成如下形式:
%
%$$
%\left\langle {{\mathbf{x}}^{\prime }\left| {\mathbf{L}}^{2}\right| \alpha }\right\rangle = - {\hslash }^{2}\left\lbrack {\frac{1}{{\sin }^{2}\theta }\frac{{\partial }^{2}}{\partial {\phi }^{2}} + \frac{1}{\sin \theta }\frac{\partial }{\partial \theta }\left( {\sin \theta \frac{\partial }{\partial \theta }}\right) }\right\rbrack \left\langle {{\mathbf{x}}^{\prime } \mid \alpha }\right\rangle . tag{3. 6.15}
%$$
%
%除了 $1/{r}^{2}$ 的因子之外,可以看到出现在这里的微商算符正是球坐标中拉普拉斯算子的角度部分.
%
%通过直接观察动能算符,以另外一种方式建立 ${\mathbf{L}}^{2}$ 算符与拉普拉斯算子的角度部分间的联系是有益的. 首先写下一个重要的算符恒等式:
%
%$$
%{\mathbf{L}}^{2} = {\mathbf{x}}^{2}{\mathbf{p}}^{2} - {\left( \mathbf{x} \cdot \mathbf{p}\right) }^{2} + i\hslash \mathbf{x} \cdot \mathbf{p}, tag{3. 6.16}
%$$
%
%其中 ${\mathbf{x}}^{2}$ 被理解为算符 $\mathbf{x} \cdot \mathbf{x}$ ,正如 ${\mathbf{p}}^{2}$ 代表算符 $\mathbf{p} \cdot \mathbf{p}$ 一样. 该式的证明是很容易的:
%
%$$
%{\mathbf{L}}^{2} = \mathop{\sum }\limits_{{ijlmk}}{\varepsilon }_{ijk}{x}_{i}{p\not{} }_{j}{\varepsilon }_{lmk}{x}_{l}{p\not{} }_{m}
%$$
%
%$$
%= \mathop{\sum }\limits_{{ijlm}}\left( {{\delta }_{il}{\delta }_{jm} - {\delta }_{im}{\delta }_{jl}}\right) {x}_{i}{p}_{j}{x}_{l}{p}_{m}
%$$
%
%$$
%= \mathop{\sum }\limits_{{ijlm}}\left\lbrack {{\delta }_{il}{\delta }_{jm}{x}_{i}\left( {{x}_{l}{p}_{j} - i\hslash {\delta }_{jl}}\right) {p}_{m} - {\delta }_{im}{\delta }_{jl}{x}_{i}{p}_{j}\left( {{p}_{m}{x}_{l} + i\hslash {\delta }_{lm}}\right) }\right\rbrack tag{3.6.17`}
%$$
%
%$$
%= {\mathbf{x}}^{2}{\mathbf{p}}^{2} - i\hslash \mathbf{x} \cdot \mathbf{p} - \mathop{\sum }\limits_{{ijlm}}{\delta }_{im}{\delta }_{jl}\left\lbrack {{x}_{i}{p}_{m}\left( {{x}_{l}{p}_{j} - i\hslash {\delta }_{jl}}\right) + i\hslash {\delta }_{lm}{x}_{i}{p}_{j}}\right\rbrack
%$$
%
%$$
%= {\mathbf{x}}^{2}{\mathbf{p}}^{2} - {\left( \mathbf{x} \cdot \mathbf{p}\right) }^{2} + i\hslash \mathbf{x} \cdot \mathbf{p}.
%$$
%
%在将上式置于 $\langle {\mathbf{x}}^{\prime }|$和 $\alpha |$中间之前,首先注意到$\rangle$
%
%$$
%\left\langle {{\mathbf{x}}^{\prime }\left| {\mathbf{x} \cdot \mathbf{p}}\right| \alpha }\right\rangle = {\mathbf{x}}^{\prime } \cdot \left( {-i\hslash {\nabla }^{\prime }\left\langle {{\mathbf{x}}^{\prime } \mid \alpha }\right\rangle }\right)
%$$
%
%$$
%= - {ihr}\frac{\partial }{\partial r}\left\langle {{\mathbf{x}}^{\prime } \mid \alpha }\right\rangle . tag{3. 6.18}
%$$
%
%同样地,
%
%$$
%\left\langle {{\mathbf{x}}^{\prime }\left| {\left( \mathbf{x} \cdot \mathbf{p}\right) }^{2}\right| \alpha }\right\rangle = - {\hslash }^{2}r\frac{\partial }{\partial r}\left( {r\frac{\partial }{\partial r}\left\langle {{\mathbf{x}}^{\prime } \mid \alpha }\right\rangle }\right) tag{3. 6.19}
%$$
%
%$$
%= - {\hslash }^{2}\left( {{r}^{2}\frac{{\partial }^{2}}{\partial {r}^{2}}\left\langle {{\mathbf{x}}^{\prime } \mid \alpha }\right\rangle + r\frac{\partial }{\partial r}\left\langle {{\mathbf{x}}^{\prime } \mid \alpha }\right\rangle }\right) .
%$$
%
%因此,
%
%$$
%\left\langle {{\mathbf{x}}^{\prime }\left| {\mathbf{L}}^{2}\right| \alpha }\right\rangle = {r}^{2}\left\langle {{\mathbf{x}}^{\prime }\left| {\mathbf{p}}^{2}\right| \alpha }\right\rangle + {\hslash }^{2}\left( {{r}^{2}\frac{{\partial }^{2}}{\partial {r}^{2}}\left\langle {{\mathbf{x}}^{\prime } \mid \alpha }\right\rangle + {2r}\frac{\partial }{\partial r}\left\langle {{\mathbf{x}}^{\prime } \mid \alpha }\right\rangle }\right) . tag{3. 6.20}
%$$
%
%按照动能算符 ${\mathbf{p}}^{2}/{2m}$ ,有
%
%$$
%\frac{1}{2m}\left\langle {{\mathbf{x}}^{\prime }\left| {\mathbf{p}}^{2}\right| \alpha }\right\rangle = - \left( \frac{{\hslash }^{2}}{2m}\right) {\nabla }^{\prime 2}\left\langle {{\mathbf{x}}^{\prime } \mid \alpha }\right\rangle tag{3. 6.21}
%$$
%
%$$
%= - \left( \frac{{\hslash }^{2}}{2m}\right) \left( {\frac{{\partial }^{2}}{\partial {r}^{2}}\left\langle {{\mathbf{x}}^{\prime } \mid \alpha }\right\rangle + \frac{2}{r}\frac{\partial }{\partial r}\left\langle {{\mathbf{x}}^{\prime } \mid \alpha }\right\rangle - \frac{1}{{\hslash }^{2}{r}^{2}}\left\langle {{\mathbf{x}}^{\prime }\left| {\mathbf{L}}^{2}\right| \alpha }\right\rangle }\right)
%$$
%
%最后一行中的前两项正是作用于 $\left\langle {{\mathbf{x}}^{\prime } \mid \alpha }\right\rangle$ 上的拉普拉斯算子的径向部分. 最后一项则一定是作用于 $\left\langle {{\mathbf{x}}^{\prime } \mid \alpha }\right\rangle$ 上的拉普拉斯算子的角度部分,这与 (3.6.15) 式完全一致.
\subsection{球谐函数}
考虑一个受到球对称势的作用的无自旋粒子. 已知波动方程在球坐标系中是可分离变量的, 能量本征函数可以写成

$$
\left\langle {{\mathbf{x}}^{\prime } \mid n, l, m}\right\rangle = {R}_{nl}\left( r\right) {Y}_{l}^{m}\left( {\theta ,\phi }\right) , tag{3. 6.22}
$$

其中位置矢量 ${\mathbf{x}}^{\prime }$ 由球坐标 $r,\theta$ 和 $\phi$ 确定,而 $n$ 表示 $l$ 和 $m$ 之外的某个量子数——例如,束缚态问题的径向量子数或一个自由粒子球面波的能量. 正如将在 3.11 节看得更清楚的, 这种形式可以被视为该问题转动不变的直接后果. 当哈密顿量是球对称的时候, $H$ 与 $L$ 。 和 ${\mathbf{L}}^{2}$ 对易,因而预期能量本征右矢也是 ${\mathbf{L}}^{2}$ 和 ${L}_{z}$ 的本征右矢. 因为在 $k = 1,2,3$ 时 ${L}_{k}$ 满足角动量对易关系,预期 ${\mathbf{L}}^{2}$ 和 ${L}_{z}$ 的本征值分别为 $l\left( {l + 1}\right) {\hslash }^{2}$ 和 ${mh} = \lbrack - l\hslash$ , $\left( -l + 1\right) \hslash,\left( l - 1\right) \hslash, l\hslash \rbrack$ .

因为角度的依赖关系对于所有球对称的问题是共同的, 可以把它孤立出来考虑

$$
\langle \widehat{\mathbf{n}} \mid l, m\rangle = {Y}_{l}^{m}\left( {\theta ,\phi }\right) = {Y}_{l}^{m}\left( \widehat{\mathbf{n}}\right) , tag{3. 6.23}
$$

其中,定义了一个方向本征右矢 $|\widehat{\mathbf{n}}\rangle$ . 基于此观点, ${Y}_{l}^{m}\left( {\theta ,\phi }\right)$ 就是在 $\theta$ 和 $\phi$ 所规定的方向 $\widehat{\mathbf{n}}$ 上找到 $l, m$ 表征的态的振幅.

假定有一个包含轨道角动量本征右矢的关系式, 能够立即写出包含球谐函数的相应的关系式. 例如, 取本征值方程

$$
{L}_{z}\left| {l, m\rangle = m\hslash }\right| l, m\rangle . tag{3. 6.24}
$$

左乘 $\langle \widehat{\mathbf{n}}|$ ,并且利用 (3.6.9) 式,得到

$$
- i\hslash \frac{\partial }{\partial \phi }\langle \widehat{\mathbf{n}}\left| {l, m\rangle = m\hslash \langle \widehat{\mathbf{n}}}\right| l, m\rangle . tag{3. 6.25}
$$

这个方程就是

$$
- i\hslash \frac{\partial }{\partial \phi }{Y}_{l}^{m}\left( {\theta ,\phi }\right) = m\hslash {Y}_{l}^{m}\left( {\theta ,\phi }\right) , tag{3. 6.26}
$$

它意味着 ${Y}_{l}^{m}\left( {\theta ,\phi }\right)$ 的 $\phi$ 依赖行为必然像 ${e}^{im\phi }$ . 同样,对应于

$$
{\mathbf{L}}^{2}\left| {l, m\rangle = l\left( {l + 1}\right) {\hslash }^{2}}\right| l, m\rangle , tag{3. 6.27}
$$

有 [见 (3.6.15) 式]

$$
\left\lbrack {\frac{1}{\sin \theta }\frac{\partial }{\partial \theta }\left( {\sin \theta \frac{\partial }{\partial \theta }}\right) + \frac{1}{{\sin }^{2}\theta }\frac{{\partial }^{2}}{\partial {\phi }^{2}} + l\left( {l + 1}\right) }\right\rbrack {Y}_{l}^{m} = 0, tag{3. 6.28}
$$

它只不过是 ${Y}_{l}^{m}$ 自身满足的偏微分方程. 正交性关系

$$
\left\langle {{l}^{\prime },{m}^{\prime } \mid l, m}\right\rangle = {\delta }_{l{l}^{\prime }}{\delta }_{m{m}^{\prime }} tag{3.6.29}
$$

导致

$$
{\int }_{0}^{2\pi }{d\phi }{\int }_{-1}^{1}d\left( {\cos \theta }\right) {Y}_{{l}^{\prime }}^{{m}^{\prime }}\left( {\theta ,\phi }\right) {Y}_{l}^{m}\left( {\theta ,\phi }\right) = {\delta }_{l{l}^{\prime }}{\delta }_{m{m}^{\prime }}, tag{3. 6.30}
$$

其中用到了方向本征右矢的完备性关系

公式

为得到 ${Y}_{l}^{m}$ ,从 $m = l$ 出发,有

$$
{L}_{ + }\left| {l, l\rangle = 0,}\right| tag{3. 6.32}
$$

由于 (3.6.13) 式, 上式导致

$$
- i\hslash {e}^{i\phi }\left\lbrack {i\frac{\partial }{\partial \theta } - \cot \theta \frac{\partial }{\partial \phi }}\right\rbrack \langle \widehat{\mathbf{n}}|l, l\rangle = 0. tag{3. 6.33}
$$

记住 $\phi$ 依赖的行为一定是 ${e}^{it\phi }$ ,可以容易地证明

$$
\langle \widehat{\mathbf{n}} \mid l, l\rangle = {Y}_{l}^{l}\left( {\theta ,\phi }\right) = {c}_{l}{e}^{il\phi }{\sin }^{l}\theta tag{3. 6.34}
$$

满足这个偏微分方程,(3.6.33),其中的 ${c}_{l}$ 是由 (3.6.30) 式确定的归一化常数*

* 当然,归一化条件 (3.6.30) 式确定不了 ${c}_{l}$ 的相因子. 插入一个因子 ${\left( -1\right) }^{l}$ ,这样当连续使用 ${L}_{ - }$ 算符达到了 $m$ $= 0$ 的态时,得到的 ${Y}_{I}^{n}$ 有着与勒让德多项式 ${P}_{I}\left( {\cos \theta }\right)$ 相同的符号,而后者的相位是由 ${P}_{I}\left( 1\right) = 1$ 确定的,见公式

(3. 6.39).

$$
{c}_{l} = \left\lbrack \frac{{\left( -1\right) }^{l}}{{2}^{l}l!}\right\rbrack \sqrt{\frac{\left\lbrack \left( 2l + 1\right) \left( 2l\right) !\right\rbrack }{4\pi }}. tag{3. 6.35}
$$

从 (3.6.34) 式出发, 可以连续运用

$$
\langle \widehat{\mathbf{n}} \mid l, m - 1\rangle = \frac{\left\langle \widehat{\mathbf{n}} \mid {L}_{ - } \mid l, m\right\rangle }{\sqrt{\left( {l + m}\right) \left( {l - m + 1}\right) }\hslash }
$$

$$
= \frac{1}{\sqrt{\left( {l + m}\right) \left( {l - m + 1}\right) }}{e}^{-{i\phi }}\left( {-\frac{\partial }{\partial \theta } + i\cot \theta \frac{\partial }{\partial \phi }}\right) \langle \widehat{\mathbf{n}} \mid l, m\rangle tag{3.6.36}
$$

以得到所有固定 $l$ 的 ${Y}_{l}^{m}$ . 因为这在许多初等量子力学教科书中都做过 (例如, Townsend, 2000),在这里不准备给出其细节. $m \geq 0$ 时的结果为

$$
{Y}_{l}^{m}\left( {\theta ,\phi }\right) = \frac{{\left( -1\right) }^{l}}{{2}^{l}l!}\sqrt{\frac{\left( 2l + 1\right) }{4\pi }\frac{\left( {l + m}\right) !}{\left( {l - m}\right) !}}{e}^{i{m}^{\phi }}\frac{1}{{\sin }^{m}\theta }\frac{{d}^{l - m}}{d{\left( \cos \theta \right) }^{l - m}}{\left( \sin \theta \right) }^{2l}, tag{3. 6.37}
$$

并且用

$$
{Y}_{l}^{-m}\left( {\theta ,\phi }\right) = {\left( -1\right) }^{m}{\left\lbrack {Y}_{l}^{m}\left( \theta ,\phi \right) \right\rbrack }^{ \cdot }. tag{3. 6.38}
$$

定义 ${Y}_{l}^{-m}$ . 无论 $m$ 是正还是负, ${Y}_{l}^{m}\left( {\theta ,\phi }\right)$ 的 $\theta$ 相关部分是 ${\left\lbrack \sin \theta \right\rbrack }^{\left| m\right| }$ 乘以一个最高幂为 $l -$ $\left| m\right|$ 的 $\cos \theta$ 的多项式. 对于 $m = 0$ ,我们得到

$$
{Y}_{l}^{0}\left( {\theta ,\phi }\right) = \sqrt{\frac{{2l} + 1}{4\pi }}{P}_{l}\left( {\cos \theta }\right) . tag{3. 6.39}
$$

单从角动量对易关系的观点来看,为什么 $l$ 不能是半整数好像不是显然的. 事实表明,几种理由都不利于半整数 $l$ 值. 首先,对于半整数的 $l$ ,并因此对于半整数的 $m$ ,在 ${2\pi }$ 转动下的波函数会获得一个负号,

$$
{e}^{{im}\left( {2\pi }\right) } = - 1, tag{3. 6.40}
$$

结果, 波函数就会不是单值的. 在 2.4 节曾经指出, 波函数必须是单值的, 这是由于要求一个态右矢用位置的本征右矢的展开是唯一的. 可以证明,如果定义为 $\mathbf{x} \times \mathbf{p}$ 的 $\mathbf{L}$ 被确认为转动的生成元,那么在 ${2\pi }$ 转动下的波函数必须获得一个正号. 这是由于一个 ${2\pi }$ 转动后的态的波函数就是没有符号改变的、原始的波函数:

$$
\left\langle {{\mathbf{x}}^{\prime }\left| {\exp \left( \frac{-i{L}_{z}{2\pi }}{\hslash }\right) \alpha }\right| = \left\langle {{x}^{\prime }\cos {2\pi } + {y}^{\prime }\sin {2\pi },{y}^{\prime }\cos {2\pi } - {x}^{\prime }\sin {2\pi },{z}^{\prime }}\right\rangle \alpha }\right\rangle tag{3. 6.41}
$$

$$
= \left\langle {{\mathbf{x}}^{\prime } \mid \alpha }\right\rangle ,
$$

其中用到了有限角度情况下的 (3.6.6) 式. 其次,假设倘若具有半整数 $l$ 的 ${Y}_{l}^{m}\left( {\theta ,\phi }\right)$ 是可能的. 为确定起见,选择最简单的情况, $l = m = \frac{1}{2}$ . 按照 (3.6.34) 式,就会有

$$
{Y}_{1/2}^{1/2}\left( {\theta ,\phi }\right) = {c}_{1/2}{e}^{{i\phi }/2}\sqrt{\sin \theta }. tag{3. 6.42}
$$

由 ${L}_{ - }$ 的性质 [见 (3.6.36) 式],就会得到

$$
{Y}_{1/2}^{-1/2}\left( {\theta ,\phi }\right) = {e}^{-{i\phi }}\left( {-\frac{\partial }{\partial \theta } + i\cot \theta \frac{\partial }{\partial \phi }}\right) \left( {{c}_{1/2}{e}^{{i\phi }/2}\sqrt{\sin \theta }}\right)
$$

$$
= - {c}_{1/2}{e}^{-{i\phi }/2}\cot \theta \sqrt{\sin \theta }\text{.} tag{3. 6.43}
$$

这个表示式是不允许的,因为在 $\theta = 0,\pi$ 处它是奇异的. 更糟糕的是,由偏微分方程

$$
\left\langle {\widehat{\mathbf{n}}\left| {L}_{ - }\right| \frac{1}{2}, - \frac{1}{2}}\right\rangle = - i\hslash {e}^{-{i\phi }}\left( {-i\frac{\partial }{\partial \theta } - \cot \theta \frac{\partial }{\partial \phi }}\right) \left\langle {\widehat{\mathbf{n}}\left| {\;\frac{1}{2}}\right. , - \frac{1}{2}}\right\rangle tag{3. 6.44}
$$

$$
= 0
$$

直接得

$$
{Y}_{1/2}^{-1/2} = {c}_{1/2}^{\prime }{e}^{-{i\phi }/2}\sqrt{\sin \theta } tag{3. 6.45}
$$

它与 (3.6.43) 式有尖锐的矛盾. 最后, 从微分方程的斯特姆-刘维尔 (Sturm-Liouville) 理论知道,(3.6.28) 式的 $l$ 为整数的解构成一个完备集. $\theta$ 和 $\phi$ 的任意函数都可以用只取整数 $l$ 和 $m$ 的 ${Y}_{l}^{m}$ 展开. 由于所有这些理由,苦思冥想半整数 $l$ 值的轨道角动量是徒劳的. (这里的半整数, 确切地讲, 应为半奇数, 因为半偶数仍为整数, 显然不会产生这些矛盾. 遵照作者的原文, 这里没有改正. ——译者注)

\subsection{球谐函数作为转动矩阵}

通过用上一节引入的转动矩阵的观点讨论球谐函数来结束关于轨道角动量的这一节. 通过将适当的转动算符作用于沿正的 $z$ 方向的方向本征右矢 $|\widehat{\mathbf{z}}\rangle$ 上,构造最普遍的方向本征右矢 $|\widehat{\mathbf{n}}\rangle$ ,可以很容易地建立所想要的、两种方法间的联系. 希望找到这样的 $\mathcal{D}\left( R\right)$ , 使得

$$
\left| {\widehat{\mathbf{n}}\rangle = \mathcal{D}\left( R\right) }\right| \widehat{\mathbf{z}}\rangle tag{3. 6.46}
$$

可以采用在 3.2 节构造 $\mathbf{\sigma } \cdot \widehat{\mathbf{n}}$ 的本征旋量时使用过的技巧. 首先绕 $y$ 轴转 $\theta$ 角,然后绕 $z$ 轴转 $\phi$ 角,见图 3.3,取 $\beta \rightarrow \theta ,\alpha \rightarrow \phi$ . 用欧拉角的符号,有

$$
\mathcal{D}\left( R\right) = \mathcal{D}\left( {\alpha = \phi ,\beta = \theta ,\gamma = 0}\right) . tag{3. 6.47}
$$

把 (3.6.46) 式写成

$$
\left| {\widehat{\mathbf{n}}\rangle = \mathop{\sum }\limits_{l}\mathop{\sum }\limits_{m}\mathcal{D}\left( R\right) }\right| l, m\rangle \langle l, m \mid \widehat{\mathbf{z}}\rangle . tag{3. 6.48}
$$

可以看到,当用 $|l, m\rangle$ 展开时, $|\widehat{\mathbf{n}}\rangle$ 包含了所有可能的 $l$ 值. 然而,当用 $\left\langle {l,{m}^{\prime }}\right|$ 左乘上述方程时,在 $l$ 求和中只有一项有贡献,即

$$
\left\langle {l,{m}^{\prime } \mid \widehat{\mathbf{n}}}\right\rangle = \mathop{\sum }\limits_{m}{\mathcal{D}}_{{m}^{\prime }m}^{\left( l\right) }\left( {\alpha = \phi ,\beta = \theta ,\gamma = 0}\right) \langle l, m \mid \widehat{\mathbf{z}}\rangle . tag{3.6.49}
$$

此时 $\langle l, m \mid \widehat{\mathbf{z}}\rangle$ 只是一个数. 事实上,它就是在 $\theta = 0$ ,但 $\phi$ 不确定时算出的 ${Y}_{l}^{m} \cdot \left( {\theta ,\phi }\right)$ . 在 $\theta = 0$ 时, $m \neq 0$ 时的 ${Y}_{l}^{m}$ 为零,它也可以直接从 $|\widehat{\mathbf{z}}\rangle$ 是 ${L}_{z}$ (它等于 $x{p}_{y} - y{p}_{x}$ ) 的本征值为零的本征右矢看到. 所以可以写出

$$
\left\langle {l, m \mid \widehat{\mathbf{z}}}\right\rangle = {Y}_{l}^{m}\left( {\theta = 0,\phi \text{ 不确定 }}\right) {\delta }_{m0}
$$

$$
= {\left. \sqrt{\frac{\left( 2l + 1\right) }{4\pi }}{P}_{l}\left( \cos \theta \right) \right| }_{\cos \theta = 1}{\delta }_{m0} tag{3. 6.50}
$$

$$
= \sqrt{\frac{\left( 2l + 1\right) }{4\pi }}{\delta }_{m0}.
$$

返回到 (3.6.49) 式, 有

$$
{Y}_{l}^{\prime \prime \prime }\left( {\theta ,\phi }\right) = \sqrt{\frac{\left( 2l + 1\right) }{4\pi }}{D}_{{m}^{\prime }0}^{\left( l\right) }\left( {\alpha = \phi ,\beta = \theta ,\gamma = 0}\right) tag{3. 6.51}
$$

或

$$
{\left. {D}_{m0}^{\left( l\right) }\left( \alpha ,\beta ,\gamma = 0\right) = \sqrt{\frac{4\pi }{\left( 2l + 1\right) }}{Y}_{l}^{m}\left( \theta ,\phi \right) \right| }_{\theta = \beta ,\phi = \alpha } tag{3. 6.52}
$$

注意 $m = 0$ 的情况是特别重要的:

$$
{\left. {d}_{00}^{\left( l\right) }\left( \beta \right) \right| }_{\beta = 0} = {P}_{l}\left( {\cos \theta }\right) . tag{3. 6.53}
$$

\section{中心势的薛定谔方程}
由下列形式的哈密顿量

$$
H = \frac{{\mathbf{p}}^{2}}{2m} + V\left( r\right) \;{r}^{2} = {\mathbf{x}}^{2} tag{3.7.1}
$$

描述的问题是物理世界中许多情况的基础. 这个哈密顿量的基本重要性在于它是球对称的. 在经典物理中, 预期轨道角动量在这样的一个系统中是守恒的. 在量子力学中这也是对的, 因为很容易证明

$$
\left\lbrack {\mathbf{L},{\mathbf{p}}^{2}}\right\rbrack = \left\lbrack {\mathbf{L},{\mathbf{x}}^{2}}\right\rbrack = 0 tag{3.7.2}
$$

因此,如果 $H$ 由 (3.7.1) 式给定,则

$$
\left\lbrack {\mathbf{L}, H}\right\rbrack = \left\lbrack {{\mathbf{L}}^{2}, H}\right\rbrack = 0 tag{3. 7.3}
$$

把这样的问题称之为中心势或中心力问题. 既使哈密顿量并不严格地是这种形式, 当考虑构建在对中心势问题做 “微小” 修正的近似方案时, 它经常是一个很好的出发点.

在这一节将讨论由 (3.7.1) 式导致的本征函数的一些一般性质以及几个有代表性的中心势问题. 为了解更多的细节, 建议读者参考诸多非常深入地探讨了这些问题的优秀教材.

\subsection{径向方程}

方程 (3.7.3) 清晰地表明,应寻找能量本征态 $\left| {\alpha \rangle = }\right| {Elm}\rangle$ ,其中

$$
H\left| {{Elm}\rangle = E}\right| {Elm}\rangle , tag{3. 7.4}
$$

$$
{\mathbf{L}}^{2}\left| {{Elm}\rangle = l\left( {l + 1}\right) {\hslash }^{2}}\right| {Elm}\rangle , tag{3. 7.5}
$$

$$
{L}_{z}\left| {{Elm}\rangle = m\hslash }\right| {Elm}\rangle . tag{3.7.6}
$$

最容易做到这一点的是,利用坐标表象并且借助 (3.6.22) 式所示的径向函数 ${R}_{EI}\left( r\right)$ 和球谐函数求解适当的本征函数的微分方程. 把 (3.7.1)、(3.7.4)、(3.7.5) 与 (3.6.21)、 (3.6.22) 式结合在一起, 得到径向方程*

$$
\left\lbrack {-\frac{{\hslash }^{2}}{{2m}{r}^{2}}\frac{d}{dr}\left( {{r}^{2}\frac{d}{dr}}\right) + \frac{l\left( {l + 1}\right) {\hslash }^{2}}{{2m}{r}^{2}} + V\left( r\right) }\right\rbrack {R}_{El}\left( r\right) = E{R}_{El}\left( r\right) . tag{3.7.7}
$$

取决于 $V\left( r\right)$ 的具体形式,可用这个方程或者它的一些变形来确定本征函数的径向部分 ${R}_{El}\left( r\right)$ 和/或能量本征值 $E$ .

事实上, 通过代换

$$
{R}_{El}\left( r\right) = \frac{{u}_{El}\left( r\right) }{r}, tag{3.7.8}
$$

可以马上获得有关角动量对本征函数影响的某些深刻理解, 该代换把 (3.7.7) 式约化为

$$
- \frac{{\hslash }^{2}}{2m}\frac{{d}^{2}{u}_{El}}{d{r}^{2}} + \left\lbrack {\frac{l\left( {l + 1}\right) {\hslash }^{2}}{{2m}{r}^{2}} + V\left( r\right) }\right\rbrack {u}_{El}\left( r\right) = E{u}_{El}\left( r\right) . tag{3.7.9}
$$

把上式与球谐函数是单独归一的实际情况结合在一起, 使得整体归一化条件变成

$$
1 = \int {r}^{2}{dr}{R}_{El}^{ * }\left( r\right) {R}_{El}\left( r\right) = \int {dr}{u}_{El}^{ * }\left( r\right) {u}_{El}\left( r\right) , tag{3.7.10}
$$

我们看到, ${u}_{El}\left( r\right)$ 可以解释为一个在 “等效势”

$$
{V}_{\text{等效 }}\left( r\right) = V\left( r\right) + \frac{l\left( {l + 1}\right) {\hslash }^{2}}{{2m}{r}^{2}} tag{3. 7.11}
$$

---

* 非常抱歉用 $m$ 既表示了 “质量”,又表示了角动量量子数. 然而,从本节上下文应当清楚哪个是哪个.

---

中运动的粒子的一维波函数. 方程 (3.7.11) 展示,如果 $l \neq 0$ ,则存在一个 “角动量势垒”,如图 3.5 所示. 从量子力学讲,这意味着除了 $s$ 态,粒子定位在原点附近的振幅 (因此概率) 是很小的. 正如稍后将看到的, 例如, 这件事在原子中有重要的物理后果.

可以做更为定量的解释. 假定势能函数 $V\left( r\right)$ 并不是这样奇异,以致 $\mathop{\lim }\limits_{{r \rightarrow 0}}{r}^{2}V\left( r\right) =$ 0 . 于是,对于很小的 $r$ 值,(3.7.9) 式变成

$$
\frac{{d}^{2}{u}_{El}}{d{r}^{2}} = \frac{l\left( {l + 1}\right) }{{r}^{2}}{u}_{El}\left( r\right) \;\left( {r \rightarrow 0}\right) , tag{3. 7.12}
$$

它的通解为 $u\left( r\right) = A{r}^{l + 1} + B{r}^{-l}$ . 很容易想到马上取 $B = 0$ ,因为 $1/{r}^{l}$ 在 $r \rightarrow 0$ 时产生严重的奇异性,特别是当 $l$ 较大时. 然而,取 $B = 0$ 还有更好的理由,该理由是源于量子力学的基础.


图 3.5 控制 “径向波函数” ${u}_{EI}\left( r\right)$ 行为的 “等效势”. 如果势能 $V\left( r\right)$ (如图虚线所示) 在原点不太奇异,则对所有 $l \neq 0$ 的态都存在一个角动量位垒,它使得一个粒子定位于原点附近是极不可能的.

考虑由 (2.4.16) 式给出的概率流. 这是一个矢量, 它的径向分量为

$$
{j}_{r} = \widehat{\mathbf{r}} \cdot \mathbf{j} = \frac{\hslash }{m}\operatorname{Im}\left( {{\psi }^{ * }\frac{\partial }{\partial r}\psi }\right) tag{3. 7.13}
$$

$$
= \frac{\hslash }{m}{R}_{El}\left( r\right) \frac{d}{dr}{R}_{El}\left( r\right) .
$$

现在,如果当 $r \rightarrow 0$ 时, ${R}_{El}\left( r\right) \rightarrow {r}^{l}$ ,则 ${j}_{r} \propto l{r}^{{2l} - 1}$ . 于是,就像它应该的那样,从中心位于原点附近的一个小球 “泄漏” 出去的概率,对于所有的 $l$ 值都是 ${4\pi }{r}^{2}{j}_{r} \propto l{r}^{{2l} + 1} \rightarrow 0$ .

然而,如果当 $r \rightarrow 0$ 时, ${R}_{El}\left( r\right) \rightarrow {r}^{-\left( {l + 1}\right) }$ ,则 ${j}_{r} \propto \left( {l + 1}\right) {r}^{-{2l} - 3}$ ,并且当 $r \rightarrow 0$ 时从小球流出的概率是 ${4\pi }{r}^{2}{j}_{r} \propto \left( {l + 1}\right) {r}^{-{2l} - 1} \rightarrow \infty$ ,甚至 $l = 0$ 时也是如此. 于是,作为 (3.7.12) 式的解,必须只选择 $u\left( r\right) \propto {r}^{l + 1}$ ; 否则,就会破坏量子力学振幅的概率解释.

所以, 有

$$
{R}_{El}\left( r\right) \rightarrow {r}^{l}\text{ 当 }r \rightarrow 0. tag{3. 7.14}
$$

这个关系式有深远的影响. 首先,它体现了图 3.5 所示的 “角动量势垒”,因为除了 $s$ 态, 波函数都趋向于零. 更实际地说, 它意味着, 比如, 在原子核的区域内找到一个原子的电子的概率遵循 ${\left( R/{a}_{0}\right) }^{2l}$ ,其中 $R < < {a}_{0}$ 是原子核的大小,而 ${a}_{0}$ 是波尔半径. 当进入原子结构研究时, 这些概念将变得明确了.

在研究大 $r$ 处趋于零的势能函数 $V\left( r\right)$ 的束缚态时,有另一种形式的径向方程可以考虑. 对于 $r \rightarrow \infty ,\left( {3.7.9}\right)$ 式变成

$$
\frac{{d}^{2}{u}_{E}}{d{r}^{2}} = {\kappa }^{2}u\;{\kappa }^{2} \equiv - {2mE}/{\hslash }^{2} > 0\;r \rightarrow \infty , tag{3. 7.15}
$$

因为对于束缚态 $E < 0$ . 这个方程的解简单地就是 ${u}_{E}\left( r\right) \propto {e}^{-{\kappa r}}$ . 它也表明,无量纲变量 $\rho \equiv$ ${\kappa r}$ 对重写径向方程将是有用的. 于是,既消除波函数短距离的奇异行为,又消除长距离的奇异行为, 而写成

$$
{u}_{El}\left( \rho \right) = {\rho }^{l + 1}{e}^{-\rho }w\left( \rho \right) , tag{3. 7.16}
$$

其中函数 $w\left( \rho \right)$ 是 “行为很好的”,且满足方程

$$
\frac{{d}^{2}w}{d{\rho }^{2}} + 2\left( {\frac{l + 1}{\rho } - 1}\right) \frac{dw}{d\rho } + \left\lbrack {\frac{V}{E} - \frac{2\left( {l + 1}\right) }{\rho }}\right\rbrack w = 0. tag{3. 7.17}
$$

(导出这个方程的处理方法留给读者.) 随后,将集中精力对于特定的函数 $V\left( {r = \rho /\kappa }\right)$ 的 (3.7.17) 式求解 $w\left( \rho \right)$ .

\subsection{自由粒子和无穷深球形势阱}

在 2.5 节已经利用笛卡尔坐标, 得到了三维自由粒子问题的解. 当然可以利用球对称性和角动量处理同样的问题. 从 (3.7.7) 式出发, 写出

$$
E \equiv \frac{{\hslash }^{2}{k}^{2}}{2m}\;\text{ 和 }\;\rho \equiv {kr} tag{3. 7.18}
$$

然后得到修改后的径向方程

$$
\frac{{d}^{2}R}{d{\rho }^{2}} + \frac{2}{\rho }\frac{dR}{d\rho } + \left\lbrack {1 - \frac{l\left( {l + 1}\right) }{{\rho }^{2}}}\right\rbrack R = 0. tag{3. 7.19}
$$

这个方程是一个著名的微分方程,它的解称之为球贝塞尔 (Bessel) 函数 ${j}_{l}\left( \rho \right)$ 和 ${n}_{l}\left( \rho \right)$ , 其中

$$
{j}_{l}\left( \rho \right) = {\left( -\rho \right) }^{l}{\left\lbrack \frac{1}{\rho }\frac{d}{d\rho }\right\rbrack }^{l}\left( \frac{\sin \rho }{\rho }\right) , tag{3. 7. 20a}
$$

$$
{n}_{l}\left( \rho \right) = - {\left( -\rho \right) }^{l}{\left\lbrack \frac{1}{\rho }\frac{d}{d\rho }\right\rbrack }^{l}\left( \frac{\cos \rho }{\rho }\right) . tag{3. 7. 20b}
$$

很容易证明,当 $\rho \rightarrow 0$ 时, ${j}_{l}\left( \rho \right) \rightarrow {\rho }^{l}$ 而 ${n}_{l}\left( \rho \right) \rightarrow {\rho }^{-l - 1}$ . 因此, ${j}_{l}\left( \rho \right)$ 对应着 (3.7.14) 式,该解是在这里唯一考虑的解 * 指出球贝塞尔函数定义在整个复平面也是有用的, 而且可以证明

$$
{j}_{l}\left( z\right) = \frac{1}{2{i}^{l}}{\int }_{-1}^{1}{ds}{e}^{izs}{P}_{l}\left( s\right) . tag{3. 7.21}
$$

前几个球贝塞尔函数为

$$
{j}_{0}\left( \rho \right) = \frac{\sin \rho }{\rho }. tag{3.7.22}
$$

$$
{j}_{1}\left( \rho \right) = \frac{\sin \rho }{{\rho }^{2}} - \frac{\cos \rho }{\rho }, tag{3. 7.23}
$$

$$
{j}_{2}\left( \rho \right) = \left\lbrack {\frac{3}{{\rho }^{3}} - \frac{1}{\rho }}\right\rbrack \sin \rho - \frac{3\cos \rho }{{\rho }^{2}}. tag{3. 7.24}
$$

---

* 在处理 “硬球散射” 问题时,原点显然被排除了,因而解 ${n}_{l}\left( \rho \right)$ 也被保留了下来. 对于一个给定的 $l$ ,这两个解的相对相位称为相移.

---

这个结果立即可以应用到一个粒子被禁闭在一个无穷深球势阱中的情况,即,在 $r < a$ 区域内其势能函数 $V\left( r\right) = 0$ ,但波函数在 $r = a$ 处被约束为零. 对于任何给定的 $l$ 值,这些约束将导致 “量子化条件” ${j}_{l}\left( {ka}\right) = 0$ ,即 ${ka}$ 等于球贝塞尔函数零点的集合. 对 $l = 0$ ,这些值显然是 ${ka} = \pi ,{2\pi },{3\pi },\cdots$ . 对于其他的 $l$ 值,能计算零点的计算机程序是很容易使用的. 可以发现

$$
{E}_{t = 0} = \frac{{\hslash }^{2}}{{2m}{a}^{2}}\left\lbrack {{\pi }^{2},{\left( 2\pi \right) }^{2},{\left( 3\pi \right) }^{2},\cdots }\right\rbrack , tag{3. 7.25}
$$

$$
{E}_{t = 1} = \frac{{\hslash }^{2}}{{2m}{a}^{2}}\left\lbrack {{4.49}^{2},{7.73}^{2},{10.90}^{2},\cdots }\right\rbrack , tag{3. 7.26}
$$

$$
{E}_{l = 2} = \frac{{\hslash }^{2}}{{2m}{a}^{2}}\left\lbrack {{5.84}^{2},{8.96}^{2},{12.25}^{2},\cdots }\right\rbrack . tag{3. 7.27}
$$

应当注意,这一系列的能级显示没有 $l$ 的简并. 的确,除了不同阶的球贝塞尔函数零点之间偶然的相等之外, 这样的简并能级是不可能的.

\subsection{各向同性谐振子}

确定哈密顿量

$$
H = \frac{{\mathbf{p}}^{2}}{2m} + \frac{1}{2}m{\omega }^{2}{r}^{2} tag{3. 7.28}
$$

的能量本征值是很简单的. 用下式引入无量纲能量 $\lambda$ 和径向坐标 $\rho$

$$
E = \frac{1}{2}\hslash {\omega \lambda }\;\text{ 和 }\;r = {\left\lbrack \frac{\hslash }{m\omega }\right\rbrack }^{1/2}\rho , tag{3. 7.29}
$$

把 (3.7.9) 式变换为

$$
\frac{{d}^{2}u}{d{\rho }^{2}} - \frac{l\left( {l + 1}\right) }{{\rho }^{2}}u\left( \rho \right) + \left( {\lambda - {\rho }^{2}}\right) u\left( \rho \right) = 0. tag{3. 7.30}
$$

虽然因为大 $r$ 处的 $V\left( r\right)$ 不趋于零因而不能使用 (3.7.16) 式,但明确地移除大 (和小) $\rho$ 处的奇异行为仍然是值得做的. 代替该式, 写出

$$
u\left( \rho \right) = {\rho }^{l + 1}{e}^{-{\rho }^{2}/2}f\left( \rho \right) . tag{3. 7.31}
$$

这就产生了函数 $f\left( \rho \right)$ 的下列微分方程:

$$
\rho \frac{{d}^{2}f}{d{\rho }^{2}} + 2\left\lbrack {\left( {l + 1}\right) - {\rho }^{2}}\right\rbrack \frac{df}{d\rho } + \left\lbrack {\lambda - \left( {{2l} + 3}\right) }\right\rbrack {\rho f}\left( \rho \right) = 0. tag{3. 7.32}
$$

通过把 $f\left( \rho \right)$ 写成一个无穷级数,即

$$
f\left( \rho \right) = \mathop{\sum }\limits_{{n = 0}}^{\infty }{a}_{n}{\rho }^{n}. tag{3. 7.33}
$$

来求解方程 (3.7.32). 把这个级数插入到微分方程中,并置 $\rho$ 的各次幂都等于零. 唯一留存下来的 ${\rho }^{0}$ 项是 $2\left( {l + 1}\right) {a}_{1}$ ,所以

$$
{a}_{1} = 0 tag{3. 7.34}
$$

正比于 ${\rho }^{1}$ 的项允许把 ${a}_{2}$ 与 ${a}_{0}$ 联系起来,接下来它们可以通过归一条件确定. 继续做下去, (3.7.32) 式变成

$$
\mathop{\sum }\limits_{{n = 2}}^{\infty }\left\{ {\left( {n + 2}\right) \left( {n + 1}\right) {a}_{n + 2} + 2\left( {l + 1}\right) \left( {n + 2}\right) {a}_{n + 2} - {2n}{a}_{n} + \left\lbrack {\lambda - \left( {{2l} + 3}\right) }\right\rbrack {a}_{n}}\right\} {\rho }^{n + 1} = 0, tag{3. 7.35}
$$

最终, 它导致递推关系

$$
{a}_{n + 2} = \frac{{2n} + {2l} + 3 - \lambda }{\left( {n + 2}\right) \left( {n + {2l} + 3}\right) }{a}_{n}. tag{3. 7.36}
$$

立即看到, $f\left( \rho \right)$ 只包含 $\rho$ 的偶次幂,因为 (3.7.34) 式和 (3.7.36) 式意味着对于奇数的 $n$ 有 ${a}_{n} = 0$ . 另外,当 $n \rightarrow \infty$ 时,有

$$
\frac{{a}_{n + 2}}{{a}_{n}} \rightarrow \frac{2}{n} = \frac{1}{q}, tag{3. 7.37}
$$

其中 $q = n/2$ 既包括奇整数也包括偶整数. 因此,对于大 $\rho ,\left( {3.7.33}\right)$ 式变成

$$
f\left( \rho \right) \rightarrow \text{ 常数 } \times \mathop{\sum }\limits_{q}\frac{1}{q!}{\left( {\rho }^{2}\right) }^{q} \propto {e}^{{\rho }^{2}}. tag{3. 7.38}
$$

换言之,由 (3.7.31) 式给出的 $u\left( \rho \right)$ 在大 $\rho$ 时将按指数增长 (因此将不能满足归一条件),除非该级数中断. 所以,对于某个偶数值的 $n = {2q}$ 有

$$
{2n} + {2l} + 3 - \lambda = 0 tag{3. 7.39}
$$

并且能量本征值为

$$
{E}_{ql} = \left( {{2q} + l + \frac{3}{2}}\right) \hslash \omega \equiv \left( {N + \frac{3}{2}}\right) \hslash \omega tag{3. 7.40}
$$

其中 $q = 0,1,2,\cdots, l = 0,1,2,\cdots$ 和 $N \equiv {2q} + l$ . 人们经常称 $N$ 为 “主” 量子数. 可以证明. $q$ 给出径向波函数的节点数.

完全不像方势阱,三维各向同性谐振子具有对量子数 $l$ 简并的能量本征值. 对 $N = 1$ 有三个态 (都是 $l = 1$ ). 对 $N = 2$ ,有五个 $l = 2$ 的态,加上一个 $q = 1$ 且 $l = 0$ 的态,总共给出六个态. 注意,对于偶数 (奇数) 的 $N$ ,只有偶数 (奇数) 的 $l$ 值是允许的. 因此,波函数的宇称是偶还是奇是随着 $N$ 值来确定的.

当势能函数是一个某种有限大小的阱时, 这些波函数是计算各种自然现象的最流行的基态. 这类方法的最大成就之一是原子核壳模型, 单独的质子和中子被描绘成在一个势能函数的势阱中独立运动, 而该势能函数是由原子核中所有核子的累积效应产生的. 图 3.6 将原子核内观测到的能级与对于各向同性谐振子和无穷深球势阱求得的能级做了比较.

把哈密顿量 (3.7.28) 式的本征态标记为 $\left| {qlm}\right\rangle$ 或 $\left| {Nlm}\right\rangle$ 是很自然的. 然而,这个哈密顿量还可以写成

$$
H = {H}_{x} + {H}_{y} + {H}_{z}, tag{3. 7.41}
$$

其中 ${H}_{i} = {a}_{i}^{ \dagger }{a}_{i} + \frac{1}{2}$ 是一个沿 $i = x, y, z$ 方向的独立的一维谐振子. 在这种方法中,把本征态标记为 $\left| {{n}_{x},{n}_{y},{n}_{z}}\right\rangle$ ,能量本征值为

$$
E = \left( {{n}_{x} + \frac{1}{2} + {n}_{x} + \frac{1}{2} + {n}_{x} + \frac{1}{2}}\right) \hslash \omega tag{3. 7.42}
$$

$$
= \left( {N + \frac{3}{2}}\right) {h\omega }
$$

其中, $N = {n}_{x} + {n}_{y} + {n}_{z}$ . 很容易从数值上证明,不管使用哪种基,对于前几个能级简并都一样. 总的说来,证明这一点并推导出从一个基变到另一个基的幺正变换矩阵 $\left\langle {{n}_{r},{n}_{y},{n}_{z} \mid {qlm}}\right\rangle$ 是一个有趣的练习. (见本章末尾的习题 3.21.)

图 3.6 原子核壳模型中的能级,取自 Haxel, Jensen 和 Suess, Zeitschrift für Physik 128 (1950) 295. 左边一列是三维各向同性谐振子的能级, 紧随着的一列是无穷深球形势阱时的能级. 其右边的两列是修改了的无穷深方势阱的能级, 先是有限壁高势阱的情况, 然后是 “磨圆了角的”方势阱的情况. 最右边能级图显示了包括了核子的自旋与轨道角动量相互作用的势阱的能级. 最后一列标出了总角动量量子数.

\subsection{库仑 (Coulomb) 势}

或许物理学中最重要的势能函数是

$$
V\left( \mathbf{x}\right) = - \frac{Z{e}^{2}}{r}, tag{3. 7.43}
$$

很明显,选择常数 $Z{e}^{2}$ 是为了使 (3.7.43) 式能表示原子序数为 $Z$ 的一个单电子原子的势能. 除了库仑力以及经典引力外, 它广泛地用在适用于很多物理系统的模型中*. 在这里, 考虑基于这样一个函数的径向方程以及得出的能量本征值.

$1/r$ 势满足导出 (3.7.17) 式的所有要求. 因此,通过确定函数 $w\left( \rho \right)$ ,寻找形式为

---

* 的确, $1/r$ 势能函数源于任意一个中间交换无质量粒子的、空间维度为三维的量子场论. 见 Zee (2010) 中的 1.6 章.

---

(3.7.16) 式的解. 定义

$$
{\rho }_{0} = {\left\lbrack \frac{2m}{-E}\right\rbrack }^{1/2}\frac{Z{e}^{2}}{\hslash } = {\left\lbrack \frac{{2m}{c}^{2}}{-E}\right\rbrack }^{1/2}{Z\alpha }, tag{3. 7.44}
$$

其中 $\alpha \equiv {e}^{2}/\hslash c \approx 1/{137}$ 是精细结构常数,则 (3.7.17) 式变成

$$
\rho \frac{{d}^{2}w}{d{\rho }^{2}} + 2\left( {l + 1 - \rho }\right) \frac{dw}{d\rho } + \left\lbrack {{\rho }_{0} - 2\left( {l + 1}\right) }\right\rbrack w\left( \rho \right) = 0. tag{3. 7.45}
$$

当然能够利用一种级数方法开始求解 (3.7.45) 式, 并导出系数的递推关系, 就像对 (3.7.32)式所做的那样. 然而, 事实证明这个解实际上已经为众所周知.

方程 (3.7.45) 可以写成库默尔 (Kummer) 方程:

$$
x\frac{{d}^{2}F}{d{x}^{2}} + \left( {c - x}\right) \frac{dF}{dx} - {aF} = 0, tag{3. 7.46}
$$

其中

$$
x = {2\rho },
$$

$$
c = 2\left( {l + 1}\right) ,
$$

和 ${2a} = 2\left( {l + 1}\right) - {\rho }_{0}$ .(3. 7.47)

(3.7.46)式的解称为合流超几何函数, 它被写成为级数

$$
F\left( {a;c;x}\right) = 1 + \frac{a}{c}\frac{x}{1!} + \frac{a\left( {a + 1}\right) }{c\left( {c + 1}\right) }\frac{{x}^{2}}{2!} + \cdots . tag{3. 7.48}
$$

因此,

$$
w\left( \rho \right) = F\left( {l + 1 - \frac{{\rho }_{0}}{2};2\left( {l + 1}\right) ;{2\rho }}\right) . tag{3. 7.49}
$$

注意,对于大的 $\rho$ ,有

$$
w\left( \rho \right) \approx \mathop{\sum }\limits_{{k\text{ 的 }N}}\frac{a\left( {a + 1}\right) \cdots }{c\left( {c + 1}\right) \cdots }\frac{{\left( 2\rho \right) }^{N}}{N!}
$$

$$
\approx \mathop{\sum }\limits_{{大的N}}\frac{{\left( N/2\right) }^{N}}{{N}^{N}}\frac{{\left( 2\rho \right) }^{N}}{N!} \approx \mathop{\sum }\limits_{{大的N}}\frac{{\left( \rho \right) }^{N}}{N!} \approx {e}^{\rho }.
$$

于是, (3.7.16) 式再一次给出一个会无限增长的径向波函数, 除非级数 (3.7.48) 式中断. 因此,对某个整数 $N$ ,必须有 $a + N = 0$ ,它导致

$$
{\rho }_{0} = 2\left( {N + l + 1}\right) , tag{3. 7.50}
$$

$$
\text{其中}N = 0,1,2\cdots
$$

$$
\text{和}l = 0,1,2,\cdots
$$

按照惯例 (并且,正如将看到的,这样做是有益的),定义主量子数 $n$ 为

$$
n \equiv N + l + 1 = 1,2,3,\cdots , tag{3. 7.51}
$$

$$
\text{其中}l = 0,1,\cdots, n - 1\text{.}
$$

可以看出, 利用 2.5 节曾描述过的生成函数技巧求解库仑势的径向方程是可能的. 见本章末尾的习题 3.22 .

通过把 (3.7.44) 式和 (3.7.50) 式组合在一起, 根据主量子数推出能量本征值, 即

$$
{\rho }_{0} = {\left\lbrack \frac{{2m}{c}^{2}}{-E}\right\rbrack }^{1/2}{Z\alpha } = {2n}, tag{3. 7.52}
$$

它导致

$$
E = - \frac{1}{2}m{c}^{2}\frac{{Z}^{2}{\alpha }^{2}}{{n}^{2}} = - {13.6}\mathrm{{eV}}\frac{{Z}^{2}}{{n}^{2}}, tag{3. 7.53}
$$

其中的数值结果是对一个单电子原子的结果,即 $m{c}^{2} = {511}\mathrm{{keV}}$ . 方程 (3.7.53) 当然就是熟知的巴耳末 (Balmer) 公式.

现在是整理各种观点的时候了. 首先, 现代量子力学预言能级性质与波尔的旧原子模型的结果存在明显的不一致. 波尔模型在角动量本征值 $l$ 与主量子数 $n$ 之间是一一对应的,事实上,基态对应于 $n = l = 1$ . 相反,对 $n = 1$ ,只有 $l = 0$ 是允许的; 而对更高的能级,一些不同的 $l$ 值也都是允许的.

其次,出现了一个自然的长度标度 ${a}_{0}$ . 因为 $\rho = {\kappa r}$ ,其中 $\kappa = \sqrt{-{2mE}/{\hslash }^{2}}$ [见 (3.7.15) 式], 有

$$
\frac{1}{\kappa } = \frac{\hslash }{mc\alpha }\frac{n}{Z} \equiv {a}_{0}\frac{n}{Z}, tag{3. 7.54}
$$

其中

$$
{a}_{0} = \frac{\hslash }{mc\alpha } = \frac{{\hslash }^{2}}{m{e}^{2}} tag{3. 7.55}
$$

被称为波尔半径. 对于一个电子, ${a}_{0} = {0.53} \times {10}^{-8}\mathrm{\;{cm}} = {0.53Å}$ . 它的确是典型的原子尺度.

最后,能量本征值 (3.7.35) 式显示了一种有趣的简并. 本征值只依赖于 $n$ ,而不依赖于 $l$ 和 $m$ . 因此,对于一个 $|{nlm}\rangle$ 态,能级的简并度为

$$
\text{简并度} = \mathop{\sum }\limits_{{l = 0}}^{{n - 1}}\left( {{2l} + 1}\right) = {n}^{2}\text{.} tag{3. 7.56}
$$

实际上, 这个简并度并不是偶然的, 而是反映了库仑势的微妙对称性. 在第 4 章将回到这个问题.

现在, 可以写出氢原子波函数的显示表达式. 回到 (3.6.22) 式, 并且引入适当的归一因子, 有

$$
{\psi }_{nlm}\left( \mathbf{x}\right) = \left\langle {\mathbf{x} \mid {nlm}}\right\rangle = {R}_{nl}\left( r\right) {Y}_{l}^{m}\left( {\theta ,\phi }\right) , tag{3. 7.57}
$$

其中

$$
{R}_{nl}\left( r\right) = \frac{1}{\left( {{2l} + 1}\right) !}{\left( \frac{2Zr}{n{a}_{0}}\right) }^{l}{e}^{-{Zr}/n{a}_{0}}{\left\lbrack {\left( \frac{2Z}{n{a}_{0}}\right) }^{3}\frac{\left( {n + l}\right) !}{{2n}\left( {n - l - 1}\right) !}\right\rbrack }^{1/2}
$$

$$
\times F\left( {-n + l + 1;{2l} + 2;{2Zr}/n{a}_{0}}\right) . tag{3. 7.58}
$$


图 3.7 库仑势情况的径向波函数,其主量子数 $n = 1$ (左) 和 $n = 2$ (右).

图 3.7 画出了 $n = 1$ 和 $n = 2$ 的径向波函数. 正如曾讨论过的,只有 $l = 0$ 的波函数在原点不为零. 还要注意, $l = 0$ 的波函数有 $n - 1$ 个节点,而满足 $l = n - 1$ 的波函数没有节点.

\section{角动量的加法}

角动量加法在现代物理的所有领域——从原子光谱学到原子核与粒子碰撞——都有重要的应用. 此外, 角动量加法的研究提供一个很好的阐明基改变概念的机会, 在第 1 章曾广泛地讨论过.

\subsection{角动量加法的一些简单例子}

在研究角动量加法的形式理论之前, 值得看一下读者可能熟悉的两个简单例子; (1) 如何把轨道角动量与自旋角动量加在一起,以及 (2) 如何把两个自旋 $\frac{1}{2}$ 粒子的自旋角动量加在一起.

前面既研究了忽略除自旋以外的一切量子力学自由度(诸如位置和动量)的自旋 $\frac{1}{2}$ 系统, 也研究考虑了空间自由度 (诸如位置和动量), 但忽略了内部自由度 (比如自旋) 的量子力学粒子. 一个具有自旋的粒子的实际描述当然必须既考虑空间自由度也要考虑内部自由度. 对于一个自旋 $\frac{1}{2}$ 的粒子,其基右矢可以视为由位置本征右矢 $\left\{ \left| {\mathbf{x}}^{\prime }\right\rangle \right\}$ 所张的无穷维右矢空间和由 $\left| {+\rangle \text{与}}\right| - \rangle$ 所张的二维自旋空间的直积空间. 更明确地说,对这种基右矢, 有

$$
\left| {{\mathbf{x}}^{\prime }, \pm }\right\rangle = \left| {\mathbf{x}}^{\prime }\right\rangle \otimes | \pm \rangle , tag{3.8.1}
$$

其中,由 $\left\{ \left| {\mathbf{x}}^{\prime }\right\rangle \right\}$ 所张空间中的任何算符与由 $| \pm \rangle$ 所张的二维空间中的任何算符都对易.

转动算符仍取 $\exp \left( {-i\mathbf{J} \cdot \widehat{\mathbf{n}}\phi /\hslash }\right)$ 的形式,但是转动生成元 $\mathbf{J}$ 现在由两部分构成,即

$$
\mathbf{J} = \mathbf{L} + \mathbf{S}. tag{3.8.2}
$$

实际上, 更明显地是把 (3.8.2) 式写成

$$
\mathbf{J} = \mathbf{L} \otimes 1 + 1 \otimes \mathbf{S}, tag{3.8.3}
$$

在 $\mathbf{L} \otimes 1$ 中的 1 表示自旋空间中的单位算符,而在 $1 \otimes \mathbf{S}$ 中的 1 表示由位置本征右矢所张的无穷维右矢空间中的单位算符. 因为 $\mathbf{L}$ 和 $\mathbf{S}$ 对易,可以写成

$$
\mathcal{D}\left( R\right) = {\mathcal{D}}^{\left( \text{ 轨道 }\right) }\left( R\right) \otimes {\mathcal{D}}^{\left( \text{ 自旋 }\right) }\left( R\right) = \exp \left( \frac{-i\mathbf{L} \cdot \widehat{\mathbf{n}}\phi }{\hslash }\right) \otimes \exp \left( \frac{-i\mathbf{S} \cdot \widehat{\mathbf{n}}\phi }{\hslash }\right) . tag{3.8.4}
$$

一个有自旋的粒子的波函数可以写成

$$
\left\langle {{\mathbf{x}}^{\prime }, \pm \mid \alpha }\right\rangle = {\psi }_{ \pm }\left( {\mathbf{x}}^{\prime }\right) . tag{3.8.5}
$$

两个分量 ${\psi }_{ \pm }$ 经常被排布成如下形式的列矩阵

$$
\left( \begin{array}{ll} {\psi }_{ + } & \left( {\mathbf{x}}^{\prime }\right) \\ {\psi }_{ - } & \left( {\mathbf{x}}^{\prime }\right) \end{array}\right) , tag{3.8.6}
$$

其中, ${\left| {\psi }_{ \pm }\left( {\mathbf{x}}^{\prime }\right) \right| }^{2}$ 表示在 ${\mathbf{x}}^{\prime }$ 处找到该粒子自旋分别为向上和向下的概率密度. 代替 $\left| {\mathbf{x}}^{\prime }\right\rangle$ 作为空间部分的基右矢,可以使用 $|n, l, m\rangle$ ,它是 ${\mathbf{L}}^{2}$ 和 ${L}_{z}$ 的本征右矢,本征值分别为 $l\left( {l + 1}\right) {\hslash }^{2}$ 和 ${m}_{l}\hslash$ . 对于自旋部分, $| \pm \rangle$ 是 ${\mathbf{S}}^{2}$ 和 ${S}_{z}$ 的本征右矢,本征值分别为 $3{\hslash }^{2}/4$ 和 $\pm \hslash /2$ . 然而正如稍后将证明的,可以利用 ${\mathbf{J}}^{2},{J}_{z},{\mathbf{L}}^{2}$ 和 ${\mathbf{S}}^{2}$ 的本征右矢作为基矢. 换句话说,可以把一个有自旋的粒子的态右矢或者用 ${\mathbf{L}}^{2},{\mathbf{S}}^{2},{L}_{z}$ 和 ${S}_{z}$ 的共同本征态展开,或者用 ${\mathbf{J}}^{2},{J}_{z},{\mathbf{L}}^{2}$ 和 ${\mathbf{S}}^{2}$ 的共同本征态展开. 下面将详细地研究这两种描述是如何关联的.

作为第二个例子,研究两个自旋 $\frac{1}{2}$ 的粒子——比方说两个电子——同时略去它们的轨道自由度. 总自旋算符通常写成

$$
\mathbf{S} = {\mathbf{S}}_{1} + {\mathbf{S}}_{2}, tag{3.8.7}
$$

但是再一次将其理解为

$$
{\mathbf{S}}_{1} \otimes 1 + 1 \otimes {\mathbf{S}}_{2}, tag{3.8.8}
$$

其中, 在第一 (第二) 项中 1 代表电子 2 (1) 的自旋空间中的单位算符. 当然有

$$
\left\lbrack {{S}_{1x},{S}_{2y}}\right\rbrack = 0 tag{3.8.9}
$$

等. 在电子 $1\left( 2\right)$ 的空间内,有通常的对易关系

$$
\left\lbrack {{S}_{1x},{S}_{1y}}\right\rbrack = i\hslash {S}_{1z},\left\lbrack {{S}_{2x},{S}_{2y}}\right\rbrack = i\hslash {S}_{2z},\cdots , tag{3.8.10}
$$

作为 (3.8.9) 式和 (3.8.10) 式的直接结果. 对总自旋算符有

$$
\left\lbrack {{S}_{x},{S}_{y}}\right\rbrack = i\hslash {S}_{z} tag{3.8.11}
$$

等.

各种自旋算符的本征值表示如下

$$
{\mathbf{S}}^{2} = {\left( {\mathbf{S}}_{1} + {\mathbf{S}}_{2}\right) }^{2} : s\left( {s + 1}\right) {\hslash }^{2}
$$

$$
{S}_{z} = {S}_{1z} + {S}_{2z} : {mh} tag{3.8.12}
$$

$$
{S}_{1z} : {m}_{1}\hslash
$$

$$
{S}_{2z} : {m}_{2}\hslash
$$

同样地,可以把与两个电子任意自旋态对应的右矢,用 ${\mathbf{S}}^{2}$ 和 ${S}_{z}$ 的本征右矢或者 ${S}_{1z}$ 和 ${S}_{2z}$ 的本征右矢展开. 这两种可能性如下:

1. 基于 ${S}_{1z}$ 和 ${S}_{2z}$ 的本征右矢的表示 $\left\{ {{m}_{1},{m}_{2}}\right\}$ :

$$
\left| {+ + \rangle ,}\right| + - \rangle ,\left| {- + \rangle \text{和}}\right| - - \rangle \text{,} tag{3.8.13}
$$

其中 $| + - \rangle$ 表示 ${m}_{1} = \frac{1}{2},{m}_{2} = - \frac{1}{2}$ ,以及等等.

2. 基于 ${\mathbf{S}}^{2}$ 和 ${S}_{z}$ 的本征右矢的表示 (或三重态-单态表示) $\{ s, m\}$ :

$$
\left| {s = 1, m = \pm 1,0\rangle ,}\right| s = 0, m = 0\rangle , tag{3.8.14}
$$

其中 $s = 1\left( {s = 0}\right)$ 被称为自旋三重态 (或单态).

注意, 每一集合有四个基右矢. 两组基右矢之间的关系如下:

$$
\left| {s = 1, m = 1\rangle = }\right| + + \rangle , tag{3.8.15a}
$$

$$
|s = 1, m = 0\rangle = \left( \frac{1}{\sqrt{2}}\right) \left( {\left| {+ - \rangle + }\right| - + \rangle }\right) , tag{3.8.15b}
$$

$$
\left| {s = 1, m = - 1\rangle = }\right| - - \rangle , tag{3.8.15c}
$$

$$
|s = 0, m = 0\rangle = \left( \frac{1}{\sqrt{2}}\right) \left( {\left| {+ - \rangle - }\right| - + \rangle }\right) . tag{3.8.15d}
$$

由 (3.8.15a) 式的右边得知,有两个自旋都向上的电子,这种情况只对应于 $s = 1, m =$ 1. 通过把阶梯算符

$$
{S}_{ - } \equiv {S}_{1 - } + {S}_{2 - } tag{3.8.16}
$$

$$
= \left( {{S}_{1x} - i{S}_{1y}}\right) + \left( {{S}_{2x} - i{S}_{2y}}\right)
$$

作用于 (3.8.15a) 式的两边,由 (3.8.15a) 式得到 (3.8.15b) 式. 在这么做的时候, 必须记住,像 ${S}_{1 - }$ 这样一个电子 1 的算符仅影响到 $| + + \rangle$ 中的第一个量等. 可以把

$$
{S}_{ - }\left| {s = 1, m = 1\rangle = \left( {{S}_{1 - } + {S}_{2 - }}\right) }\right| + + \rangle tag{3.8.17}
$$

写成

$$
\sqrt{\left( {1 + 1}\right) \left( {1 - 1 + 1}\right) }\left| {s = 1, m = 0\rangle = \sqrt{\left( {\frac{1}{2} + \frac{1}{2}}\right) \left( {\frac{1}{2} - \frac{1}{2} + 1}\right) } \times }\right| - + \rangle tag{3.8.18}
$$

$$
+ \sqrt{\left( {\frac{1}{2} + \frac{1}{2}}\right) \left( {\frac{1}{2} - \frac{1}{2} + 1}\right) }| + - \rangle ,
$$

它立即导致 (3.8.15b) 式. 同样地, 通过把 (3.8.16) 式再次作用于 (3.8.15b) 式, 可得到 $|s = 1, m = - 1\rangle$ . 最后,通过要求它与其他三个右矢,特别是与 (3.8.15b) 式正交, 可得到 (3.8.15d) 式.

出现在 (3.8.15) 各式右边的系数是克莱布什-戈丹 (Clebsch-Gordan) 系数最简单的例子,稍后将进一步讨论. 它们只不过是一些把 $\left\{ {{m}_{1},{m}_{2}}\right\}$ 基与 $\{ s, m\}$ 基联系起来的变换矩阵的矩阵元. 用另一种方法导出这些系数是有益的. 假定,利用 $\left\{ {{m}_{1},{m}_{2}}\right\}$ 基写出对应于

$$
{\mathbf{S}}^{2} = {\mathbf{S}}_{1}^{2} + {\mathbf{S}}_{2}^{2} + 2{\mathbf{S}}_{1} \cdot {\mathbf{S}}_{2} tag{3.8.19}
$$

$$
= {\mathbf{S}}_{1}^{2} + {\mathbf{S}}_{2}^{2} + 2{S}_{1z}{S}_{2z} + {S}_{1 + }{S}_{2 - } + {S}_{1 - }{S}_{2 + }
$$

的 $4 \times 4$ 矩阵. 这个方阵显然不是对角的,因为 ${S}_{1 + }$ 这样的一个算符把 $\left| {- + \rangle \text{与}}\right| + + \rangle$ 联系了起来. 把这个矩阵对角化的幺正矩阵使基右矢 $\left| {{m}_{1},{m}_{2}}\right\rangle$ 变成基右矢 $|s, m\rangle$ . 这个幺正矩阵的矩阵元就是这个问题的克莱布什-戈丹系数. 本书鼓励读者详细地算出所有这些系数.

\subsection{角动量加法的形式理论}

通过考虑这些简单例子获得了一些物理的理解之后, 现在能够更系统地研究角动量加法形式理论. 考虑在不同子空间的两个角动量算符 ${\mathbf{J}}_{1}$ 和 ${\mathbf{J}}_{2}.{\mathbf{J}}_{1}\left( {\mathbf{J}}_{2}\right)$ 的各分量满足通常的角动量对易关系:

$$
\left\lbrack {{J}_{1i},{J}_{1j}}\right\rbrack = i\hslash {\varepsilon }_{ijk}{J}_{1k} tag{3.8.20a}
$$

和

$$
\left\lbrack {{J}_{2i},{J}_{2j}}\right\rbrack = i\hslash {\varepsilon }_{ijk}{J}_{2k}. tag{3.8.20b}
$$

然而, 任何一对取自不同子空间的算符之间有

$$
\left\lbrack {{J}_{1k},{J}_{2l}}\right\rbrack = 0 tag{3.8.21}
$$

既影响子空间 1 又影响子空间 2 的无穷小转动算符可写为

$$
\left( {1 - \frac{i{\mathbf{J}}_{1} \cdot \widehat{\mathbf{n}}{\delta \phi }}{\hslash }}\right) \otimes \left( {1 - \frac{i{\mathbf{J}}_{2} \cdot \widehat{\mathbf{n}}{\delta \phi }}{\hslash }}\right) = 1 - \frac{i\left( {{\mathbf{J}}_{1} \otimes 1 + 1 \otimes {\mathbf{J}}_{2}}\right) \cdot \widehat{\mathbf{n}}{\delta \phi }}{\hslash }. tag{3.8.22}
$$

把总角动量定义为

$$
\mathbf{J} \equiv {\mathbf{J}}_{1} \otimes 1 + 1 \otimes {\mathbf{J}}_{2}, tag{3.8.23}
$$

更常见地被写成

$$
\mathbf{J} = {\mathbf{J}}_{1} + {\mathbf{J}}_{2}, tag{3.8.24}
$$

有限角度下的 (3.8.22) 式为

$$
{\mathcal{D}}_{1}\left( R\right) \otimes {\mathcal{D}}_{2}\left( R\right) = \exp \left( \frac{-i{\mathbf{J}}_{1} \cdot \widehat{\mathbf{n}}\phi }{\hslash }\right) \otimes \exp \left( \frac{-i{\mathbf{J}}_{2} \cdot \widehat{\mathbf{n}}\phi }{\hslash }\right) . tag{3.8.25}
$$

注意相同的转动轴和同样的转角的出现.

极为重要的是要注意,作为 (3.8.20) 式和 (3.8.21) 式的一个直接后果,总的 $\mathbf{J}$ 满足角动量对易关系

$$
\left\lbrack {{J}_{i},{J}_{j}}\right\rbrack = i\hslash {\varepsilon }_{ijk}{J}_{k} tag{3.8.26}
$$

换句话说, $\mathbf{J}$ 是一个 3.1 节意义上的角动量. 这在物理上是合理的,因为 $\mathbf{J}$ 是整个系统的生成元. 在 3.5 节所得到的一切——例如, ${\mathbf{J}}^{2}$ 和 ${J}_{z}$ 的本征值谱以及阶梯算符的矩阵元 ——对总的 $\mathbf{J}$ 也都成立.

至于基右矢, 有两种选择.

选择 $\mathrm{A} : {\mathrm{J}}_{1}^{2},{\mathrm{\;J}}_{2}^{2},{J}_{1z}$ 和 ${J}_{2z}$ 的共同本征右矢. 把这些右矢表示为 $\left| {{j}_{1}{j}_{2};{m}_{1}{m}_{2}}\right\rangle$ ,显然, 这四个算符彼此对易. 定义它们的方程为

$$
{\mathbf{J}}_{1}^{2}\left| {{j}_{1}{j}_{2};{m}_{1}{m}_{2}}\right\rangle = {j}_{1}\left( {{j}_{1} + 1}\right) {\hslash }^{2}\left| {{j}_{1}{j}_{2};{m}_{1}{m}_{2}}\right\rangle , tag{3.8.27a}
$$

$$
{J}_{1z}\left| {{j}_{1}{j}_{2};{m}_{1}{m}_{2}}\right\rangle = {m}_{1}\hslash \left| {{j}_{1}{j}_{2};{m}_{1}{m}_{2}}\right\rangle , tag{3.8.27b}
$$

$$
{\mathbf{J}}_{2}^{2}\left| {{j}_{1}{j}_{2};{m}_{1}{m}_{2}}\right\rangle = {j}_{2}\left( {{j}_{2} + 1}\right) {\hslash }^{2}\left| {{j}_{1}{j}_{2};{m}_{1}{m}_{2}}\right\rangle ,
$$

(3. ${8.27}\mathrm{c}$ )

$$
{J}_{2z}\left| {{j}_{1}{j}_{2};{m}_{1}{m}_{2}}\right\rangle = {m}_{2}\hslash \left| {{j}_{1}{j}_{2};{m}_{1}{m}_{2}}\right\rangle .
$$

(3. ${8.27}\mathrm{\;d}$ )

选择 $\mathrm{B} : {\mathbf{J}}^{2},{\mathbf{J}}_{1}^{2},{\mathbf{J}}_{2}^{2}$ 和 ${J}_{z}$ 的共同本征右矢. 首先注意到这一组算符相互对易. 特别是, 有

$$
\left\lbrack {{\mathbf{J}}^{2},{\mathbf{J}}_{1}^{2}}\right\rbrack = 0, tag{3.8.28}
$$

只要把 ${\mathbf{J}}^{2}$ 写成

$$
{\mathbf{J}}^{2} = {\mathbf{J}}_{1}^{2} + {\mathbf{J}}_{2}^{2} + 2{J}_{1z}{J}_{2z} + {J}_{1 + }{J}_{2 - } + {J}_{1 - }{J}_{2 + }. tag{3.8.29}
$$

(3.8.28) 式就很容易证明. 用 $\left| {{j}_{1},{j}_{2};{jm}}\right\rangle$ 代表选择 $\mathrm{B}$ 的基右矢:

$$
{\mathbf{J}}_{1}^{2}\left| {{j}_{1}{j}_{2};{jm}}\right\rangle = {j}_{1}\left( {{j}_{1} + 1}\right) {\hslash }^{2}\left| {{j}_{1}{j}_{2};{jm}}\right\rangle , tag{3.8.30a}
$$

$$
{\mathbf{J}}_{2}^{2}\left| {{j}_{1}{j}_{2};{jm}}\right\rangle = {j}_{2}\left( {{j}_{2} + 1}\right) {\hslash }^{2}\left| {{j}_{1}{j}_{2};{jm}}\right\rangle , tag{3. 8. 30b}
$$

$$
{\mathbf{J}}^{2}\left| {{j}_{1}{j}_{2};{jm}}\right\rangle = j\left( {j + 1}\right) {\hslash }^{2}\left| {{j}_{1}{j}_{2};{jm}}\right\rangle , tag{3.8.30c}
$$

$$
{J}_{z}\left| {{j}_{1}{j}_{2};{jm}}\right\rangle = m\hslash \left| {{j}_{1}{j}_{2};{jm}}\right\rangle . tag{3.8.30d}
$$

通常 ${j}_{1},{j}_{2}$ 是不言自明的,因而基右矢简单地写成 $|j, m\rangle$ .

非常重要的是要注意, 尽管

$$
\left\lbrack {{\mathbf{J}}^{2},{J}_{z}}\right\rbrack = 0, tag{3.8.31}
$$

却有

$$
\left\lbrack {{\mathbf{J}}^{2},{J}_{1z}}\right\rbrack \neq 0,\;\left\lbrack {{\mathbf{J}}^{2},{J}_{2z}}\right\rbrack \neq 0, tag{3.8.32}
$$

实际上读者利用 (3.8.29) 式很容易证明它们. 这意味着,不能把 ${\mathbf{J}}^{2}$ 添加到选择 $\mathrm{A}$ 的算符集合中. 同样,不能把 ${J}_{1\varepsilon }$ 和/或 ${J}_{2\varepsilon }$ 添加到选择 $\mathrm{B}$ 的算符集合中. 有两个可能的基右矢集合, 它们对应着已经构造的相互相容可观测量的两个极大集合.

考虑在 1.5 节意义上联系着两组基的幺正变换:

$$
\left| {{j}_{1}{j}_{2};{jm}}\right\rangle = \mathop{\sum }\limits_{{m}_{1}}\mathop{\sum }\limits_{{m}_{2}}\left| {{j}_{1}{j}_{2};{m}_{1}{m}_{2}}\right\rangle \left\langle {{j}_{1}{j}_{2};{m}_{1}{m}_{2} \mid {j}_{1}{j}_{2};{jm}}\right\rangle , tag{3.8.33}
$$

其中用到了

$$
\mathop{\sum }\limits_{{m}_{1}}\mathop{\sum }\limits_{{m}_{2}}\left| {{j}_{1}{j}_{2};{m}_{1}{m}_{2}}\right\rangle \left\langle {{j}_{1}{j}_{2};{m}_{1}{m}_{2}}\right| = 1 tag{3. 8.34}
$$

该式右边是给定 ${j}_{1}$ 和 ${j}_{2}$ 的右矢空间中的单位算符. 这个变换矩阵矩阵元 $\left\langle {{j}_{1}{j}_{2};{m}_{1}{m}_{2}}\right\rangle$ $\left. {{j}_{1}{j}_{2};{jm}}\right\rangle$ 就是克莱布什-戈丹系数.

现在准备研究的克莱布什-戈丹系数具有许多重要的性质. 首先, 除非

$$
m = {m}_{1} + {m}_{2}. tag{3.8.35}
$$

否则该系数为零. 要证明这一点, 注意到,

$$
\left( {{J}_{z} - {J}_{1z} - {J}_{2z}}\right) \left| {{j}_{1}{j}_{2};{jm}}\right\rangle = 0. tag{3.8.36}
$$

用 $\left\langle {{j}_{1}{j}_{2};{m}_{1}{m}_{2}}\right|$ 左乘上式,得到

$$
\left( {m - {m}_{1} - {m}_{2}}\right) \left\langle {{j}_{1}{j}_{2};{m}_{1}{m}_{2} \mid {j}_{1}{j}_{2};{jm}}\right\rangle = 0, tag{3.8.37}
$$

它证明了上述论断. 钦佩狄拉克符号的能力! 正如已经做到的那样, 用狄拉克括号形式写出克莱布什-戈丹系数真的是很值得的.

其次, 除非满足

$$
\left| {{j}_{1} - {j}_{2}}\right| \leq j \leq {j}_{1} + {j}_{2}, tag{3.8.38}
$$

否则该系数为零. 从角动量加法的矢量模型看来,这个性质似乎很显然,在我们把 $\mathrm{J}$ 视为 ${\mathbf{J}}_{1}$ 与 ${\mathbf{J}}_{2}$ 的矢量和. 然而,通过如下方法核实一下这一点还是值得的,即: 证明若 (3.8.38) 式成立,则 $\left\{ \left| {{j}_{1}{j}_{2};{m}_{1}{m}_{2}}\right\rangle \right\}$ 所张空间的维数与 $\left\{ \left| {{j}_{1}{j}_{2};{jm}}\right\rangle \right\}$ 所张空间的维数是相同. 通过对 $\left( {{m}_{1},{m}_{2}}\right)$ 计数的方式,得到

$$
N = \left( {2{j}_{1} + 1}\right) \left( {2{j}_{2} + 1}\right) tag{3.8.39}
$$

因为对给定的 ${j}_{1}$ ,有 $2{j}_{1} + 1$ 个可能的 ${m}_{1}$ 值; 类似的说法对另一个角动量 ${j}_{2}$ 也是对的. 至于对 $\left( {j, m}\right)$ 的计数方式,注意到对每个 $j$ ,有 ${2j} + 1$ 个态,而按照 (3.8.38) 式, $j$ 本身从 ${j}_{1} - {j}_{2}$ 变到 ${j}_{1} + {j}_{2}$ ,其中不失普遍性假定了 ${j}_{1} \geq {j}_{2}$ . 因此,得到

$$
N = \mathop{\sum }\limits_{{j = {j}_{1} - {j}_{2}}}^{{{j}_{1} + {j}_{2}}}\left( {{2j} + 1}\right)
$$

$$
= \frac{1}{2}\left\lbrack {\left\{ {2\left( {{j}_{1} - {j}_{2}}\right) + 1}\right\} + \left\{ {2\left( {{j}_{1} + {j}_{2}}\right) + 1}\right\} }\right\rbrack \left( {2{j}_{2} + 1}\right) tag{3.8.40}
$$

$$
= \left( {2{j}_{1} + 1}\right) \left( {2{j}_{2} + 1}\right) \text{.}
$$

因为这两种计数方法给出了相同的 $N$ 值,得出 (3.8.38) 式是完全自洽 *.

克莱布什-戈丹系数形成一个幺正矩阵. 另外, 按照约定, 该矩阵的矩阵元都取作实数. 它的直接结果是逆系数 $\left\langle {{j}_{1}{j}_{2};{jm} \mid {j}_{1}{j}_{2};{m}_{1}{m}_{2}}\right\rangle$ 与 $\left\langle {{j}_{1}{j}_{2};{m}_{1}{m}_{2} \mid {j}_{1}{j}_{2};{jm}}\right\rangle$ 相同. 一个实的幺正矩阵是正交矩阵, 所以, 有正交条件

$$
\mathop{\sum }\limits_{j}\mathop{\sum }\limits_{m}\left\langle {{j}_{1}{j}_{2};{m}_{1}{m}_{2} \mid {j}_{1}{j}_{2};{jm}}\right\rangle \left\langle {{j}_{1}{j}_{2};{m}_{1}^{\prime }{m}_{2}^{\prime } \mid {j}_{1}{j}_{2};{jm}}\right\rangle = {\delta }_{{m}_{1}{m}_{1}^{\prime }}{\delta }_{{m}_{2}{m}_{2}^{\prime }}, tag{3.8.41}
$$

由 $\left\{ \left| {{j}_{1}{j}_{2};{m}_{1}{m}_{2}}\right\rangle \right\}$ 的正交性与克莱布什-戈丹系数的实数性,上式显然成立. 同样, 还有

$$
\mathop{\sum }\limits_{{m}_{1}}\mathop{\sum }\limits_{{m}_{2}}\left\langle {{j}_{1}{j}_{2};{m}_{1}{m}_{2} \mid {j}_{1}{j}_{2};{jm}}\right\rangle \left\langle {{j}_{1}{j}_{2};{m}_{1}{m}_{2} \mid {j}_{1}{j}_{2};{j}^{\prime }{m}^{\prime }}\right\rangle = {\delta }_{j{j}^{\prime }}{\delta }_{m{m}^{\prime }}. tag{3.8.42}
$$

作为上式的特殊情况,可以设 ${j}^{\prime } = j,{m}^{\prime } = m = {m}_{1} + {m}_{2}$ . 那么,得到

$$
\mathop{\sum }\limits_{{m}_{1}}\mathop{\sum }\limits_{{m}_{2}}{\left| \left\langle {j}_{1}{j}_{2};{m}_{1}{m}_{2} \mid {j}_{1}{j}_{2};jm\right\rangle \right| }^{2} = 1, tag{3.8.43}
$$

它正是 $\left| {{j}_{1}{j}_{2};{jm}}\right\rangle$ 的归一化条件.

一些作者采用了略微不同的克莱布什-戈丹系数符号. 代替 $\left\langle {{j}_{1}{j}_{2};{m}_{1}{m}_{2} \mid {j}_{1}{j}_{2};{jm}}\right\rangle$ , 有时用 $\left\langle {{j}_{1}{m}_{1}{j}_{2}{m}_{2} \mid {j}_{1}{j}_{2}{jm}}\right\rangle, C\left( {{j}_{1}{j}_{2}j;{m}_{1}{m}_{2}m}\right) ,{C}_{{j}_{1}{j}_{2}}\left( {{jm};{m}_{1}{m}_{2}}\right)$ 等. 它们还可以用维格纳 $3 - j$ 符号写出来,在文献中偶尔会见到:

$$
\left\langle {{j}_{1}{j}_{2};{m}_{1}{m}_{2} \mid {j}_{1}{j}_{2};{jm}}\right\rangle = {\left( -1\right) }^{{j}_{1} - {j}_{2} + m}\sqrt{{2j} + 1}\left( \begin{matrix} {j}_{1} & {j}_{2} & j \\ {m}_{1} & {m}_{2} & - m \end{matrix}\right) . tag{3.8.44}
$$

\subsection{克莱布什-戈丹系数的递推关系}

在 ${j}_{1},{j}_{2}$ 和 $j$ 固定的情况下,具有不同的 ${m}_{1}$ 和 ${m}_{2}$ 的系数彼此通过递推关系联系起来. 从

$$
{J}_{ \pm }\left| {{j}_{1}{j}_{2};{jm}}\right\rangle = \left( {{j}_{1 \pm } + {j}_{2 \pm }}\right) \mathop{\sum }\limits_{{m}_{1}}\mathop{\sum }\limits_{{m}_{2}}\left| {{j}_{1}{j}_{2};{m}_{1}{m}_{2}}\right\rangle \left\langle {{j}_{1}{j}_{2};{m}_{1}{m}_{2} \mid {j}_{1}{j}_{2};{jm}}\right\rangle . tag{3.8.45}
$$

出发. 利用 (3.5.39) 式和 (3.5.40) 式,得到 (在 ${m}_{1} \rightarrow {m}_{1}^{\prime },{m}_{2} \rightarrow {m}_{2}^{\prime }$ 的情况下)

$$
\sqrt{\left( {j \mp m}\right) \left( {j \pm m + 1}\right) }\left| {{j}_{1}{j}_{2};j, m \pm 1}\right\rangle
$$

$$
= \mathop{\sum }\limits_{{m}_{1}^{\prime }}\mathop{\sum }\limits_{{m}_{2}^{\prime }}\left( {\sqrt{\left( {{j}_{1} \mp {m}_{1}^{\prime }}\right) \left( {{j}_{1} \pm {m}_{1}^{\prime } + 1}\right) }\left| {{j}_{1}{j}_{2};{m}_{1}^{\prime } \pm 1,{m}_{2}^{\prime }}\right\rangle }\right. tag{3.8.46}
$$

$$
\left. {+\sqrt{\left( {{j}_{2} \mp {m}_{2}^{\prime }}\right) \left( {{j}_{2} \pm {m}_{2}^{\prime } + 1}\right) }\left| {{j}_{1}{j}_{2};{m}_{1}^{\prime },{m}_{2}^{\prime } \pm 1}\right\rangle }\right)
$$

$$
\times \left\langle {{j}_{1}{j}_{2};{m}_{1}^{\prime }{m}_{2}^{\prime } \mid {j}_{1}{j}_{2};{jm}}\right\rangle .
$$

下一步是左乘 $\left\langle {{j}_{1}{j}_{2};{m}_{1}{m}_{2}}\right|$ 并使用正交归一性,它意味着,右边第一项的非零贡献只可能为

$$
{m}_{1} = {m}_{1}^{\prime } \pm 1,\;{m}_{2} = {m}_{2}^{\prime } tag{3.8.47}
$$

而第二项的非零贡献为

$$
{m}_{1} = {m}_{1}^{\prime },\;{m}_{2} = {m}_{2}^{\prime } \pm 1 tag{3.8.48}
$$

用这种方式, 得到所要求的递推关系:

$$
\sqrt{\left( {j \mp m}\right) \left( {j \pm m + 1}\right) }\left\langle {{j}_{1}{j}_{2};{m}_{1}{m}_{2} \mid {j}_{1}{j}_{2};j, m \pm 1}\right\rangle
$$

$$
= \sqrt{\left( {{j}_{1} \mp {m}_{1} + 1}\right) \left( {{j}_{1} \pm {m}_{1}}\right) }\left\langle {{j}_{1}{j}_{2};{m}_{1} \mp 1,{m}_{2} \mid {j}_{1}{j}_{2};{jm}}\right\rangle tag{3.8.49}
$$

$$
+ \sqrt{\left( {{j}_{2} \mp {m}_{2} + 1}\right) \left( {{j}_{2} \pm {m}_{2}}\right) }\left\langle {{j}_{1}{j}_{2};{m}_{1},{m}_{2} \mp 1 \mid {j}_{1}{j}_{2};{jm}}\right\rangle .
$$

重要的是要注意,因为 ${J}_{t}$ 算符移动了 $m$ 值,克莱布什-戈丹系数的非零条件 (3.8.35) 式现在变成 [当作用于 (3.8.49) 式时]

$$
{m}_{1} + {m}_{2} = m \pm 1. tag{3. 8.50}
$$

通过在 ${m}_{1}{m}_{2}$ 平面上观察 (3.8.49) 式,可以领会递推关系的意义. ${J}_{ + }$ 的递推关系 (取上面的符号) 说明,在 $\left( {{m}_{1},{m}_{2}}\right)$ 处的系数与在 $\left( {{m}_{1} - 1,{m}_{2}}\right)$ 及 $\left( {{m}_{1},{m}_{2} - 1}\right)$ 处的系数有关,如图 3.8a 所示. 同样地, ${J}_{ - }$ 的递推关系 (取下面的符号) 也把三个系数联系起来,它们的 ${m}_{1},{m}_{2}$ 值在图 3.8b 中给出.

递推关系 (3.8.49) 和归一化条件 (3.8.43) 一起几乎唯一地确定了所有的克莱布什- 戈丹系数 *. (“几乎唯一” 是因为某些符号约定还必须规定.) 策略如下,返回到 ${m}_{1}{m}_{2}$ 平面,仍然对固定的 ${j}_{1},{j}_{2}$ 和 $j$ ,画出允许区的边界,它由

$$
\left| {m}_{1}\right| \leq {j}_{1},\;\left| {m}_{2}\right| \leq {j}_{2},\; - j \leq {m}_{1} + {m}_{2} \leq j tag{3.8.51}
$$

确定,(见图 3.9a). 从右上角出发,用 $A$ 表示它. 因为在 $A$ 附近开始工作,一个更详细地 “地图” 已就绪; 见图 3.9b. 由 $\left( {{m}_{1},{m}_{2} + 1}\right)$ 对应的 $A$ 开始,使用 ${J}_{ - }$ 递推关系

---

* 克莱布什-戈丹和拉卡 (Racah) 系数、重耦合以及类似的系数的详细讨论已经给出, 例如. Edmonds(1960) 的书.


图 3.8 显示通过递推关系 (3.8.49) 式关联的克莱布什-戈丹系数的 ${m}_{1}{m}_{2}$ 平面


图 3.9 用递推关系求克莱布什-戈丹系数

(3.8.49) 式 (靠下的符号). 现在观察到这个递推关系只把 $A$ 与 $B$ 联系起来,因为对应于 $\left( {{m}_{1} + 1,{m}_{2}}\right)$ 的点是被 ${m}_{1} \leq {j}_{1}$ 禁戒的. 作为结果,可以借助 $A$ 的克莱布什-戈丹系数求得 $B$ 的克莱布什-戈丹系数. 下一步,构建一个由 $A, B$ 和 $D$ 组成的 ${J}_{ + }$ 三角形. 这使得一经确定了 $A$ 的系数,就能够得到 $D$ 的系数. 可以用这种方式继续做下去: 知道了 $B$ 和 $D$ ,可以得到 $E$ ; 知道了 $B$ 和 $E$ ,可以得到 $C$ ,等等. 只要有足够的耐心,就可借助出发点 $A$ 的系数,得到每一点的克莱布什-戈丹系数. 对于整体的归一化,使用 (3.8.43) 式. 最后的整体符号按照惯例固定下来 (见下面的例子).

作为一个重要的实例,考虑一个自旋 $\frac{1}{2}$ 单粒子的轨道和自旋角动量相加的问题. 有

$$
{j}_{1} = l\text{ (整数),}{m}_{1} = {m}_{l}\text{,}
$$

$$
{j}_{2} = s = \frac{1}{2},\;{m}_{2} = {m}_{s} = \pm \frac{1}{2}. tag{3. 8.52}
$$

$j$ 的允许值由

$$
j = l \pm \frac{1}{2},\;l > 0;\;j = \frac{1}{2}, l = 0, tag{3. 8.53}
$$

给定,所以,对于每个 $l$ ,有两个可能的 $j$ 值. 例如,对于 $l = 1$ ( $p$ 态),得到用光谱学符号表示的 ${p}_{3/2}$ 和 ${p}_{1/2}$ ,其中的下脚标指的是 $j$ . 这个问题的 ${m}_{1}{m}_{2}$ 平面,或许用 ${m}_{l}{m}_{s}$ 平面更好一些,特别简单. 可允许的点只形成两行: 上面一行对应于 ${m}_{s} = \frac{1}{2}$ ,而下面一行对应于 ${m}_{s} = - \frac{1}{2}$ ; 见图 3.10. 具体地说,计算 $j = l + \frac{1}{2}$ 的情形. 因为 ${m}_{s}$ 不可能超过 $\frac{1}{2}$ ,能够这样使用 ${J}_{ - }$ 递推关系,确保总是留在上面的一行 $\left( {{m}_{2} = {m}_{s} = \frac{1}{2}}\right)$ ,同时每当 ${m}_{l}$ 改变一个单位时,考虑一个新的 ${J}_{ - }$ 三角形. 在书写克莱布什-戈丹系数时忽略掉 ${j}_{1} = l,{j}_{2} = \frac{1}{2}$ , 从 (3.7.49) 式 (靠下的符号) 得到

$$
\sqrt{\left( {l + \frac{1}{2} + m + 1}\right) \left( {l + \frac{1}{2} - m}\right) }\left\langle {m - \frac{1}{2},\frac{1}{2} \mid l + \frac{1}{2}, m}\right\rangle tag{3.8.54}
$$

$$
= \sqrt{\left( {l + m + \frac{1}{2}}\right) \left( {l - m - \frac{1}{2}}\right) }\left\langle {m + \frac{1}{2},\frac{1}{2} \mid l + \frac{1}{2}, m + 1}\right\rangle ,
$$

其中用到了

$$
{m}_{1} = {m}_{t} = m - \frac{1}{2},\;{m}_{2} = {m}_{s} = \frac{1}{2}. tag{3. 8.55}
$$


图 3.10 求解 ${j}_{1} = l$ 和 ${j}_{2} = s = \frac{1}{2}$ 的克莱布什-戈丹系数用到的递推关系

用这种方式, 可以水平移动一个单位:

$$
\left\langle {m - \frac{1}{2},\frac{1}{2}\left| {\;l + \frac{1}{2}}\right., m}\right\rangle = \sqrt{\frac{l + m + \frac{1}{2}}{l + m + \frac{3}{2}}}\left\langle {m + \frac{1}{2},\frac{1}{2}\left| {\;l + \frac{1}{2}}\right., m + 1}\right\rangle . tag{3.8.56}
$$

反过来,可用 $\left\{ {m + \frac{3}{2},\frac{1}{2}\left| {l + \frac{1}{2}, m + 2}\right\rangle }\right.$ 表示出 $\left\{ {m + \frac{1}{2},\frac{1}{2}\left| {l + \frac{1}{2}, m + 1}\right\rangle }\right.$ 等. 显然, 这个程序可以继续下去,直到 ${m}_{l}$ 达到最大的可能的值 $l$ :

$$
\left\langle {m - \frac{1}{2},\frac{1}{2}\left| {\;l + \frac{1}{2}, m}\right. }\right\rangle = \sqrt{\frac{l + m + \frac{1}{2}}{l + m + \frac{3}{2}}}\sqrt{\frac{l + m + \frac{3}{2}}{l + m + \frac{5}{2}}}\left\langle {m + \frac{3}{2},\frac{1}{2}\left| {\;l + \frac{1}{2}}\right., m + 2}\right\rangle
$$

$$
= \sqrt{\frac{l + m + \frac{1}{2}}{l + m + \frac{3}{2}}}\sqrt{\frac{l + m + \frac{3}{2}}{l + m + \frac{5}{2}}}\sqrt{\frac{l + m + \frac{5}{2}}{l + m + \frac{7}{2}}}
$$

$$
\times \left\langle {m + \frac{5}{2},\frac{1}{2}\left| {\;l + \frac{1}{2}}\right., m + 3}\right\rangle
$$

$\vdots$

$$
= \sqrt{\frac{l + m + \frac{1}{2}}{{2l} + 1}}\left\langle {l,\frac{1}{2}\left| {l + \frac{1}{2}, l + \frac{1}{2}}\right| }\right\rangle . tag{3.8.57}
$$

考虑 ${m}_{l}$ 和 ${m}_{s}$ 都取最大值的角动量组态一一即分别为 $l$ 和 $\frac{1}{2}$ . 总的 $m = {m}_{l} + {m}_{s}$ 是 $l + \frac{1}{2}$ , 它仅当 $j = l + \frac{1}{2}$ 而不是 $j = l - \frac{1}{2}$ 时才是可能的. 所以 $\left. {\mid {m}_{l} = l,{m}_{s} = \frac{1}{2}}\right\rangle$ 一定在最多差一个相因子的情况下等于 $\left| {j = l + \frac{1}{2}, m = l + \frac{1}{2}\rangle \text{. 按照惯例,取这个相因子为正实数. 使用这}}\right|$ 种选择, 有

$$
\left\langle {l,\frac{1}{2}\left| {\;l + \frac{1}{2}}\right., l + \frac{1}{2}}\right\rangle = 1 tag{3.8.58}
$$

返回到 (3.8.57) 式, 最终得到

$$
\left\langle {m - \frac{1}{2},\frac{1}{2}\left| {\;l + \frac{1}{2}}\right., m}\right\rangle = \sqrt{\frac{l + m + \frac{1}{2}}{{2l} + 1}}. tag{3.8.59}
$$

但是这还只是这个故事的四分之一, 仍须确定下式中问号的值:

$$
\left| {j = l + \frac{1}{2}, m\rangle = \sqrt{\frac{l + m + \frac{1}{2}}{{2l} + 1}}\;{m}_{l} = m - \frac{1}{2},{m}_{s} = \frac{1}{2}}\right|
$$

$$
+ ?\left| {{m}_{l} = m + \frac{1}{2},{m}_{s} = - \frac{1}{2}}\right\rangle ,
$$

$$
j = l - \frac{1}{2}, m\rangle = ?\left| {{m}_{l} = m - \frac{1}{2},{m}_{s} = \frac{1}{2}\rangle + ?}\right| {m}_{l} = m + \frac{1}{2},{m}_{s} = - \frac{1}{2}\rangle . tag{3.8.60}
$$

注意到,由于正交性,预期具有固定 $m$ 值的、从 $\left( {{m}_{l},{m}_{s}}\right)$ 基到 $\left( {j, m}\right)$ 基的变换矩阵有下列形式:

$$
\left( \begin{matrix} \cos \alpha & \sin \alpha \\ - \sin \alpha & \cos \alpha \end{matrix}\right) . tag{3.8.61}
$$

与 (3.8.60) 式对照表明, $\cos \alpha$ 就是 (3.8.59) 式本身,所以可以很容易地确定 $\sin \alpha$ ,至多有一个符号有歧义:

$$
{\sin }^{2}\alpha = 1 - \frac{\left( l + m + \frac{1}{2}\right) }{\left( 2l + 1\right) } = \frac{\left( l - m + \frac{1}{2}\right) }{\left( 2l + 1\right) }. tag{3.8.62}
$$

要求 $\left\{ {{m}_{l} = m + \frac{1}{2},{m}_{s} = - \frac{1}{2}\left| {j = l + \frac{1}{2}, m}\right\rangle }\right.$ 必须是正的,因为所有的 $j = l + \frac{1}{2}$ 的态都可以通过 ${J}_{ - }$ 算符逐次作用于 $\left| {j = l + \frac{1}{2}, m = l + \frac{1}{2}}\right\rangle$ 得到,而按照惯例, ${J}_{ - }$ 的矩阵元总是

正的. 所以这个 $2 \times 2$ 变换矩阵 (3.8.61) 只能为

$$
\left( \begin{array}{ll} \sqrt{\frac{l + m + \frac{1}{2}}{{2l} + 1}} & \sqrt{\frac{l - m + \frac{1}{2}}{{2l} + 1}} \\ - \sqrt{\frac{l - m + \frac{1}{2}}{{2l} + 1}} & \sqrt{\frac{l + m + \frac{1}{2}}{{2l} + 1}} \end{array}\right) . tag{3. 8.63}
$$

把二分量形式的自旋角函数定义如下:

$$
{y}_{l}^{j = l \pm 1/2, m} = \pm \sqrt{\frac{l \pm m + \frac{1}{2}}{{2l} + 1}}{Y}_{l}^{m - 1/2}\left( {\theta ,\phi }\right) {\chi }_{ + }
$$

$$
+ \sqrt{\frac{l \pm m + \frac{1}{2}}{{2l} + 1}}{Y}_{l}^{m + 1/2}\left( {\theta ,\phi }\right) {\chi }_{ - } tag{3.8.64}
$$

$$
= \frac{1}{\sqrt{{2l} + 1}}\left( \begin{array}{l} \pm \sqrt{l \pm m + \frac{1}{2}}{Y}_{l}^{m - 1/2}\left( {\theta ,\phi }\right) \\ \sqrt{l \mp m + \frac{1}{2}}{Y}_{l}^{m + 1/2}\left( {\theta ,\phi }\right) \end{array}\right) .
$$

通过它们的结构可看到,它们是 ${\mathbf{L}}^{2},{\mathbf{S}}^{2},{\mathbf{J}}^{2}$ 和 ${J}_{z}$ 的共同本征函数. 它们还是 $\mathbf{L} \cdot \mathbf{S}$ 的本征函数,但是 $\mathbf{L} \cdot \mathbf{S}$ 只不过是

$$
\mathbf{L} \cdot \mathbf{S} = \left( \frac{1}{2}\right) \left( {{\mathbf{J}}^{2} - {\mathbf{L}}^{2} - {\mathbf{S}}^{2}}\right) , tag{3. 8.65}
$$

因此是不独立的. 的确, 它的本征值可以很容易地计算

$$
\left( \frac{{\hslash }^{2}}{2}\right) \left\lbrack {j\left( {j + 1}\right) - l\left( {l + 1}\right) - \frac{3}{4}}\right\rbrack = \left\{ \begin{array}{ll} \frac{l{\hslash }^{2}}{2} & \text{ 对于 }j = l + \frac{1}{2}, \\ - \frac{\left( {l + 1}\right) {\hslash }^{2}}{2} & \text{ 对于 }j = l - \frac{1}{2}. \end{array}\right. tag{3.8.66}
$$

\subsection{克莱布什-戈丹系数和转动矩阵}

可以从转动矩阵观点出发讨论角动量加法. 在本征值为 ${j}_{1}$ 的角动量本征右矢所张的右矢空间中考虑转动算符 ${\mathcal{D}}^{\left( {j}_{1}\right) }\left( R\right)$ . 同样,考虑 ${\mathcal{D}}^{\left( {j}_{2}\right) }\left( R\right)$ . 在适当选择了基右矢之后的意义上,乘积 ${\mathcal{D}}^{\left( {j}_{1}\right) } \otimes {\mathcal{D}}^{\left( {j}_{2}\right) }$ 是可约的,它的矩阵表示可以取如下形式

(3.8.67)

用群论符号, 可以将其写成

$$
{\mathcal{D}}^{\left( {j}_{1}\right) } \otimes {\mathcal{D}}^{\left( {j}_{2}\right) } = {\mathcal{D}}^{\left( {j}_{1} + {j}_{2}\right) } \oplus {\mathcal{D}}^{\left( {j}_{1} + {j}_{2} - 1\right) } \oplus \cdots \oplus {\mathcal{D}}^{\left( {j}_{1} - {j}_{2}\right) }. tag{3. 8.68}
$$

(原书该式有重要错误, 等式右边也写成了张量积的形式. 现已订正. 一译者注) 借助于转动矩阵的矩阵元, 有一个称之为克莱布什-戈丹级数的重要展开:

$$
{\mathcal{D}}_{{m}_{1}{m}_{1}^{\prime }}^{\left( {j}_{1}\right) }\left( R\right) {\mathcal{D}}_{{m}_{2}{m}_{2}^{\prime }}^{\left( {j}_{2}\right) }\left( R\right) = \mathop{\sum }\limits_{j}\mathop{\sum }\limits_{m}\mathop{\sum }\limits_{{m}^{\prime }}\left\langle {{j}_{1}{j}_{2};{m}_{1}{m}_{2} \mid {j}_{1}{j}_{2};{jm}}\right\rangle tag{3.8.69}
$$

$$
\times \left\langle {{j}_{1}{j}_{2};{m}^{\prime }{}_{1}{m}^{\prime }{}_{2} \mid {j}_{1}{j}_{2};j{m}^{\prime }}\right\rangle {\mathcal{D}}_{m{m}^{\prime }}^{\left( j\right) }\left( R\right)
$$

其中对 $j$ 的求和从 $\left| {{j}_{1} - {j}_{2}}\right|$ 到 ${j}_{1} + {j}_{2}$ . 这个方程的证明如下. 首先,注意 (3.8.69) 式的左边与

$$
\left\langle {{j}_{1}{j}_{2};{m}_{1}{m}_{2}\left| {\mathcal{D}\left( R\right) }\right| {j}_{1}{j}_{2};{m}^{\prime }{}_{1}{m}^{\prime }{}_{2}}\right\rangle = \left\langle {{j}_{1}{m}_{1}\left| {\mathcal{D}\left( R\right) }\right| {j}_{1}{m}^{\prime }{}_{1}}\right\rangle \left\langle {{j}_{2}{m}_{2}\left| {\mathcal{D}\left( R\right) }\right| {j}_{2}{m}^{\prime }{}_{2}}\right\rangle
$$

$$
= {\mathcal{D}}_{{m}_{1}{m}_{1}^{\prime }}^{\left( {j}_{1}\right) }\left( R\right) {\mathcal{D}}_{{m}_{2}{m}_{2}^{\prime }}^{\left( {j}_{2}\right) }\left( R\right) . tag{3.8.70}
$$

是一样的. 但是,同样的矩阵元也可以通过插入在 $\left( {j, m}\right)$ 基中的态的完备集来计算. 于是

$$
\left\langle {{j}_{1}{j}_{2};{m}_{1}{m}_{2} \mid \mathcal{D}\left( R\right) \mid {j}_{1}{j}_{2};{m}_{1}^{\prime }{m}_{2}^{\prime }}\right\rangle
$$

$$
= \mathop{\sum }\limits_{j}\mathop{\sum }\limits_{m}\mathop{\sum }\limits_{{j}^{\prime }}\mathop{\sum }\limits_{{m}^{\prime }}\left\langle {{j}_{1}{j}_{2};{m}_{1}{m}_{2} \mid {j}_{1}{j}_{2};{jm}}\right\rangle \left\langle {{j}_{1}{j}_{2};{jm} \mid \mathcal{D}\left( R\right) \mid {j}_{1}{j}_{2};{j}^{\prime }{m}^{\prime }}\right\rangle
$$

$$
\times \left\langle {{j}_{1}{j}_{2};{j}^{\prime }{m}^{\prime } \mid {j}_{1}{j}_{2};{m}_{1}^{\prime }{m}_{2}^{\prime }}\right\rangle
$$

$$
= \mathop{\sum }\limits_{j}\mathop{\sum }\limits_{m}\mathop{\sum }\limits_{{j}^{\prime }}\mathop{\sum }\limits_{{m}^{\prime }}\left\langle {{j}_{1}{j}_{2};{m}_{1}{m}_{2} \mid {j}_{1}{j}_{2};{jm}}\right\rangle {\mathcal{D}}_{m{m}^{\prime }}^{\left( j\right) }\left( R\right) {\delta }_{j{j}^{\prime }}
$$

$$
\times \left\langle {{j}_{1}{j}_{2};{m}^{\prime }{}_{1}{m}^{\prime }{}_{2} \mid {j}_{1}{j}_{2};{j}^{\prime }{m}^{\prime }}\right\rangle , tag{3.8.71}
$$

它正是 (3.8.69) 式的右边.

作为 (3.8.69) 式的一个有趣的应用, 推导一个包含三个球谐函数的重要积分公式. 首先,回忆一下由 (3.6.52) 式给出的 ${D}_{m0}^{\left( l\right) }$ 和 ${Y}_{l}^{m * }$ 间的联系. 令 (3.8.69) 式中的 ${j}_{1} \rightarrow$ ${l}_{1},{j}_{2} \rightarrow {l}_{2},{m}_{1}^{\prime } \rightarrow 0,{m}_{2}^{\prime } \rightarrow 0$ (因此, ${m}^{\prime } \rightarrow 0$ ),在求复共轭之后,得到,

$$
{Y}_{{l}_{1}}^{{m}_{1}}\left( {\theta ,\phi }\right) {Y}_{{l}_{2}}^{{m}_{2}}\left( {\theta ,\phi }\right) = \frac{\sqrt{\left( {2{l}_{1} + 1}\right) \left( {2{l}_{2} + 1}\right) }}{4\pi }\mathop{\sum }\limits_{{l}^{\prime }}\mathop{\sum }\limits_{{m}^{\prime }}\left\langle {{l}_{1}{l}_{2};{m}_{1}{m}_{2} \mid {l}_{1}{l}_{2};{l}^{\prime }{m}^{\prime }}\right\rangle
$$

$$
\text{* 非常抱歉用}m\text{既表示了 “质量”,又表示了角动量量子数. 然而,从本节上下文应当清楚哪个是哪个.} tag{3.8.72}
$$

式子的两边乘以 ${Y}_{l}^{m * }\left( {\theta ,\phi }\right)$ ,然后对立体角求积分. 由于球谐函数的正交性,求和消失了, 留下的是

$$
\int {d\Omega }{Y}_{l}^{m * }\left( {\theta ,\phi }\right) {Y}_{{l}_{1}}^{{m}_{1}}\left( {\theta ,\phi }\right) {Y}_{{l}_{2}}^{{m}_{2}}\left( {\theta ,\phi }\right) tag{3.8.73}
$$

$$
= \sqrt{\frac{\left( {2{l}_{1} + 1}\right) \left( {2{l}_{2} + 1}\right) }{{4\pi }\left( {{2l} + 1}\right) }}\left\langle {{l}_{1}{l}_{2};{00} \mid {l}_{1}{l}_{2};{l0}}\right\rangle \left\langle {{l}_{1}{l}_{2};{m}_{1}{m}_{2} \mid {l}_{1}{l}_{2};{lm}}\right\rangle .
$$

平方根因子乘以第一个克莱布什-戈丹系数不依赖于取向——即 ${m}_{1}$ 和 ${m}_{2}$ . 第二个克莱布什-戈丹系数是一个适用于 ${l}_{1}$ 和 ${l}_{2}$ 相加得到总 $l$ 的系数. 结果表明,方程 (3.8.73) 是 3.11 节推导出的维格纳-埃卡特 (Wigner-Eckart) 定理的一种特殊情况. 这个公式在计算原子及原子核光谱学中的多极矩阵元时极为有用.

\section{角动量的施温格振子模型}
\subsection{角动量和无耦合振子}

在角动量代数和两个独立 (即, 无耦合的) 振子的代数之间存在着一个种非常有趣的联系, 它是在施温格的一篇短文中提出来的, 请见 Biedenharn 和 Van Dam (1965), 229 页. 考虑两个简谐振子, 把它们分别叫作加号型和减号型. 加号型振子的湮灭和产生算符分别用 ${a}_{ + }$ 和 ${a}_{ + }^{ \dagger }$ 表示; 同样地,减号型振子的湮灭和产生算符分别用 ${a}_{ - }$ 和 ${a}_{ - }^{ \dagger }$ 表示. 粒子数算符 ${N}_{ + }$ 和 ${N}_{ - }$ 定义如下:

$$
{N}_{ + } \equiv {a}_{ + }^{ \dagger }{a}_{ + },\;{N}_{ - } \equiv {a}_{ - }^{ \dagger }{a}_{ - }. tag{3.9.1}
$$

假定对于同一种类型的振子,在 $a,{a}^{ \dagger }$ 和 $N$ 之间通常的对易关系成立 (见 2.3 节).

$$
\left\lbrack {{a}_{ + },{a}_{ + }^{ \dagger }}\right\rbrack = 1,\;\left\lbrack {{a}_{ - },{a}_{ - }^{ \dagger }}\right\rbrack = 1, tag{3.9.2a}
$$

$$
\left\lbrack {{N}_{ + },{a}_{ + }}\right\rbrack = - {a}_{ + },\;\left\lbrack {{N}_{ - },{a}_{ - }}\right\rbrack = - {a}_{ - }, tag{3.9.2b}
$$

$$
\left\lbrack {{N}_{ + },{a}_{ + }^{ \dagger }}\right\rbrack = {a}_{ + }^{ \dagger },\;\left\lbrack {{N}_{ - },{a}_{ - }^{ \dagger }}\right\rbrack = {a}_{ - }^{ \dagger }.
$$

(3. ${9.2c}$ )

然而, 还假定任何一对不同振子的算符对易:

$$
\left\lbrack {{a}_{ + },{a}_{ - }^{ \dagger }}\right\rbrack = \left\lbrack {{a}_{ - },{a}_{ + }^{ \dagger }}\right\rbrack = 0 tag{3.9.3}
$$

等等. 因此, 正是在这种意义上, 这两种振子是无耦合的.

因为借助于 (3.9.3) 式, ${N}_{ + }$ 和 ${N}_{ - }$ 对易,构建 ${N}_{ + }$ 和 ${N}_{ - }$ 的共同本征态,其本征值分别为 ${n}_{ + }$ 和 ${n}_{ - }$ . 所以,有下列 ${N}_{ \pm }$ 的本征方程:

$$
{N}_{ + }\left| {{n}_{ + },{n}_{ - }}\right\rangle = {n}_{ + }\left| {{n}_{ + },{n}_{ - }}\right\rangle ,\;{N}_{ - }\left| {{n}_{ + },{n}_{ - }}\right\rangle = {n}_{ - }\left| {{n}_{ + },{n}_{ - }}\right\rangle . tag{3.9.4}
$$

完全类似于 (2.3.16) 式和 (2.3.17) 式,产生算符 ${a}_{ \pm }^{ \dagger }$ 和湮灭算符 ${a}_{ \pm }$ 作用在 $\left| {{n}_{ + },{n}_{ - }}\right\rangle$ 上, 有:

$$
{a}_{ + }^{ \dagger }\left| {{n}_{ + },{n}_{ - }}\right\rangle = \sqrt{{n}_{ + } + 1}\left| {{n}_{ + } + 1,{n}_{ - }}\right\rangle ,\;{a}_{ - }^{ \dagger }\left| {{n}_{ + },{n}_{ - }}\right\rangle = \sqrt{{n}_{ - } + 1}\left| {{n}_{ + },{n}_{ - } + 1}\right\rangle , tag{3.9.5a}
$$

$$
{a}_{ + }\left| {{n}_{ + },{n}_{ - }}\right\rangle = \sqrt{{n}_{ + }}\left| {{n}_{ + } - 1,{n}_{ - }}\right\rangle ,\;{a}_{ - }\left| {{n}_{ + },{n}_{ - }}\right\rangle = \sqrt{{n}_{ - }}\left| {{n}_{ + },{n}_{ - } - 1}\right\rangle .
$$

(3. ${9.5}\mathrm{\;b}$ )

通过把 ${a}_{ + }^{ \dagger }$ 和 ${a}_{ - }^{ \dagger }$ 连续作用于用

$$
{a}_{ + }\left| {0,0\rangle = 0,\;{a}_{ - }}\right| 0,0\rangle = 0. tag{3. 9.6}
$$

定义的真空右矢上,可得到 ${N}_{ + }$ 和 ${N}_{ - }$ 最普遍的本征右矢. 以这样的方法得到

$$
\left| {{n}_{ + },{n}_{ - }}\right\rangle = \frac{{\left( {a}_{ + }^{ \dagger }\right) }^{{n}_{ + }}{\left( {a}_{ - }^{ \dagger }\right) }^{{n}_{ - }}}{\sqrt{{n}_{ + }!}\sqrt{{n}_{ - }!}}|0,0\rangle tag{3.9.7}
$$

接着, 定义

$$
{J}_{ + } \equiv \hslash {a}_{ + }^{ \dagger }{a}_{ - },\;{J}_{ - } \equiv \hslash {a}_{ - }^{ \dagger }{a}_{ + } tag{3.9.8a}
$$

和

$$
{J}_{z} \equiv \left( \frac{\hslash }{2}\right) \left( {{a}_{ + }^{ \dagger }{a}_{ + } - {a}_{ - }^{ \dagger }{a}_{ - }}\right) = \left( \frac{\hslash }{2}\right) \left( {{N}_{ + } - {N}_{ - }}\right) . tag{3.9.8b}
$$

可以很容易地证明, 这些算符满足通常形式的角动量对易关系

$$
\left\lbrack {{J}_{z},{J}_{ \pm }}\right\rbrack = \pm \hslash {J}_{ \pm }, tag{3.9.9a}
$$

$$
\left\lbrack {{J}_{ + },{J}_{ - }}\right\rbrack = 2\hslash {J}_{z}. tag{3.9.9b}
$$

例如, 证明 (3.9.9) 式

$$
{\hslash }^{2}\left\lbrack {{a}_{ + }^{ \dagger }{a}_{ - },{a}_{ - }^{ \dagger }{a}_{ + }}\right\rbrack = {\hslash }^{2}{a}_{ + }^{ \dagger }{a}_{ - }{a}_{ - }^{ \dagger }{a}_{ + } - {\hslash }^{2}{a}_{ - }^{ \dagger }{a}_{ + }{a}_{ + }^{ \dagger }{a}_{ - }
$$

$$
= {\hslash }^{2}{a}_{ + }^{ \dagger }\left( {{a}_{ - }^{ \dagger }{a}_{ - } + 1}\right) {a}_{ + } - {\hslash }^{2}{a}_{ - }^{ \dagger }\left( {{a}_{ + }^{ \dagger }{a}_{ + } + 1}\right) {a}_{ - } tag{3.9.10}
$$

$$
= {\hslash }^{2}\left( {{a}_{ + }^{ \dagger }{a}_{ + } - {a}_{ - }^{ \dagger }{a}_{ - }}\right) = 2\hslash {J}_{z}.
$$

定义总 $N$ 为

$$
N \equiv {N}_{ + } + {N}_{ - } = {a}_{ + }^{ \dagger }{a}_{ + } + {a}_{ - }^{ \dagger }{a}_{ - }, tag{3.9.11}
$$

还可以证明

$$
{\mathbf{J}}^{2} \equiv {J}_{z}^{2} + \left( \frac{1}{2}\right) \left( {{J}_{ + }{J}_{ - } + {J}_{ - }{J}_{ + }}\right) tag{3.9.12}
$$

$$
= \left( \frac{{\hslash }^{2}}{2}\right) N\left( {\frac{N}{2} + 1}\right) ,
$$

把它留作一个练习.

如何从物理上解释所有这些结果? 把自旋向上 $\left( {m = \frac{1}{2}}\right)$ 与具有一个量子单位的加号型振子以及把自旋向下 $\left( {m = - \frac{1}{2}}\right)$ 与具有一个量子单位的减号型振子联系起来. 如果喜欢的话,可以把每一个量子单位的加 (减) 号型振子想象为一个自旋向上 (下) 的自旋 $\frac{1}{2}$ 的 “粒子”. 本征值 ${n}_{ + }$ 和 ${n}_{ - }$ 正好分别是自旋向上和自旋向下的数目. ${J}_{ + }$ 的意思是,它消灭一个单位的自旋角动量 $z$ 分量为 $- \hslash /2$ 的向下自旋,并且产生一个单位的自旋角动量 $z$ 分量为 $+ \hslash /2$ 的向上自旋; 因此角动量的 $z$ 分量增加了 $\hslash$ . 同样, ${J}_{ - }$ 消灭一个单位的向上自旋,并且产生一个单位的向下自旋,因此角动量的 $z$ 分量减少了 $h$ . 至于说 ${J}_{z}$ 算符,它只不过是计算了 $\hslash /2$ 乘以 ${n}_{ + }$ 和 ${n}_{ - }$ 之差,它正好就是总角动量的 $z$ 分量. 使用 (3.9.5) 式,就可以很容易地考察 ${J}_{ \pm }$ 和 ${J}_{z}$ 怎样作用在 $\left| {{n}_{ + },{n}_{ - }}\right\rangle$ 上

$$
{J}_{ + }\left| {{n}_{ + },{n}_{ - }}\right\rangle = \hslash {a}_{ + }^{ \dagger }{a}_{ - }\left| {{n}_{ + },{n}_{ - }}\right\rangle = \sqrt{{n}_{ - }\left( {{n}_{ + } + 1}\right) }\hslash \left| {{n}_{ + } + 1,{n}_{ - } - 1}\right\rangle , tag{3.9.13a}
$$

$$
{J}_{ - }\left| {{n}_{ + },{n}_{ - }}\right\rangle = \hslash {a}_{ - }^{ \dagger }{a}_{ + }\left| {{n}_{ + },{n}_{ - }}\right\rangle = \sqrt{{n}_{ + }\left( {{n}_{ - } + 1}\right) }\hslash \left| {{n}_{ + } - 1,{n}_{ - } + 1}\right\rangle . tag{3.9.13b}
$$

$$
{J}_{z}\left| {{n}_{ + },{n}_{ - }}\right\rangle = \left( \frac{\hslash }{2}\right) \left( {{N}_{ + } - {N}_{ - }}\right) \left| {{n}_{ + },{n}_{ - }}\right\rangle = \left( \frac{1}{2}\right) \left( {{n}_{ + } - {n}_{ - }}\right) \hslash \left| {{n}_{ + },{n}_{ - }}\right\rangle .
$$

(3. ${9.13}\mathrm{c}$ )

注意在所有这些运算中,对应于自旋 $\frac{1}{2}$ 粒子总数的 ${n}_{ + }$ 和 ${n}_{ - }$ 之和保持不变.

现在注意到, 只要做如下代换

$$
{n}_{ + } \rightarrow j + m,\;{n}_{ - } \rightarrow j - m. tag{3.9.14}
$$

(3.9.13a)、(3.9.13b) 和 (3.9.13c) 就能约化成在 3.5 节导出的、熟悉的 ${J}_{ \pm }$ 和 ${J}_{ * }$ 算符. 此时 (3.9.13a) 和 (3.9.13b) 中的平方根因子变成

$$
\sqrt{{n}_{ - }\left( {{n}_{ + } + 1}\right) } \rightarrow \sqrt{\left( {j - m}\right) \left( {j + m + 1}\right) }, tag{3.9.15}
$$

$$
\sqrt{{n}_{ + }\left( {{n}_{ - } + 1}\right) } \rightarrow \sqrt{\left( {j + m}\right) \left( {j - m + 1}\right) },
$$

它们正是 (3.5.39) 和 (3.5.41) 式中的平方根因子.

还要注意,(3.9.12) 式定义的 ${\mathbf{J}}^{2}$ 算符的本征值变成

$$
\left( \frac{{\hslash }^{2}}{2}\right) \left( {{n}_{ + } + {n}_{ - }}\right) \left\lbrack {\frac{\left( {n}_{ + } + {n}_{ - }\right) }{2} + 1}\right\rbrack \rightarrow {\hslash }^{2}j\left( {j + 1}\right) . tag{3.9.16}
$$

所有这些可能不是太奇怪的,因为已经证明了,用振子算符所构建的 ${J}_{ \pm }$ 和 ${\mathbf{J}}^{2}$ 算符满足通常的角动量对易关系. 但是以一种明确的方式看一下振子矩阵元与角动量矩阵元之间的联系是有益的. 不管怎么说, 使用

$$
j \equiv \frac{\left( {n}_{ + } + {n}_{ - }\right) }{2},\;m \equiv \frac{\left( {n}_{ + } - {n}_{ - }\right) }{2} tag{3.9.17}
$$

代替 ${n}_{ + }$ 和 ${n}_{ - }$ 来表征 ${\mathbf{J}}^{2}$ 和 ${J}_{z}$ 的共同本征右矢是很自然的. 按照 (3.9.13a) 式, ${J}_{ + }$ 的作用把 ${n}_{ + }$ 变成 ${n}_{ + } + 1$ ,而把 ${n}_{ - }$ 变成 ${n}_{ - } - 1$ ,它意味着 $j$ 不变,而 $m$ 变成 $m + 1$ . 同样地, ${J}_{ - }$ 算符把 ${n}_{ + }$ 变成 ${n}_{ + } - 1$ ,而把 ${n}_{ - }$ 变成 ${n}_{ - } - 1$ ,使 $m$ 减少一个单位而 $j$ 不改变. 现在可以像 (3.9.7) 式一样把最普遍的 ${N}_{ + }$ 和 ${N}_{ - }$ 的本征右矢写成:

$$
\left| {j, m\rangle = \frac{{\left( {a}_{ + }^{ \dagger }\right) }^{j + m}{\left( {a}_{ - }^{ \dagger }\right) }^{j - m}}{\sqrt{\left( {j + m}\right) !\left( {j - m}\right) !}}}\right| 0\rangle , tag{3. 9.18}
$$

其中用 $|0\rangle$ 表示真空右矢,而早些时候用 $|0,0\rangle$ 表示.

(3.9.18)式的一个特殊情况是令人感兴趣的. 令 $m = j$ ,物理上它意味着对于一个给定的 $j,{J}_{z}$ 的本征值尽可能地大. 有

$$
\left| {j, j\rangle = \frac{{\left( {a}_{ + }^{ \dagger }\right) }^{2j}}{\sqrt{\left( {2j}\right) !}}}\right| 0\rangle . tag{3.9.19}
$$

可以把这个态想象为 ${2j}$ 个自旋 $\frac{1}{2}$ 的粒子构成的,这些粒子的自旋都指向正 $z$ 方向.

一般来说,注意到,一个高 $j$ 值的复杂客体可以视为由一些简单的自旋 $\frac{1}{2}$ 的粒子构成,其中 $j + m$ 个粒子的自旋向上,而剩下的 $j - m$ 个粒子的自旋向下. 显然,尽管不能总是把一个角动量为 $j$ 的对象从字面上看作是一个自旋 $\frac{1}{2}$ 粒子的组合系统,这个图像还是非常方便的. 所说的一切归结为: 就转动下的变换性质而言,可以把角动量为 $j$ 的任何客体视为一个由 ${2j}$ 个自旋 $\frac{1}{2}$ 的粒子以 (3.9.18) 式所示的方式构成的复合系统.

按照前一节展开的角动量加法的观点,可以把 ${2j}$ 个自旋 $\frac{1}{2}$ 粒子的自旋相加,得到角动量 $j, j - 1, j - 2,\cdots$ 的态. 作为一个简单的例子,可以把两个自旋 $\frac{1}{2}$ 粒子的自旋角动量相加得到一个总角动量为 1 , 和一个总角动量为零的态. 然而, 在施温格的振子方案中, 从 ${2j}$ 个自旋 $\frac{1}{2}$ 的粒子出发,仅能得到角动量为 $j$ 的态. 用第 7 章阐述的置换对称性的语言来讲,用这种方法只能构成完全对称的态. 这里出现的原始的自旋 $\frac{1}{2}$ 粒子实际上是些玻色子! 如果目的是考察由 $j$ 和 $m$ 表征的态在转动下的性质,不过问这样的态一开始是如何构成的话, 这种方法已经足够.

熟悉原子核与粒子物理中同位旋的读者可能注意到, 这里所做的提供了一种对同位旋形式体系的新见解. 消灭一个单位的减号型振子, 接着产生一个单位的加号型振子的算符 ${J}_{ + }$ 完全类似于同位旋阶梯算符 ${T}_{ + }$ (有时用 ${I}_{ + }$ 表示),它湮灭一个中子 (同位旋向下), 接着产生一个质子 (同位旋向上),于是使同位旋的 $z$ 分量升高一个单位. 相反, ${J}_{z}$ 类似于 ${T}_{z}$ ,它只不过计算了原子核中质子数与中子数的差.

\subsection{转动矩阵的显式表达式}

施温格的方案可以以一种非常简单的方式推导最早由维格纳利用类似 (但并不全同)

方法求得的转动矩阵的解析形式的公式. 把转动算符 $\mathcal{D}\left( R\right)$ 作用到写成 (3.9.18) 式形式的 $|j, m\rangle$ 上. 用欧拉角符号,唯一非平凡转动是绕 $y$ 轴的第二次转动,所以把注意力指向

$$
\mathcal{D}\left( R\right) = {\left. \mathcal{D}\left( \alpha ,\beta ,\gamma \right) \right| }_{\alpha = \gamma = 0} = \exp \left( \frac{-i{J}_{y}\beta }{\hslash }\right) . tag{3.9.20}
$$

有

$$
\mathcal{D}\left( R\right) \left| {j, m\rangle = \frac{{\left\lbrack \mathcal{D}\left( R\right) {a}_{ + }^{ \dagger }{\mathcal{D}}^{-1}\left( R\right) \right\rbrack }^{j + m}{\left\lbrack \mathcal{D}\left( R\right) {a}_{ - }^{ \dagger }{\mathcal{D}}^{-1}\left( R\right) \right\rbrack }^{j - m}}{\sqrt{\left( {j + m}\right) !\left( {j - m}\right) !}}\mathcal{D}\left( R\right) }\right| 0\rangle . tag{3.9.21}
$$

现在, $\mathcal{D}\left( R\right)$ 作用于 $|0\rangle$ 上,只不过重新产生出 $|0\rangle$ ,因为由 (3.9.6) 式,在指数 (3.9.20) 的展开中, 只有领头项 1 有贡献. 所以

$$
\mathcal{D}\left( R\right) {a}_{ \pm }^{ \dagger }{\mathcal{D}}^{-1}\left( R\right) = \exp \left( \frac{-i{J}_{y}\beta }{\hslash }\right) {a}_{ \pm }^{ \dagger }\exp \left( \frac{i{J}_{y}\beta }{\hslash }\right) . tag{3.9.22}
$$

于是, 可以使用 (2.3.47) 式. 在该式中令

$$
G \rightarrow \frac{-{J}_{y}}{\hslash },\;\lambda \rightarrow \beta tag{3. 9.23}
$$

意识到必须详细了解各种对易关系, 即

$$
\left\lbrack {\frac{-{J}_{y}}{\hslash },{a}_{ + }^{ \dagger }}\right\rbrack = \left( \frac{1}{2i}\right) \left\lbrack {{a}_{ - }^{ \dagger }{a}_{ + },{a}_{ + }^{ \dagger }}\right\rbrack = \left( \frac{1}{2i}\right) {a}_{ - }^{ \dagger } tag{3. 9.24}
$$

$$
\left\lbrack {\frac{-{J}_{y}}{\hslash },\left\lbrack {\frac{-{J}_{y}}{\hslash },{a}_{ + }^{ \dagger }}\right\rbrack }\right\rbrack = \left\lbrack {\frac{-{J}_{y}}{\hslash },\frac{{a}_{ - }^{ \dagger }}{2i}}\right\rbrack = \left( \frac{1}{4}\right) {a}_{ + }^{ \dagger }
$$

以及等等. 显然,得到的永远是 ${a}_{ + }^{ \dagger }$ 或者是 ${a}_{ - }^{ \dagger }$ . 把各项分别收集起来,得到

$$
\mathcal{D}\left( R\right) {a}_{ + }^{ \dagger }{\mathcal{D}}^{-1}\left( R\right) = {a}_{ + }^{ \dagger }\cos \left( \frac{\beta }{2}\right) + {a}_{ - }^{ \dagger }\sin \left( \frac{\beta }{2}\right) . tag{3.9.25}
$$

同样地,

$$
\mathcal{D}\left( R\right) {a}_{ - }^{ \dagger }{\mathcal{D}}^{-1}\left( R\right) = {a}_{ - }^{ \dagger }\cos \left( \frac{\beta }{2}\right) - {a}_{ + }^{ \dagger }\sin \left( \frac{\beta }{2}\right) . tag{3.9.26}
$$

实际上这个结果并不奇怪. 毕竟,基本的自旋向上的态在绕 $y$ 轴的转动下变换成

$$
{a}_{ + }^{ \dagger }\left| {0\rangle \rightarrow \cos \left( \frac{\beta }{2}\right) {a}_{ + }^{ \dagger }}\right| 0\rangle + \sin \left( \frac{\beta }{2}\right) {a}_{ - }^{ \dagger }|0\rangle tag{3.9.27}
$$

把 (3.9.25) 式和 (3.9.26) 式代入 (3.9.21) 式, 并回忆二项式定理

$$
{\left( x + y\right) }^{N} = \mathop{\sum }\limits_{k}\frac{N!{x}^{N - k}{y}^{k}}{\left( {N - k}\right) !k!}, tag{3. 9.28}
$$

得到

$$
\mathcal{D}\left( {\alpha = 0,\beta ,\gamma = 0}\right) |j, m\rangle = \mathop{\sum }\limits_{k}\mathop{\sum }\limits_{l}\frac{\left( {j + m}\right) !\left( {j - m}\right) !}{\left( {j + m - k}\right) !k!\left( {j - m - l}\right) !l!}
$$

$$
\times \frac{{\left\lbrack {a}_{ + }^{ \dagger }\cos \left( \beta /2\right) \right\rbrack }^{j + m - k}{\left\lbrack {a}_{ - }^{ \dagger }\sin \left( \beta /2\right) \right\rbrack }^{k}}{\sqrt{\left( {j + m}\right) !\left( {j - m}\right) !}} tag{3.9.29}
$$

$$
\times {\left\lbrack -{a}_{ + }^{ \dagger }\sin \left( \beta /2\right) \right\rbrack }^{j - m - l}{\left\lbrack {a}_{ - }^{ \dagger }\cos \left( \beta /2\right) \right\rbrack }^{l}|0\rangle .
$$

可以把 (3.9.29) 式与

$$
\mathcal{D}\left( {\alpha = 0,\beta ,\gamma = 0}\right) \left| {j, m\rangle = \mathop{\sum }\limits_{{m}^{\prime }}}\right| j,{m}^{\prime }\rangle {d}_{{m}^{\prime }m}^{\left( j\right) }\left( \beta \right)
$$

$$
= \mathop{\sum }\limits_{{m}^{\prime }}{d}_{{m}^{\prime }m}^{\left( j\right) }\left( \beta \right) \frac{{\left( {a}_{ + }^{ \dagger }\right) }^{j + {m}^{\prime }}{\left( {a}_{ - }^{ \dagger }\right) }^{j - {m}^{\prime }}}{\sqrt{\left( {j + {m}^{\prime }}\right) !\left( {j - {m}^{\prime }}\right) !}}|0\rangle . tag{3.9.30}
$$

相比较. 让 (3.9.29) 式和 (3.9.30) 式中 ${a}_{ + }^{ \dagger }$ 各次幂的系数相等,就可得到 ${d}_{m \mid m}^{\left( j\right) }\left( \beta \right)$ 的显式形式. 特别是,在 (3.9.30) 式中把 ${a}_{ + }^{ \dagger }$ 升高到 $j + {m}^{\prime }$ 次幂与 (3.9.29) 式中 ${a}_{ + }^{ \dagger }$ 升高到 ${2j} - k - l$ 次幂相比较,这样就能确定

$$
l = j - k - {m}^{\prime }. tag{3. 9.31}
$$

寻求固定 ${m}^{\prime }$ 的 ${d}_{{m}^{\prime }m}\left( \beta \right)$ . (3.9.29) 式中对 $k$ 求和与对 $l$ 求和彼此不相独立,利用 (3.9.31) 式消去 $l$ 而剩下 $k$ . 至于 ${a}_{ - }^{ + }$ 的幂次,注意到,在 (3.9.30) 式中 ${a}_{ - }^{ + }$ 升高到 $j - {m}^{\prime }$ 次幂与强加了 (3.9.31) 式而在 (3.9.29) 式中 ${a}_{ - }^{ \dagger }$ 升高到 $k + l$ 次幂自动相匹配. 最后一步是让 $\cos \left( {\beta /2}\right) ,\sin \left( {\beta /2}\right)$ 与 -1 的指数相等,分别是

$$
j + m - k + l = {2j} - {2k} + m - {m}^{\prime },
$$

(3. ${9.32a}$ )

$$
k + j - m - l = {2k} - m + {m}^{\prime }, tag{3. 9.32b}
$$

$$
j - m - l = k - m + {m}^{\prime },
$$

(3. ${9.32}\mathrm{c}$ )

其中用 (3.9.31) 式消掉了 $l$ . 用这样的方法求得 ${d}_{m, m}^{\left( j\right) }\left( \beta \right)$ 的维格纳公式:

$$
{d}_{{m}^{\prime }m}^{\left( j\right) }\left( \beta \right) = \mathop{\sum }\limits_{k}{\left( -1\right) }^{k - m + {m}^{\prime }}\frac{\sqrt{\left( {j + m}\right) !\left( {j - m}\right) !\left( {j + {m}^{\prime }}\right) !\left( {j - {m}^{\prime }}\right) !}}{\left( {j + m - k}\right) !k!\left( {j - k - {m}^{\prime }}\right) !\left( {k - m + {m}^{\prime }}\right) !} tag{3. 9.33}
$$

$$
\times {\left( \cos \frac{\beta }{2}\right) }^{{2j} - {2k} + m - {m}^{\prime }}{\left( \sin \frac{\beta }{2}\right) }^{{2k} - m + {m}^{\prime }},
$$

其中,只要分母中的各阶乘的宗量没有一个为负,就要对 $k$ 求和.

\section{自旋关联测量和贝尔不等式}
\subsection{自旋单态中的关联}

在 3.8 节遇到的角动量相加的最简单的例子涉及由自旋 $\frac{1}{2}$ 粒子构成的一个组合系统. 在这一节以这种系统为例说明量子力学中最使人吃惊的一个结果.

考虑处在一个自旋单态的双电子系统一一这就是说, 它具有零总自旋. 已经看到过, 其态右矢可以写成 [见 (3.8.15d)]:

$$
|\text{自旋单态}\rangle = \left( \frac{1}{\sqrt{2}}\right) \left( {\left| {\widehat{\mathbf{z}}+;\widehat{\mathbf{z}} - \rangle - }\right| \widehat{\mathbf{z}}-;\widehat{\mathbf{z}} + \rangle }\right) \text{,} tag{3.10.1}
$$

这里已经明显地标明了量子化的取向. 回忆一下 $|\widehat{\mathbf{z}} + ;\widehat{\mathbf{z}} - \rangle$ 意味着电子 1 是在自旋向上的态而电子 2 是在自旋向下的态. 对于 $|\widehat{\mathbf{z}} - ;\widehat{\mathbf{z}} + \rangle$ 类似的说法也是对的.

假定测量其中一个电子的自旋分量. 很清楚,有 ${50}\%$ 的机会得到自旋向上或者自旋向下,因为该组合系统可以以相等的概率处于 $|\widehat{\mathbf{z}} + ;\widehat{\mathbf{z}} - \rangle$ 或 $|\widehat{\mathbf{z}} - ;\widehat{\mathbf{z}} + \rangle$ . 但是,如果成员之一被证明是处在自旋向上的态上, 则另一个必然是处在自旋向下的态上, 反之亦然. 当电子 1 的自旋分量被证明是向上的,则测量仪器就挑选了 (3.10.1) 式的第一项 $|\widehat{\mathbf{z}} + ;\widehat{\mathbf{z}} - \rangle$ ; 接着电子 2 自旋分量的测量一定确认该组合系统的态右矢由 $|\widehat{\mathbf{z}} + ;\widehat{\mathbf{z}} - \rangle$ 给出.

值得注意的是, 如果这两个粒子飞开, 即使它们分开得很远, 而且不再发生相互作用, 这一类关联却仍能继续存在, 只要当它们飞开时, 它们的自旋态不发生任何变化. 对于一个 $J = 0$ 的系统自发解体为两个自旋为 $\frac{1}{2}$ 的、没有相对轨道角动量的粒子肯定属于这种情况,因为在解体过程中角动量一定守恒. 这种情况的一个例子是 $\eta$ 介子 (质量为

${549}\mathrm{{MeV}}/{c}^{2}$ ) 到一对 $\mu$ 子的稀有衰变

$$
\eta \rightarrow {\mu }^{ + } + {\mu }^{ - } tag{3.10.2}
$$

不幸的是,它的衰变分支比仅约为 $6 \times {10}^{-6}$ . 更现实一点,在低动能质子-质子散射中,将在第 7 章讨论的泡利原理迫使相互作用的质子处于 ${S}_{0}$ 态 (轨道角动量为 0,自旋单态), 散射后的质子自旋态一定会按照 (3.10.1) 式指出的方式相关联, 即使它们被分隔开一段宏观距离之后.



图 3.11 自旋单态中的自旋关联

更形象化一些,考虑沿相反方向运动的两个自旋 $\frac{1}{2}$ 粒子,如图 3.11 所示. 观察者 $\mathrm{A}$ 专门测量 (向右飞的) 粒子 1 的 ${S}_{z}$ ,而观察者 $B$ 专门测量 (向左飞的) 粒子 2 的 ${S}_{z}$ . 具体说来,假定观察者 $\mathrm{A}$ 发现粒子 1 的 ${S}_{z}$ 为正. 接着他或她甚至在 $\mathrm{B}$ 做任何测量之前就可以肯定地预言出 $\mathrm{B}$ 的测量结果: $\mathrm{B}$ 一定会发现粒子 2 的 ${S}_{z}$ 是负的. 另一方面,如果 $\mathrm{A}$ 不做任何测量,则 $\mathrm{B}$ 有 ${50}\%$ 对 ${50}\%$ 的机会得到 ${S}_{z} +$ 或 ${S}_{z} -$ .

这本身或许还不是这么奇特. 人们可以说, “它只不过像个小罐, 已知里面装有一个黑球和一个白球,当随便从罐里抓出一个球时,有 ${50}\%$ 的机会得到黑球或者白球. 但是如果拿出来的第一个球是黑球时, 那时就可以肯定地预言, 第二个球将是白的. "

事实证明, 这种类比过于简单了. 实际的量子力学情况比这个情况复杂得多! 这是因为观察者可能选择测 ${S}_{u}$ 代替 ${S}_{z}$ . 这同样的一对 “量子力学球” 或者可以用黑与白,或者可以用蓝与红来分析.

现在回忆一下,对于一个单个的自旋 $\frac{1}{2}$ 系统, ${S}_{r}$ 的本征右矢与 ${S}_{z}$ 的本征右矢之间有如下关系:

$$
\left| {\widehat{\mathbf{x}} \pm \rangle = \left( \frac{1}{\sqrt{2}}\right) \left( {\left| {\widehat{\mathbf{z}} + \rangle \pm }\right| \widehat{\mathbf{z}} - \rangle }\right) ,\;}\right| \widehat{\mathbf{z}} \pm \rangle = \left( \frac{1}{\sqrt{2}}\right) \left( {\left| {\widehat{\mathbf{x}} + \rangle \pm }\right| \widehat{\mathbf{x}} - \rangle }\right) . tag{3.10.3}
$$

现在回到组合系统,可以通过选择 $x$ 方向作为量子化的轴,把自旋单态右矢 (3.10.1) 重写为

$$
|\text{自旋单态}\rangle = \left( \frac{1}{2}\right) \left( {\left| {\widehat{\mathbf{x}} - ;\widehat{\mathbf{x}} + \rangle - }\right| \widehat{\mathbf{x}}+;\widehat{\mathbf{x}} - \rangle }\right) \text{.} tag{3.10.4}
$$

除去在任何情况只不过是一个约定的整体符号外, 或许可以直接从 (3.10.1) 式猜出这个形式, 因为自旋单态不具有任何空间的优势方向. 现在假定, 观察者 $\mathrm{A}$ 可以通过改变他或她的自旋分析器的取向来选择测量粒子 1 的 ${S}_{x}$ 还是 ${S}_{z}$ ,而观察者 $B$ 则永远专门测量粒子 2 的 ${S}_{x}$ . 如果 $\mathrm{A}$ 确定粒子 1 的 ${S}_{z}$ 为正,显然 $\mathrm{B}$ 有 ${50}\%$ 对 ${50}\%$ 的机会得到 ${S}_{x} +$ 或 ${S}_{x} -$ ; 尽管已知粒子 2 的 ${S}_{z}$ 肯定为负,但它的 ${S}_{x}$ 是完全不确定的. 另一方面,假定 $\mathrm{A}$ 也选择测量 ${S}_{r}$ . 如果观察者 $\mathrm{A}$ 确定粒子 1 的 ${S}_{x}$ 为正,则毫无例外,观察者 $\mathrm{B}$ 将会测到粒子 2 的 ${S}_{x}$ 为负. 最后,如果 A 选择不做任何测量,当然, B 将有 ${50}\%$ 对 ${50}\%$ 机会得到 ${S}_{r}$ 十或 ${S}_{r}$ 一. 综上所述:

1. 如果 $\mathrm{A}$ 测量 ${S}_{z}$ 而 $\mathrm{B}$ 测量 ${S}_{r}$ ,则两个测量之间有完全随机的关联.

2. 如果 A 测量 ${S}_{x}$ 而 B 测量 ${S}_{x}$ ,则两个测量之间有 ${100}\%$ (相反符号的) 关联.

3. 如果 $\mathrm{A}$ 不做任何测量,则 $\mathrm{B}$ 的测量显示随机的结果.

表 3.1 自旋关联测量


表 3.1 显示了,当 $\mathrm{B}$ 和 $\mathrm{A}$ 都被允许选择测量 ${S}_{x}$ 或者 ${S}_{z}$ 时,所有可能的这样的测量结果. 这些考虑表明, $\mathrm{B}$ 的测量结果表现出对 $\mathrm{A}$ 决定做哪一种测量的依赖性: 即 $\mathrm{A}$ 测量 ${S}_{r}$ 、 测量 ${S}_{z}$ 或者不做任何测量. 再次注意, $\mathrm{A}$ 和 $\mathrm{B}$ 可能分开几英里远,没有任何通讯或相互作用的可能性. 观察者 $\mathrm{A}$ 可以决定,在两个粒子分开足够远时,他或她的分析器如何取向. 而粒子 2 就好像 “知道” 了粒子 1 的哪一个自旋分量被测量了.

正统量子力学对这种情况做如下解释. 测量是一次选择 (或过滤) 过程. 当测出粒子 1 的 ${S}_{z}$ 为正时,则挑选了 $|\widehat{\mathbf{z}} + ;\widehat{\mathbf{z}} - \rangle$ 分量. 接着对另一个粒子的 ${S}_{z}$ 的测量只不过要确认这个系统仍然处在 $|\widehat{\mathbf{z}} + \mathbf{i}\widehat{\mathbf{z}} - \rangle$ . 必须承认,对该系统的一部分处于什么状态的测量将被看作对整个系统的测量.

\subsection{爱因斯坦局域性原理和贝尔不等式}

许多物理学家对于前述自旋关联的正统解释感到不舒服. 这种感觉体现在下面频繁引用的爱因斯坦的评述中, 称它为爱因斯坦局域性原理: “但是, 在我看来, 我们确实应该紧紧抓住一个假定: 系统 ${S}_{2}$ 的真实情况和系统 ${S}_{1}$ 做了些什么无关,它们是空间分离的. ” 因为这个问题最早是在 1935 年爱因斯坦, 波多尔斯基 (B. Podolsky) 和罗森 (N. Rosen) 的一篇文章中讨论的, 因此有时称之为爱因斯坦-波多尔斯基-罗森 (Einstein-Podolsky-Rosen) 佯谬. *

---

* 精确的历史事实是,爱因斯坦-波多尔斯基-罗森的原文处理 $x$ 和 $p$ 的测量. 利用组合的 $\frac{1}{2}$ 系统为例说明爱因斯坦-波多尔斯基-罗森佯谬始于玻姆.

---

有些人曾争辩过, 在这里遇到的困难是量子力学的概率解释所固有的, 微观层次的动力学行为出现了概率性的结果仅仅是因为某些还不知道的参量——所谓的隐变量——还没有被确定下来. 这里的目的不是要讨论基于隐变量或其他一些考虑的量子力学的各种替代理论. 相反, 试问, 这样的理论给出了不同于量子力学的预言吗? 直到 1964 年之前, 人们一直认为, 可以这样编造替代理论, 以致给不出任何不同于通常量子力学的而又能被实验证实的预言. 整个争论应该属于形而上学领域而不属于物理学. 当时, J. S. Bell 指出, 基于爱因斯坦的局域性原理的替代理论, 实际上预言了在自旋关联实验中可观测量间的一个可以检验的不等式关系, 它与量子力学的预言不相符.

在一个由维格纳构想出来的简单模型框架下推导出贝尔不等式, 该模型纳入了各种替代理论的基本特点. 这个模型的支持者们赞同不可能同时确定 ${S}_{r}$ 和 ${S}_{z}$ . 然而,当有大量自旋 $\frac{1}{2}$ 的粒子时,赋予它们当中的某部分成员具有下列的一些性质:

如果测量 ${S}_{z}$ ,确定无疑地得到加号.

如果测量 ${S}_{r}$ ,确定无疑地得到减号.

满足这一性质的粒子被称为属于 $\left( {\widehat{\mathbf{z}} + ,\widehat{\mathbf{x}} - }\right)$ 类型. 注意,并不能断言,可以同时测量分别为十和一的 ${S}_{z}$ 和 ${S}_{x}$ . 当测量 ${S}_{z}$ 时,不测量 ${S}_{x}$ ,反之亦然. 赋予沿不止一个方向的确定的自旋分量值, 应理解为只有一个或另一个分量可以实际被测量. 尽管这种方法与量子力学的方法根本不同,只要属于 $\left( {\widehat{\mathbf{z}}+,\widehat{\mathbf{x}} + }\right)$ 类型的粒子像 $\left( {\widehat{\mathbf{z}}+,\widehat{\mathbf{x}} - }\right)$ 类型的粒子一样多,则对在自旋向上 $\left( {{S}_{z} + }\right)$ 的态上所做的 ${S}_{z}$ 和 ${S}_{x}$ 测量的量子力学预言可以重现.

现在检验这个模型怎样解释在组合自旋单态系统上测量自旋关联的结果. 显然, 对于一对特殊的粒子, 粒子 1 和粒子 2 之间一定有一个完美的匹配, 以保证总角动量为零: 如果粒子 1 是 $\left( {\widehat{\mathbf{z}}+,\widehat{\mathbf{x}} - }\right)$ 类型的,则粒子 2 一定是 $\left( {\widehat{\mathbf{z}}-,\widehat{\mathbf{x}} + }\right)$ 类型的,等等. 如果具有相等的布居——即各为 ${25}\%$ ——的粒子 1 和粒子 2 有如下匹配:

$$
\text{粒子 1 粒子 2}
$$

$$
\left( {\widehat{\mathbf{z}}+,\widehat{\mathbf{x}} - }\right) \leftrightarrow \left( {\widehat{\mathbf{z}}-,\widehat{\mathbf{x}} + }\right) , tag{3.10.5a}
$$

$$
\left( {\widehat{\mathbf{z}}+,\widehat{\mathbf{x}} + }\right) \leftrightarrow \left( {\widehat{\mathbf{z}}-,\widehat{\mathbf{x}} - }\right) , tag{3.10.5b}
$$

$$
\left( {\widehat{\mathbf{z}}-,\widehat{\mathbf{x}} + }\right) \leftrightarrow \left( {\widehat{\mathbf{z}}+,\widehat{\mathbf{x}} - }\right) ,
$$

(3.10. ${5c}$ )

$$
\left( {\widehat{\mathbf{z}}-,\widehat{\mathbf{x}} - }\right) \leftrightarrow \left( {\widehat{\mathbf{z}}+,\widehat{\mathbf{x}} + }\right) tag{3.10.5d}
$$

则诸如表 3.1 所示的关联测量结果就可以重新产生. 这里隐含着一个非常重要的假设. 假定特殊的一对粒子是属于 (3.10.5a) 类型的,并且观察者 $\mathrm{A}$ 决定测量粒子 1 的 ${S}_{z}$ ; 那时,他或她必定得到一个加号,不管 B 决定测量 ${S}_{z}$ 或 ${S}_{x}$ . 正是在这种意义上,爱因斯坦的局域性原理被包含在这个模型中: $\mathrm{A}$ 的结果是预先确定的,它与 $\mathrm{B}$ 选择测量什么无关.

到此为止所考虑的例子中, 这个模型在重现量子力学预言方面已经获得了成功. 现在考虑该模型能导致不同于通常量子力学预言的更为复杂的情况. 这一次, 从三个单位矢量 $\widehat{\mathbf{a}},\widehat{\mathbf{b}}$ 和 $\widehat{\mathbf{c}}$ 开始,一般说来它们不是相互正交的. 设想其中的一个粒子属于某确定类型,比如 $\left( {\widehat{a}-,\widehat{b} + ,\widehat{c} + }\right)$ ,它意味着: 如果测量 $S \cdot \widehat{a}$ ,肯定得到负号; 如果量 $S \cdot \widehat{b}$ ,肯定得到正号; 如果测量 $\mathbf{S} \cdot \widehat{\mathbf{c}}$ ,也肯定得到正号. 再一次,在为了保证总角动量为零,另一个粒子必定属于 $\left( {\widehat{a}+,\widehat{b}-,\widehat{c} - }\right)$ 类型的意义上,一定有一个完美的匹配. 在任意给定的事例中, 所考虑的粒子对一定是表 3.2 所示的八种类型之一. 这八种可能性互不相容并且没有交集. 每一类的布居数标明在第一列中.

假定观察者 $\mathrm{A}$ 发现 ${\mathbf{S}}_{1} \cdot \widehat{\mathbf{a}}$ 是正号而观察者 $\mathrm{B}$ 发现 ${\mathbf{S}}_{2} \cdot \widehat{\mathbf{b}}$ 也是正号,从表 3.2 显然可见,这一对粒子属于类型 3 或者类型 4,所以,在这种情况中粒子对的数目是 ${N}_{3} + {N}_{1}$ . 因为 ${N}_{i}$ 是半正定的,必须有不等式

$$
{N}_{3} + {N}_{4} \leq \left( {{N}_{2} + {N}_{4}}\right) + \left( {{N}_{3} + {N}_{7}}\right) . tag{3.10.6}
$$

设 $P\left( {\widehat{\mathbf{a}}+;\widehat{\mathbf{b}} + }\right)$ 是在一种随机选择下,观察者 $\mathrm{A}$ 测量 ${\mathrm{S}}_{1} \cdot \widehat{\mathbf{a}}$ 是正号,并且观察者 $\mathrm{B}$ 测量 ${\mathrm{S}}_{2} \cdot \widehat{\mathrm{b}}$ 也是正号,等等的概率.

表 3.2 在一些替代理论中匹配的自旋分量

<table><thead><tr><th>布居数</th><th>粒子 1</th><th>粒子 2</th></tr></thead><tr><td>${N}_{1}$</td><td>$\left( {\widehat{\mathbf{a}} + ,\widehat{\mathbf{b}} + ,\widehat{\mathbf{c}} + }\right)$</td><td>$\left( {\widehat{\mathbf{a}}-,\widehat{\mathbf{b}} - ,\widehat{\mathbf{c}} - }\right)$</td></tr><tr><td>${N}_{2}$</td><td>$\left( {\widehat{\mathbf{a}} + \widehat{\mathbf{b}} + \widehat{\mathbf{c}} - }\right)$</td><td>$\left( {\widehat{\mathbf{a}}-,\widehat{\mathbf{b}}-,\widehat{\mathbf{c}} + }\right)$</td></tr><tr><td>${N}_{3}$</td><td>$\left( {\widehat{\mathbf{a}} + ,\widehat{\mathbf{b}} - ,\widehat{\mathbf{c}} + }\right)$</td><td>$\left( {\widehat{\mathbf{a}} - ,\widehat{\mathbf{b}} + ,\widehat{\mathbf{c}} - }\right)$</td></tr><tr><td>${N}_{4}$</td><td>$\left( {\widehat{\mathbf{a}} + \widehat{\mathbf{b}} - \widehat{\mathbf{c}} - \widehat{\mathbf{c}} - }\right)$</td><td>$\left( {\widehat{\mathbf{a}}-,\widehat{\mathbf{b}} + ,\widehat{\mathbf{c}} + }\right)$</td></tr><tr><td>${N}_{5}$</td><td>$\left( {\widehat{\mathbf{a}} - \widehat{\mathbf{b}} + \widehat{\mathbf{c}} + }\right)$</td><td>$\left( {\widehat{\mathbf{a}}+,\widehat{\mathbf{b}}-,\widehat{\mathbf{c}} - }\right)$</td></tr><tr><td>${N}_{6}$</td><td>$\left( {\widehat{\mathbf{a}} - \widehat{\mathbf{b}} + \widehat{\mathbf{c}} - }\right)$</td><td>$\left( {\widehat{\mathbf{a}}+,\widehat{\mathbf{b}} - ,\widehat{\mathbf{c}} + }\right)$</td></tr><tr><td>${N}_{i}$</td><td>$\left( {\widehat{\mathbf{a}} - \widehat{\mathbf{b}} - \widehat{\mathbf{c}} + }\right)$</td><td>$\left( {\widehat{\mathbf{a}} + ,\widehat{\mathbf{b}} + ,\widehat{\mathbf{c}} - }\right)$</td></tr><tr><td>${N}_{8}$</td><td>$\left( {\widehat{\mathbf{a}}-,\widehat{\mathbf{b}}-,\widehat{\mathbf{c}} - }\right)$</td><td>$\left( {\widehat{\mathbf{a}} + \widehat{\mathbf{b}} + \widehat{\mathbf{b}} + \widehat{\mathbf{c}} + }\right)$</td></tr></table>

显然有

$$
P\left( {\widehat{\mathbf{a}}+;\widehat{\mathbf{b}} + }\right) = \frac{\left( {N}_{3} + {N}_{4}\right) }{\mathop{\sum }\limits_{i}^{8}{N}_{i}}. tag{3.10.7}
$$

用类似的方式. 得到

$$
P\left( {\widehat{\mathbf{a}}+;\widehat{\mathbf{c}} + }\right) = \frac{\left( {N}_{2} + {N}_{1}\right) }{\mathop{\sum }\limits_{i}^{8}{N}_{i}}\text{ 和 }\;P\left( {\widehat{\mathbf{c}}+;\widehat{\mathbf{b}} + }\right) = \frac{\left( {N}_{3} + {N}_{7}\right) }{\mathop{\sum }\limits_{i}^{8}{N}_{i}}. tag{3.10.8}
$$

正定性条件 (3.10.6) 式现在变成

$$
P\left( {\widehat{\mathbf{a}}+;\widehat{\mathbf{b}} + }\right) \leq P\left( {\widehat{\mathbf{a}}+;\widehat{\mathbf{c}} + }\right) + P\left( {\widehat{\mathbf{c}}+;\widehat{\mathbf{b}} + }\right) . tag{3.10.9}
$$

这就是贝尔不等式, 它源自爱因斯坦局域性原理.

\subsection{量子力学和贝尔不等式}

现在返回到量子力学世界. 在量子力学中,没有谈到粒子对的某一部分,比如 ${N}_{3}/$ $\mathop{\sum }\limits_{i}^{8}{N}_{i}$ ,属于类型 3 . 相反,用同样的右矢 (3.10.1) 式表征所有的自旋单态系统; 用 3.4 节的语言讲, 在这里所涉及的是一个纯系综. 利用这个右矢和所发展的量子力学规则, 可以无疑义地计算在不等式 (3.10.9) 的三项中的每一项.

先求 $P\left( {\widehat{\mathbf{a}}+;\widehat{\mathbf{b}} + }\right)$ . 假定观察者 $\mathrm{A}$ 发现 ${\mathbf{S}}_{1} \cdot \widehat{\mathbf{a}}$ 是正的,由于早先讨论过的 ${100}\%$ (相反的符号) 关联, $\mathrm{B}$ 测量 ${\mathbf{S}}_{2} \cdot \widehat{\mathbf{a}}$ 将肯定地产生负号. 但是,为计算 $P\left( {\widehat{\mathbf{a}}+;\widehat{\mathbf{b}} + }\right)$ ,必须考虑一个新的量子化轴 $\widehat{\mathbf{b}}$ ,它与 $\widehat{\mathbf{a}}$ 夹角为 ${\theta }_{ub}$ ,见图 3.12. 按照 3.2 节的公式形式,当已知粒子 2 处在 ${\mathbf{S}}_{2} \cdot \widehat{\mathbf{a}}$ 的负本征值的本征右矢上时,测量 ${\mathbf{S}}_{2} \cdot \widehat{\mathbf{b}}$ 产生 + 的概率为

$$
{\cos }^{2}\left\lbrack \frac{\left( \pi - {\theta }_{ab}\right) }{2}\right\rbrack = {\sin }^{2}\left( \frac{{\theta }_{ab}}{2}\right) . tag{3.10.10}
$$

结果得到

$$
P\left( {\widehat{\mathbf{a}}+;\widehat{\mathbf{b}} + }\right) = \left( \frac{1}{2}\right) {\sin }^{2}\left( \frac{{\theta }_{ab}}{2}\right) , tag{3.10.11}
$$

其中因子 $\frac{1}{2}$ 来自初始得到 ${\mathbf{S}}_{1} \cdot \widehat{\mathbf{a}}$ 为正的概率. 利用 (3.10.11) 式和它在 (3.10.9) 式其他两项上的推广, 可以把贝尔不等式写成

$$
{\sin }^{2}\left( \frac{{\theta }_{ab}}{2}\right) \leq {\sin }^{2}\left( \frac{{\theta }_{ac}}{2}\right) + {\sin }^{2}\left( \frac{{\theta }_{cb}}{2}\right) . tag{3.10.12}
$$



图 3.12 $P\left( {\widehat{a}+;\widehat{b} + }\right)$ 的求值

现在来证明, 从几何观点出发, 不等式 (3.10.12) 式不是永远可能的. 为了简单起见, 选择 $\widehat{\mathbf{a}},\widehat{\mathbf{b}}$ 和 $\widehat{\mathbf{c}}$ 处在同一平面上,而且设 $\widehat{\mathbf{c}}$ 把 $\widehat{\mathbf{a}}$ 和 $\widehat{\mathbf{b}}$ 确定的两个方向一分为二:

$$
{\theta }_{ab} = {2\theta },\;{\theta }_{ac} = {\theta }_{cb} = \theta . tag{3.10.13}
$$

于是, 不等式 (3.10.12) 在

$$
0 < \theta < \frac{\pi }{2}. tag{3.10.14}
$$

时被破坏. 例如取 $\theta = \pi /4$ 时,就会得到

$$
{0.500} \leq {0.292}?? tag{3.10.15}
$$

因此, 量子力学预言与贝尔不等式不相容. 在量子力学和满足爱因斯坦局域性原理的替代理论之间, 存在一个真实、可观测——在可被实验检验的意义上——的差别.

目前已经完成了几个检验贝尔不等式的实验. 最近的评述请见《贝尔不等式检验: 比以往任何时候都更理想》, 源自 A. Aspect 的 Nature 398 (1999) 189. 在其中的一个实验中, 测量了在低能质子-质子散射中, 末态质子间的自旋关联. 而其他所有的实验都是在一个激发原子 $\left( {\mathrm{{Ca}},\mathrm{{Hg}},\cdots }\right)$ 的级联跃迁

$$
\left( {j = 0}\right) \overset{\gamma }{ \rightarrow }\left( {j = 1}\right) \overset{\gamma }{ \rightarrow }\left( {j = 0}\right) , tag{3.10.16}
$$

中或者在电子偶素(一个 ${e}^{ + }{e}^{ - }$ 的 ${}^{1}{S}_{0}$ 束缚态)的衰变中,对一对光子间的光子-极化关联进行测量. 基于 1.1 节发展出来的类比

$$
{S}_{z} + \rightarrow \widehat{\varepsilon }\;\text{ 沿 }x\text{ 方向,} tag{3.10.17a}
$$

$$
{S}_{z} \rightarrow \widehat{\varepsilon }\;\text{沿}y\text{方向,} tag{3.10.17b}
$$

${S}_{x} + \rightarrow \widehat{\varepsilon }$ 沿 ${45}^{ \circ }$ 斜线方向,(3.10.17c)

${S}_{x} \rightarrow \widehat{\varepsilon }$ 沿 ${135}^{ \circ }$ 斜线方向.(3.10.17d)

研究光子-极化关联应当同样地好. 所有最近的精确实验结果都决定性地确证贝尔不等式被破坏了, 有一种情况竟超过了九个标准偏差. 此外, 在所有这些实验中, 这个不等式是以这样的一种方式被破坏的, 即量子力学预言在误差范围之内得到满足. 在这场论战中, 量子力学取得了辉煌的胜利.

量子力学预言被证实并不意味着现在这整个课题都是无聊的. 虽然有实验的结论, 但这类测量的很多方面使人感到心里不舒服. 特别是考虑下面的一点: 在观察者 $\mathrm{A}$ 对粒子 1 做了一次测量以后, 粒子 2-一原则上, 它可能离开粒子 1 许多光年——怎么会 “知道” 它的自旋应怎样取向, 以使清楚地列在表 3.1 中的那些引人瞩目的关联得以实现呢? 在其中一个检验贝尔不等式的实验当中 [由阿斯派克特 (A. Aspect) 和他的合作者完成的], 分析器的设置变化要非常地快, 直到任何一类传播速度比光速慢的影响来不及到达 $\mathrm{B}$ 之前, A 就已经做出了测量什么的决定.

通过展示下述看法来结束这一节: 尽管有这些奇特之处, 仍然不可能用自旋关联的测量在两个宏观分离的点之间传送任何有用的信息. 特别是, 超光速 (比光速还快) 的通信是不可能的.

假定 $\mathrm{A}$ 和 $\mathrm{B}$ 事先都同意测量 ${S}_{z}$ ,然后,不用问 $\mathrm{A},\mathrm{B}$ 就精确地知道 $\mathrm{A}$ 得到的是什么. 但是, 这并不意味着 A 和 B 正在通信联系, B 只不过观测一个正号与负号的随机系列. 这里面显然没有包含任何有用的信息. 只有当 $\mathrm{B}$ 与 $\mathrm{A}$ 相聚在一起并且比较了记录(或计算机工作表) 才能证实由量子力学预言的令人瞩目的关联.

或许有人会认为, 如果 A 和 B 中的一个突然改变了他或她的分析器的取向, 他们就能够通信. 假定一开始 $\mathrm{A}$ 同意测量 ${S}_{z}$ ,而 $\mathrm{B}$ 测量 ${S}_{r}.\mathrm{\;A}$ 的测量结果与 $\mathrm{B}$ 的测量结果完全没有关联, 因此不存在被传递的信息. 但是之后, 假定 $\mathrm{A}$ 突然破坏了他或她的许诺,没有告知 B,就开始测量 ${S}_{r}$ . 现在,在 A 的结果与 B 的结果之间有了完全的关联. 然而, B 没有任何办法推断 $\mathrm{A}$ 已经改变了他或她的分析器的取向. $\mathrm{B}$ 通过只看他或她自己的笔记本仅能继续看到 “十” 和 “一” 的随机系列. 于是, 再次没有被传递的信息.

\section{张量算符}
\subsection{矢量算符}

书中已经使用了诸如 $\mathbf{x},\mathbf{p},\mathbf{S}$ 和 1 等符号,但到目前为止,还没有系统地讨论过它们的转动性质. 它们都是矢量算符, 但在转动之下它们有什么样的性质呢? 在本节, 基于矢量算符与角动量算符的对易关系, 给出它们精确的量子力学定义. 然后推广到具有更复杂变换性质的张量算符, 并推导出关于矢量算符和张量算符矩阵元的一个重要定理.

大家都知道,经典物理中的矢量是一个具有三个分量的量,在转动之下根据定义像 ${V}_{i}$ $\rightarrow \mathop{\sum }\limits_{j}{R}_{ij}{V}_{j}$ 一样变换. 合理的做法是,要求量子力学中的一个矢量算符 $V$ 的期待值在转动之下像一个经典矢量一样变换. 具体地说, 假定在转动之下当态右矢按照

$$
\left| {\alpha \rangle \rightarrow \mathcal{D}\left( R\right) }\right| \alpha \rangle , tag{3.11.1}
$$

变化时, $\mathbf{V}$ 的期待值变化如下:

$$
\left\langle {\alpha \left| {V}_{i}\right| \alpha }\right\rangle \rightarrow \left\langle {\alpha \left| {{\mathcal{D}}^{ \dagger }\left( R\right) {V}_{i}\mathcal{D}\left( R\right) }\right| \alpha }\right\rangle = \mathop{\sum }\limits_{j}{R}_{ij}\left\langle {\alpha \left| {V}_{j}\right| \alpha }\right\rangle , tag{3.11.2}
$$

这必须对任意的一个右矢都对. 因此, 作为一个算符方程式

$$
{\mathcal{D}}^{ + }\left( R\right) {V}_{i}\mathcal{D}\left( R\right) = \mathop{\sum }\limits_{j}{R}_{ij}{V}_{j} tag{3.11.3}
$$

必须成立,其中 ${R}_{ij}$ 是与转动对应的 $3 \times 3$ 矩阵.

下面考虑一种具体的情况, 即一个无穷小转动. 当转动是无穷小时, 有

$$
\mathcal{D}\left( R\right) = 1 - \frac{{i\varepsilon }\mathbf{J} \cdot \widehat{\mathbf{n}}}{\hslash }. tag{3.11.4}
$$

可以把 (3.11.3) 式写成

$$
{V}_{i} + \frac{\varepsilon }{i\hslash }\left\lbrack {{V}_{i},\mathbf{J} \cdot \widehat{\mathbf{n}}}\right\rbrack = \mathop{\sum }\limits_{j}{R}_{ij}\left( {\widehat{\mathbf{n}};\varepsilon }\right) {V}_{j}, tag{3.11.5}
$$

特别是,对于沿 $z$ 轴的 $\widehat{\mathbf{n}}$ ,有

$$
R\left( {\widehat{\mathbf{z}};\varepsilon }\right) = \left( \begin{matrix} 1 & - \varepsilon & 0 \\ \varepsilon & 1 & 0 \\ 0 & 0 & 1 \end{matrix}\right) , tag{3.11.6}
$$

所以

$$
i = 1 : \;{V}_{x} + \frac{\varepsilon }{i\hslash }\left\lbrack {{V}_{x},{J}_{z}}\right\rbrack = {V}_{x} - \varepsilon {V}_{y} tag{3.11.7a}
$$

$$
i = 2 : \;{V}_{y} + \frac{\varepsilon }{i\hslash }\left\lbrack {{V}_{y},{J}_{z}}\right\rbrack = \varepsilon {V}_{x} + {V}_{y} tag{3.11.7b}
$$

$$
i = 3 : \;{V}_{z} + \frac{\varepsilon }{ih}\left\lbrack {{V}_{z},{J}_{z}}\right\rbrack = {V}_{z}. tag{3.11.7c}
$$

这意味着, $\mathbf{V}$ 必须满足对易关系:

$$
\left\lbrack {{V}_{i},{J}_{j}}\right\rbrack = i{\varepsilon }_{ijk}\hslash {V}_{k}. tag{3.11.8}
$$

显然,在一个有限转动下 $\mathbf{V}$ 的行为完全由上述对易关系确定; 只要将目前已非常熟悉

的公式 (2.3.47) 应用于

$$
\exp \left( \frac{i{J}_{j}\phi }{\hslash }\right) {V}_{i}\exp \left( \frac{-i{J}_{j}\phi }{\hslash }\right) . tag{3.11.9}
$$

只需计算

$$
\left\lbrack {{J}_{j},\left\lbrack {{J}_{j},\left\lbrack {\cdots \left\lbrack {{J}_{j},{V}_{i}}\right\rbrack \cdots }\right\rbrack }\right\rbrack }\right\rbrack \text{.} tag{3.11.10}
$$

就像自旋情况下的 (3.2.7) 式一样,多重对易关系仍然返回 ${V}_{t}$ 或者 ${V}_{k}\left( {k \neq i, j}\right)$ .

用 (3.11.8) 式作为一个矢量算符的定义属性. 注意, 角动量对易关系是 (3.11.8) 式中令 ${V}_{i} \rightarrow {J}_{i},{V}_{k} \rightarrow {J}_{k}$ 的一种特殊情况. 其他的一些特殊情况是 $\left\lbrack {y,{L}_{z}}\right\rbrack = {ihx},\left\lbrack {x,{L}_{z}}\right\rbrack$ $= - i\hslash y,\left\lbrack {{p}_{x},{L}_{z}}\right\rbrack = - i\hslash {p}_{y}$ 和 $\left\lbrack {{p}_{y},{L}_{z}}\right\rbrack = i\hslash {p}_{x}$ ,这些都可以直接证明.

\subsection{笛卡尔张量与不可约张量}

在经典物理中,通常在 $3 \times 3$ 正交矩阵 $R$ 所规定的转动下,通过把 ${V}_{i} \rightarrow {\sum }_{j}{R}_{ij}{V}_{j}$ 推广成:

$$
{T}_{ijk}\cdots \rightarrow \mathop{\sum }\limits_{{i}^{\prime }}\mathop{\sum }\limits_{{j}^{\prime }}\mathop{\sum }\limits_{{k}^{\prime }}\cdots {R}_{i{i}^{\prime }}{R}_{j{j}^{\prime }}\cdots {T}_{{i}^{\prime }{j}^{\prime }{k}^{\prime }}\cdots tag{3.11.11}
$$

来定义一个张量 ${T}_{yk}\ldots$ 指标的数目称为张量的秩. 这样的张量叫笛卡尔张量.

最简单的秩为 2 的笛卡尔张量的例子是由两个矢量 $\mathbf{U}$ 和 $\mathbf{V}$ 构成的并矢. 人们只要取 $\mathbf{U}$ 的一个笛卡尔分量和 $\mathbf{V}$ 的一个笛卡尔分量,然后把它们放到一起:

$$
{T}_{ij} \equiv {U}_{i}{V}_{j}. tag{3.11.12}
$$

注意, 总共有九个分量, 在转动下显然像 (3.11.11) 式那样变换.

像 (3.11.12) 式那样的一个笛卡尔张量遇到的麻烦是, 它是可约的一一即它可以分解为几个成员, 它们在转动下的变换是不同的. 具体地说, 对 (3.11.12) 式中的并矢, 有

$$
{U}_{i}{V}_{j} = \frac{\mathbf{U} \cdot \mathbf{V}}{3}{\delta }_{ij} + \frac{\left( {U}_{i}{V}_{j} - {U}_{j}{V}_{i}\right) }{2} + \left( {\frac{{U}_{i}{V}_{j} + {U}_{j}{V}_{i}}{2} - \frac{\mathbf{U} \cdot \mathbf{V}}{3}{\delta }_{ij}}\right) . tag{3.11.13}
$$

其右边的第一项 $\mathbf{U} \cdot \mathbf{V}$ 是一个在转动下不变的标量积. 第二项是一个反对称张量,它可以写成一个矢量积 ${\varepsilon }_{ijk}{\left( \mathbf{U} \times \mathbf{V}\right) }_{k}$ . 它一共有 3 个独立的分量. 最后一项是一个有 $5( = 6 - 1$ , 其中的 1 来自无迹条件) 个独立分量的 $3 \times 3$ 对称无迹张量. 核对独立分量的个数:

$$
3 \times 3 = 1 + 3 + 5. tag{3.11.14}
$$

注意到,在 (3.11.14) 式右边出现的数恰好是角动量分别为 $l = 0, l = 1$ 和 $l = 2$ 成员的多重数. 这就表明并矢已被分解成能像 $l = 0, l = 1$ 和 $l = 2$ 的球谐函数一样变换的张量. 事实上, (3.11.13) 式是一个演示笛卡尔张量约化成不可约球张量的最简单的非平凡实例.

在介绍球张量的精确定义之前,先给出一个秩为 $k$ 的球张量的例子. 假定取球谐函数 ${Y}_{l}^{m}\left( {\theta ,\phi }\right)$ . 它可以写成 ${Y}_{l}^{m}\left( \widehat{\mathbf{n}}\right)$ ,其中 $\widehat{\mathbf{n}}$ 的取向由 $\theta$ 和 $\phi$ 来表征. 现在用某个矢量 $\mathbf{V}$ 代替 $\widehat{\mathbf{n}}$ . 结果是,有了一个秩为 $k$ (代替 $l$ ) 磁量子数为 $q$ (代替 $m$ ) 的球张量,即

$$
{T}_{q}^{\left( k\right) } = {Y}_{l = k}^{m = q}\left( \mathbf{V}\right) . tag{3.11.15}
$$

具体地说,在 $k = 1$ 的情况下,取 $l = 1$ 的球谐函数,并用 ${V}_{z}$ 代替 $\left( {z/r}\right) = {\left( \widehat{\mathbf{n}}\right) }_{z}$ ,等等.

$$
{Y}_{l}^{0} = \sqrt{\frac{3}{4\pi }}\cos \theta = \sqrt{\frac{3}{4\pi }}\frac{z}{r} \rightarrow {T}_{0}^{\left( 1\right) } = \sqrt{\frac{3}{4\pi }}{V}_{z}, tag{3.11.16}
$$

$$
{Y}_{1}^{\pm 1} = \mp \sqrt{\frac{3}{4\pi }}\frac{x \pm {iy}}{\sqrt{2}r} \rightarrow {T}_{\pm 1}^{\left( 1\right) } = \sqrt{\frac{3}{4\pi }}\left( {\mp \frac{{V}_{x} \pm i{V}_{y}}{\sqrt{2}}}\right) .
$$

显然. 这可以推广到更高的 $k$ . 例如

$$
{Y}_{2}^{\pm 2} = \sqrt{\frac{15}{32\pi }}\frac{{\left( x \pm iy\right) }^{2}}{{r}^{2}} \rightarrow {T}_{\pm 2}^{\left( 2\right) }\sqrt{\frac{15}{32\pi }}{\left( {V}_{x} \pm i{V}_{y}\right) }^{2}. tag{3.11.17}
$$

${T}_{q}^{\left( k\right) }$ 像 ${Y}_{l}^{\left( m\right) }$ 一样是不可约的. 由于这个原因,用球张量比用笛卡尔张量更令人满意.

要看这样构造的球张量的变换,先来评述在转动下 ${Y}_{l}^{m}$ 如何变换. 首先,对于方向本征右矢. 有

$$
\left| {\widehat{\mathbf{n}}\rangle \rightarrow \mathcal{D}\left( R\right) }\right| \widehat{\mathbf{n}}\rangle \equiv \left| {\widehat{\mathbf{n}}}^{\prime }\right\rangle , tag{3.11.18}
$$

它定义了转动后的本征右矢 $\left| {\widehat{\mathbf{n}}}^{\prime }\right\rangle$ . 希望检查 ${Y}_{l}^{m}\left( {\widehat{\mathbf{n}}}^{\prime }\right) = \left\langle {{\widehat{\mathbf{n}}}^{\prime } \mid l, m}\right\rangle$ 怎样用 ${Y}_{l}^{m}\left( \widehat{\mathbf{n}}\right)$ 表示. 从

$$
\mathcal{D}\left( {R}^{-1}\right) \left| {l, m\rangle = \mathop{\sum }\limits_{{m}^{\prime }}}\right| l,{m}^{\prime }\rangle {\mathcal{D}}_{{m}^{\prime }m}^{\left( l\right) }\left( {R}^{-1}\right) tag{3.11.19}
$$

出发,用 $\left\langle \widehat{\mathbf{n}}\right|$ 从左边收缩,并利用 (3.11.18) 式,很容易看到:

$$
{Y}_{l}^{m}\left( {\widehat{\mathbf{n}}}^{\prime }\right) = \mathop{\sum }\limits_{{m}^{\prime }}{Y}_{l}^{{m}^{\prime }}\left( \widehat{\mathbf{n}}\right) {\mathcal{D}}_{{m}^{\prime }m}^{\left( l\right) }\left( {R}^{-1}\right) . tag{3.11.20}
$$

如果有一个算符能像 ${Y}_{l}^{m}\left( \mathbf{V}\right)$ 那样作用,则预期

$$
{\mathcal{D}}^{ \dagger }\left( R\right) {Y}_{l}^{m}\left( \mathbf{V}\right) \mathcal{D}\left( R\right) = \mathop{\sum }\limits_{{m}^{\prime }}{Y}_{l}^{{m}^{\prime }}\left( \mathbf{V}\right) {\mathcal{D}}_{m{m}^{\prime }}^{\left( l\right) }\left( R\right) tag{3.11.21}
$$

是合理的,这里用转动算符的幺正性重写了 ${\mathcal{D}}_{{m}^{\prime }m}^{\left( l\right) }\left( {R}^{-1}\right)$

所有这些工作就是为了诱导出球张量的定义. 现在考虑量子力学中的球张量. 受 (3.11.21) 式的启发,定义一个有 ${2k} + 1$ 个分量的 $k$ 秩球张量算符

$$
{\mathcal{D}}^{ + }\left( R\right) {T}_{q}^{\left( k\right) }\mathcal{D}\left( R\right) = \mathop{\sum }\limits_{{{q}^{\prime } = - k}}^{k}{\mathcal{D}}_{q{q}^{\prime }}^{\left( k\right) }{}^{ * }{T}_{{q}^{\prime }}^{\left( k\right) } tag{3.11.22a}
$$

或等价地

$$
\mathcal{D}\left( R\right) {T}_{q}^{\left( k\right) }{\mathcal{D}}^{ \dagger }\left( R\right) = \mathop{\sum }\limits_{{{q}^{\prime } = - k}}^{k}{\mathcal{D}}_{{q}^{\prime }q}^{\left( k\right) }\left( R\right) {T}_{q}^{\left( k\right) } tag{3.11.22b}
$$

不管 ${T}_{q}^{\left( k\right) }$ 是否可以写成 ${Y}_{l = k}^{m = q}\left( \mathbf{V}\right)$ ,这个定义都成立. 例如,尽管 $\left( {{U}_{x} + i{U}_{y}}\right) \left( {{V}_{x} + i{V}_{y}}\right)$ 是一个二秩球张量的 $q = + 2$ 分量,与 ${\left( {V}_{x} + i{V}_{y}\right) }^{2}$ 不同,它不能写成 ${Y}_{k}^{q}\left( \mathbf{V}\right)$ .

通过考虑 (3.11.22b) 式的无穷小形式, 可以得到球张量更为方便的定义, 即

$$
\left( {1 + \frac{i\mathbf{J} \cdot \widehat{\mathbf{n}}\varepsilon }{\hslash }}\right) {T}_{q}^{\left( k\right) }\left( {1 - \frac{i\mathbf{J} \cdot \widehat{\mathbf{n}}\varepsilon }{\hslash }}\right) = \mathop{\sum }\limits_{{{q}^{\prime } = - k}}^{k}{T}_{{q}^{\prime }}^{\left( k\right) }\left\langle {k{q}^{\prime }\left| \left( {1 + \frac{i\mathbf{J} \cdot \widehat{\mathbf{n}}\varepsilon }{\hslash }}\right) \right| {kq}}\right\rangle tag{3.11.23}
$$

或

$$
\left\lbrack {\mathbf{J} \cdot \widehat{\mathbf{n}},{T}_{q}^{\left( k\right) }}\right\rbrack = \mathop{\sum }\limits_{{q}^{\prime }}{T}_{q}^{\left( k\right) }\left\langle {k{q}^{\prime }\left| {\mathbf{J} \cdot \widehat{\mathbf{n}}}\right| {kq}}\right\rangle . tag{3.11.24}
$$

取 $\widehat{\mathbf{n}}$ 沿 $\widehat{\mathbf{z}}$ 和 ( $\widehat{\mathbf{x}} \pm i\widehat{\mathbf{y}}$ ) 方向,并且利用 ${J}_{z}$ 和 ${J}_{ \pm }$ 的非零矩阵元 [见 (3.5.35b) 和 (3.5.41) 式], 得到

$$
\left\lbrack {{J}_{z},{T}_{q}^{\left( k\right) }}\right\rbrack = \hslash q{T}_{q}^{\left( k\right) } tag{3.11.25a}
$$

和

$$
\left\lbrack {{J}_{ \pm },{T}_{q}^{\left( k\right) }}\right\rbrack = \hslash \sqrt{\left( {k \mp q}\right) \left( {k \pm q + 1}\right) }{T}_{q \pm 1}^{\left( k\right) }, tag{3.11.25b}
$$

代替 (3.11.22) 式, 这些对易关系可看作球张量的定义.

\subsection{张量积}

书中已经多次用到了笛卡尔张量的语言. 的确, 用它们构造了标量、矢量、反对称张量和无迹对称张量. 例如, 请见 (3.11.13) 式. 当然, 也可以利用球张量语言 (Baym 1969, 第 17 章), 例如

$$
{T}_{0}^{\left( 0\right) } = \frac{-\mathbf{U} \cdot \mathbf{V}}{3} = \frac{\left( {U}_{+1}{V}_{-1} + {U}_{-1}{V}_{+1} - {U}_{0}{V}_{0}\right) }{3},
$$

$$
{T}_{q}^{\left( 1\right) } = \frac{{\left( \mathbf{U} \times \mathbf{V}\right) }_{q}}{i\sqrt{2}},
$$

$$
{T}_{\pm 2}^{\left( 2\right) } = {U}_{\pm 1}{V}_{\pm 1}, tag{3.11.26}
$$

$$
{T}_{\pm 1}^{\left( 2\right) } = \frac{{U}_{\pm 1}{V}_{0} + {U}_{0}{V}_{\pm 1}}{\sqrt{2}},
$$

$$
{T}_{0}^{\left( 2\right) } = \frac{{U}_{+1}{V}_{-1} + 2{U}_{0}{V}_{0} + {U}_{-1}{V}_{+1}}{\sqrt{6}},
$$

其中 ${U}_{q}\left( {V}_{q}\right)$ 是一秩球张量的第 $q$ 个分量,对应着矢量 $\mathbf{U}\left( \mathbf{V}\right)$ . 通过与 ${Y}_{l}^{m}$ 比较并记住 ${U}_{+1}$ $= - \left( {{U}_{x} + i{U}_{y}}\right) /\sqrt{2},{U}_{-1} = \left( {{U}_{x} - i{U}_{y}}\right) /\sqrt{2},{U}_{0} = {U}_{z}$ ,可以检验前述的变换性质. 对 ${V}_{\pm {1.0}}$ 可做类似的检验. 例如

$$
{Y}_{2}^{0} = \sqrt{\frac{5}{16\pi }}\frac{3{z}^{2} - {r}^{2}}{{r}^{2}},
$$

其中 $3{z}^{2} - {r}^{2}$ 可以写成

$$
2{z}^{2} + 2\left\lbrack {-\frac{\left( x + iy\right) }{\sqrt{2}}\frac{\left( x - iy\right) }{\sqrt{2}}}\right\rbrack ,
$$

因此, ${Y}_{2}^{0}$ 正是 ${T}_{0}^{\left( 2\right) }$ 在 $\mathbf{U} = \mathbf{V} = \mathbf{r}$ 时的特例.

一个更为系统的构成张量积的方法如下所示. 从一个定理开始:

定理 3.1 令 ${X}_{{q}_{1}}^{\left( {k}_{1}\right) }$ 和 ${Z}_{{q}_{2}}^{\left( {k}_{2}\right) }$ 分别为 ${k}_{1}$ 秩和 ${k}_{2}$ 秩的不可约球张量. 则

$$
{T}_{q}^{\left( k\right) } = \mathop{\sum }\limits_{{q}_{1}}\mathop{\sum }\limits_{{q}_{2}}\left\langle {{k}_{1}{k}_{2};{q}_{1}{q}_{2} \mid {k}_{1}{k}_{2};{kq}}\right\rangle {X}_{{q}_{1}}^{\left( {k}_{1}\right) }{Z}_{{q}_{2}}^{\left( {k}_{2}\right) } tag{3.11.27}
$$

是一个 $k$ 秩球 (不可约) 张量.

证明 必须证明在转动下 ${T}_{q}^{\left( k\right) }$ 一定按照 (3.11.22) 式变换.

$$
{\mathcal{D}}^{ + }\left( R\right) {T}_{q}^{\left( k\right) }\mathcal{D}\left( R\right) = \mathop{\sum }\limits_{{q}_{1}}\mathop{\sum }\limits_{{q}_{2}}\left\langle {{k}_{1}{k}_{2};{q}_{1}{q}_{2} \mid {k}_{1}{k}_{2};{kq}}\right\rangle
$$

$$
\times {\mathcal{D}}^{ \dagger }\left( R\right) {X}_{{q}_{1}}^{\left( {k}_{1}\right) }\mathcal{D}\left( R\right) {\mathcal{D}}^{ \dagger }\left( R\right) {Z}_{{q}_{2}}^{\left( {k}_{2}\right) }\mathcal{D}\left( R\right)
$$

$$
= \mathop{\sum }\limits_{{q}_{1}}\mathop{\sum }\limits_{{q}_{2}}\mathop{\sum }\limits_{{q}_{1}^{\prime }}\mathop{\sum }\limits_{{q}_{2}^{\prime }}\left\langle {{k}_{1}{k}_{2};{q}_{1}{q}_{2} \mid {k}_{1}{k}_{2};{kq}}\right\rangle
$$

$$
\times {X}_{{q}_{1}^{\left( 1\right) }}^{\left( {k}_{1}\right) }{\mathcal{D}}_{{q}_{1}^{\left( 1\right) }{q}_{1}}^{\left( {k}_{1}\right) }\left( {R}^{-1}\right) {Z}_{{q}_{2}^{\left( 2\right) }}^{\left( {k}_{2}\right) }{\mathcal{D}}_{{q}_{2}^{\left( 2\right) }{q}_{2}}^{\left( {k}_{2}\right) }\left( {R}^{-1}\right)
$$

$$
= \mathop{\sum }\limits_{{k}^{\prime }}\mathop{\sum }\limits_{{q}_{1}}\mathop{\sum }\limits_{{q}_{2}}\mathop{\sum }\limits_{{q}_{1}^{\prime }}\mathop{\sum }\limits_{{q}_{2}^{\prime }}\mathop{\sum }\limits_{{q}^{\prime }}\mathop{\sum }\limits_{{q}^{\prime }}\mathop{\sum }\limits_{{q}^{\prime }}\left\langle {{k}_{1}{k}_{2};{q}_{1}{q}_{2} \mid {k}_{1}{k}_{2};{kq}}\right\rangle
$$

$$
\times \left\langle {{k}_{1}{k}_{2};{q}_{1}^{\prime }{q}_{2}^{\prime } \mid {k}_{1}{k}_{2};{k}^{\prime \prime }{q}^{\prime }}\right\rangle
$$

$$
\times \left\langle {{k}_{1}{k}_{2};{q}_{1}{q}_{2} \mid {k}_{1}{k}_{2};{k}^{\prime \prime }{q}^{\prime \prime }}\right\rangle {\mathcal{D}}_{{q}^{\prime }{q}^{\prime \prime }}^{\left( {k}^{\prime \prime }\right) }\left( {R}^{-1}\right) {X}_{{q}_{1}^{\prime }}^{\left( {k}_{1}\right) }{Z}_{{q}_{2}^{\prime \prime }}^{\left( {k}_{2}\right) },
$$

其中用到了克莱布什-戈丹系数公式 (3.8.69) 式. 上述表达式变成

$$
= \mathop{\sum }\limits_{{k}^{\prime }}\mathop{\sum }\limits_{{q}_{1}^{\prime }}\mathop{\sum }\limits_{{q}_{2}^{\prime }}\mathop{\sum }\limits_{{q}^{\prime \prime }}\mathop{\sum }\limits_{{q}^{\prime }}{\delta }_{k{k}^{\prime \prime }}{\delta }_{q{q}^{\prime \prime }}\left\langle {{k}_{1}{k}_{2};{q}_{1}^{\prime }{q}_{2}^{\prime } \mid {k}_{1}{k}_{2};{k}^{\prime \prime }{q}^{\prime }}\right\rangle {\mathcal{D}}_{{q}^{\prime }{q}^{\prime \prime }}^{\left( {k}^{\prime \prime }\right) }\left( {R}^{-1}\right) {X}_{{q}_{1}}^{\left( {k}_{1}\right) }{Z}_{{q}_{2}}^{\left( {k}_{2}\right) }
$$

其中用到了克莱布什-戈丹系数的正交性 (3.8.42) 式. 最后, 这个表达式约化为

$$
= \mathop{\sum }\limits_{{q}^{\prime }}\left( {\mathop{\sum }\limits_{{q}_{1}^{\prime }}\mathop{\sum }\limits_{{q}_{2}^{\prime }}\left\langle {{k}_{1}{k}_{2};{q}_{1}^{\prime }{q}_{2}^{\prime } \mid {k}_{1}{k}_{2};k{q}^{\prime }}\right\rangle {X}_{{q}_{1}^{\prime }}^{\left( {k}_{1}\right) }{Z}_{{q}_{2}}^{\left( {k}_{2}\right) }}\right) {D}_{{q}^{\prime }q}^{\left( k\right) }\left( {R}^{-1}\right)
$$

$$
= \mathop{\sum }\limits_{{q}^{\prime }}{T}_{q}^{\left( k\right) }{D}_{{q}^{\prime }q}^{\left( k\right) }\left( {R}^{-1}\right) = \mathop{\sum }\limits_{{q}^{\prime }}{D}_{q{q}^{\prime }}^{\left( k\right) }\left( R\right) {T}_{q}^{\left( k\right) }.
$$

上述表明, 可以通过把两个张量算符相乘构造更高或更低秩的张量算符. 此外, 用两个张量构造张量积的方式完全类似于把两个角动量相加构造一个角动量本征态的方式, 如果令 ${k}_{1,2} \rightarrow {j}_{1,2}$ 和 ${q}_{1,2} \rightarrow {m}_{1,2}$ ,则出现完全相同的克莱布什-戈丹系数.

\subsection{张量算符的矩阵元及维格纳-埃卡特定理}

在考虑电磁场与原子和原子核相互作用时, 经常需要计算张量算符对于角动量本征态的矩阵元. 一些这类的例子将在第 5 章给出. 一般来说, 计算这样的矩阵元是一件艰巨的动力学任务. 然而, 这些矩阵元具有一些纯粹来自运动学或几何学考虑的性质, 现在来讨论它们.

首先,有一个非常简单的 $m$ 选择定则:

$$
\left\langle {{\alpha }^{\prime },{j}^{\prime }{m}^{\prime }\left| {T}_{q}^{\left( k\right) }\right| \alpha ,{jm}}\right\rangle = 0,\;\text{ 除非 }{m}^{\prime } = q + m. tag{3.11.28}
$$

证明 利用 (3.11.25a) 式, 有

$$
\left\langle {{\alpha }^{\prime },{j}^{\prime }{m}^{\prime }\left| \left( {\left\lbrack {{J}_{z},{T}_{q}^{\left( k\right) }}\right\rbrack - \hslash q{T}_{q}^{\left( k\right) }}\right) \right| \alpha ,{jm}}\right\rangle = \left\lbrack {\left( {{m}^{\prime } - m}\right) \hslash - \hslash q}\right\rbrack
$$

$$
\times \left\langle {{\alpha }^{\prime },{j}^{\prime }{m}^{\prime }\left| {T}_{q}^{\left( k\right) }\right| \alpha ,{jm}}\right\rangle = 0,
$$

因此

$$
\left\langle {{\alpha }^{\prime },{j}^{\prime }{m}^{\prime }\left| {T}_{q}^{\left( k\right) }\right| \alpha ,{jm}}\right\rangle = 0,\;\text{ 除非 }{m}^{\prime } = q + m.
$$

另一种看到这种性质的方法是注意 ${T}_{q}^{\left( k\right) } \mid \alpha ,{jm}\rangle$ 在转动下的变换性质,即(3.11.29)


如果现在令 $\mathbf{D}$ 为绕 $z$ 轴的一个转动算符,得到 [请见 (3.11.22b) 式和 (3.1.16) 式]

$$
\mathcal{D}\left( {\widehat{\mathbf{z}},\phi }\right) {T}_{q}^{\left( k\right) }\left| {\alpha ,{jm}\rangle = {e}^{-{iq\phi }}{e}^{-{im\phi }}{T}_{q}^{\left( k\right) }}\right| \alpha ,{jm}\rangle , tag{3.11.30}
$$

除非 $q + m = {m}^{\prime }$ ,否则它与 $\left| {{\alpha }^{\prime },{j}^{\prime }{m}^{\prime }}\right\rangle$ 正交.

下面将证明量子力学中的最重要的定理之一, 维格纳-埃卡特定理.

定理 3.2 维格纳-埃卡特定理. 张量算符对于角动量本征态的矩阵元满足

$$
\left\langle {{\alpha }^{\prime },{j}^{\prime }{m}^{\prime }\left| {T}_{q}^{\left( k\right) }\right| \alpha ,{jm}}\right\rangle = \left\langle {{jk};{mq} \mid {jk};{j}^{\prime }{m}^{\prime }}\right\rangle \frac{\left\langle {\alpha }^{\prime }{j}^{\prime }\begin{Vmatrix}{T}^{\left( k\right) }\end{Vmatrix}\alpha j\right\rangle }{\sqrt{{2j} + 1}}, tag{3.11.31}
$$

其中的双线矩阵元不依赖于 $m,{m}^{\prime }$ 和 $q$ .

在阐述该定理的证明之前, 先看一下它的意义. 首先, 可以看到这个矩阵元被写成两个因子之积. 第一个因子是一个关于 $j$ 和 $k$ 相加得到 ${j}^{\prime }$ 的克莱布什-戈丹系数. 它只依赖于几何学一即依赖系统相对于 $z$ 轴的取向方式. 不涉及张量算符任何特殊的性质. 第二个因子确实依赖于动力学,例如, $\alpha$ 可以表示径向量子数,因而它的计算可能包含求径向积分. 另一方面,它完全不依赖于确定该物理系统取向的磁量子数 $m,{m}^{\prime }$ 和 $q$ . 要计算具有 $m,{m}^{\prime }$ 和 ${q}^{\prime }$ 各种组合的 $\left\langle {{\alpha }^{\prime },{j}^{\prime }{m}^{\prime }\left| {T}_{q}^{\left( k\right) }\right| \alpha ,{jm}}\right\rangle$ ,只要知道它们中的一个就够了,所有其他的矩阵元都能用几何学关联起来, 因为它们都正比于克莱布什-戈丹系数, 后者是已知的. 共同的比例因子是 $\left\langle {{\alpha }^{\prime }{j}^{\prime }\left| \right| {T}^{\left( k\right) }\left| \right| {\alpha j}}\right\rangle$ ,它不涉及任何几何特点.

张量算符矩阵元的选择定则可以马上从角动量相加的选择定则中读取出来. 的确, 从克莱布什-戈丹系数非零的要求,立即得到前面导出的 $m$ 选择定则 (3.11.28) 式,以及三角形关系

$$
\left| {j - k}\right| \leq {j}^{\prime } \leq j + k tag{3.11.32}
$$

下面证明这个定理.

证明 利用 (3.11.25b) 式, 有

$$
\left\langle {{\alpha }^{\prime },{j}^{\prime }{m}^{\prime }\left| \left\lbrack {{J}_{ \pm },{T}_{q}^{\left( k\right) }}\right\rbrack \right| \alpha ,{jm}}\right\rangle = \hslash \sqrt{\left( {k \mp q}\right) \left( {k \pm q + 1}\right) } < {\alpha }^{\prime },{j}^{\prime }{m}^{\prime }\left| {T}_{q \pm 1}^{\left( k\right) }\right| \alpha ,{jm}\rangle , tag{3.11.33}
$$

或者利用 (3.5.39) 式和 (3.5.40) 式, 有

$$
\sqrt{\left( {{j}^{\prime } \pm {m}^{\prime }}\right) \left( {{j}^{\prime } \mp {m}^{\prime } + 1}\right) }\left\langle {{\alpha }^{\prime },{j}^{\prime },{m}^{\prime } \mp 1\left| {T}_{q}^{\left( k\right) }\right| \alpha ,{jm}}\right\rangle
$$

$$
= \sqrt{\left( {j \mp m}\right) \left( {j \pm m + 1}\right) }\left\langle {{\alpha }^{\prime },{j}^{\prime }{m}^{\prime }\left| {T}_{q}^{\left( k\right) }\right| \alpha, j, m \pm 1}\right\rangle
$$

$$
+ \sqrt{\left( {k \mp q}\right) \left( {k \pm q + 1}\right) }\left\langle {{\alpha }^{\prime },{j}^{\prime }{m}^{\prime }\left| {T}_{q \pm 1}^{\left( k\right) }\right| \alpha ,{jm}}\right\rangle . tag{3.11.34}
$$

把该式与克莱布什-戈丹系数的递推关系 (3.8.49) 式比较. 注意如果做代换 ${j}^{\prime } \rightarrow j,{m}^{\prime } \rightarrow$

$m, j \rightarrow {j}_{1}, m \rightarrow {m}_{1}, k \rightarrow {j}_{2}$ 和 $q \rightarrow {m}_{2}$ ,则有惊人的相似性. 这两个递推关系都具有 $\mathop{\sum }\limits_{j}{a}_{ij}{x}_{j} = 0$ 形式,即它们都是一阶齐次线性方程组,而且有相同的系数 ${a}_{ij}$ . 每当

$$
\mathop{\sum }\limits_{j}{a}_{ij}{x}_{j} = 0,\;\mathop{\sum }\limits_{j}{a}_{ij}{y}_{j} = 0, tag{3.11.35}
$$

不能把 ${x}_{j}$ (或 ${y}_{j}$ ) 的解分别都求出来,但可以求解它们的比,所以

$$
\frac{{x}_{j}}{{x}_{k}} = \frac{{y}_{j}}{{y}_{k}}\;\text{ 或 }{x}_{j} = c{y}_{j}. tag{3.11.36}
$$

其中 $c$ 是一个普适的比例因子. 注意到克莱布什-戈丹系数的递推关系 (3.8.49) 式中的 $\left\langle {{j}_{1}{j}_{2};{m}_{1},{m}_{2} \pm 1 \mid {j}_{1}{j}_{2};{jm}}\right\rangle$ 对应于 $\left\langle {{\alpha }^{\prime },{j}^{\prime }{m}^{\prime }\left| {T}_{q \pm 1}^{\left( k\right) }\right| \alpha ,{jm}}\right\rangle$ ,得到

$\left\langle {{\alpha }^{\prime },{j}^{\prime }{m}^{\prime }\left| {T}_{q + \pm 1}^{\left( k\right) }\right| \alpha ,{jm}}\right\rangle = \left( {\text{ 不依赖于 }m, q\text{ 和 }{m}^{\prime }\text{ 的普适比例常数 }}\right)$

$$
\left\langle {{jk};{mq} \pm 1 \mid {jk};{j}^{\prime }{m}^{\prime }}\right\rangle , tag{3.11.37}
$$

于是证明了该定理.

现在看一下维格纳-埃卡特定理的两个简单例子.

例 3.5 0 秩张量,即标量 ${T}_{0}^{\left( 0\right) } = S$ . 一个标量算符的矩阵元满足

$$
\left\langle {{\alpha }^{\prime },{j}^{\prime }{m}^{\prime }\left| S\right| \alpha ,{jm}}\right\rangle = {\delta }_{j{j}^{\prime }}{\delta }_{m{m}^{\prime }}\frac{\left\langle {\alpha }^{\prime }{j}^{\prime }\parallel S\parallel \alpha j\right\rangle }{\sqrt{{2j} + 1}} tag{3.11.38}
$$

因为 $S$ 作用于 $|\alpha ,{jm}\rangle$ 就像加上一个零角动量. 于是,标量算符不可能改变 $j, m$ 的值.

例 3.6 利用球张量语言,矢量算符是一个一秩张量. $\mathbf{V}$ 的球分量可以写成 ${V}_{q = \pm {1.0}}$ ,所以有选择定则

$$
{\Delta m} \equiv {m}^{\prime } - m = \pm 1,0\;{\Delta j} \equiv {j}^{\prime } - j = \left\{ \begin{array}{l} \pm 1 \\ 0 \end{array}\right. tag{3.11.39}
$$

此外, $0 \rightarrow 0$ 的跃迁是禁戒的. 这个选择定则在辐射理论中是非常重要的. 它是在发射光子的长波极限下求得的偶极选择定则.

$j = {j}^{\prime }$ 时的维格纳-埃卡特定理一一当将其用于矢量算符时一一有特别简单的形式,由于一些明显的理由, 它经常被称之为投影定理.

定理 3.3 投影定理

$$
\left\langle {{\alpha }^{\prime }, j{m}^{\prime }\left| {V}_{q}\right| \alpha ,{jm}}\right\rangle = \frac{\left\langle {\alpha }^{\prime }, jm\left| \mathbf{J} \cdot \mathbf{V}\right| \alpha, jm\right\rangle }{{\hslash }^{2}j\left( {j + 1}\right) }\left\langle {j{m}^{\prime }\left| {J}_{q}\right| {jm}}\right\rangle , tag{3.11.40}
$$

其中, 类似在 (3.11.26) 式之后的讨论, 选择

$$
{J}_{\pm 1} = \mp \frac{1}{\sqrt{2}}\left( {{J}_{x} \pm i{J}_{y}}\right) = \mp \frac{1}{\sqrt{2}}{J}_{ \pm },\;{J}_{0} = {J}_{z}. tag{3.11.41}
$$

证明 注意 (3.11.26) 式, 使用维格纳-埃卡特定理 (3.11.31) 式, 有

$$
\left\langle {{\alpha }^{\prime },{jm}\left| {\mathbf{J} \cdot \mathbf{V}}\right| \alpha ,{jm}}\right\rangle = \left\langle {{\alpha }^{\prime },{jm}\left| \left( {{J}_{0}{V}_{0} - {J}_{+1}{V}_{-1} - {J}_{-1}{V}_{+1}}\right) \right| \alpha ,{jm}}\right\rangle
$$

$$
= {mh}\left\langle {{\alpha }^{\prime },{jm}\left| {V}_{0}\right| \alpha ,{jm}}\right\rangle + \frac{\hslash }{\sqrt{2}}\sqrt{\left( {j + m}\right) \left( {j - m + 1}\right) }
$$

$$
\times \left\langle {{\alpha }^{\prime },{jm} - 1\left| {V}_{-1}\right| \alpha ,{jm}}\right\rangle tag{3.11.42}
$$

$$
- \frac{\hslash }{\sqrt{2}}\sqrt{\left( {j - m}\right) \left( {j + m + 1}\right) }\left\langle {{\alpha }^{\prime },{jm} + 1\left| {V}_{+1}\right| \alpha ,{jm}}\right\rangle
$$

,

$$
= {c}_{jm}\left\langle {{\alpha }^{\prime }j\parallel \mathbf{v}\parallel {\alpha j}}\right\rangle
$$

其中 ${c}_{jm}$ 不依赖于 $\alpha ,{\alpha }^{\prime }$ 和 $\mathbf{V}$ ,并且 ${V}_{0, \pm 1}$ 的矩阵元都正比于双线矩阵元(有时也称为约化矩阵元). 此外, ${c}_{jm}$ 不依赖于 $m$ ,因为 $\mathbf{J} \cdot \mathbf{V}$ 是一个标量算符,所以也可以把它写做 ${c}_{j}$ . 因为 ${c}_{j}$ 不依赖于 $\mathbf{V}$ ,即使令 $\mathbf{V} \rightarrow \mathbf{J}$ 和 ${\alpha }^{\prime } \rightarrow \alpha ,\left( {3.11.42}\right)$ 式仍成立,即

$$
\left\langle {\alpha ,{jm}\left| {\mathbf{J}}^{2}\right| \alpha ,{jm}}\right\rangle = {c}_{j}\langle {\alpha j}\parallel \mathbf{J}\parallel {\alpha j}\rangle . tag{3.11.43}
$$

返回到应用于 ${V}_{q}$ 和 ${J}_{q}$ 的维格纳-埃卡特定理,有

$$
\frac{\left\langle {\alpha }^{\prime }, j{m}^{\prime }\left| {V}_{q}\right| \alpha, jm\right\rangle }{\left\langle \alpha, j{m}^{\prime }\left| {J}_{q}\right| \alpha, jm\right\rangle } = \frac{\left\langle {\alpha }^{\prime }j\parallel \mathbf{V}\parallel \alpha j\right\rangle }{\langle {\alpha j}\parallel \mathbf{J}\parallel {\alpha j}\rangle } tag{3.11.44}
$$

但是我们可以利用 (3.11.42) 式和 (3.11.43) 式把 (3.11.44) 式的右边写成 $\left\langle {{\alpha }^{\prime },{jm}}\right| \mathbf{J}$ - $\mathbf{V}\left| {\alpha ,{jm}}\right\rangle /\left\langle {\alpha ,{jm}{\left| {\mathbf{J}}^{2}\right| }_{\alpha },{jm}}\right\rangle$ . 此外,(3.11.43) 式的左边正是 $j\left( {j + 1}\right) {\hslash }^{2}$ . 所以

$$
\left\langle {{\alpha }^{\prime }, j{m}^{\prime }\left| {V}_{q}\right| \alpha ,{jm}}\right\rangle = \frac{\left\langle {\alpha }^{\prime }, jm\left| \mathbf{J} \cdot \mathbf{V}\right| \alpha, jm\right\rangle }{{\hslash }^{2}j\left( {j + 1}\right) }\left\langle {j{m}^{\prime }\left| {J}_{q}\right| {jm}}\right\rangle , tag{3.11.45}
$$

投影定理得证.

习题

$$
H = \frac{1}{2}\left( {\frac{{K}_{1}^{2}}{{I}_{1}} + \frac{{K}_{2}^{2}}{{I}_{2}} + \frac{{K}_{3}^{2}}{{I}_{3}}}\right) ,
$$

---

随后的章节将给出这个定理的一些应用.

3.1 求 ${\sigma }_{y} = \left( \begin{matrix} 0 & - i \\ i & 0 \end{matrix}\right)$ 的本征值和本征矢. 假定一个电子处在自旋态 $\left( \begin{array}{l} \alpha \\ \beta \end{array}\right)$ 上. 如果测得 ${s}_{y}$ ,结果为 $\hslash /2$ 的概率是什么?

3.2 对于一个自旋 $\frac{1}{2}$ 的粒子,在存在一个磁场 $\mathbf{B} = {B}_{x}\widehat{\mathbf{x}} + {B}_{y}\widehat{\mathbf{y}} + {B}_{z}\widehat{\mathbf{z}}$ 情况下,通过利用泡利矩阵显式结构, 求哈密顿量

$$
H = - \frac{2\mu }{\hslash }\mathbf{S} \cdot \mathbf{B}
$$

的本征值.

3.3 考虑由

$$
U = \frac{{a}_{0} + i\mathbf{\sigma } \cdot \mathbf{a}}{{a}_{0} - i\mathbf{\sigma } \cdot \mathbf{a}}.
$$

定义的 $2 \times 2$ 矩阵,其中 ${a}_{0}$ 是一个实数,而 $\mathbf{a}$ 是一个有着实分量的三维矢量.

(a) 证明 $U$ 是幺正的和么模的.

(b) 一般地说,一个 $2 \times 2$ 幺正么模矩阵表示一个三维转动. 借助于 ${a}_{0},{a}_{1},{a}_{2}$ 和 ${a}_{3}$ ,求适合于 $U$ 的转动轴和转角.

3.4 在存在一个沿 $z$ 方向的均匀磁场情况下,一个电子-正电子系统的时间相关哈密顿量可以写成

$$
H = A{\mathbf{S}}^{\left( {r}^{ - }\right) } \cdot {\mathbf{S}}^{\left( {r}^{ + }\right) } + \left( \frac{eB}{mc}\right) \left( {{S}_{z}^{\left( {r}^{ - }\right) } - {S}_{z}^{\left( {r}^{ + }\right) }}\right) ,
$$

假定系统的自旋函数由 ${\chi }_{ + }^{\left( {e}^{ - }\right) }{\chi }_{ - }^{\left( {e}^{ + }\right) }$ 给出.

(a) 在 $A \rightarrow 0,{eB}/{mc} \neq 0$ 的极限下,它是 $H$ 的本征函数吗? 如果它是,能量本征值是什么? 如果它不是, $H$ 的期待值是什么?

(b) 当 ${eB}/{mc} \rightarrow 0, A \neq 0$ 时,求解同样的问题.

3.5 考虑一个自旋为 1 的粒子. 求

$$
{S}_{z}\left( {{S}_{z} + h}\right) \left( {{S}_{z} - h}\right) \;\text{ 和 }{S}_{x}\left( {{S}_{x} + h}\right) \left( {{S}_{x} - h}\right)
$$

的矩阵元.

3.6 设一个刚体的哈密顿量为

---

其中 $\mathbf{K}$ 是本体坐标架中的角动量. 从这个表示式求 $\mathbf{K}$ 的海森伯运动方程,然后在相应的极限下求欧拉运动方程.

3.7 令 $U = {e}^{i{\theta }_{3}a}{e}^{i{\theta }_{2}\beta }{e}^{i{\theta }_{3}\gamma }$ ,其中 $\left( {\alpha ,\beta ,\gamma }\right)$ 是欧拉角. 为了使 $U$ 表示一个转动 $\left( {\alpha ,\beta ,\gamma }\right) ,{G}_{k}$ 必须满足的对易关系是什么? 把 $\mathbf{G}$ 与角动量算符联系起来.

3.8 下列方程式的意义是什么?

$$
{U}^{-1}{A}_{k}U = \sum {R}_{kl}{A}_{l},
$$

其中. $\mathbf{A}$ 的三个分量都是矩阵. 由这个方程式证明. 矩阵元 $\left\langle {m\left| {A}_{k}\right| n}\right\rangle$ 像一个矢量一样变换.

3.9 考虑一个由下式表示的欧拉转动序列

$$
{\mathcal{D}}^{\left( 12\right) }\left( {\alpha ,\beta ,\gamma }\right) = \exp \left( \frac{-i{\sigma }_{3}\alpha }{2}\right) \exp \left( \frac{-i{\sigma }_{2}\beta }{2}\right) \exp \left( \frac{-i{\sigma }_{3}\gamma }{2}\right)
$$

$$
= \left( \begin{matrix} {e}^{-i\left( {\alpha + \gamma }\right) /2}\cos \frac{\beta }{2} & - {e}^{-i\left( {\alpha - \gamma }\right) /2}\sin \frac{\beta }{2} \\ {e}^{i\left( {\alpha - \gamma }\right) /2}\sin \frac{\beta }{2} & {e}^{i\left( {\alpha + \gamma }\right) /2}\cos \frac{\beta }{2} \end{matrix}\right) .
$$

由于转动的群性质,预期这一序列操作等价于绕某个轴转一个 $\theta$ 角的单一转动. 求 $\theta$ .

3. 10 (a) 考虑全同制备的自旋 $\frac{1}{2}$ 系统的一个纯系综. 假定期待值 $\left\langle {S}_{x}\right\rangle$ 和 $\left\langle {S}_{z}\right\rangle$ 已知,而 $\left\langle {S}_{y}\right\rangle$ 的符号也已知. 证明如何确定态矢量. 为什么不必知道 $\left\langle {S}_{y}\right\rangle$ 的大小?

(b) 考虑一个自旋 $\frac{1}{2}$ 系统的混合系综. 假定系综平均值 $\left\lbrack {S}_{x}\right\rbrack ,\left\lbrack {S}_{y}\right\rbrack$ 和 $\left\lbrack {S}_{z}\right\rbrack$ 都是已知的. 证明如何可以构造表征这个系综的 $2 \times 2$ 密度矩阵.

3.11 (a) 证明密度算符 $\rho$ (在薛定谔绘景中) 的时间演化由下式给定

$$
\rho \left( t\right) = u\left( {t,{t}_{0}}\right) \rho \left( {t}_{0}\right) {u}^{ + }\left( {t,{t}_{0}}\right) .
$$

(b) 假定在 $t = 0$ 时有一个纯系综. 证明只要时间演化由薛定谔方程控制,则它不可能演化成一个混合系综.

3.12 考虑自旋为 1 系统的一个系综. 密度矩阵现在是一个 $3 \times 3$ 矩阵. 为了表征这个密度矩阵,需要多少独立的 (实) 参量? 除了 $\left\lbrack {S}_{r}\right\rbrack ,\left\lbrack {S}_{v}\right\rbrack$ 和 $\left\lbrack {S}_{z}\right\rbrack$ . 为了完全表征这个系综还必须知道什么?

3.13 一个角动量本征态 $\left| {j, m = {m}_{\text{最大 }} = j}\right\rangle$ 绕 $y$ 轴转一个无穷小角度 $\varepsilon$ . 不使用 ${d}_{mim}^{\left( t\right) }$ 函数的显式表达式, 求在其原来态上发现这个新的转动后的态的概率表示式,直到 ${\varepsilon }^{2}$ 量级项.

3.14 已知 $3 \times 3$ 矩阵 ${G}_{i}\left( {i = 1,2,3}\right)$ ,其矩阵元由下式给出:

$$
{\left( {G}_{i}\right) }_{jk} = - i\hslash {\varepsilon }_{ijk},
$$

其中 $j$ 和 $k$ 是行和列指标,证明它满足角动量对易关系. 把 ${G}_{i}$ 与比较常用的角动量算符 ${J}_{i}$ ,在 ${J}_{3}$ 取为对角的情况下的 $3 \times 3$ 表示联系起来,实现该联系的变换矩阵的物理 (或几何) 意义是什么? 把得到的结果与无穷小转动下的

$$
\mathbf{V} \rightarrow \mathbf{V} + \widehat{\mathbf{n}}{\delta \phi } \times \mathbf{V}
$$

联系起来. (注: 这个问题可能有助于理解光子的自旋.)

3.15 (a) 令 $\mathbf{J}$ 是角动量. (它可以是轨道角动量 $\mathbf{L}$ ,自旋 $\mathbf{S}$ ,或 ${\mathbf{J}}_{\text{①}}$ .) 利用 ${J}_{t},{J}_{y},{J}_{z}\left( {{J}_{ \pm } \equiv {J}_{t} \pm i{J}_{y}}\right)$ 满足通常角动量对易关系的事实, 证明

$$
{\mathbf{J}}^{2} = {J}_{z}^{2} + {J}_{ + }{J}_{ - } - \hslash {J}_{z}.
$$

(b) 利用 (a) (或其他方式) 推导出在

$$
{J}_{ - }{\psi }_{jm} = {c}_{ - }{\psi }_{j, m - 1}
$$

中的系数 ${c}_{ - }$ 的 “著名” 的表示式.

3.16 证明轨道角动量算符 $\mathrm{L}$ 与算符 ${\mathbf{p}}^{2}$ 和 ${\mathbf{x}}^{2}$ 都对易,即证明 (3.7.2) 式.

3.17 一个受到球对称势 $V\left( r\right)$ 作用的粒子的波函数由下式给出:

$$
\psi \left( \mathbf{x}\right) = \left( {x + y + {3z}}\right) f\left( r\right) .
$$

(a) $\psi$ 是 ${\mathbf{L}}^{2}$ 的一个本征函数吗? 如果是的话, $l$ 值是什么? 如果不是,当测量 ${\mathbf{L}}^{2}$ 时能得到什么样可能的 $l$ 值?

(b) 在各种 ${m}_{t}$ 态上找到该粒子的概率是什么?

(c) 假定以某种方式知道 $\psi \left( \mathbf{x}\right)$ 是一个能量本征函数,本征值为 $E$ . 指出怎样可以找到 $V\left( r\right)$ .

3.18 已知一个在球对称势中的粒子处在 ${\mathbf{L}}^{2}$ 和 ${L}_{z}$ 的本征态,本征值分别为 $l\left( {l + 1}\right) {\hslash }^{2}$ 和 $m\hslash$ . 证明在 $|{lm}\rangle$ 态之间的期待值满足

$$
\left\langle {L}_{x}\right\rangle = \left\langle {L}_{y}\right\rangle = 0,\;\left\langle {L}_{x}^{2}\right\rangle = \left\langle {L}_{y}^{2}\right\rangle = \frac{\left\lbrack l\left( l + 1\right) {\hslash }^{2} - {m}^{2}{\hslash }^{2}\right\rbrack }{2}.
$$

半经典地解释这个结果.

3.19 假定轨道角动量允许有一个半整数的 $l$ 值,比如 $\frac{1}{2}$ . 通常,从

$$
{L}_{ + }{Y}_{1,2,1,2}\left( {\theta ,\phi }\right) = 0,
$$

可以导出

$$
{Y}_{1/2,1/2}\left( {\theta ,\phi }\right) \propto {e}^{{i\phi }/2}\sqrt{\sin \theta }.
$$

现在尝试 (a) 用 ${L}_{ - }$ 作用于 ${Y}_{1/2,1/2}\left( {\theta ,\phi }\right)$ ; (b) 用 ${L}_{ - }{Y}_{1/2, - 1/2}\left( {\theta ,\phi }\right) = 0$ ,构造 ${Y}_{1/2, - 1/2}\left( {\theta ,\phi }\right)$ . 证明这两种做法会导致矛盾的结果. (这给出了轨道角动量不能为一个半奇数 $l$ 值的论据.)

3.20 考虑一个轨道角动量的本征态 $|l = 2, m = 0\rangle$ . 假定这个态绕 $y$ 轴转了 $\beta$ 角. 求在 $m = 0, \pm 1$ 和 $\pm 2$ 的态上找到这个新态的概率. (在附录 $\mathrm{B}$ 的 $\mathrm{B}{.5}$ 节中给出的 $l = 0,1$ 和 2 的球谐函数可能是有用的.)

3.21 本题的目的是借助于笛卡尔本征态 $\left| {n,{n}_{y}{n}_{z}}\right\rangle$ . 确定写成 ${\mathbf{L}}^{2}$ 和 ${L}_{z}$ 本征态的三维各向同性谐振子的简并本征态.

(a) 证明角动量算符由下式给出:

$$
{L}_{i} = {ih}{\varepsilon }_{ijk}{a}_{j}{a}_{k}^{ + }
$$

$$
{\mathbf{L}}^{2} = {\hslash }^{2}\left\lbrack {N\left( {N + 1}\right) - {a}_{k}^{ \dagger }{a}_{k}^{ \dagger }{a}_{j}{a}_{j}}\right\rbrack ,
$$

其中隐含了对于重复指标的求和, ${\varepsilon }_{ijk}$ 是全反对称符号,而 $N \equiv {a}_{i}^{ \dagger }{a}_{j}$ 计数了总的量子数日.

(b) 使用这些关系式,把态 $\left| {{qlm}\rangle = }\right| {01m}\rangle, m = 0, \pm 1$ ,借助于能量简并的三个本征态 $\left| {{n}_{x}{n}_{y}{n}_{z}}\right\rangle$ 表示出来. 在坐标空间中表述你的答案, 并且检查角度和径向依赖关系都是正确的.

(c) 对 $\left| {{qlm}\rangle = }\right| {200}\rangle$ 重复以上两问.

(d) 对 $\left| {{qlm}\rangle = }\right| {02m}\rangle$ ,在 $m = 0,1$ 和 2 的情况下,重复 (a) 和 (b) 两问.

3.22 遵照下列步骤,证明库默尔方程 (3.7.46) 可以用拉盖尔多项式 ${L}_{n}\left( x\right)$ 写出来,后者按照母函数定义为

$$
g\left( {x, t}\right) = \frac{{e}^{-x/\left( {1 - t}\right) }}{1 - t} = \mathop{\sum }\limits_{{n = 0}}^{\infty }{L}_{n}\left( x\right) \frac{{t}^{n}}{n!}
$$

其中 $0 < t < 1.{2.5}$ 节中关于厄米多项式的母函数的讨论将会很有帮助.

(a) 证明 ${L}_{n}\left( 0\right) = n$ ! 和 ${L}_{0}\left( x\right) = 1$ .

(b) 把 $g\left( {x, t}\right)$ 对 $x$ 求微商,证明

$$
{L}^{\prime }{}_{n}\left( x\right) - n{L}^{\prime }{}_{n - 1}\left( x\right) = - n{L}_{n - 1}\left( x\right) ,
$$

并求出前几个拉盖尔多项式.

(c) 把 $g\left( {x, t}\right)$ 对 $t$ 求微商,证明

$$
{L}_{n + 1}\left( x\right) - \left( {{2n} + 1 - x}\right) {L}_{n}\left( x\right) + {n}^{2}{L}_{n - 1}\left( x\right) = 0.
$$

(d) 现在, 证明库默尔方程可以通过推导

$$
x{L}^{\prime \prime }{}_{n}\left( x\right) + \left( {1 - x}\right) {L}^{\prime }{}_{n}\left( x\right) + n{L}_{n}\left( x\right) = 0,
$$

而求解,而且把 $n$ 与氢原子的主量子数联系起来.

(e) 定义缔合拉盖尔多项式为

$$
{L}_{n}^{k}\left( x\right) = {\left( -1\right) }^{k}\frac{{d}^{k}}{d{x}^{k}}\left\lbrack {{L}_{n + k}\left( x\right) }\right\rbrack
$$

证明. 它满足下列方程

$$
x{L}_{n}^{{k}^{ * }}\left( x\right) + \left( {k + 1 - x}\right) {L}_{n}^{{k}^{ * }}\left( x\right) + n{L}_{n}^{k}\left( x\right) = 0
$$

事实上, 可以证明这个方程就是库默尔方程. 这就是说. 氢原子的径向波函数是用缔合拉盖尔多项式表示的. [ (e) 小题是作者在勘误表中要求加上的. 一译者注]

3.23 在角动量的施温格方案中, 算符

$$
{K}_{ + } \equiv {a}_{ + }^{ \dagger }{a}_{ - }^{ \dagger }\;\text{ 和 }\;{K}_{ - } \equiv {a}_{ + }{a}_{ - }
$$

的物理意义是什么? 给出 ${K}_{ \pm }$ 的非零矩阵元.

3.24 通过把 ${j}_{1} = 1$ 和 ${j}_{2} = 1$ 相加,求出所形成的 $j = 2,1,0$ 的所有 9 个 $|j, m\rangle$ 态. 利用简化符号,以 $\pm$ , 0 分别代表 ${m}_{1,2} = \pm 1,0$ ,写出 $|j, m\rangle$ 的显式式,例如

$$
\left| {1,1\rangle = \frac{1}{\sqrt{2}}}\right| + 0\rangle - \frac{1}{\sqrt{2}}|0 + \rangle
$$

可以利用阶梯算符 ${J}_{ \pm }$ ,或递推关系以及正交性. 找一个克莱布什-戈丹系数表用来做比较,检验你的结果.

3.25 (a) 对于任意的 $j$ (整数或半奇数),求

$$
\mathop{\sum }\limits_{{m = - j}}^{j}{\left| {d}_{m{m}^{\prime }}^{\left( j\right) }\left( \beta \right) \right| }^{2}m.
$$

并核对 $j = \frac{1}{2}$ 时所得的答案.

(b) 证明,对于任意的 $j$

$$
\mathop{\sum }\limits_{{m = - j}}^{j}{m}^{2}{\left| {d}_{{m}^{\prime }m}^{\left( j\right) }\left( \beta \right) \right| }^{2} = \frac{1}{2}j\left( {j + 1}\right) {\sin }^{2}\beta + {m}^{\prime 2}\frac{1}{2}\left( {3{\cos }^{2}\beta - 1}\right) .
$$

[提示: 这可以用许多方法证明. 例如,可以利用球 (不可约) 张量语言检查 ${J}_{z}^{2}$ 的转动性质.]

3.26 (a) 考虑一个 $j = 1$ 的系统. 明确地把

$$
\left\langle {j = 1,{m}^{\prime }\left| {J}_{y}\right| j = 1, m}\right\rangle
$$

写成 $3 \times 3$ 矩阵形式.

(b) 证明,只有 $j = 1$ 时,才可以合理地用

$$
1 - i\left( \frac{{J}_{y}}{\hslash }\right) \sin \beta - {\left( \frac{{J}_{y}}{\hslash }\right) }^{2}\left( {1 - \cos \beta }\right)
$$

代替 ${e}^{-i{J}_{y}{\beta }^{i/h}}$ .

(c) 利用 (b) 证明

$$
{d}^{\left( j = 1\right) }\left( \beta \right) = \left( \begin{matrix} \left( \frac{1}{2}\right) \left( {1 + \cos \beta }\right) & - \left( \frac{1}{\sqrt{2}}\right) \sin \beta & \left( \frac{1}{2}\right) \left( {1 - \cos \beta }\right) \\ \left( \frac{1}{\sqrt{2}}\right) \sin \beta & \cos \beta & - \left( \frac{1}{\sqrt{2}}\right) \sin \beta \\ \left( \frac{1}{2}\right) \left( {1 - \cos \beta }\right) & \left( \frac{1}{\sqrt{2}}\right) \sin \beta & \left( \frac{1}{2}\right) \left( {1 + \cos \beta }\right) \end{matrix}\right) .
$$

3. 27 借助

$$
{\mathcal{D}}_{mn}^{\prime }\left( {\alpha \beta \gamma }\right) = \langle {\alpha \beta \gamma } \mid {jmn}\rangle
$$

中的级数把矩阵元 $\left\langle {{\alpha }_{2}{\beta }_{2}{\gamma }_{2}\left| {J}_{3}^{2}\right| {\alpha }_{1}{\beta }_{1}{\gamma }_{1}}\right\rangle$ 表示出来. (作者在勘误表中认为这个题应该删掉,因为 $|{jmn}\rangle$ 代表什么,并不清楚. 一译者注)

3.28 考虑由两个自旋 $\frac{1}{2}$ 的粒子组成的一个系统. 观察者 $\mathrm{A}$ 专门测量其中一个粒子的自旋分量 $\left( {s}_{1z}\right.$ , ${s}_{1, t}$ ,等等),同时观察者 B 测量另一个粒子的自旋分量. 假定已知系统处在自旋单态,即 ${S}_{\& } = 0$ .

(a) 当观察者 $\mathrm{B}$ 不做任何测量时,观察者 $\mathrm{A}$ 得到 ${s}_{1z} = \hslash /2$ 的概率是什么? 对于 ${s}_{1x} = \hslash /2$ 求解同样问题.

(b) 观察者 $\mathrm{B}$ 肯定地确认粒子 2 的自旋处于 ${s}_{2z} = \hslash /2$ 态. 如果观察者 $\mathrm{A}$ (i) 测量 ${s}_{1z}$ ; (ii) 测 ${s}_{1z}$ , 则对观察者 $\mathrm{A}$ 的测量结果能给出的结论是什么? 解释你的答案.

3.29 考虑一个秩为 1 的球张量 (即一个矢量)

$$
{V}_{\pm 1}^{\left( 1\right) } = \mp \frac{{V}_{x} \pm i{V}_{y}}{\sqrt{2}},\;{V}_{0}^{\left( 1\right) } = {V}_{z}.
$$

利用习题 3.26 给出的 ${d}^{\left( j = 1\right) }$ 的表示式,求

$$
\mathop{\sum }\limits_{{q}^{\prime }}{d}_{q{q}^{\prime }}^{\left( 1\right) }\left( \beta \right) {V}_{{q}^{\prime }}^{\left( 1\right) },
$$

并证明这结果正是从绕 $y$ 轴转动时 ${V}_{x, y, z}$ 的变换性质所预期的.

3.30 (a) 用两个不同的矢量 $\mathbf{U} = \left( {{U}_{x},{U}_{y},{U}_{z}}\right)$ 和 $\mathbf{V} = \left( {{V}_{x},{V}_{y},{V}_{z}}\right)$ 构造一个秩为 1 的球张量. 明确地用 ${U}_{x, y, z}$ 和 ${V}_{x, y, z}$ 写出 ${T}_{\pm 1,0}^{\left( 1\right) }$ .

(b) 用两个不同的矢量 $\mathbf{U}$ 和 $\mathbf{V}$ 构造一个秩为 2 的球张量. 明确地用 ${U}_{x, y, z}$ 和 ${V}_{x, y, z}$ 写出 ${T}_{\pm 2, \pm 1,0}^{\left( 2\right) }$ .

3.31 考虑一个无自旋粒子被一个中心力位势束缚于一个固定的中心.

(a) 尽可能只用维格纳-埃卡特定理建立起矩阵元

$$
\left\langle {{n}^{\prime },{l}^{\prime },{m}^{\prime }}\right\rangle \mp \frac{1}{\sqrt{2}}\left( {x \pm {iy}}\right) \left| {n, l, m\rangle \;\text{ 和 }\;\left\langle {{n}^{\prime },{l}^{\prime },{m}^{\prime }}\right\rangle z}\right| n, l, m\rangle
$$

的关系. 肯定地指出在什么样的条件下这些矩阵元非零.

(b) 利用波函数 $\psi \left( \mathbf{x}\right) = {R}_{nl}\left( r\right) {Y}_{l}^{m}\left( {\theta ,\phi }\right)$ 求解同样的问题.

3.32 (a) 把 ${xy},{xz}$ 和 $\left( {{x}^{2} - {y}^{2}}\right)$ 写成一个秩为 2 的球 (不可约) 张量的分量.

(b) 期待值

$$
Q \equiv e\left\langle {\alpha, j, m = j\left| \left( {3{z}^{2} - {r}^{2}}\right) \right| \alpha, j, m = j}\right\rangle
$$

被称为四级矩. 利用 $Q$ 和适当的克莱布什-戈丹系数,求

$$
e\left\langle {\alpha, j,{m}^{\prime }\left| \left( {{x}^{2} - {y}^{2}}\right) \right| \alpha, j, m = j}\right\rangle ,
$$

其中 ${m}^{\prime } = j, j - 1, j - 2,\cdots$ .

3.33 一个处于原点的自旋为 $\frac{3}{2}$ 的原子核受到一个外部非均匀电场的作用. 基本的电四极矩相互作用可以取为

$$
{H}_{\mathrm{{int}}} = \frac{eQ}{{2s}\left( {s - 1}\right) {\hslash }^{2}}\left\lbrack {{\left( \frac{{\partial }^{2}\phi }{\partial {x}^{2}}\right) }_{0}{S}_{x}^{2} + {\left( \frac{{\partial }^{2}\phi }{\partial {y}^{2}}\right) }_{0}{S}_{y}^{2} + {\left( \frac{{\partial }^{2}\phi }{\partial {z}^{2}}\right) }_{0}{S}_{z}^{2}}\right\rbrack ,
$$

其中 $\phi$ 是满足拉普拉斯方程的静电势,而坐标轴的选取. 使得:

$$
{\left( \frac{{\partial }^{2}\phi }{\partial x\;\partial y}\right) }_{0} = {\left( \frac{{\partial }^{2}\phi }{\partial y\;\partial z}\right) }_{0} = {\left( \frac{{\partial }^{2}\phi }{\partial x\;\partial z}\right) }_{0} = 0.
$$

证明相互作用可以写成

$$
A\left( {3{S}_{z}^{2} - {\mathbf{S}}^{2}}\right) + B\left( {{S}_{ + }^{2} + {S}_{ - }^{2}}\right) .
$$

并借助于 ${\left( {\partial }^{2}\phi /\partial {x}^{2}\right) }_{0}$ 等表示 $A$ 和 $B$ . (借助 $|m\rangle$ ,其中 $m = \pm \frac{3}{2}, \pm \frac{1}{2}$ ) 确定能量本征右矢和相应的能量本征值. 有简并存在吗?



	
	
	
	
	
\ifx\allfiles\undefined
\end{document}
	\else
	\fi
