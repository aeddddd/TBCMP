\chapter*{前言}
\addcontentsline{toc}{chapter}{Preface} % Add the preface to the table of contents as a chapter



\textbf{2025/4/20编: 更改了模板,彻底调整了之前的一些遗留问题}\\

本书第一章提供了大部分数学内容的概述,主要是使初学者意识到:\textbf{一味的学数学打基础是有害无益的},读者可以参阅后续章节思考究竟实际用到了多少数学内容.对于想要正常学习的读者,建议从第二章开始正文,可以将第一章当做对相应数学领域的粗浅了解,同时,对于真的想要深入学习下去的读者,也提供了相应的推荐书目来参考.

\marginnote[]{无论我怎么强调数学不过是辅助,但总是有人不由自主的去钻牛角尖,于是我在第一章准备了多到离谱的数学内容供这些人阅读,\textbf{如果想要正常思路阅读下去,请从第二章开始.}}
很多初学者在初次自学较高深的物理的时候往往会犯一个错误:被同等级的高深的数学所迷惑,认为学好这些物理离不开这些数学.但事实上,学好物理确实离不开数学,但仅仅是一小部分的数学.例如,学好量子力学离不开线性代数,但经常有很多物理系的学生拐去学习泛函分析,李代数,辛几何之类的内容,而这些数学内容往往只有一小部分应用在物理上面,打着先学完这些数学再开始学习物理的想法,只会让物理的学习一拖再拖.我们应当意识到,数学不过是物理的工具,\textbf{切勿舍本逐末}.

但这并不意味着数学不重要,离开数学的物理不亚于纸上谈兵,毫无意义.而对于本讲义,前面的部分并不十分紧要,仅作为开启凝聚态场论的先备知识,将重要的部分提取出来,作为先导知识.%第三章是为后续的大量使用做铺垫,第四章开始是文章的主体部分,最后的几章为目前在做的研究方向的总结.

前言仅供参考,不代表最终前言内容和书中内容,该前言已经是第四版.




\begin{flushright}
	\textit{530 Group}
\end{flushright}
