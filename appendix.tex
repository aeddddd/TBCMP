\ifx\allfiles\undefined

% 如果有这一部分另外的package,在这里加上
% 没有的话不需要

\begin{document}
	\else
	\fi
\chapter{单位制}
我们从小学就逐步接触一些单位,常见的如米(m),千克(kg),秒(s)等是国际统一使用的\textbf{标准度量系统(国际单位制)}.相应的,像是国内经常接触的斤,公里,亩,美国\footnote{包括美国、开曼群岛、伯利兹等极少数国家和地区}常用的华氏度等,则是生活中使用的独立度量系统,大多数度量系统都和标准度量系统之间存在换算关系.而且生活中使用的度量单位大多比较局限,对于相干度较低的单位往往是不涉及的.\\
对于初中和高中的物理学习,我们已经熟练使用国际单位制(SI\footnote{法语 Système International d'Unités,简称SI})来解决一些简单的物理问题.但是,就像生活中使用的单位制一样,人们出于方便的角度对于一些物理场景也构建出一些新的单位制.这些单位制能够简化相关的物理问题.\\物理上使用的单位制与国际单位制的转换往往比较复杂,使用时建议标注使用了哪个单位制.\\
\begin{remark}
	在这个附录中,电磁单位制与自然单位制独立分为两节,但是按照较广义的自然单位制的定义\footnote{区别于粒子物理的``自然单位制"和普朗克单位制},电磁单位制也属于其中的一类,特此说明.
\end{remark}
\section{电磁单位制}
相比于我们常用的国际单位制,也称为MKSA单位制(即米,千克,秒,安培),我们在电磁中常用的高斯单位制被称为CGS单位制(即厘米,克,秒).\\接下来为了避免混乱,列举高斯单位制所常用的单位:电荷$statC$,电势$statV$,力$dyne$\footnote{中文音译为达因},磁感应强度$gauss$,磁场强度$oersted$,磁通量$mx$,能量$erg$.\\

相比于自然单位制直接将值赋为1,高斯单位制就比较保守,它根据我们熟知的库仑定律,通过定义$1\mathrm{A}=0.1c\cdot\mathrm{\d yne}^{\frac12},1\mathrm{C}=0.1c\cdot\rm{\d yne}^{\frac12}\cdot s$来达到简化的操作.\\
\begin{table}[htbp]
	\centering
	\caption{一些简单对应关系}
	\begin{tabular}{|c|c|c|c|}
		\hline
		& SI      & Gaussian        & G/SI                     \\ \hline
		E          & $V/m$   & $statV/m$       & $\sqrt{4\pi\epsilon_0}$  \\ \hline
		V          & $V$     & $statV$         & $\sqrt{4\pi\epsilon_0}$  \\ \hline
		D          & $C/m^2$ & $statC/cm^2$    & $\sqrt{4\pi/\epsilon_0}$ \\ \hline
		q          & $C$     & $statC$         & $1/\sqrt{4\pi\epsilon_0}$ \\ \hline
		P          & $C/m^2$ & $statC/cm^2$    & $1/\sqrt{4\pi\epsilon_0}$ \\ \hline
		I          & $A$     & $statC/s$       & $1/\sqrt{4\pi\epsilon_0}$ \\ \hline
		B          & $T$     & $Gauss$         & $\sqrt{4\pi/\mu_0}$      \\ \hline
		A          & $Wb/m$  & $Gauss\cdot cm$ & $\sqrt{4\pi/\mu_0}$      \\ \hline
		H          & $A/m$   & $oersted$       & $\sqrt{4\pi\mu_0}$       \\ \hline
		$\epsilon$ & $F/m$   & 1               & $1/\epsilon_0$           \\ \hline
		$\mu$      & $H/m$   & 1               & $1/\mu_0$                \\ \hline
	\end{tabular}
\end{table}
\section{自然单位制}
我们熟知,国际单位制的7个基本单位是通过物理常数所定义的,那么,如果我们把其中\textbf{一个或几个}的定义值改为1,那么就又可以构造出来一套度量系统.这其中\textbf{显而易见的优点}是直接导致原本含有大量常数的公式可以被写成更加简洁方便的形式.在物理学里,自然单位制就是一种建立于此类方法的计量单位制度.例如,电荷的自然单位是基本电荷${\displaystyle e}$,速度的自然单位是光速${\displaystyle c}$,角动量的自然单位是约化普朗克常数${\displaystyle \hbar }$,电阻的自然单位是自由空间阻抗${\displaystyle Z_{0}}$,质量的自然单位则有电子质量${\displaystyle m_{e}}$与质子质量${\displaystyle m_{p}}$等.\\

事实上,对于单位的改动,我们至少要求不会导致无量纲常数的值发生改变,如精细结构常数.
$${\displaystyle \alpha ={\frac {e^{2}k_{e}}{\hbar c}}={\frac {e^{2}}{\hbar c(4\pi \epsilon _{0})}}={\frac {1}{137.035999074}}=7.2973525698\cdot 10^{-3}}$$
这个常数就要求不能同时把${\displaystyle e},{\displaystyle \hbar },{\displaystyle c},{\displaystyle k_{e}}$同时为1.
\subsection{普朗克单位制}
普朗克单位制几乎是最常使用的单位制,它的定义只依赖于最基本的性质.普朗克单位选择将真空光速${\displaystyle c}$,万有引力常数${\displaystyle G}$,约化普朗克常数${\displaystyle \hbar }$,真空电容率${\displaystyle \epsilon _{0}}$,玻尔兹曼常数${\displaystyle k_{B}}$定为1\footnote{普朗克洛伦兹-亥维赛单位制将${\displaystyle 4\pi G},{\displaystyle \epsilon _{0}}$定为1,普朗克高斯单位制将${\displaystyle G},{\displaystyle 4\pi \epsilon _{0}}$定为1}.\\
类比国际单位制,普朗克单位制也有一些基本单位(如常常出现在各种科普作品中的普朗克长度,普朗克时间等)和导出单位(普朗克面积,普朗克动量等).具体列表可参考相关wiki\href{https://zh.wikipedia.org/wiki/%E6%99%AE%E6%9C%97%E5%85%8B%E5%96%AE%E4%BD%8D%E5%88%B6}{普朗克单位制},这里不做展开.
\subsection{``自然单位制"(粒子物理)}
在粒子物理中,自然单位制特指${\displaystyle \hbar =c=k_{B}=1}$情况下的单位制.通常会根据情况选择使用洛伦兹-亥维赛单位制或高斯单位制来确定电荷定义.
\subsection{其他单位制}
\subsubsection*{史东纳单位制}
第一次出现的单位制,已经不再使用.规定了${\displaystyle c=G=e={\frac {1}{4\pi \epsilon _{0}}}=k_{B}=1}$.
\subsubsection*{原子单位制}
这类单位制是特别为了简易表达原子物理学和分子物理学的方程而精心设计,在本篇中仅做介绍.\\
原子单位制分为两种:哈特里原子单位制和里德伯原子单位制.哈特里原子单位制比里德伯原子单位制常见.两者的主要区别在于质量单位与电荷单位的选取.\\
哈特里原子单位制的基本单位为${\displaystyle e=m_{e}=\hbar ={\frac {1}{4\pi \epsilon _{0}}}=k_{B}=1}$,${\displaystyle c={\frac {1}{\alpha }}}$.\\
里德伯原子单位制的基本单位为${\displaystyle {\frac {e}{\sqrt {2}}}=2m_{e}=\hbar ={\frac {1}{4\pi \epsilon _{0}}}=k_{B}=1}$,${\displaystyle c={\frac {2}{\alpha }}}$.
\chapter{固体物理中的一些概念}
对于物理研究,把它放在合适的空间下能够简化问题.对于坐标空间(正格子,基矢)和动量空间(倒格子,倒格矢)来讲,相当于从两个角度来描写\textbf{同一}事物.在之后对于晶格的分析中,我们常常要在动量空间上分析这一问题.\\
\textit{如果对于物理形式较为敏感,应该会容易的想到``两个角度描写同一事物"的表述和傅里叶变换有很大的相似性.实际上,坐标空间和动量空间互为傅里叶变换.如果对于量子力学有一定了解或已经阅读过关于表象变换的内容,对这一部分会有更深的体会.}
\section*{一些需要了解的概念}
为了能够便于理解接下来的内容,以下是需要了解的概念.
\begin{enumerate}
	\item \textbf{格矢}:联系任两个晶格点的向量
	\item \textbf{布拉维晶格 Bravais lattices}:由同种原子构成的晶胞,多种原子构成的晶胞可以视为几个布拉维晶格的叠加.
	\item 待补充
\end{enumerate}


\chapter{$\delta_{ij},\varepsilon_{ijk}$和爱因斯坦求和约定与$\delta$函数}
\section{克罗内克符号$\delta_{ij}$}
克罗内克符号是一类二元函数,其通常定义为以下形式
\begin{equation}
	\delta _{ij}=\left\{{\begin{matrix}1&(i=j)\\0&(i\neq j)\end{matrix}}\right.\,\!
\end{equation}
其具有筛选性(和投影算符类似)
\begin{equation}
	\sum _{i=-\infty }^{\infty }\delta _{ij}a_{i}=a_{j}\,\!
\end{equation}
其具有和$\delta$函数共同的部分性质,而$\delta$函数也正是源于克罗内克符号.
\section{列维西维塔符号$\varepsilon_{ijk}$}
列维-奇维塔符号,对于正整数 $n$ ,它以$1, 2, ..., n $所形成排列的奇偶性来定义.其他名称包括排列符号、反对称符号与交替符号.

而$\varepsilon_{ijk}$的值由下角标$ijk$决定,当存在任意两个角标相同时值取$0$,当全部指标都不相等时,角标的逆序数为偶数取$1$,为奇数则取$0$.
\subsection*{二维形式}
\begin{equation}
	\varepsilon_{ij}={
		\begin{cases}+1&\text{当}\left(i,j\right)=\left(1,2\right)\\
			-1&\text{当}\left(i,j\right)=\left(2,1\right)\\
			0&\text{当}i=j
		\end{cases}
	}\,
\end{equation}
二维较为少见,仅作为了解.
\subsection*{三维形式}
我们经常看到的列维西维塔符号常常是三维形式的,即如下
\begin{equation}
	\varepsilon _{ijk}={
		\begin{cases}
			+1&\text{当}(i,j,k)=(1,2,3),(2,3,1),(3,1,2)\\
			-1&\text{当}(i,j,k)=(3,2,1),(2,1,3),(1,3,2)\\
			0&\text{当}i=j,j=k\text{或}k=i
		\end{cases}
	}\,
\end{equation}
\subsection*{性质}
两个列维-奇维塔符号的积,可以用一个以克罗内克符号表示的行列式求得
\begin{equation}
	\varepsilon _{ijk\dots }\varepsilon _{mnl\dots }={\begin{vmatrix}\delta _{im}&\delta _{in}&\delta _{il}&\dots \\\delta _{jm}&\delta _{jn}&\delta _{jl}&\dots \\\delta _{km}&\delta _{kn}&\delta _{kl}&\dots \\\vdots &\vdots &\vdots \\\end{vmatrix}}
\end{equation}
也可以用来表示行列式和向量内积,对于一个$3\times3$的方阵$A$,有表示
\begin{equation}
	\det(A)=\sum _{i,j,k=1}^{3}\varepsilon _{ijk}\,a_{1i}\,a_{2j}\,a_{3k}
\end{equation}
对于向量内积,有
\begin{equation}
	{\boldsymbol {a}}\times {\boldsymbol {b}}={\begin{vmatrix}{\boldsymbol {e}}_{1}&{\boldsymbol {e}}_{2}&{\boldsymbol {e}}_{3}\\a_{1}&a_{2}&a_{3}\\b_{1}&b_{2}&b_{3}\\\end{vmatrix}}=\sum _{1\leq i,j,k\leq 3}\varepsilon _{ijk}\,a_{i}b_{j}\,{\boldsymbol {e}}_{k}
\end{equation}
\section{爱因斯坦求和约定}
爱因斯坦求和约定是一种标记的约定,即重复角标意味着求和,一般未指定的情况下就是由$1$至$3$.
\section{$\delta$函数}
我们首先明确,$\delta$函数并不是通常意义的函数,其更准确的称呼是广义函数.其在$x=0$处取值为正无穷,在$x\ne0$处取值为$0$,但是在全定义域上的积分值为$1$,当然,我们所学的黎曼积分并不支持这一操作,我们需要\textbf{勒贝格积分}来处理它,当然,这里并不会讲述测度论的内容,仅仅作为了解即可.
\subsection{定义}
我们从其定义开始这部分内容.
\begin{equation}
	\delta (x)={\begin{cases}+\infty ,&x=0\\0,&x\neq 0\end{cases}}
\end{equation}
且同时满足
\begin{equation}
	\int _{-\infty }^{\infty }\delta (x)\,\d x=1
\end{equation}
而对于复变函数中,一切在域$D$中闭包的全纯函数,我们可以用柯西积分公式来表示$\delta$函数
\begin{equation}
	\delta _{z}[f]=f(z)={\frac {1}{2\pi i}}\oint _{\partial D}{\frac {f(\zeta )\,\d\zeta }{\zeta -z}}.
\end{equation}
\subsection{性质}
首先,回忆克罗内克符号的部分,很容易联想到$\delta$函数也具有筛选的性质:
\begin{equation}
	\int _{-\infty }^{\infty }f(x)\delta (x-x_{0})\,\d x=f(x_{0})
\end{equation}
以及(高维的情况与之类似,不再重复,相关证明是显然的)
\begin{equation}
	\begin{aligned}
		\delta (\alpha x)&={\frac {\delta (x)}{|\alpha |}}\\
		\delta (-x)&=\delta (x)\\
		x\delta (x)&=0\\
		\int _{-\infty }^{\infty }\delta (\xi -x)\delta (x-\eta )\,\mathrm {d} x&=\delta (\xi -\eta )
	\end{aligned}
\end{equation}
狄拉克$\delta$分布是在包含所有平方可积函数的希尔伯特空间$L_2$上所稠密定义的一个无界线性泛函.在许多应用中,可以对$L_2$的某个子空间赋予更强的拓扑,使得$\delta$函数能够定义一个有界线性算子.
\chapter{mathematica的基本用法}
1
\chapter{临时章节}
\begin{equation}
	\begin{array}{ll|l}
		\texttt{"normal"}      &\texttt{}         & ABCDEFGHIJKLMNOPQRSTUVWXYZ\\
		\texttt{"blackboard"}  &\texttt{mathbb}  &\mathbb{ABCDEFGHIJKLMNOPQRSTUVWXYZ}\\
		\texttt{"boldface"}    &\texttt{mathbf}  &\mathbf{ABCDEFGHIJKLMNOPQRSTUVWXYZ}\\
		\texttt{"typewriter"}  &\texttt{mathtt}  &\mathtt{ABCDEFGHIJKLMNOPQRSTUVWXYZ}\\
		\texttt{"roman"}       &\texttt{mathrm}  &\mathrm{ABCDEFGHIJKLMNOPQRSTUVWXYZ}\\
		\texttt{"sans-serif"}  &\texttt{mathsf}  &\mathsf{ABCDEFGHIJKLMNOPQRSTUVWXYZ}\\
		\texttt{"calligraphic"}&\texttt{mathcal} &\mathcal{ABCDEFGHIJKLMNOPQRSTUVWXYZ}\\
		\texttt{"script"}      &\texttt{mathscr} &\mathscr{ABCDEFGHIJKLMNOPQRSTUVWXYZ}\\
		\texttt{"fraktur"}     &\texttt{mathfrak}&\mathfrak{ABCDEFGHIJKLMNOPQRSTUVWXYZ}\\
	\end{array}
\end{equation}
\begin{equation}
	\begin{aligned}
		\mathcal{L}_{\mathrm{StandardModel}}&=-\frac12 \partial_{\nu} g_{\mu}^{a} \partial_{\nu} g_{\mu}^{a}-g_{s} f^{a b c} \partial_{\mu} g_{\nu}^{a} g_{\mu}^{b} g_{\nu}^{c}-\frac{1}{4} g_{s}^{2} f^{a b c} f^{a d e} g_{\mu}^{b} g_{\nu}^{c} g_{\mu}^{d} g_{\nu}^{e}+\frac12 i g_{s}^{2}(\bar{q}_{i}^{\sigma} \gamma^{\mu} q_{j}^{\sigma}) g_{\mu}^{a}+\bar{G}^{a} \partial^{2} G^{a}+&\\&g_{s} f^{a b c} \partial_{\mu} \bar{G}^{a} G^{b} g_{\mu}^{c}-\partial_{\nu} W_{\mu}^{+} \partial_{\nu} W_{\mu}^{-}-M^{2} W_{\mu}^{+} W_{\mu}^{-}-\frac12 \partial_{\nu} Z_{\mu}^{0} \partial_{\nu} Z_{\mu}^{0}-\frac{1}{2 c_{w}^{2}} M^{2} Z_{\mu}^{0} Z_{\mu}^{0}-\frac12 \partial_{\mu} A_{\nu} \partial_{\mu} A_{\nu}-&\\&\frac12 \partial_{\mu} H \partial_{\mu} H-\frac12 m_{h}^{2} H^{2}-\partial_{\mu} \phi^{+} \partial_{\mu} \phi^{-}-M^{2} \phi^{+} \phi^{-}-\frac12 \partial_{\mu} \phi^{0} \partial_{\mu} \phi^{0}-\frac{1}{2 c_{w}^{2}} M \phi^{0} \phi^{0}-\beta_{h}[\frac{2 M^{2}}{g^{2}}+\frac{2 M}{g} H+&\\&\frac12(H^{2}+\phi^{0} \phi^{0}+2 \phi^{+} \phi^{-})]+\frac{2 M^{4}}{g^{2}} \alpha_{h}-i g c_{w}[\partial_{\nu} Z_{\mu}^{0}(W_{\mu}^{+} W_{\nu}^{-}-W_{\nu}^{+} W_{\mu}^{-})-Z_{\nu}^{0}(W_{\mu}^{+} \partial_{\nu} W_{\mu}^{-}-W_{\mu}^{-} \partial_{\nu} W_{\mu}^{+})+&\\&Z_{\mu}^{0}(W_{\nu}^{+} \partial_{\nu} W_{\mu}^{-}-W_{\nu}^{-} \partial_{\nu} W_{\mu}^{+})]-i g s_{w}[\partial_{\nu} A_{\mu}(W_{\mu}^{+} W_{\nu}^{-}-W_{\nu}^{+} W_{\mu}^{-})-A_{\nu}(W_{\mu}^{+} \partial_{\nu} W_{\mu}^{-}-\frac12 g^{2} W_{\mu}^{+} W_{\nu}^{-} W_{\mu}^{+} W_{\nu}^{-}+&\\&g^{2} c_{w}^{2}(Z_{\mu}^{0} W_{\mu}^{+} Z_{\nu}^{0} W_{\nu}^{-}-Z_{\mu}^{0} Z_{\mu}^{0} W_{\nu}^{+} W_{\nu}^{-})+g^{2} s_{w}^{2}(A_{\mu} W_{\mu}^{+} A_{\nu} W_{\nu}^{-}-_{\mu} A_{\mu} W_{\nu}^{+} W_{\nu}^{-})+Ag^{2} s_{w} c_{w}[A_{\mu} Z_{\nu}^{0}(W_{\mu}^{+} W_{\nu}^{-}-&\\&W_{\nu}^{+} W_{\mu}^{-})-2 A_{\mu} Z_{\mu}^{0} W_{\nu}^{+} W_{\nu}^{-}]-g \alpha[H^{3}+H \phi^{0} \phi^{0}+2 H \phi^{+} \phi^{-}]-\frac{1}{8} g^{2} \alpha_{h}[H^{4}+(\phi^{0})^{4}+4(\phi^{+} \phi^{-})^{2}+4(\phi^{0})^{2} \phi^{+} \phi^{-}+&\\&4 H^{2} \phi^{+} \phi^{-}+2(\phi^{0})^{2} H^{2}]-g M W_{\mu}^{+} W_{\mu}^{-} H-\frac12 g \frac{M}{c_{w}^{2}} Z_{\mu}^{0} Z_{\mu}^{0} H-\frac12 i g[W_{\mu}^{+}(\phi^{0} \partial_{\mu} \phi^{-}-\phi^{-} \partial_{\mu} \phi^{0})-W_{\mu}^{-}(\phi^{0} \partial_{\mu} \phi^{+}-&\\&\phi^{+} \partial_{\mu} \phi^{0})]+\frac12 g[W_{\mu}^{+}(H \partial_{\mu} \phi^{-}-\phi^{-} \partial_{\mu} H)-W_{\mu}^{-}(H \partial_{\mu} \phi^{+}-\phi^{+} \partial_{\mu} H)]+\frac12 g \frac{1}{c_{w}}(Z_{\mu}^{0}(H \partial_{\mu} \phi^{0}-\phi^{0} \partial_{\mu} H)-&\\&i g_{c_{w}}^{s_{w}^{2}} M Z_{\mu}^{0}(W_{\mu}^{+} \phi^{-}-W_{\mu}^{-} \phi^{+})+i g s_{w} M A_{\mu}(W_{\mu}^{+} \phi^{-}-W_{\mu}^{-} \phi^{+})-i g \frac{1-2 c_{w}^{2}}{2 c_{w}} Z_{\mu}^{0}(\phi^{+} \partial_{\mu} \phi^{-}-\phi^{-} \partial_{\mu} \phi^{+})+igs_{w} A_{\mu}&\\&(\phi^{+} \partial_{\mu} \phi^{-}-\phi^{-} \partial_{\mu} \phi^{+})-\frac{1}{4} g^{2} W_{\mu}^{+} W_{\mu}^{-}[H^{2}+(\phi^{0})^{2}+2 \phi^{+} \phi^{-}]-\frac{1}{4} g^{2} \frac{1}{c_{w}^{2}} Z_{\mu}^{0} Z_{\mu}^{0}[H^{2}+(\phi^{0})^{2}+2(2 s_{w}^{2}-1)^{2} \phi^{+} \phi^{-}]-&\\&\frac12 g^{2} \frac{s_{w}^{2}}{c_{w}} Z_{\mu}^{0} \phi^{0}(W_{\mu}^{+} \phi^{-}+W_{\mu}^{-} \phi^{+})-\frac12 i g^{2} \frac{s_{w}^{2}}{c_{w}} Z_{\mu}^{0} H(W_{\mu}^{+} \phi^{-}-W_{\mu}^{-} \phi^{+})+\frac12 g^{2} s_{w} A_{\mu} \phi^{0}(W_{\mu}^{+} \phi^{-}+W_{\mu}^{-} \phi^{+})+\frac12 i g^{2} s_{w} A_{\mu} H&\\&(W_{\mu}^{+} \phi^{-}-W_{\mu}^{-} \phi^{+})-g^{2} \frac{s_{w}}{c_{w}}(2 c_{w}^{2}-1) Z_{\mu}^{0} A_{\mu} \phi^{+} \phi^{-}-g^{1} s_{w}^{2} A_{\mu} A_{\mu} \phi^{+} \phi^{-}-\bar{e}^{\lambda}(\gamma \partial+m_{e}^{\lambda}) e^{\lambda}-\bar{\nu}^{\lambda} \gamma \partial \nu^{\lambda}-\bar{u}_{j}^{\lambda}(\gamma \partial+&\\&m_{u}^{\lambda}) u_{j}^{\lambda}-\bar{d}_{j}^{\lambda}(\gamma \partial+m_{d}^{\lambda}) d_{j}^{\lambda}+i g s_{w} A_{\mu}[-(\bar{e}^{\lambda} \gamma^{\mu} e^{\lambda})+\frac{2}{3}(\bar{u}_{j}^{\lambda} \gamma^{\mu} u_{j}^{\lambda})-\frac{1}{3}(\bar{d}_{j}^{\lambda} \gamma^{\mu} d_{j}^{\lambda})]+\frac{i g}{4 c_{w}} Z_{\mu}^{0}[(\bar{\nu}^{\lambda} \gamma^{\mu}(1+\gamma^{5}) \nu^{\lambda})+&\\&(\bar{e}^{\lambda} \gamma^{\mu}(4 s_{w}^{2}-1-\gamma^{5}) e^{\lambda})+(\bar{u}_{j}^{\lambda} \gamma^{\mu}(\frac{4}{3} s_{w}^{2}-1-\gamma^{5}) u_{j}^{\lambda})+(\bar{d}_{j}^{\lambda} \gamma^{\mu}(1-\frac{8}{3} s_{w}^{2}-\gamma^{5}) d_{j}^{\lambda})]+\frac{i g}{2 \sqrt{2}} W_{\mu}^{+}[(\bar{\nu}^{\lambda} \gamma^{\mu}(1+\gamma^{5}) e^{\lambda})+&\\&(\bar{u}_{j}^{\lambda} \gamma^{\mu}(1+\gamma^{5}) C_{\lambda \kappa} d_{j}^{\kappa})]+\frac{i g}{2 \sqrt{2}} W_{\mu}^{-}[(\bar{e}^{\lambda} \gamma^{\mu}(1+\gamma^{5}) \nu^{\lambda})+(\bar{d}_{j}^{\kappa} C_{\lambda \kappa}^{\dagger} \gamma^{\mu}(1+\gamma^{5}) u_{j}^{\lambda})]+\frac{i g}{2 \sqrt{2}} \frac{m_{e}^{\lambda}}{M}[-\phi^{+}(\bar{\nu}^{\lambda}(1-\gamma^{5}) e^{\lambda})+&\\&\phi^{-}(\bar{e}^{\lambda}(1+\gamma^{5}) \nu^{\lambda})]-\frac{\partial}{2} \frac{m_{e}^{\lambda}}{M}[H(\bar{e}^{\lambda} e^{\lambda})+i \phi^{0}(\bar{e}^{\lambda} \gamma^{5} e^{\lambda})]+\frac{i g}{2 M \sqrt{2}} \phi^{+}[-m_{d}^{\kappa}(\bar{u}_{j}^{\lambda} C_{\lambda \kappa}(1-\gamma^{5}) d_{j}^{\kappa})+m_{u}^{\lambda}(\bar{u}_{j}^{\lambda} C_{\lambda \kappa}(1+&\\&\gamma^{5}) d_{j}^{\kappa}]+\frac{i g}{2 M \sqrt{2}} \phi^{-}[m_{d}^{\lambda}(\bar{d}_{j}^{\lambda} C_{\lambda \kappa}^{\dagger}(1+\gamma^{5}) u_{j}^{\kappa})-m_{u}^{\kappa}(\bar{d}_{j}^{\lambda} C_{\lambda \kappa}^{\dagger}(1-\gamma^{5}) u_{j}^{\kappa}]-\frac{g}{2} \frac{m_{u}^{\lambda}}{M} H(\bar{u}_{j}^{\lambda} u_{j}^{\lambda})-\frac{g}{2} \frac{m_{d}^{\lambda}}{M} H(\bar{d}_{j}^{\lambda} d_{j}^{\lambda})+&\\&\frac{i g}{2} \frac{m_{u}^{\lambda}}{M} \phi^{0}(\bar{u}_{j}^{\lambda} \gamma^{5} u_{j}^{\lambda})-\frac{i g}{2} \frac{m_{d}^{\lambda}}{M} \phi^{0}(\bar{d}_{j}^{\lambda} \gamma^{5} d_{j}^{\lambda})+\bar{X}^{+}(\partial^{2}-M^{2}) X^{+}+\bar{X}^{-}(\partial^{2}-M^{2}) X^{-}+\bar{X}^{0}(\partial^{2}-\frac{M^{2}}{c_{w}^{2}}) X^{0}+\bar{Y} \partial^{2} Y+&\\&i g c_{w} W_{\mu}^{+}(\partial_{\mu} \bar{X}^{0} X^{-}-\partial_{\mu} \bar{X}^{+} X^{0})+i g s_{w} W_{\mu}^{+}(\partial_{\mu} \bar{Y} X^{-}-\partial_{\mu} \bar{X}^{+} Y)+i g c_{w} W_{\mu}^{-}(\partial_{\mu} \bar{X}-X^{0}-\partial_{\mu} \bar{X}^{0} X^{+})+i g s_{w} W_{\mu}^{-}(\partial_{\mu} \bar{X}-&\\&Y-\partial_{\mu} \bar{Y} X^{+})+i g c_{w} Z_{\mu}^{0}(\partial_{\mu} \bar{X}^{+} X^{+}-\partial_{\mu} \bar{X}^{-} X^{-})+i g s_{w} A_{\mu}(\partial_{\mu} \bar{X}^{+} X^{+}-\partial_{\mu} \bar{X}-X^{-})-\frac12 g M[\bar{X}+X^{+} H+\bar{X}^{-} X^{-} H+&\\&\frac{1}{c_{w}^{2}} \bar{X}^{0} X^{0} H]+\frac{1-2 c_{w}^{2}}{2 c_{w}} i g M[\bar{X}^{+} X^{0} \phi^{+}-\bar{X}^{-} X^{0} \phi^{-}]+\frac{1}{2 c_{w}} i g M[\bar{X}^{0} X^{-} \phi^{+}-\bar{X}^{0} X^{+} \phi^{-}]+ig M s_{w}[\bar{X}^{0} X^{-} \phi^{+}-&\\&\bar{X}^{0} X^{+} \phi^{-}]+\frac12 i g M[\bar{X}^{+} X^{+} \phi^{0}-\bar{X}^{-} X^{-} \phi^{0}]
	\end{aligned}
\end{equation}
\chapter{答案及解析}
\section{第一章}
\begin{enumerate}
	\item 证明:如果$X=|\beta\rangle\langle\alpha|$,那么则有$X^\dagger=|\alpha\rangle\langle\beta|$.
	\begin{proof}
		由左右矢运算规则可得,也可以利用矩阵形式进行运算.
		$$(|\beta\rangle\langle\alpha|)^\dagger=\langle\alpha|^\dagger|\beta\rangle^\dagger=|\alpha\rangle\langle\beta|$$
	\end{proof}
	\item 判断
	\begin{enumerate}
		\item 对
		\item 错
		\item 对
	\end{enumerate}
	\item 证明:
	$$
	\left\lbrack {{AB},{CD}}\right\rbrack = - {AC}\{ D, B\} + A\{ C, B\} D - C\{ D, A\} B + \{ C, A\} {D B}.
	$$
	\begin{proof}
	$$
	\begin{aligned}
	AC\{D,B\}&=ACDB+ACBD,\\
	A\{C,B\}D&=ACBD+ABCD,\\
	C\{D,A\}B&=CDAB+CADB,\\ \{C,A\}DB&=CADB+ACDB.\\
	\text{于是有}-AC\{D,B\}+A\{C,B\}D-C\{D,A\}B+\{C,A\}DB&=-ACDB+ABCD-CDAB+ACDB\\&=ABCD-CDAB=[AB,CD]
	\end{aligned}
	$$
	\end{proof}
	\item 假定一个 $2 \times 2$ 矩阵 $X$ (不一定是厄米或幺正矩阵) 被写成
	$$
	X = {a}_{0} + \mathbf{\sigma } \cdot \mathbf{a},
	$$
	其中 ${a}_{0}$ 和 ${a}_{1.2.3}$ 都是数.
	\begin{enumerate}
		\item ${a}_{0}$ 和 ${a}_{k}\left( {k = 1,2,3}\right)$ 与 $\operatorname{tr}\left( X\right)$ 和 $\operatorname{tr}\left( {{\sigma }_{k}X}\right)$ 有什么样的关系?
		\begin{proof}
			我们先分别计算$\mathrm{Tr}(X)$和$\mathrm{Tr}(\sigma_{k}X)$,对于前者(我们首先需要强调,$a_0$是一个数,得到的结果是矩阵,这里的加法我们视为$X = {a}_{0}\cdot \textbf{1} + \mathbf{\sigma } \cdot \mathbf{a}$,并且这里的乘法显然是矩阵乘法而不是内积.)我们有
			\begin{equation}
			\mathrm{Tr}(X)=a_{0} \mathrm{Tr}(\textbf{1})+\sum_{\ell} \mathrm{Tr}(\sigma_{\ell})a_{\ell}=2a_{0}
			\end{equation}
			其中我们知道$\mathrm{Tr}(A+B)=\mathrm{Tr}(A)+\mathrm{Tr}(B)$,而且我们不需要理会后面那一项,因为我们知道泡利矩阵的迹为$0$,不要忘记单位矩阵的迹是$2$,自然得出结果.\\
			下面我们看后面部分
			\begin{equation}
				\mathrm{Tr}(\sigma_{k}X)=a_{0}\mathrm{Tr}(\sigma_{k})+\sum_{\ell}\mathrm{Tr}(\sigma_{k}\sigma_{\ell})a_{\ell}=\frac{1}{2}\sum_{\ell}\mathrm{Tr}(\sigma_{k}\sigma_{\ell}+\sigma_{\ell}\sigma_{k})a_{\ell}
			\end{equation}
			其中$a_{0}\mathrm{Tr}(\sigma_{k})$这一项同样为$0$,后一项我们利用了迹的循环性质(泡利矩阵为$2\times2$的方阵):$\mathrm{Tr}(ABCD)=\mathrm{Tr}(BCDA)=\mathrm{Tr}(CDAB)$,并且注意到可以利用克罗内克符号将其表现为
			\begin{equation}
				\frac{1}{2}\sum_{\ell}\mathrm{Tr}(\sigma_{k}\sigma_{\ell}+\sigma_{\ell}\sigma_{k})a_{\ell}=\sum_{\ell}\delta_{k\ell}\mathrm{Tr}(\textbf{1})a_{\ell}=2a_{k}
			\end{equation}
			注意$\frac12$被约去了,且我们知道$\sum_{\ell}\delta_{k\ell}$具有筛选性质(参照附录),现在我们一并给出最终结果.
			\begin{equation}
				a_{0}=\frac{1}{2}\mathrm{Tr}(X)\text{和}a_{k}=\frac{1}{2}\mathrm{Tr}(\sigma_{k}X)
			\end{equation}
		\end{proof}
		\item 利用矩阵元 ${X}_{ij}$ 求出 ${a}_{0}$ 和 ${a}_{k}$.
		\begin{proof}
			直接写为矩阵元形式带入即可得到答案
			\begin{equation}
				a_{0}=(X_{11}+X_{22})/2,a_{1}=(X_{12}+X_{21})/2, a_{2}=i(-X_{21}+X_{12})/2,a_{3}=(X_{11}-X_{22})/2
			\end{equation}
		\end{proof}
	\end{enumerate}
	\item 证明一个 $2 \times 2$ 矩阵 $\mathbf{\sigma } \cdot \mathbf{a}$ 的行列式在如下变换中不变:
	$$
	\sigma \cdot \mathbf{a} \rightarrow \sigma \cdot {\mathbf{a}}^{\prime } \equiv \exp \left( \frac{{i\sigma } \cdot \widehat{\mathbf{n}}\phi }{2}\right) \sigma \cdot \textbf{a}\operatorname{exp}\left( \frac{-{i\sigma } \cdot \widehat{\mathbf{n}}\phi }{2}\right) .
	$$
	当 $\widehat{\mathbf{n}}$ 沿 $z$ 正方向时,利用 ${a}_{k}$ 求出 ${a}_{k}^{\prime }$ 并解释你的结果.
	\begin{proof}
		事实上,这一题需要一定对于旋转矩阵和SO(2)群的了解,下面使用了其中的部分结论,如果你并不了解这些内容,可以参照后续章节.\\
		我们知道,$\det(A\cdot B)=\det(A)\det(B)$,自然有$\det(\sigma\cdot\textbf{a})=-a_z^2-(a_x^2+a_y^2)=-|\textbf{a}|^2$,下面存在关系
		\begin{equation}
			\det\left[\exp\left(\pm\frac{i\sigma\cdot\hat{\mathbf{n}}\phi}{2}\right)\right]=1
		\end{equation}
		这个关系可以从两个角度来解释,我们可以认为通过一个旋转变换后矩阵的行列式不变,也可以尝试证明它:
		\begin{equation}
			\det\left[\exp\left(\pm\frac{i\sigma\cdot\hat{\mathbf{n}}\phi}{2}\right)\right]=\exp\left(\mathrm{Tr}\left(\pm\frac{i\sigma\cdot\hat{\mathbf{n}}\phi}{2}\right)\right)=\exp\left(\pm\frac{i\phi}2 \mathrm{Tr}\left(\sigma\cdot\hat{\mathbf{n}}\right)\right)=1
		\end{equation}
		自然证明题干.
		第二种方法是对$\exp\left(\pm\frac{i\sigma\cdot\hat{\mathbf{n}}\phi}{2}\right)$进行泰勒展开,并注意到$(\mathbf{\sigma } \cdot \mathbf{a})^n$当$n$为偶数时为$1$,奇数时仍为$\mathbf{\sigma } \cdot \mathbf{a}$,我们有
		\begin{equation}
			\begin{aligned}
				\exp\biggl({\frac{-i\sigma\cdot{\hat{\mathbf{n}}}\phi}{2}}\biggr)& = \Big[ 1-\frac{(\sigma\cdot\hat{\mathbf{n}})^{2}}{2 !}\Big(\frac{\phi}{2}\Big)^{2}+\frac{(\sigma\cdot\hat{\mathbf{n}})^{1}}{4 !}\Big(\frac{\phi}{2}\Big)^{1}-\cdots\Big] \\
				&-i\Big[(\sigma\cdot\hat{\mathbf{n}}) \frac{\phi}{2}-\frac{(\sigma\cdot\hat{\mathbf{n}})^{3}}{3!}\Big(\frac{\phi}{2}\Big)^{3}+\cdots\Big] \\
				&=\Big[ 1-\frac{1}{2 !}\Big(\frac{\phi}{2}\Big)^{2}+\frac{1}{4 !}\Big(\frac{\phi}{2}\Big)^{4}-\cdots\Big]-i\Big[(\sigma\cdot\hat{\mathbf{n}}) \frac{\phi}{2}-\frac{(\sigma\cdot\hat{\mathbf{n}})}{3!}\Big(\frac{\phi}{2}\Big)^{3}+\cdots\Big]\\
				&= \mathbf{1}\cos\biggl(\frac{\phi}{2}\biggr) - i\mathbf{\sigma} \cdot \mathbf{\hat{n}}\sin\biggl(\frac{\phi}{2}\biggr).
			\end{aligned}
		\end{equation}
		并将其写为矩阵形式
		\begin{equation}
			\exp\left(\frac{-i\boldsymbol{\sigma}\cdot\hat{\mathbf{n}}\phi}2\right)=
			\begin{pmatrix}
				\cos\left(\frac\phi2\right)-in_z\sin\left(\frac\phi2\right)&(-in_x-n_y)\sin\left(\frac\phi2\right)\\(-in_x+n_y)\sin\left(\frac\phi2\right)&\cos\left(\frac\phi2\right)+in_z\sin\left(\frac\phi2\right)
			\end{pmatrix}
		\end{equation}
		并注意到题干给出$\widehat{\mathbf{n}}$ 沿 $z$ 正方向,自然$n_x,n_y=0,n_z=1$再次发现其行列式恰好为$1$,并可以继续求得$a^{\prime}_k$,后续为基本的矩阵乘法,不单独列出,仅给出答案:
		\begin{equation}
			a^{\prime}_1=a_1\cos\phi+a_2\sin\phi,a^{\prime}_2=a_2\cos\phi-a_1\sin\phi,a^{\prime}_3=a_3
		\end{equation}
		现在我们不难看出,在$z$-方向上,显然我们刚刚执行了一个绕$z$轴的旋转(将矢量$\textbf{a}$绕$z$轴旋转$\phi$的角度).
	\end{proof}
	\item 利用左矢-右矢代数规则证明或计算下列各式
	\begin{enumerate}
		\item $\operatorname{tr}\left( {XY}\right) = \operatorname{tr}\left( {YX}\right)$ ,其中 $X$ 和 $Y$ 都是算符.
		\begin{proof}
			我们将其写成左右矢形式,并插入一个单位算符:$\mathrm{Tr}(XY)\equiv\sum_a\langle a|XY|a\rangle=\sum_a\sum_b\langle a|X|b\rangle\langle b|Y|a\rangle$,指标交换转置后即证$\mathrm{Tr}(XY)=\sum_b\sum_a\langle b|Y|a\rangle\langle a|X|b\rangle=\sum_b\langle b|YX|b\rangle=\mathrm{Tr}(YX)$
		\end{proof}
		\item ${\left( XY\right) }^{ \dagger } = {Y}^{ \dagger }{X}^{ \dagger }$ ,其中 $X$ 和 $Y$ 都是算符.
		\begin{proof}
			注意到对偶关系即证.
		\end{proof}
		\item 在左矢-右矢形式下 $\exp \left\lbrack {{if}\left( A\right) }\right\rbrack =$ ? 其中 $A$ 是厄米算符,其本征值是已知的.
		\begin{proof}
			$\exp[if(A)]=\sum_{a}\exp[if(A)]|a\rangle\langle a|=\sum_{a}\exp[if(a)]|a\rangle\langle a|$
		\end{proof}
		\item $\sum {a}^{\prime }{\psi }_{a}^{\prime }\left( {\mathbf{x}}^{\prime }\right) {\psi }_{a}\left( {\mathbf{x}}^{\prime \prime }\right)$ ,其中 ${\psi }_{a}\left( {\mathbf{x}}^{\prime }\right) = \left\langle {{\mathbf{x}}^{\prime } | {a}^{\prime }}\right\rangle$ .
		\begin{proof}
			$\sum_{a}\psi_{a}^{*}(\mathbf{x}')\psi_{a}(\mathbf{x}'')=\sum_{a}\langle\mathbf{x}'|a\rangle^{*}\langle\mathbf{x}''|a\rangle=\sum_{a}\langle\mathbf{x}''|a\rangle\langle a|\mathbf{x}'\rangle=\langle\mathbf{x}''|\mathbf{x}'\rangle=\delta(\mathbf{x}''-\mathbf{x}')$
		\end{proof}
	\end{enumerate}
	\item
	\begin{enumerate}
		\item 考虑两个右矢 $\left| {\alpha \rangle \text{和}}\right| \beta \rangle$ . 假定 $\left\langle {{\alpha }^{\prime } | \alpha }\right\rangle ,\left\langle {{\alpha }^{\prime \prime } | \alpha }\right\rangle ,\cdots$ 和 $\left\langle {{\alpha }^{\prime } | \beta }\right\rangle ,\left\langle {{\alpha }^{\prime \prime } | \beta }\right\rangle ,\cdots$ 均为已知,其中 $\left| {a}^{\prime }\right\rangle ,\left| {a}^{\prime \prime }\right\rangle ,\cdots$ 组成基右矢的完备集. 求在该基下算符 $|\alpha \rangle \langle \beta |$ 的矩阵表示.
		\begin{proof}
			第一问较为简单,我们只需要写出矩阵元$X=|\alpha \rangle \langle \beta |$,$X_{ij}=\langle a_i|\alpha \rangle \langle \beta |a_j\rangle=\langle a_i|\alpha \rangle \langle a_j |\beta\rangle^*$即可.
		\end{proof}
		\item 现在考虑一个自旋 $\frac{1}{2}$ 系统,设 $|\alpha \rangle$ 和 $|\beta \rangle$ 分别为 $\left| {{s}_{z} = \hbar /2}\right\rangle$ 和 $\left| {{s}_{x} = \hbar /2}\right\rangle$ 态. 写出在通常 $\left( {s}_{z}\right.$ 对角) 的基下,与 $\left| {\alpha \rangle \langle \beta }\right|$ 对应的方阵的显示式.
		\begin{proof}
			我们仅需把之前的结论带入即可
			\begin{equation}
				\left.|S_z=\hbar/2\rangle\langle S_x=\hbar/2|\doteq\frac{1}{\sqrt{2}}\left(\begin{array}{cc}1&1\\0&0\end{array}\right.\right)
			\end{equation}
		\end{proof}
	\end{enumerate}
	\item 假定 $\left| {i\rangle \text{和}}\right| j\rangle$ 都是某厄米算符 $A$ 的本征右矢. 在什么条件下 $\left| {i\rangle + }\right| j\rangle$ 也是 $A$ 的一个本征右矢. 证明答案的正确性.
	\begin{proof}
		仅当本征值相同时正确.
	\end{proof}
	\item 利用 $\left| {+\rangle \text{和}}\right| - \rangle$ 的正交性证明
	$$
	\left\lbrack {{S}_{i},{S}_{j}}\right\rbrack = i{\varepsilon }_{ijk}\hbar {S}_{k},\;\left\{ {{S}_{i},{S}_{j}}\right\} = \left( \frac{{\hbar }^{2}}{2}\right) {\delta }_{ij},
	$$
	其中
	$$
	{S}_{x} = \frac{\hbar }{2}\left( {\left| {+\rangle \langle - }\right| + \left| {-\rangle \langle + }\right| }\right) ,\;{S}_{y} = \frac{i\hbar }{2}\left( {-\left| {+\rangle \langle - }\right| + \left| {-\rangle \langle + }\right| }\right) ,
	{S}_{z} = \frac{\hbar }{2}\left( {\left| {+\rangle \langle + }\right| - \left| {-\rangle \langle - }\right| }\right) .
	$$
	\begin{proof}
		本题仅需要理解各个符号的含义并带入即可,一个可以有一定作用的方法是令$\langle +|-\rangle=\langle -|+\rangle=0,\langle +|+\rangle=\langle -|-\rangle=1$,这样可以减少其中的运算量.
	\end{proof}
	\item 一个双态系统其哈密顿算符由下式给出
	$$
	H = a\left( {\left| {1\rangle \langle 1}\right| - \left| {2\rangle \langle 2}\right| + \left| {1\rangle \langle 2}\right| + \left| {2\rangle \langle 1}\right| }\right) ,
	$$
	其中 $a$ 是一个数,其量纲为能量. 求能量的本征值和相应的能量本征右矢 (作为 $|1\rangle$ 和 $|2\rangle$ 的线性组合).
	\begin{proof}	
		我们先将其写成矩阵表示的形式
		\begin{equation}
			H\doteq\left[\begin{array}{rr}a&a\\a&-a\end{array}\right]
		\end{equation}
		本征值 $E$ 满足 $( a- E) ( - a- E) - a^2= - 2a^2+ E^2= 0$ 即 $E= \pm a\sqrt {2}.$ 令$x_1$ 和 $x_2$为本征右矢的两个元素.对于$E=+a\sqrt{2}\equiv E^{(1)},(1-\sqrt{2})x_{1}^{(1)}+x_{2}^{(1)}=0$,而对于$E=-a\sqrt2\equiv E^{(2)},(1+\sqrt2)x_1^{(2)}+x_2^{(2)}=0.$ 于是本征态可以被表示为
		\begin{equation}
			\left.|E^{(1)}\rangle\doteq N^{(1)}\left[\begin{array}{c}1\\\sqrt{2}-1\end{array}\right.\right]\quad\text{ 及 }\quad|E^{(2)}\rangle\doteq N^{(2)}\left[\begin{array}{c}-1\\\sqrt{2}+1\end{array}\right]
		\end{equation}
		其中 $N^{(1)^2}=1/(4-2\sqrt{2})$ 和 $N^{(2)^2}=1/(4+2\sqrt{2}).$
	\end{proof}
	\item 已知一个自旋 $\frac{1}{2}$ 的系统处于 $\mathbf{S} \cdot \widehat{\mathbf{n}}$ 的一个本征态,其本征值为 $\hbar/2$ ,其中 $\widehat{\mathbf{n}}$ 为 ${xz}$ 平面上的一个单位矢量,与正 $z$ 轴夹 $\gamma$ 角.
	\begin{enumerate}
		\item 假定已测得 ${S}_{x}$ . 得到 $+ \hbar /2$ 的概率是什么?
		\begin{proof}
			这里我们直接给出本征值为$\hbar/2$的$\mathbf{S} \cdot \widehat{\mathbf{n}}$态的构造,可以自行尝试构造.
			\begin{equation}
				\cos\left(\frac{\beta}{2}\right)|+\rangle+\sin\left(\frac{\beta}{2}\right)e^{i\alpha}|-\rangle 
			\end{equation}
			其中$\alpha$和$\beta$分别对应球坐标系下的$(r,\theta,\varphi)$中的后两项,在本题中$\alpha=0,\beta=\gamma$,于是有
			\begin{equation}
				\left|\left[\frac{1}{\sqrt{2}}\langle+|+\frac{1}{\sqrt{2}}\langle-|\right]\left[\cos\frac{\gamma}{2}|+\rangle+\sin\frac{\gamma}{2}|-\rangle\right]\right|^2=\frac{1}{2}\left[\sqrt{\frac{1+\cos\gamma}{2}}+\sqrt{\frac{1-\cos\gamma}{2}}\right]^2=\frac{1+\sin\gamma}{2}
			\end{equation}
		\end{proof}
		\item 计算 ${S}_{x}$ 的弥散度,即
		$$
		\left\langle {\left( {S}_{x} - \left\langle {S}_{x}\right\rangle \right) }^{2}\right\rangle \text{.}
		$$
		(为了让你自己放心起见,验证在 $\gamma = 0,\pi /2$ 和 $\pi$ 等特殊情况下的答案.)
		\begin{proof}
			由第9题我们容易得出$S^2_x=\frac{\hbar^2}4$并且$S_x$的期望
			\begin{equation}
				\left|\left[\frac{1}{\sqrt{2}}\langle+|+\frac{1}{\sqrt{2}}\langle-|\right]\left[\cos\frac{\gamma}{2}|+\rangle+\sin\frac{\gamma}{2}|-\rangle\right]\right|^2=\frac{1}{2}\left[\sqrt{\frac{1+\cos\gamma}{2}}+\sqrt{\frac{1-\cos\gamma}{2}}\right]^2=\frac{1+\sin\gamma}{2}
			\end{equation}
			于是原式为
			\begin{equation}
				\langle(S_{x}-\langle S_{x}\rangle)^{2}\rangle=\hbar^{2}(1-\sin^{2}\gamma)/4=\hbar^{2}\cos^{2}\gamma/4=\hbar^{2}/4,0,\hbar^{2}4\text{分别对于}\gamma=0,\pi/2,\pi.
			\end{equation}
		\end{proof}
	\end{enumerate}
	\item 设 $A$ 和 $B$ 是两个可观测量. 假定 $A$ 和 $B$ 的共同本征右矢 $\left\{ \left| {{a}^{\prime },{b}^{\prime }}\right\rangle \right\}$ 构成一组正交完备的基右矢集合. 我们是否总可以得出结论
	$$
	\left\lbrack {A, B}\right\rbrack = 0?
	$$
	如果你的答案是可以, 证明这一论断. 如果你的答案是不可以,举出一个反例.
	\begin{proof}
		可以,我们利用单位算符即可得出
		\begin{equation}
			AB=AB\textbf{1}=AB\sum_{a^\prime,b^\prime}|a^\prime,b^\prime\rangle\langle a^\prime,b^\prime|=A\sum_{a^\prime,b^\prime}b^\prime|a^\prime,b^\prime\rangle\langle a^\prime,b^\prime|=\sum_{a^\prime,b^\prime}b^\prime a^\prime|a^\prime,b^\prime\rangle\langle a^\prime,b^\prime|=BA
		\end{equation}
		我们发现完备性具有强大的作用.需要着重指出的是,为了涵盖完整的集合组,求和必须同时涵盖 a 和 b .
	\end{proof}
	\item 两个厄米算符反对易
	$$
	\{ A, B\} = {AB} + {BA} = 0.
	$$
	能够存在一个 $A$ 和 $B$ 的同时 (即,共同) 的本征右矢吗? 证明或举例说明你的论断.
	\begin{proof}
		由反对易关系,$AB=-BA$,如果存在共同本征右矢,需要满足$AB|a,b\rangle=ab|a,b\rangle=BA|a,b\rangle $,则必须有$ab=-ba$,我们同时注意到$a,b$为实数,则除非$a,b$至少有一个为$0$.
	\end{proof}
	\item 已知两个可观测量 ${A}_{1}$ 和 ${A}_{2}$ 均不显含时间,且相互不对易
	$$
	\left\lbrack {{A}_{1},{A}_{2}}\right\rbrack \neq 0,
	$$
	我们还知道 ${A}_{1}$ 和 ${A}_{2}$ 均与哈密顿量对易:
	$$
	\left\lbrack {{A}_{1}, H}\right\rbrack = 0.\;\left\lbrack {{A}_{2}, H}\right\rbrack = 0.
	$$
	证明, 在一般情况下, 能量本征态是简并的. 存在例外吗? 
	\begin{proof}
		略
	\end{proof}
	\item 找出使不确定度乘积
	$$
	\left\langle {\left( \Delta {S}_{x}\right) }^{2}\right\rangle \left\langle {\left( \Delta {S}_{y}\right) }^{2}\right\rangle 
	$$
	取最大值的右矢 $\left| {+\rangle \text{和}}\right| - \rangle$ 的线性组合. 直接证明,你所找到的这个线性组合不破坏 ${S}_{x}$ 和 ${S}_{y}$ 的不确定度关系.
	\begin{proof}
		略
	\end{proof}
	\item 对于一个禁闭于两个刚性壁之间的一维粒子,
	$$
	V = \left\{ \begin{array}{ll} 0 & \text{ 对于 }0 < x < a, \\ \infty & \text{ 其他 } \end{array}\right.
	$$
	求出 $x - p$ 不确定度乘积 $\left\langle {\left( \Delta x\right) }^{2}\right\rangle \left\langle {\left( \Delta p\right) }^{2}\right\rangle$ 的值. 对基态和激发态都进行求解.
	\begin{proof}
		略
	\end{proof}
	\item 考虑一个三维右矢空间. 如果某一组正交的右矢集合,比如 $\left| {1\rangle ,}\right| 2\rangle$ 和 $|3\rangle$ ,用作基右矢,算符 $A$ 和 $B$ 由
	$$
	A \doteq \left( \begin{matrix} a & 0 & 0 \\ 0 & - a & 0 \\ 0 & 0 & - a \end{matrix}\right) ,\;B \doteq \left( \begin{matrix} b & 0 & 0 \\ 0 & 0 & - {ib} \\ 0 & {ib} & 0 \end{matrix}\right)
	$$
	表示,其中 $a$ 和 $b$ 都是实数.
	\begin{enumerate}
		\item 显然, $A$ 展示了一个简并的谱. $B$ 也展示了简并的谱吗?
		\begin{proof}
			略
		\end{proof}
		\item 证明 $A$ 和 $B$ 对易.
		\begin{proof}
			略
		\end{proof}
		\item 找到一组新的正交归一右矢集合,它们是 $A$ 和 $B$ 的共同本征右矢. 具体确定在这三个本征右矢的每一个本征右矢上 $A$ 和 $B$ 的本征值. 你确定的本征值能完全地表征每个本征右矢吗?
		\begin{proof}
			略
		\end{proof}
	\end{enumerate}
	\item 构造一个变换矩阵,它把 ${S}_{z}$ 对角的基和 ${S}_{x}$ 对角的基联系起来. 证明你的结果与下列的普遍关系式自洽:
	$$
	U = \mathop{\sum }\limits_{r}\left| {b}^{\left( r\right) }\right\rangle \left\langle {a}^{\left( r\right) }\right| .
	$$
	\begin{proof}
		略
	\end{proof}
	\item 一个有限的 (空间) 位移的平移算符由
	$$
	\mathcal{J}\left( \mathbf{I}\right) = \exp \left( \frac{-i\mathbf{p} \cdot \mathbf{I}}{\hbar }\right) ,
	$$
	给出,其中 $\mathbf{p}$ 是动量算符.
	\begin{enumerate}
		\item 求
		$$
		\left\lbrack {{x}_{i}, f\left( 1\right) }\right\rbrack \text{.}
		$$
		\begin{proof}
			略
		\end{proof}
		\item 利用上一问, 展示期望 $\langle \mathbf{x}\rangle$ 在平移下如何改变.
		\begin{proof}
			略
		\end{proof}
	\end{enumerate}
	\item 
	\begin{enumerate}
		\item 证明下列各式:
		\begin{enumerate}
			\item $\left\langle {{p}^{\prime }\left| x\right| \alpha }\right\rangle = i\hbar \frac{\partial }{\partial {p}^{\prime }}\left\langle {{p}^{\prime } | \alpha }\right\rangle$
			\begin{proof}
				略
			\end{proof}
			\item $\left\langle {{\left. \beta \right| }_{x} | \alpha }\right\rangle = \int \d{p}^{\prime }{\phi }_{\beta }\left( {p}^{\prime }\right) i\hbar \frac{\partial }{\partial {p}^{\prime }}{\phi }_{a}\left( {p}^{\prime }\right)$ . 其中 ${\phi }_{a}\left( {p}^{\prime }\right) = \left\langle {{p}^{\prime } | \alpha }\right\rangle$ 和 ${\phi }_{\beta }\left( {p}^{\prime }\right) = \left\langle {{p}^{\prime } | \beta }\right\rangle$ 都是动量空间波函数.
			\begin{proof}
				略
			\end{proof}
		\end{enumerate}
		\item $$
		\exp \left( \frac{ix\Xi }{\hbar }\right)
		$$
		
		的物理意义是什么,其中 $x$ 是位置算符,而 $\Xi$ 是某个量纲为动量的数? 证明你的答案的正确性.
		\begin{proof}
			略
		\end{proof}
	\end{enumerate}
\end{enumerate}
\section{第二章}
1
\chapter{致谢/参考}
\section{致谢}
感谢elegantbook所提供的模板,\href{https://elegantlatex.org/}{https://elegantlatex.org/}.\\

\section{参考}
本文主要参考的书籍和期刊如下:
\begin{enumerate}
	\item Modern Quantum Mechanics 2nd.J.J.Sakurai
	\item Quantum Field Theory in Condensed Matter Physics 2nd.Alexei M.Tsvellk
	\item Entanglement in Many-Body Systems
	\item 物理学家用李群李代数
	\item 物理学中的泛函分析
	\item Nicolas Dupuis - Field Theory of Condensed Matter and Ultracold Gases
	\item Conformal Field Theory A.N. Schellekens
\end{enumerate}


	
	
	
	
\ifx\allfiles\undefined
\end{document}
	\else
	\fi
