\ifx\allfiles\undefined

% 如果有这一部分另外的package,在这里加上
% 没有的话不需要

\begin{document}
	\else
	\fi
\chapter{二次量子化}
\begin{introduction}
	\item 角动量对易关系
	\item $\frac12$自旋
	\item SO(3)和SU(2)
	\item 系综
\end{introduction}
这一章关注角动量理论和相关论题的系统处理. 在现代物理中角动量理论的重要性怎么强调也不过分. 在分子、原子及核谱学中, 彻底理解角动量是极为重要的; 角动量的考虑不仅在散射和碰撞问题中, 也在束缚态问题中起着重要的作用. 此外, 角动量概念还有着重要的推广——核物理中的同位旋,粒子物理中的 $\mathrm{{SU}}\left( 3\right) ,\mathrm{{SU}}\left( 2\right) \otimes \mathrm{U}\left( 1\right)$ 等等.

\section{密度算符和纯系综与混合系综}
\subsection{极化束与非极化束}

至此, 已发展的量子力学形式关于一个全同制备的物理系统的系综, 即一个集合作出了统计的预言. 更精确地说,在这样的一个系综内假定所有的成员都由同样的态右矢 $|\alpha \rangle$ 所表征. 通过 $\mathrm{{SG}}$ 过滤器的一束银原子就是这种系综的一个好例子. 束流中的每个原子其自旋指向相同的方向, 即由该过滤器的非均匀磁场确定的方向. 我们迄今还没有讨论过如何用量子力学描述这样的一个系综,其中的某些物理系统,比如 ${60}\%$ ,由 $|\alpha \rangle$ 表征,而其余的 ${40}\%$ 由某个其他的右矢 $|\beta \rangle$ 表征.

为了生动地说明到此为止发展的形式体系的不完备性, 让我们考虑直接由热炉子飞出的银原子, 它们将受到斯特恩-盖拉赫类型的过滤器的作用. 基于对称性, 我们预期这样的原子具有随机的自旋取向; 换句话说, 没有任何优势的方向与这样的一个原子系综相联系. 按照到此为止发展的形式体系,一个自旋 $\frac{1}{2}$ 系统的最普遍的态右矢由下式给出

$$
\left| {\alpha \rangle = {c}_{ + }}\right| + \rangle + {c}_{ - }| - \rangle tag{3. 4.1}
$$

这个方程能够描写有着随机自旋取向的集团吗? 显然答案是不能. (3.4.1) 式表征一个这样的态右矢,它的自旋指向某个确定的方向,即沿 $\widehat{\mathbf{n}}$ 方向,它的极角和方位角分别为 $\beta$ 和 $\alpha$ ,它们可以通过求解

$$
\frac{{c}_{ + }}{{c}_{ - }} = \frac{\cos \left( {\beta /2}\right) }{{e}^{i\alpha }\sin \left( {\beta /2}\right) }, tag{3. 4.2}
$$

得到, 见 (3.2.52) 式.

为了应对这类情况, 我们引入分数布居或概率权重的概念. 一个有着完全随机自旋取向的银原子系综可以看作是这样的一团银原子,其中 ${50}\%$ 的系综成员由 $| + \rangle$ 表征,而其余的 ${50}\%$ 由 $| - \rangle$ 表征. 我们通过指定

$$
{w}_{ + } = {0.5},\;{w}_{ - } = {0.5}, tag{3.4.3}
$$

其中 ${w}_{ + }$ 和 ${w}_{ - }$ 分别为自旋向上和向下的分数布居,来确定这样的一个系综. 因为这样的一束束流不存在任何优势方向,可合理地预期这同一系综也可以被看作 $\left| {{S}_{x}; + }\right\rangle$ 和 ${S}_{x}$ ; 一) 一半对一半的混合. 不久就会见到实现这一点所需要的数学形式.

---

* 这当然依赖于我们的约定: ${S}_{y}$ (或更普遍的 ${J}_{y}$ ) 的矩阵元取为纯虚的.

---

非常重要的是要注意,我们只不过引进了两个实数 ${w}_{ + }$ 和 ${w}_{ - }$ . 没有任何有关自旋向上和自旋向下的右矢间的相对相位信息. 通常, 我们称这样的情况为自旋向上和自旋向下状态的非相干混合. 在这里我们正在做的是要把我们在相干线性叠加时所做的清楚地区分开——例如,

$$
\left( \frac{1}{\sqrt{2}}\right) + \rangle + \left( \frac{1}{\sqrt{2}}\right) - \rangle , tag{3. 4.4}
$$

在那里, $\left| {+\rangle \text{和}}\right| - \rangle$ 之间的相位关系包含了有关在 ${xy}$ 平面中自旋取向的至关重要的信息, 在这种情况下,自旋沿正 $x$ 方向. 一般说来,我们不应把 ${w}_{ + }$ 和 ${w}_{ - }$ 与 ${\left| {c}_{ + }\right| }^{2}$ 和 ${\left| {c}_{ - }\right| }^{2}$ 混淆. 与 ${w}_{ + }$ 和 ${w}_{ - }$ 联系起来的概率概念更靠近经典概率论中遇到的那种. 在处理直接来自热炉子的银原子所遇到的情况可以与一个毕业班中 ${50}\%$ 的毕业生是男生,其余的 ${50}\%$ 是女生的情况相比. 当我们随机地挑选出一个学生时, 这位特定的学生是男生 (或女生) 的概率是 0.5 . 究竟谁听说过一个学生可视为男生与女生以一种特定的相位关系线性相关叠加吗?

直接从炉子飞出来的银原子束流是一个完全随机系综的例子; 该束流被称为非极化的, 因为其自旋取向没有优势方向. 相比之下, 通过了一个有选择的斯特恩-盖拉赫类测量的束流是一个纯系综的例子; 该束流称为极化束, 因为这个系综的所有成员由一个单一的共同右矢所表征, 它描述自旋指向某确定方向的一个态. 为了领会完全随机系综和纯系综之间的区别, 让我们考虑一个可以转动的 SG 仪器, 在那里通过转动该仪器就可以改变非均匀磁场 $\mathbf{B}$ 的方向. 当直接从炉子飞出来的完全非极化的束流遇到这样一个仪器时,不管仪器可能是什么样的取向, 我们总是得到两个强度相等的出射束流. 相比之下, 如果一个极化束流遭遇到这样一个仪器时, 两个出射束流的相对强度随着仪器的转动而变化. 对于某个特定的取向, 实际上的强度之比变为 1 到 0 间的值. 事实上, 第 1 章所发展的形式体系告诉我们,这个相对强度只不过是 ${\cos }^{2}\left( {\beta /2}\right)$ 和 ${\sin }^{2}\left( {\beta /2}\right)$ ,其中 $\beta$ 是原子的自旋方向和 SG 仪器中的非均匀磁场方向之间的夹角.

一个完全随机系综和一个纯系综可以看成所谓混合系综的两个极端. 在一个混合系综中,其成员的某个部分,例如 ${70}\%$ 由一个态右矢 $|\alpha \rangle$ 表征,而其余的 ${30}\%$ 用 $|\beta \rangle$ 表征. 在这样的情况下,该束流称为部分极化. 这里的 $|\alpha \rangle$ 和 $|\beta \rangle$ 甚至不需要是正交的,例如, 有 ${70}\%$ 的自旋沿正 $x$ 方向,而 ${30}\%$ 的自旋沿负 $z$ 方向 *.

\subsection{系综平均和密度算符}

现在介绍冯$\cdot$诺依曼 (J. von Neumann) 在 1927 年首先提出来的密度算符形式, 它可定量地描述纯系综以及混合系综的物理情况. 在这里, 我们的一般性讨论不只限于自旋 $\frac{1}{2}$ 系统,但是为了举例说明的目的,我们会反复回到自旋 $\frac{1}{2}$ 的系统.

一个纯系综被定义为一个这样的物理系统的集合,它的每个成员都由同样的右矢 $|\alpha \rangle$ 表征. 相比之下,在一个混合系综内,相对布居数为 ${w}_{1}$ 的那部分成员由 $\left| {\alpha }^{\left( 1\right) }\right\rangle$ 表征,而相对布居数为 ${w}_{2}$ 的其他部分成员由 $\left| {\alpha }^{\left( 2\right) }\right\rangle$ 表征,等等. 粗略地讲,一个混合系综可以看作纯系综的混合, 正如它的名字所示. 布居数受到满足归一条件的约束, 即

$$
\mathop{\sum }\limits_{i}{w}_{i} = 1 tag{3. 4.5}
$$

---

* 在文献中, 称之为纯的或混合的系综经常指的是纯态和混态. 然而, 在本书中我们使用态意味着用一个确定的态右矢 $|\alpha \rangle$ 描写的物理系统.

---

正如我们在前面提到的, $\left| {\alpha }^{\left( 1\right) }\right\rangle$ 和 $\left| {\alpha }^{\left( 2\right) }\right\rangle$ 是不需要正交的. 此外在 (3.4.5) 式中,对 $i$ 求和中的项数不需要与右矢空间的维数 $N$ 一致,它可以很容易超过 $N$ . 例如,对 $N = 2$ 的自旋 $\frac{1}{2}$ 系统,可以考虑有 ${40}\%$ 的态具有沿正 $z$ 方向的自旋, ${30}\%$ 的态具有沿正 $x$ 方向的自旋,而其余 ${30}\%$ 的态具有沿负 $y$ 方向的自旋.

假定在一个混合系综上测量某可观测量 $A$ . 我们可以问: 当测量的次数很多时, $A$ 的平均测量值是多少. 答案由 $A$ 的系综平均值确定,它被定义为

$$
\left\lbrack A\right\rbrack \equiv \mathop{\sum }\limits_{i}{w}_{i}\left\langle {{\alpha }^{\left( i\right) }\left| A\right| {\alpha }^{\left( i\right) }}\right\rangle tag{3. 4.6}
$$

$$
= \mathop{\sum }\limits_{i}\mathop{\sum }\limits_{{a}^{\prime }}{w}_{i}{\left| \left\langle {a}^{\prime } \mid {\alpha }^{\left( i\right) }\right\rangle \right| }^{2}{a}^{\prime },
$$

其中 $\left| {\alpha }^{\prime }\right\rangle$ 是 $A$ 的一个本征右矢. 回想 $\left\langle {{\alpha }^{\left( i\right) }\left| A\right| {\alpha }^{\left( i\right) }}\right\rangle$ 是通常量子力学中 $A$ 在 $\left| {\alpha }^{\left( i\right) }\right\rangle$ 上的期待值. (3.4.6) 式告诉我们,这些期待值必须再用相应的分数布居数 ${w}_{i}$ 加权. 注意,概率概念是如何进来两次的: 第一次是在 $A$ 的一个本征态 $\left| {\alpha }^{\prime }\right\rangle$ 中找到态 $\left| {\alpha }^{\left( i\right) }\right\rangle$ 的量子力学概率 ${\left| \left\langle {a}^{\prime }\right\rangle {\alpha }^{\left( i\right) }\right| }^{2}$ ,而第二次是在系综中找到一个由 $\left| {\alpha }^{\left( i\right) }\right\rangle$ 表征的量子力学态的概率因子 *.

现在可以利用一个更一般的基 $\left\{ \left| {b}^{\prime }\right\rangle \right\}$ 改写系综平均值 (3.4.6) 式:

$$
\left\lbrack A\right\rbrack = \mathop{\sum }\limits_{i}{w}_{i}\mathop{\sum }\limits_{{b}^{\prime }}\mathop{\sum }\limits_{{b}^{\prime \prime }}\left\langle {{\alpha }^{\left( i\right) }\left| {{b}^{\prime }\rangle \left\langle {{b}^{\prime }\left| A\right| {b}^{\prime \prime }}\right\rangle \left\langle {b}^{\prime \prime }\right| {\alpha }^{\left( i\right) }}\right| }\right\rangle tag{3.4.7}
$$

$$
= \mathop{\sum }\limits_{{b}^{\prime }}\mathop{\sum }\limits_{{b}^{\prime \prime }}\left( {\mathop{\sum }\limits_{i}{w}_{i}\left\langle {{b}^{\prime \prime }\left| {\alpha }^{\left( i\right) }\right\rangle \left\langle {{\alpha }^{\left( i\right) } \mid {b}^{\prime }}\right\rangle }\right\rangle \left\langle {{b}^{\prime }\left| A\right| {b}^{\prime \prime }}\right\rangle }\right) .
$$

在对 ${b}^{\prime }\left( {b}^{\prime \prime }\right)$ 求和中的项数就是右矢空间的维数,而对 $i$ 求和中的项数依赖于这个混合系综是怎么被看作纯系综的混合的. 注意,在这种形式中,不依赖于特定可观测量 $A$ 的系综的基本性质被分离了出来. 这诱使我们把密度算符 $\rho$ 定义如下:

$$
\rho \equiv \mathop{\sum }\limits_{i}{w}_{i}\left| {\alpha }^{\left( i\right) }\right\rangle \left\langle {\alpha }^{\left( i\right) }\right| . tag{3.4.8}
$$

相应的密度矩阵的矩阵元有下列形式:

$$
\left\langle {{b}^{\prime \prime }\left| \rho \right| {b}^{\prime }}\right\rangle = \mathop{\sum }\limits_{i}{w}_{i}\left\langle {{b}^{\prime \prime }\left| {\alpha }^{\left( i\right) }\right\rangle \left\langle {\alpha }^{\left( i\right) }\right| {b}^{\prime }}\right\rangle . tag{3.4.9}
$$

密度算符包含了我们可能得到的, 关于所论及系综的所有有物理意义的信息. 回到 (3.4.7) 式, 系综平均可以写成

$$
\left\lbrack A\right\rbrack = \mathop{\sum }\limits_{{b}^{\prime }}\mathop{\sum }\limits_{{b}^{\prime \prime }}\left\langle {{b}^{\prime \prime }\left| \rho \right| {b}^{\prime }}\right\rangle \left\langle {{b}^{\prime }\left| A\right| {b}^{\prime \prime }}\right\rangle tag{3. 4. 10}
$$

$$
= \operatorname{tr}\left( {\rho A}\right) \text{.}
$$

因为矩阵的迹不依赖于表象, $\operatorname{tr}\left( {\rho A}\right)$ 可以利用任何方便的基计算. 因此,(3.4.10) 式是一个极为有用的关系.

密度算符有两个性质值得记住. 首先, 密度算符是厄米的, 这从 (3.4.8) 式显然可见. 第二, 密度算符满足归一化条件

- 在文献中相当常见, 系综平均也称为期待值. 然而, 在本书中, 术语期待值是指在一个纯系综中进行测量时得到的平均测量值.

$$
\operatorname{tr}\left( \rho \right) = \mathop{\sum }\limits_{i}\mathop{\sum }\limits_{{b}^{\prime }}{w}_{i}\left\langle {{b}^{\prime }\left| {{\alpha }^{\left( i\right) }\rangle \left\langle {{\alpha }^{\left( i\right) } \mid {b}^{\prime }}\right\rangle }\right| }\right\rangle
$$

$$
= \mathop{\sum }\limits_{i}{w}_{i}\left\langle {{\alpha }^{\left( i\right) } \mid {\alpha }^{\left( i\right) }}\right\rangle tag{3. 4.11}
$$

$$
= 1\text{.}
$$

由于厄米性和归一条件,对于维数为 2 的自旋 $\frac{1}{2}$ 的系统,密度算符或相应的密度矩阵由三个实参量表征. 4 个实数可以表征一个 $2 \times 2$ 厄米矩阵. 然而,由于归一化条件,只有 3 个是独立的. 所需要的 3 个数是 $\left\lbrack {S}_{x}\right\rbrack ,\left\lbrack {S}_{y}\right\rbrack$ 和 $\left\lbrack {S}_{z}\right\rbrack$ ,读者可以证明知道了这三个系综平均值, 足以重建密度算符. 一个混合系综形成的方式可以是相当复杂的. 我们可以把有所有类型的 $\left| {\alpha }^{\left( i\right) }\right\rangle$ 所表征的纯系综用适当的 ${w}_{i}$ 混合起来; 而对自旋 $\frac{1}{2}$ 的系统,三个实数完全可以表征所涉及的系综. 这强烈地暗示, 一个混合系综能以许多种不同的方式分解为纯系综. 在本章末尾有一个习题说明了这一点.

一个纯系综可以通过对某个 $\left| {\alpha }^{\left( i\right) }\right\rangle$ 一一例如, $i = n$ 时, ${w}_{i} = 1$ ,而对所有的其他可能的态右矢 ${w}_{i} = 0$ 来确定,因此相应的密度算符可写成不求和的

$$
\rho = \left| {\alpha }^{\left( n\right) }\right\rangle \left\langle {\alpha }^{\left( n\right) }\right| tag{3. 4.12}
$$

显然, 一个纯系综的密度算符是幂等的. 即

$$
{\rho }^{2} = \rho tag{3. 4.13}
$$

或, 等价地

$$
\rho \left( {\rho - 1}\right) = 0. tag{3. 4.14}
$$

于是, 除了 (3.4.11) 式以外, 只对纯系综有

$$
\operatorname{tr}\left( {\rho }^{2}\right) = 1\text{.} tag{3. 4.15}
$$

纯系综密度算符的本征值是 0 或 1,这一点可以通过在 (3.4.14) 式的 $\rho$ 和 $\rho - 1$ 之间插入一个基右矢完全集合看到,该集合能对角化厄米算符 $\rho$ . 因此,当对角化时,一个纯系综的密度矩阵看起来一定像

$$
\rho \doteq \left( \begin{matrix} 0 & & & & & & & & 0 \\ & 0 & & & & & & & \\ & & \ddots & & & & & & \\ & & & 0 & & & & & \\ & & & & 1 & & & & \\ & & & & & 0 & & & \\ & & & & & & 0 & & \\ & & & & & & & 0 & \\ & & & & & & & & \ddots \\ 0 & & & & & & & & 0 \end{matrix}\right) \text{ (对角形式) } tag{3. 4.16}
$$

可以证明,当系综是纯的时候, $\operatorname{tr}\left( {\rho }^{2}\right)$ 取最大值; 对于一个混合系综, $\operatorname{tr}\left( {\rho }^{2}\right)$ 是一个小于 1 的正数.

给定一个密度算符, 让我们看一下在一些具体的基中, 怎样构建相应的密度矩阵. 为此, 首先回顾

$$
\left| {\alpha \rangle \langle \alpha }\right| = \mathop{\sum }\limits_{{b}^{\prime }}\mathop{\sum }\limits_{{b}^{\prime \prime }}\left| {b}^{\prime }\right\rangle \left\langle {{b}^{\prime } \mid \alpha }\right\rangle \left\langle {\alpha \mid {b}^{\prime \prime }}\right\rangle \left\langle {b}^{\prime \prime }\right| . tag{3. 4.17}
$$

这表明,在外积的意义上,可以通过把 $\left\langle {{b}^{\prime } \mid {\alpha }^{\left( i\right) }}\right\rangle$ 形成的列矩阵与 $\left\langle {{\alpha }^{\left( i\right) } \mid {b}^{\prime \prime }}\right\rangle$ (当然,它等于 $\left\langle {{b}^{\prime \prime } \mid {\alpha }^{\left( i\right) }{\rangle }^{ * }\text{. }}\right\rangle$ 形成的行矩阵组合在一起,形成一个对应于 $\left| {{\alpha }^{\left( i\right) }\rangle \left\langle {\alpha }^{\left( i\right) }\right. }\right|$ 的方阵. 最后一步,就像 (3.4.8) 式指出的,用权重因子 ${w}_{i}$ 对这样的方阵求和. 正如所期待的,最后的形式与 (3.4.9) 式一致.

研究几个例子是很有益的,它们都涉及自旋 $\frac{1}{2}$ 的系统.

例 3.1 一束 ${S}_{z} +$ 的完全极化束流

$$
\rho = \left| {+\rangle \langle + }\right| \pm \left( \begin{array}{l} 1 \\ 0 \end{array}\right) \left( {1,0}\right) tag{3. 4.18}
$$

$$
= \left( \begin{array}{ll} 1 & 0 \\ 0 & 0 \end{array}\right)
$$

例 3.2 一束 ${S}_{x} \pm$ 的完全极化束流

$$
\rho = \left| {{S}_{x}; \pm }\right\rangle \left\langle {{S}_{x}; \pm }\right| = \left( \frac{1}{\sqrt{2}}\right) \left( {\left| {+\rangle \pm }\right| - \rangle }\right) \left( \frac{1}{\sqrt{2}}\right) \left( {\langle + \left| {\pm \langle - }\right| }\right)
$$

$$
\doteq \left( \begin{matrix} \frac{1}{2} & \pm \frac{1}{2} \\ \pm \frac{1}{2} & \frac{1}{2} \end{matrix}\right) tag{3. 4.19}
$$

例 3.1 和 3.2 的系综都是纯的.

例 3.3 一束非极化束流. 它可以看作一个自旋向上的系综和一个自旋向下的系综以相等权重 (各为 ${50}\%$ ) 非相干混合:

$$
\rho = \left( \frac{1}{2}\right) \left| {+\rangle \langle + }\right| + \left( \frac{1}{2}\right) \left| {-\rangle \langle - }\right|
$$

$$
\doteq \left( \begin{array}{ll} \frac{1}{2} & 0 \\ 0 & \frac{1}{2} \end{array}\right) , tag{3. 4.20}
$$

它正好是一个单位矩阵除以 2 . 正如早些时候我们曾指出的, 同样的这个系综还可以看作是一种具有相等权重的 ${S}_{x}$ 十系综和 ${S}_{x}$ 一系综的非相干混合. 令人欣慰的是,我们的形式自动地满足预期

$$
\left( \begin{matrix} \frac{1}{2} & 0 \\ 0 & \frac{1}{2} \end{matrix}\right) = \frac{1}{2}\left( \begin{matrix} \frac{1}{2} & \frac{1}{2} \\ \frac{1}{2} & \frac{1}{2} \end{matrix}\right) + \frac{1}{2}\left( \begin{matrix} \frac{1}{2} & - \frac{1}{2} \\ - \frac{1}{2} & \frac{1}{2} \end{matrix}\right) , tag{3. 4.21}
$$

由例 3.2 可以看到,右手边的两项就是 ${S}_{x}$ 十和 ${S}_{x}$ 一两个纯系综的密度矩阵. 因为这种情况下的 $\rho$ 正是单位算符除以 2 (维数),我们有

$$
\operatorname{tr}\left( {\rho {S}_{x}}\right) = \operatorname{tr}\left( {\rho {S}_{y}}\right) = \operatorname{tr}\left( {\rho {S}_{z}}\right) = 0, tag{3. 4.22}
$$

其中用到了 ${S}_{k}$ 是无迹的. 于是,对于 $\mathbf{S}$ 的系综平均值,我们有

$$
\left\lbrack \mathbf{S}\right\rbrack = 0. tag{3. 4.23}
$$

这是合理的,因为在一个自旋 $\frac{1}{2}$ 系统的完全随机系综中不应当存在自旋的优势方向.

例 3.4 作为部分极化束的例子,考虑两个纯系综的 75-25 混合,一个是 ${S}_{z} +$ 而另一

个是 ${S}_{x} +$ :

$$
w\left( {{S}_{z} + }\right) = {0.75},\;w\left( {{S}_{x} + }\right) = {0.25}. tag{3. 4.24}
$$

相应的 $\rho$ 可以表示为

$$
\rho \doteq \frac{3}{4}\left( \begin{array}{ll} 1 & 0 \\ 0 & 0 \end{array}\right) + \frac{1}{4}\left( \begin{array}{ll} \frac{1}{2} & \frac{1}{2} \\ \frac{1}{2} & \frac{1}{2} \end{array}\right) tag{3. 4.25}
$$

$$
= \left( \begin{array}{ll} \frac{7}{8} & \frac{1}{8} \\ \frac{1}{8} & \frac{1}{8} \end{array}\right) ,
$$

从上式得到

$$
\left\lbrack {S}_{x}\right\rbrack = \frac{\hslash }{8},\;\left\lbrack {S}_{y}\right\rbrack = 0,\;\left\lbrack {S}_{z}\right\rbrack = \frac{3\hslash }{8}. tag{3. 4.26}
$$

把证明这个系综能以异于 (3.4.24) 式的方式分解作为一个练习留给读者.

\subsection{系综的时间演化}

作为一个时间函数,密度算符 $\rho$ 如何变化? 让我们假定在某个 ${t}_{0}$ 时刻密度算符由下式给定

$$
\rho \left( {t}_{0}\right) = \mathop{\sum }\limits_{i}{w}_{i}\left| {\alpha }^{\left( i\right) }\right\rangle \left\langle {\alpha }^{\left( i\right) }\right| . tag{3. 4.27}
$$

如果系综保持不受扰动,我们就不可能改变分数布居数 ${w}_{i}$ . 所以 $\rho$ 的变化只是由态右矢 $\left| {\alpha }^{\left( i\right) }\right\rangle$ 的时间演化所控制:

$$
{t}_{0}\text{时}\left. {\left. {\alpha }^{\left( i\right) }\right\rangle \rightarrow \mid {\alpha }^{\left( i\right) },{t}_{0};t}\right\rangle \text{.} tag{3. 4.28}
$$

由于 $\left| {{\alpha }^{\left( i\right) },{t}_{0};t}\right\rangle$ 满足薛定谔方程,我们得到

$$
i\hslash \frac{\partial \rho }{\partial t} = \mathop{\sum }\limits_{i}{w}_{i}\left( {H\left| {{\alpha }^{\left( i\right) },{t}_{0};t}\right\rangle \left\langle {{\alpha }^{\left( i\right) },{t}_{0};t}\right\rangle - \left| {{\alpha }^{\left( i\right) },{t}_{0};t}\right\rangle \left\langle {{\alpha }^{\left( i\right) },{t}_{0};t}\right| H}\right) tag{3. 4.29}
$$

$$
= - \left\lbrack {\rho, H}\right\rbrack \text{.}
$$

除了符号相反,该公式的样子很像海森伯运动方程! 这一点不会带来烦恼,因为 $\rho$ 不是一个海森伯绘景的动力学可观测量. 相反, $\rho$ 是由薛定谔绘景的态右矢和态左矢构成的,它们的时间演化都是遵从薛定谔方程的.

有意思的是, (3.4.29) 式可以看作是经典统计力学中刘维尔定理

$$
\frac{\partial {\rho }_{\text{经典 }}}{\partial t} = - {\left\lbrack {\rho }_{\text{经典 }}, H\right\rbrack }_{\text{经典 }} tag{3.4.30}
$$

的量子力学类似定理,其中的 ${\rho }_{\text{经典 }}$ 是相空间中代表点的密度 *. 因此,对出现在 (3.4.29) 式中的 $\rho$ ,密度算符的名字的确是合适的. 对于某个可观测量 $A$ 的系综平均值,(3.4.10) 式的经典类似公式由下式给定:

$$
{A}_{\text{平均 }} = \frac{\int {\rho }_{\text{经典 }}A\left( {q, p}\right) d{\Gamma }_{q, p}}{\int {\rho }_{\text{经典 }}d{\Gamma }_{q, p}}, tag{3. 4.31}
$$

---

* 记住,一个经典的纯态是用相空间 $\left( {{q}_{1},\cdots ,{q}_{f},{p}_{1},\cdots ,{p}_{f}}\right)$ 中每个时刻的一个单独运动的点表示. 另一方面,一个经典的统计态用我们的非负的密度函数 ${\rho }_{\text{轻典 }}\left( {{q}_{1},\cdots ,{q}_{f},{p}_{1},\cdots ,{p}_{f}, t}\right)$ 描写,它使得在 $t$ 时刻在间隔 $d{q}_{1},\cdots, d{p}_{f}$ 内找到一个系统的概率是 ${\rho }_{\text{经典 }}d{q}_{1},\cdots, d{p}_{f}$ .

---

其中的 $d{\Gamma }_{q.p}$ 表示相空间中的体积元.

\subsection{连续的推广} 

至此, 已经考虑了右矢空间中的密度算符, 在那里基右矢是用某个可观测量的分立的本征值标记的. 密度矩阵的概念可以推广到使用的基右矢用连续的本征值标记. 特别是, 让我们考虑由位置本征右矢 $\left| {\mathbf{x}}^{\prime }\right\rangle$ 所张的右矢空间. (3.4.10) 式的类似表达式由下式给定

$$
\left\lbrack A\right\rbrack = \int {d}^{3}{x}^{\prime }\int {d}^{3}{x}^{\prime \prime }\left\langle {{\mathbf{x}}^{\prime \prime }\left| \rho \right| {\mathbf{x}}^{\prime }}\right\rangle \left\langle {{\mathbf{x}}^{\prime }\left| A\right| {\mathbf{x}}^{\prime \prime }}\right\rangle . tag{3. 4.32}
$$

这里的密度矩阵实际上是 ${\mathbf{x}}^{\prime }$ 和 ${\mathbf{x}}^{\prime \prime }$ 的一个函数,即

$$
\left\langle {{\mathbf{x}}^{\prime \prime }\left| \rho \right| {\mathbf{x}}^{\prime }}\right\rangle = \left\langle {{\mathbf{x}}^{\prime \prime }\left| \left( {\mathop{\sum }\limits_{i}{w}_{i}\left| {\alpha }^{\left( i\right) }\right\rangle \left\langle {\alpha }^{\left( i\right) }\right| }\right) \right| {\mathbf{x}}^{\prime }}\right\rangle tag{3. 4.33}
$$

$$
= \mathop{\sum }\limits_{i}{w}_{i}{\psi }_{i}\left( {\mathbf{x}}^{\prime \prime }\right) {\psi }_{i}^{ * }\left( {\mathbf{x}}^{\prime }\right) ,
$$

其中 ${\psi }_{i}$ 是相应于态右矢 $\left| {\alpha }^{\left( i\right) }\right\rangle$ 的波函数. 注意,该矩阵的对角元 (即 ${\mathbf{x}}^{\prime } = {\mathbf{x}}^{\prime \prime }$ ) 正是概率密度加权后的求和. 再次印证了术语密度矩阵的确是合适的.

在连续态情况下也一样, 重要的是要记住同一个混合系综可以按不同方式分解为纯系综. 例如, 一个 “真实的” 粒子束流既可以看作平面波态 (单能量的自由粒子态) 的混合也可以看作波包态的混合.

\subsection{量子统计力学}

我们以简略讨论密度算符公式形式与统计力学之间的联系结束这一节. 让我们先记下完全随机的系综和纯系综的一些性质. 一个完全随机系综的密度矩阵在任何表象看起来都像

$$
\rho \doteq \frac{1}{N}\left( \begin{array}{llllll} 1 & & & & 0 & \\ & 1 & & & & \\ & & 1 & & & \\ & & & \ddots & & \\ & & & & 1 & \\ & & & & & 1 \\ 0 & & & & & 1 \end{array}\right) tag{3. 4.34}
$$

[比较例 3.3 与 (3.4.20) 式]. 这从对应于写出密度矩阵所用的基右矢的所有的态是同等布居的就可以看到. 相比之下,在 $\rho$ 是对角化的基中,对一个纯系综密度算符的矩阵表示我们有 (3.4.16) 式. 而 (3.4.34) 和 (3.4.16) 这两个对角矩阵, 都满足归一化要求 (3.4.11), 它们看上去不可能有太大差别. 假如我们能够以某种方式构建一个量, 表征这种巨大差异的话, 将会是我们所希望的.

因此, 用

$$
\sigma = - \operatorname{tr}\left( {\rho \ln \rho }\right) . tag{3. 4.35}
$$

定义一个称为 $\sigma$ 的量. 算符 $\rho$ 的对数似乎十分令人生畏,但是如果采用使 $\rho$ 对角化的基, 则 (3.4.35) 式的意义就十分清楚了:

$$
\sigma = - \mathop{\sum }\limits_{k}{\rho }_{kk}^{\left( \text{ 对角 }\right) }\ln {\rho }_{kk}^{\left( \text{ 对角 }\right) }. tag{3. 4.36}
$$

由于每个元素 ${\rho }_{kk}^{\left( \text{ 对 }\right) }$ 都是 0 和 1 之间的实数, $\sigma$ 必然是半正定的. 对于一个完全随机的系综 (3.4.34), 有

$$
\sigma = - \mathop{\sum }\limits_{{k = 1}}^{N}\frac{1}{N}\ln \left( \frac{1}{N}\right) = \ln N. tag{3. 4.37}
$$

与之相比, 对于纯系综 (3.4.16), 有

$$
\sigma = 0 tag{3. 4.38}
$$

其中对 (3.4.36) 式的每一项, 都使用了

$$
{\rho }_{kk}^{\left( \text{ 对角 }\right) } = 0\;\text{ 或 }\;\ln {\rho }_{kk}^{\left( \text{ 对角 }\right) } = 0 tag{3. 4.39}
$$

现在从物理上论证, $\sigma$ 可以被看作一个无序的定量测度. 一个纯系综是一个最大程度的有序系综, 因为它的所有的成员都用同样的量子力学的态右矢表征, 它很像在一支训练有素部队中齐步走的士兵. 按照 (3.4.38) 式,对这样的一个系综, $\sigma$ 为零. 在另一个极端, 一个完全随机系综, 其中所有的量子力学态是同等可能的, 它可以比作一群喝醉了酒的士兵四面八方地游荡. 根据 (3.4.37) 式, $\sigma$ 很大. 的确,稍后将证明,在归一条件

$$
\mathop{\sum }\limits_{k}{\rho }_{kk} = 1 tag{3.4.40}
$$

的约束下, $\sigma$ 最大的可能值是 $\ln N$ . 在热力学中我们知道,一个称为熵的量是量度无序的. 结果是, $\sigma$ 与表示系综每个成员的熵 $S$ 通过

$$
S = {k\sigma } tag{3. 4.41}
$$

相关联,其中 $k$ 是一个可确定为玻尔兹曼 (Boltzman) 常数的普适常数. 事实上, (3.4.41) 式可以作为量子统计力学中熵的定义.

现在证明,对于一个处于热平衡的系综,怎样可以求得密度算符 $\rho$ . 所做的基本假定是,在哈密顿量的系综平均值有某一指定值的约束下,自然界趋向于把 $\sigma$ 极大化. 证明这个假定的正确性, 将使我们卷入一场关于平衡是如何作为与环境相互作用的结果而建立起来的微妙的讨论, 它已超出了本书的范围. 不管怎么样, 一旦热平衡得以确立, 我们预期

$$
\frac{\partial \rho }{\partial t} = 0 tag{3. 4.42}
$$

由于 (3.4.29) 式,上式意味着 $\rho$ 与 $H$ 可以同时对角化. 所以,在写出 (3.4.36) 式时使用的右矢可以被取作能量本征右矢. 采用这一选择, ${\rho }_{kk}$ 则表示一个能量本征值为 ${E}_{k}$ 的能量本征态的分数布居数.

通过要求

$$
{\delta \sigma } = 0 tag{3. 4.43}
$$

极大化 $\sigma$ . 然而,必须考虑 $H$ 的系综平均值具有某一指定值的约束. 用统计力学的说法, $\left\lbrack H\right\rbrack$ 被确定为每个组分的内能,用 $U$ 表示

$$
\left\lbrack H\right\rbrack = \operatorname{tr}\left( {\rho H}\right) = U. tag{3. 4.44}
$$

此外, 不应该忘记归一化约束 (3.4.40) 式. 因此, 基本任务是要求 (3.4.43) 式受到如下两个约束:

$$
\delta \left\lbrack H\right\rbrack = \mathop{\sum }\limits_{k}\delta {\rho }_{kk}{E}_{k} = 0
$$

(3.4. ${45a}$ )

和

$$
\delta \left( {\operatorname{tr}\rho }\right) = \mathop{\sum }\limits_{k}\delta {\rho }_{kk} = 0.
$$

(3. 4. ${45}\mathrm{\;b}$ )

可使用拉格朗日乘子法快捷地完成这个任务. 得到

$$
\mathop{\sum }\limits_{k}\delta {\rho }_{kk}\left\lbrack {\left( {\ln {\rho }_{kk} + 1}\right) + \beta {E}_{k} + \gamma }\right\rbrack = 0, tag{3.4.46}
$$

且仅当

$$
{\rho }_{kk} = \exp \left( {-\beta {E}_{k} - \gamma - 1}\right) . tag{3. 4.47}
$$

时,它才对任意变分都是可能的. 常数 $\gamma$ 可以用归一条件 (3.4.40) 式消掉,最终结果为

$$
{\rho }_{kk} = \frac{\exp \left( {-\beta {E}_{k}}\right) }{\mathop{\sum }\limits_{l}^{N}\exp \left( {-\beta {E}_{l}}\right) }. tag{3. 4.48}
$$

它直接给出一个能量本征值为 ${E}_{k}$ 的能量本征态的分数布居数. 要始终理解求和是对分立的能量本征态进行的; 如果有简并存在, 还必须对具有相同能量本征值的态求和.

密度矩阵元 (3.4.48) 式对于统计力学中所谓的正则系综是适用的. 假如试图在没有内能约束 (3.4.45a) 的条件下极大化 $\sigma$ ,就会相反地得到

$$
{\rho }_{kk} = \frac{1}{N},\;\left( {\text{ 不依赖于 }k}\right) , tag{3.4.49}
$$

它是适用于一个完全随机系综的密度矩阵元. 把 (3.4.48) 式与 (3.4.49) 式比较, 我们推断,一个完全随机的系综可以看作一个正则系综在 $\beta \rightarrow 0$ 时的极限 (即物理上高温极限). (3.4.48) 式的分母为统计力学中的配分函数

$$
Z = \mathop{\sum }\limits_{k}^{N}\exp \left( {-\beta {E}_{k}}\right) tag{3. 4.50}
$$

它还可以写成

$$
Z = \operatorname{tr}\left( {e}^{-{\beta H}}\right) . tag{3. 4.51}
$$

知道了在能量基中给出的 ${\rho }_{kk}$ ,可以把密度算符写成

$$
\rho = \frac{{e}^{-{\beta H}}}{Z}. tag{3. 4.52}
$$

这是最基本的方程,所需的一切都可以从它得出来. 可以立即计算任何一个可观测量 $A$ 的系综平均值:

$$
\left\lbrack A\right\rbrack = \frac{\operatorname{tr}\left( {{e}^{-{\beta H}}A}\right) }{Z}
$$

$$
= \frac{\left\lbrack \mathop{\sum }\limits_{k}^{N}\langle A{\rangle }_{k}\exp \left( -\beta {E}_{k}\right) \right\rbrack }{\mathop{\sum }\limits_{k}^{N}\exp \left( {-\beta {E}_{k}}\right) }. tag{3. 4.53}
$$

特别是, 求得每组分的内能为

$$
U = \frac{\left\lbrack \mathop{\sum }\limits_{k}^{N}{E}_{k}\exp \left( -\beta {E}_{k}\right) \right\rbrack }{\mathop{\sum }\limits_{k}^{N}\exp \left( {-\beta {E}_{k}}\right) } tag{3. 4.54}
$$

$$
= - \frac{\partial }{\partial \beta }\left( {\ln Z}\right) ,
$$

它是每个统计力学的学生都熟知的一个公式.

参量 $\beta$ 与温度 $T$ 的关系如下

$$
\beta = \frac{1}{kT}, tag{3. 4. 55}
$$

其中 $k$ 是玻尔兹曼常数. 通过把简谐振子系综平均 $\left\lbrack H\right\rbrack$ 与在经典极限下内能的预期值 ${kT}$ 相比较,确信这个关系是有益的,这被留作一个练习. 已经阐明在高温极限下,一个正则系综变成一个所有的能量本征态被同等布居的完全随机的系综. 在相反的低温极限下 $\left( {\beta \rightarrow \infty }\right) ,\left( {3.4.48}\right)$ 式告诉我们,一个正则系综变成一个只有基态被布居的纯系综.

作为一个例证考虑一个自旋 $\frac{1}{2}$ 的系统组成的正则系综. 它的每一个成员都具有 ${eh}/$ $2{m}_{e}c$ 的磁矩,受到一个沿 $z$ 方向均匀磁场的作用. 与该问题相关的哈密顿量已经给出 [见 (3.2.16) 式]. 因为 $H$ 和 ${S}_{z}$ 对易,这个正则系综的密度矩阵在 ${S}_{z}$ 基中是对角的. 于是

$$
\rho \doteq \frac{\left( \begin{matrix} {e}^{-\beta \hslash \omega /2} & 0 \\ 0 & {e}^{\beta \hslash \omega /2} \end{matrix}\right) }{Z}, tag{3. 4.56}
$$

其中的配分函数恰为

$$
Z = {e}^{-\beta \hslash \omega /2} + {e}^{\beta \hslash \omega /2}. tag{3. 4.57}
$$

由此得出

$$
\left\lbrack {S}_{x}\right\rbrack = \left\lbrack {S}_{y}\right\rbrack = 0,\;\left\lbrack {S}_{z}\right\rbrack = - \left( \frac{\hslash }{2}\right) \tanh \left( \frac{\beta \hslash \omega }{2}\right) . tag{3. 4.58}
$$

磁矩分量的系综平均值正是 $e/{m}_{e}c$ 乘以 $\left\lbrack {S}_{z}\right\rbrack$ . 顺磁磁化率 $\chi$ 可由下式计算

$$
\left( \frac{e}{{m}_{e}c}\right) \left\lbrack {S}_{z}\right\rbrack = {\chi B} tag{3. 4.59}
$$

用这种方法得到 $\chi$ 的布里渊公式:

$$
\chi = \left( \frac{\left| e\right| \hslash }{2{m}_{e}{cB}}\right) \tanh \left( \frac{\beta \hslash \omega }{2}\right) . tag{3. 4.60}
$$

\section{自旋关联测量和贝尔不等式}
\subsection{自旋单态中的关联}

在 3.8 节遇到的角动量相加的最简单的例子涉及由自旋 $\frac{1}{2}$ 粒子构成的一个组合系统. 在这一节以这种系统为例说明量子力学中最使人吃惊的一个结果.

考虑处在一个自旋单态的双电子系统一一这就是说, 它具有零总自旋. 已经看到过, 其态右矢可以写成 [见 (3.8.15d)]:

$$
|\text{自旋单态}\rangle = \left( \frac{1}{\sqrt{2}}\right) \left( {\left| {\widehat{\mathbf{z}}+;\widehat{\mathbf{z}} - \rangle - }\right| \widehat{\mathbf{z}}-;\widehat{\mathbf{z}} + \rangle }\right) \text{,} tag{3.10.1}
$$

这里已经明显地标明了量子化的取向. 回忆一下 $|\widehat{\mathbf{z}} + ;\widehat{\mathbf{z}} - \rangle$ 意味着电子 1 是在自旋向上的态而电子 2 是在自旋向下的态. 对于 $|\widehat{\mathbf{z}} - ;\widehat{\mathbf{z}} + \rangle$ 类似的说法也是对的.

假定测量其中一个电子的自旋分量. 很清楚,有 ${50}\%$ 的机会得到自旋向上或者自旋向下,因为该组合系统可以以相等的概率处于 $|\widehat{\mathbf{z}} + ;\widehat{\mathbf{z}} - \rangle$ 或 $|\widehat{\mathbf{z}} - ;\widehat{\mathbf{z}} + \rangle$ . 但是,如果成员之一被证明是处在自旋向上的态上, 则另一个必然是处在自旋向下的态上, 反之亦然. 当电子 1 的自旋分量被证明是向上的,则测量仪器就挑选了 (3.10.1) 式的第一项 $|\widehat{\mathbf{z}} + ;\widehat{\mathbf{z}} - \rangle$ ; 接着电子 2 自旋分量的测量一定确认该组合系统的态右矢由 $|\widehat{\mathbf{z}} + ;\widehat{\mathbf{z}} - \rangle$ 给出.

值得注意的是, 如果这两个粒子飞开, 即使它们分开得很远, 而且不再发生相互作用, 这一类关联却仍能继续存在, 只要当它们飞开时, 它们的自旋态不发生任何变化. 对于一个 $J = 0$ 的系统自发解体为两个自旋为 $\frac{1}{2}$ 的、没有相对轨道角动量的粒子肯定属于这种情况,因为在解体过程中角动量一定守恒. 这种情况的一个例子是 $\eta$ 介子 (质量为

${549}\mathrm{{MeV}}/{c}^{2}$ ) 到一对 $\mu$ 子的稀有衰变

$$
\eta \rightarrow {\mu }^{ + } + {\mu }^{ - } tag{3.10.2}
$$

不幸的是,它的衰变分支比仅约为 $6 \times {10}^{-6}$ . 更现实一点,在低动能质子-质子散射中,将在第 7 章讨论的泡利原理迫使相互作用的质子处于 ${S}_{0}$ 态 (轨道角动量为 0,自旋单态), 散射后的质子自旋态一定会按照 (3.10.1) 式指出的方式相关联, 即使它们被分隔开一段宏观距离之后.



图 3.11 自旋单态中的自旋关联

更形象化一些,考虑沿相反方向运动的两个自旋 $\frac{1}{2}$ 粒子,如图 3.11 所示. 观察者 $\mathrm{A}$ 专门测量 (向右飞的) 粒子 1 的 ${S}_{z}$ ,而观察者 $B$ 专门测量 (向左飞的) 粒子 2 的 ${S}_{z}$ . 具体说来,假定观察者 $\mathrm{A}$ 发现粒子 1 的 ${S}_{z}$ 为正. 接着他或她甚至在 $\mathrm{B}$ 做任何测量之前就可以肯定地预言出 $\mathrm{B}$ 的测量结果: $\mathrm{B}$ 一定会发现粒子 2 的 ${S}_{z}$ 是负的. 另一方面,如果 $\mathrm{A}$ 不做任何测量,则 $\mathrm{B}$ 有 ${50}\%$ 对 ${50}\%$ 的机会得到 ${S}_{z} +$ 或 ${S}_{z} -$ .

这本身或许还不是这么奇特. 人们可以说, “它只不过像个小罐, 已知里面装有一个黑球和一个白球,当随便从罐里抓出一个球时,有 ${50}\%$ 的机会得到黑球或者白球. 但是如果拿出来的第一个球是黑球时, 那时就可以肯定地预言, 第二个球将是白的. "

事实证明, 这种类比过于简单了. 实际的量子力学情况比这个情况复杂得多! 这是因为观察者可能选择测 ${S}_{u}$ 代替 ${S}_{z}$ . 这同样的一对 “量子力学球” 或者可以用黑与白,或者可以用蓝与红来分析.

现在回忆一下,对于一个单个的自旋 $\frac{1}{2}$ 系统, ${S}_{r}$ 的本征右矢与 ${S}_{z}$ 的本征右矢之间有如下关系:

$$
\left| {\widehat{\mathbf{x}} \pm \rangle = \left( \frac{1}{\sqrt{2}}\right) \left( {\left| {\widehat{\mathbf{z}} + \rangle \pm }\right| \widehat{\mathbf{z}} - \rangle }\right) ,\;}\right| \widehat{\mathbf{z}} \pm \rangle = \left( \frac{1}{\sqrt{2}}\right) \left( {\left| {\widehat{\mathbf{x}} + \rangle \pm }\right| \widehat{\mathbf{x}} - \rangle }\right) . tag{3.10.3}
$$

现在回到组合系统,可以通过选择 $x$ 方向作为量子化的轴,把自旋单态右矢 (3.10.1) 重写为

$$
|\text{自旋单态}\rangle = \left( \frac{1}{2}\right) \left( {\left| {\widehat{\mathbf{x}} - ;\widehat{\mathbf{x}} + \rangle - }\right| \widehat{\mathbf{x}}+;\widehat{\mathbf{x}} - \rangle }\right) \text{.} tag{3.10.4}
$$

除去在任何情况只不过是一个约定的整体符号外, 或许可以直接从 (3.10.1) 式猜出这个形式, 因为自旋单态不具有任何空间的优势方向. 现在假定, 观察者 $\mathrm{A}$ 可以通过改变他或她的自旋分析器的取向来选择测量粒子 1 的 ${S}_{x}$ 还是 ${S}_{z}$ ,而观察者 $B$ 则永远专门测量粒子 2 的 ${S}_{x}$ . 如果 $\mathrm{A}$ 确定粒子 1 的 ${S}_{z}$ 为正,显然 $\mathrm{B}$ 有 ${50}\%$ 对 ${50}\%$ 的机会得到 ${S}_{x} +$ 或 ${S}_{x} -$ ; 尽管已知粒子 2 的 ${S}_{z}$ 肯定为负,但它的 ${S}_{x}$ 是完全不确定的. 另一方面,假定 $\mathrm{A}$ 也选择测量 ${S}_{r}$ . 如果观察者 $\mathrm{A}$ 确定粒子 1 的 ${S}_{x}$ 为正,则毫无例外,观察者 $\mathrm{B}$ 将会测到粒子 2 的 ${S}_{x}$ 为负. 最后,如果 A 选择不做任何测量,当然, B 将有 ${50}\%$ 对 ${50}\%$ 机会得到 ${S}_{r}$ 十或 ${S}_{r}$ 一. 综上所述:

1. 如果 $\mathrm{A}$ 测量 ${S}_{z}$ 而 $\mathrm{B}$ 测量 ${S}_{r}$ ,则两个测量之间有完全随机的关联.

2. 如果 A 测量 ${S}_{x}$ 而 B 测量 ${S}_{x}$ ,则两个测量之间有 ${100}\%$ (相反符号的) 关联.

3. 如果 $\mathrm{A}$ 不做任何测量,则 $\mathrm{B}$ 的测量显示随机的结果.

表 3.1 自旋关联测量


表 3.1 显示了,当 $\mathrm{B}$ 和 $\mathrm{A}$ 都被允许选择测量 ${S}_{x}$ 或者 ${S}_{z}$ 时,所有可能的这样的测量结果. 这些考虑表明, $\mathrm{B}$ 的测量结果表现出对 $\mathrm{A}$ 决定做哪一种测量的依赖性: 即 $\mathrm{A}$ 测量 ${S}_{r}$ 、 测量 ${S}_{z}$ 或者不做任何测量. 再次注意, $\mathrm{A}$ 和 $\mathrm{B}$ 可能分开几英里远,没有任何通讯或相互作用的可能性. 观察者 $\mathrm{A}$ 可以决定,在两个粒子分开足够远时,他或她的分析器如何取向. 而粒子 2 就好像 “知道” 了粒子 1 的哪一个自旋分量被测量了.

正统量子力学对这种情况做如下解释. 测量是一次选择 (或过滤) 过程. 当测出粒子 1 的 ${S}_{z}$ 为正时,则挑选了 $|\widehat{\mathbf{z}} + ;\widehat{\mathbf{z}} - \rangle$ 分量. 接着对另一个粒子的 ${S}_{z}$ 的测量只不过要确认这个系统仍然处在 $|\widehat{\mathbf{z}} + \mathbf{i}\widehat{\mathbf{z}} - \rangle$ . 必须承认,对该系统的一部分处于什么状态的测量将被看作对整个系统的测量.

\subsection{爱因斯坦局域性原理和贝尔不等式}

许多物理学家对于前述自旋关联的正统解释感到不舒服. 这种感觉体现在下面频繁引用的爱因斯坦的评述中, 称它为爱因斯坦局域性原理: “但是, 在我看来, 我们确实应该紧紧抓住一个假定: 系统 ${S}_{2}$ 的真实情况和系统 ${S}_{1}$ 做了些什么无关,它们是空间分离的. ” 因为这个问题最早是在 1935 年爱因斯坦, 波多尔斯基 (B. Podolsky) 和罗森 (N. Rosen) 的一篇文章中讨论的, 因此有时称之为爱因斯坦-波多尔斯基-罗森 (Einstein-Podolsky-Rosen) 佯谬. *

---

* 精确的历史事实是,爱因斯坦-波多尔斯基-罗森的原文处理 $x$ 和 $p$ 的测量. 利用组合的 $\frac{1}{2}$ 系统为例说明爱因斯坦-波多尔斯基-罗森佯谬始于玻姆.

---

有些人曾争辩过, 在这里遇到的困难是量子力学的概率解释所固有的, 微观层次的动力学行为出现了概率性的结果仅仅是因为某些还不知道的参量——所谓的隐变量——还没有被确定下来. 这里的目的不是要讨论基于隐变量或其他一些考虑的量子力学的各种替代理论. 相反, 试问, 这样的理论给出了不同于量子力学的预言吗? 直到 1964 年之前, 人们一直认为, 可以这样编造替代理论, 以致给不出任何不同于通常量子力学的而又能被实验证实的预言. 整个争论应该属于形而上学领域而不属于物理学. 当时, J. S. Bell 指出, 基于爱因斯坦的局域性原理的替代理论, 实际上预言了在自旋关联实验中可观测量间的一个可以检验的不等式关系, 它与量子力学的预言不相符.

在一个由维格纳构想出来的简单模型框架下推导出贝尔不等式, 该模型纳入了各种替代理论的基本特点. 这个模型的支持者们赞同不可能同时确定 ${S}_{r}$ 和 ${S}_{z}$ . 然而,当有大量自旋 $\frac{1}{2}$ 的粒子时,赋予它们当中的某部分成员具有下列的一些性质:

如果测量 ${S}_{z}$ ,确定无疑地得到加号.

如果测量 ${S}_{r}$ ,确定无疑地得到减号.

满足这一性质的粒子被称为属于 $\left( {\widehat{\mathbf{z}} + ,\widehat{\mathbf{x}} - }\right)$ 类型. 注意,并不能断言,可以同时测量分别为十和一的 ${S}_{z}$ 和 ${S}_{x}$ . 当测量 ${S}_{z}$ 时,不测量 ${S}_{x}$ ,反之亦然. 赋予沿不止一个方向的确定的自旋分量值, 应理解为只有一个或另一个分量可以实际被测量. 尽管这种方法与量子力学的方法根本不同,只要属于 $\left( {\widehat{\mathbf{z}}+,\widehat{\mathbf{x}} + }\right)$ 类型的粒子像 $\left( {\widehat{\mathbf{z}}+,\widehat{\mathbf{x}} - }\right)$ 类型的粒子一样多,则对在自旋向上 $\left( {{S}_{z} + }\right)$ 的态上所做的 ${S}_{z}$ 和 ${S}_{x}$ 测量的量子力学预言可以重现.

现在检验这个模型怎样解释在组合自旋单态系统上测量自旋关联的结果. 显然, 对于一对特殊的粒子, 粒子 1 和粒子 2 之间一定有一个完美的匹配, 以保证总角动量为零: 如果粒子 1 是 $\left( {\widehat{\mathbf{z}}+,\widehat{\mathbf{x}} - }\right)$ 类型的,则粒子 2 一定是 $\left( {\widehat{\mathbf{z}}-,\widehat{\mathbf{x}} + }\right)$ 类型的,等等. 如果具有相等的布居——即各为 ${25}\%$ ——的粒子 1 和粒子 2 有如下匹配:

$$
\text{粒子 1 粒子 2}
$$

$$
\left( {\widehat{\mathbf{z}}+,\widehat{\mathbf{x}} - }\right) \leftrightarrow \left( {\widehat{\mathbf{z}}-,\widehat{\mathbf{x}} + }\right) , tag{3.10.5a}
$$

$$
\left( {\widehat{\mathbf{z}}+,\widehat{\mathbf{x}} + }\right) \leftrightarrow \left( {\widehat{\mathbf{z}}-,\widehat{\mathbf{x}} - }\right) , tag{3.10.5b}
$$

$$
\left( {\widehat{\mathbf{z}}-,\widehat{\mathbf{x}} + }\right) \leftrightarrow \left( {\widehat{\mathbf{z}}+,\widehat{\mathbf{x}} - }\right) ,
$$

(3.10. ${5c}$ )

$$
\left( {\widehat{\mathbf{z}}-,\widehat{\mathbf{x}} - }\right) \leftrightarrow \left( {\widehat{\mathbf{z}}+,\widehat{\mathbf{x}} + }\right) tag{3.10.5d}
$$

则诸如表 3.1 所示的关联测量结果就可以重新产生. 这里隐含着一个非常重要的假设. 假定特殊的一对粒子是属于 (3.10.5a) 类型的,并且观察者 $\mathrm{A}$ 决定测量粒子 1 的 ${S}_{z}$ ; 那时,他或她必定得到一个加号,不管 B 决定测量 ${S}_{z}$ 或 ${S}_{x}$ . 正是在这种意义上,爱因斯坦的局域性原理被包含在这个模型中: $\mathrm{A}$ 的结果是预先确定的,它与 $\mathrm{B}$ 选择测量什么无关.

到此为止所考虑的例子中, 这个模型在重现量子力学预言方面已经获得了成功. 现在考虑该模型能导致不同于通常量子力学预言的更为复杂的情况. 这一次, 从三个单位矢量 $\widehat{\mathbf{a}},\widehat{\mathbf{b}}$ 和 $\widehat{\mathbf{c}}$ 开始,一般说来它们不是相互正交的. 设想其中的一个粒子属于某确定类型,比如 $\left( {\widehat{a}-,\widehat{b} + ,\widehat{c} + }\right)$ ,它意味着: 如果测量 $S \cdot \widehat{a}$ ,肯定得到负号; 如果量 $S \cdot \widehat{b}$ ,肯定得到正号; 如果测量 $\mathbf{S} \cdot \widehat{\mathbf{c}}$ ,也肯定得到正号. 再一次,在为了保证总角动量为零,另一个粒子必定属于 $\left( {\widehat{a}+,\widehat{b}-,\widehat{c} - }\right)$ 类型的意义上,一定有一个完美的匹配. 在任意给定的事例中, 所考虑的粒子对一定是表 3.2 所示的八种类型之一. 这八种可能性互不相容并且没有交集. 每一类的布居数标明在第一列中.

假定观察者 $\mathrm{A}$ 发现 ${\mathbf{S}}_{1} \cdot \widehat{\mathbf{a}}$ 是正号而观察者 $\mathrm{B}$ 发现 ${\mathbf{S}}_{2} \cdot \widehat{\mathbf{b}}$ 也是正号,从表 3.2 显然可见,这一对粒子属于类型 3 或者类型 4,所以,在这种情况中粒子对的数目是 ${N}_{3} + {N}_{1}$ . 因为 ${N}_{i}$ 是半正定的,必须有不等式

$$
{N}_{3} + {N}_{4} \leq \left( {{N}_{2} + {N}_{4}}\right) + \left( {{N}_{3} + {N}_{7}}\right) . tag{3.10.6}
$$

设 $P\left( {\widehat{\mathbf{a}}+;\widehat{\mathbf{b}} + }\right)$ 是在一种随机选择下,观察者 $\mathrm{A}$ 测量 ${\mathrm{S}}_{1} \cdot \widehat{\mathbf{a}}$ 是正号,并且观察者 $\mathrm{B}$ 测量 ${\mathrm{S}}_{2} \cdot \widehat{\mathrm{b}}$ 也是正号,等等的概率.



显然有

$$
P\left( {\widehat{\mathbf{a}}+;\widehat{\mathbf{b}} + }\right) = \frac{\left( {N}_{3} + {N}_{4}\right) }{\mathop{\sum }\limits_{i}^{8}{N}_{i}}. tag{3.10.7}
$$

用类似的方式. 得到

$$
P\left( {\widehat{\mathbf{a}}+;\widehat{\mathbf{c}} + }\right) = \frac{\left( {N}_{2} + {N}_{1}\right) }{\mathop{\sum }\limits_{i}^{8}{N}_{i}}\text{ 和 }\;P\left( {\widehat{\mathbf{c}}+;\widehat{\mathbf{b}} + }\right) = \frac{\left( {N}_{3} + {N}_{7}\right) }{\mathop{\sum }\limits_{i}^{8}{N}_{i}}. tag{3.10.8}
$$

正定性条件 (3.10.6) 式现在变成

$$
P\left( {\widehat{\mathbf{a}}+;\widehat{\mathbf{b}} + }\right) \leq P\left( {\widehat{\mathbf{a}}+;\widehat{\mathbf{c}} + }\right) + P\left( {\widehat{\mathbf{c}}+;\widehat{\mathbf{b}} + }\right) . tag{3.10.9}
$$

这就是贝尔不等式, 它源自爱因斯坦局域性原理.

\subsection{量子力学和贝尔不等式}

现在返回到量子力学世界. 在量子力学中,没有谈到粒子对的某一部分,比如 ${N}_{3}/$ $\mathop{\sum }\limits_{i}^{8}{N}_{i}$ ,属于类型 3 . 相反,用同样的右矢 (3.10.1) 式表征所有的自旋单态系统; 用 3.4 节的语言讲, 在这里所涉及的是一个纯系综. 利用这个右矢和所发展的量子力学规则, 可以无疑义地计算在不等式 (3.10.9) 的三项中的每一项.

先求 $P\left( {\widehat{\mathbf{a}}+;\widehat{\mathbf{b}} + }\right)$ . 假定观察者 $\mathrm{A}$ 发现 ${\mathbf{S}}_{1} \cdot \widehat{\mathbf{a}}$ 是正的,由于早先讨论过的 ${100}\%$ (相反的符号) 关联, $\mathrm{B}$ 测量 ${\mathbf{S}}_{2} \cdot \widehat{\mathbf{a}}$ 将肯定地产生负号. 但是,为计算 $P\left( {\widehat{\mathbf{a}}+;\widehat{\mathbf{b}} + }\right)$ ,必须考虑一个新的量子化轴 $\widehat{\mathbf{b}}$ ,它与 $\widehat{\mathbf{a}}$ 夹角为 ${\theta }_{ub}$ ,见图 3.12. 按照 3.2 节的公式形式,当已知粒子 2 处在 ${\mathbf{S}}_{2} \cdot \widehat{\mathbf{a}}$ 的负本征值的本征右矢上时,测量 ${\mathbf{S}}_{2} \cdot \widehat{\mathbf{b}}$ 产生 + 的概率为

$$
{\cos }^{2}\left\lbrack \frac{\left( \pi - {\theta }_{ab}\right) }{2}\right\rbrack = {\sin }^{2}\left( \frac{{\theta }_{ab}}{2}\right) . tag{3.10.10}
$$

结果得到

$$
P\left( {\widehat{\mathbf{a}}+;\widehat{\mathbf{b}} + }\right) = \left( \frac{1}{2}\right) {\sin }^{2}\left( \frac{{\theta }_{ab}}{2}\right) , tag{3.10.11}
$$

其中因子 $\frac{1}{2}$ 来自初始得到 ${\mathbf{S}}_{1} \cdot \widehat{\mathbf{a}}$ 为正的概率. 利用 (3.10.11) 式和它在 (3.10.9) 式其他两项上的推广, 可以把贝尔不等式写成

$$
{\sin }^{2}\left( \frac{{\theta }_{ab}}{2}\right) \leq {\sin }^{2}\left( \frac{{\theta }_{ac}}{2}\right) + {\sin }^{2}\left( \frac{{\theta }_{cb}}{2}\right) . tag{3.10.12}
$$



图 3.12 $P\left( {\widehat{a}+;\widehat{b} + }\right)$ 的求值

现在来证明, 从几何观点出发, 不等式 (3.10.12) 式不是永远可能的. 为了简单起见, 选择 $\widehat{\mathbf{a}},\widehat{\mathbf{b}}$ 和 $\widehat{\mathbf{c}}$ 处在同一平面上,而且设 $\widehat{\mathbf{c}}$ 把 $\widehat{\mathbf{a}}$ 和 $\widehat{\mathbf{b}}$ 确定的两个方向一分为二:

$$
{\theta }_{ab} = {2\theta },\;{\theta }_{ac} = {\theta }_{cb} = \theta . tag{3.10.13}
$$

于是, 不等式 (3.10.12) 在

$$
0 < \theta < \frac{\pi }{2}. tag{3.10.14}
$$

时被破坏. 例如取 $\theta = \pi /4$ 时,就会得到

$$
{0.500} \leq {0.292}?? tag{3.10.15}
$$

因此, 量子力学预言与贝尔不等式不相容. 在量子力学和满足爱因斯坦局域性原理的替代理论之间, 存在一个真实、可观测——在可被实验检验的意义上——的差别.

目前已经完成了几个检验贝尔不等式的实验. 最近的评述请见《贝尔不等式检验: 比以往任何时候都更理想》, 源自 A. Aspect 的 Nature 398 (1999) 189. 在其中的一个实验中, 测量了在低能质子-质子散射中, 末态质子间的自旋关联. 而其他所有的实验都是在一个激发原子 $\left( {\mathrm{{Ca}},\mathrm{{Hg}},\cdots }\right)$ 的级联跃迁

$$
\left( {j = 0}\right) \overset{\gamma }{ \rightarrow }\left( {j = 1}\right) \overset{\gamma }{ \rightarrow }\left( {j = 0}\right) , tag{3.10.16}
$$

中或者在电子偶素(一个 ${e}^{ + }{e}^{ - }$ 的 ${}^{1}{S}_{0}$ 束缚态)的衰变中,对一对光子间的光子-极化关联进行测量. 基于 1.1 节发展出来的类比

$$
{S}_{z} + \rightarrow \widehat{\varepsilon }\;\text{ 沿 }x\text{ 方向,} tag{3.10.17a}
$$

$$
{S}_{z} \rightarrow \widehat{\varepsilon }\;\text{沿}y\text{方向,} tag{3.10.17b}
$$

${S}_{x} + \rightarrow \widehat{\varepsilon }$ 沿 ${45}^{ \circ }$ 斜线方向,(3.10.17c)

${S}_{x} \rightarrow \widehat{\varepsilon }$ 沿 ${135}^{ \circ }$ 斜线方向.(3.10.17d)

研究光子-极化关联应当同样地好. 所有最近的精确实验结果都决定性地确证贝尔不等式被破坏了, 有一种情况竟超过了九个标准偏差. 此外, 在所有这些实验中, 这个不等式是以这样的一种方式被破坏的, 即量子力学预言在误差范围之内得到满足. 在这场论战中, 量子力学取得了辉煌的胜利.

量子力学预言被证实并不意味着现在这整个课题都是无聊的. 虽然有实验的结论, 但这类测量的很多方面使人感到心里不舒服. 特别是考虑下面的一点: 在观察者 $\mathrm{A}$ 对粒子 1 做了一次测量以后, 粒子 2-一原则上, 它可能离开粒子 1 许多光年——怎么会 “知道” 它的自旋应怎样取向, 以使清楚地列在表 3.1 中的那些引人瞩目的关联得以实现呢? 在其中一个检验贝尔不等式的实验当中 [由阿斯派克特 (A. Aspect) 和他的合作者完成的], 分析器的设置变化要非常地快, 直到任何一类传播速度比光速慢的影响来不及到达 $\mathrm{B}$ 之前, A 就已经做出了测量什么的决定.

通过展示下述看法来结束这一节: 尽管有这些奇特之处, 仍然不可能用自旋关联的测量在两个宏观分离的点之间传送任何有用的信息. 特别是, 超光速 (比光速还快) 的通信是不可能的.

假定 $\mathrm{A}$ 和 $\mathrm{B}$ 事先都同意测量 ${S}_{z}$ ,然后,不用问 $\mathrm{A},\mathrm{B}$ 就精确地知道 $\mathrm{A}$ 得到的是什么. 但是, 这并不意味着 A 和 B 正在通信联系, B 只不过观测一个正号与负号的随机系列. 这里面显然没有包含任何有用的信息. 只有当 $\mathrm{B}$ 与 $\mathrm{A}$ 相聚在一起并且比较了记录(或计算机工作表) 才能证实由量子力学预言的令人瞩目的关联.

或许有人会认为, 如果 A 和 B 中的一个突然改变了他或她的分析器的取向, 他们就能够通信. 假定一开始 $\mathrm{A}$ 同意测量 ${S}_{z}$ ,而 $\mathrm{B}$ 测量 ${S}_{r}.\mathrm{\;A}$ 的测量结果与 $\mathrm{B}$ 的测量结果完全没有关联, 因此不存在被传递的信息. 但是之后, 假定 $\mathrm{A}$ 突然破坏了他或她的许诺,没有告知 B,就开始测量 ${S}_{r}$ . 现在,在 A 的结果与 B 的结果之间有了完全的关联. 然而, B 没有任何办法推断 $\mathrm{A}$ 已经改变了他或她的分析器的取向. $\mathrm{B}$ 通过只看他或她自己的笔记本仅能继续看到 “十” 和 “一” 的随机系列. 于是, 再次没有被传递的信息.



习题

3.1 求 ${\sigma }_{y} = \left( \begin{matrix} 0 & - i \\ i & 0 \end{matrix}\right)$ 的本征值和本征矢. 假定一个电子处在自旋态 $\left( \begin{array}{l} \alpha \\ \beta \end{array}\right)$ 上. 如果测得 ${s}_{y}$ ,结果为 $\hslash /2$ 的概率是什么?

3.2 对于一个自旋 $\frac{1}{2}$ 的粒子,在存在一个磁场 $\mathbf{B} = {B}_{x}\widehat{\mathbf{x}} + {B}_{y}\widehat{\mathbf{y}} + {B}_{z}\widehat{\mathbf{z}}$ 情况下,通过利用泡利矩阵显式结构, 求哈密顿量

$$
H = - \frac{2\mu }{\hslash }\mathbf{S} \cdot \mathbf{B}
$$

的本征值.

3.3 考虑由

$$
U = \frac{{a}_{0} + i\mathbf{\sigma } \cdot \mathbf{a}}{{a}_{0} - i\mathbf{\sigma } \cdot \mathbf{a}}.
$$

定义的 $2 \times 2$ 矩阵,其中 ${a}_{0}$ 是一个实数,而 $\mathbf{a}$ 是一个有着实分量的三维矢量.

(a) 证明 $U$ 是幺正的和么模的.

(b) 一般地说,一个 $2 \times 2$ 幺正么模矩阵表示一个三维转动. 借助于 ${a}_{0},{a}_{1},{a}_{2}$ 和 ${a}_{3}$ ,求适合于 $U$ 的转动轴和转角.

3.4 在存在一个沿 $z$ 方向的均匀磁场情况下,一个电子-正电子系统的时间相关哈密顿量可以写成

$$
H = A{\mathbf{S}}^{\left( {r}^{ - }\right) } \cdot {\mathbf{S}}^{\left( {r}^{ + }\right) } + \left( \frac{eB}{mc}\right) \left( {{S}_{z}^{\left( {r}^{ - }\right) } - {S}_{z}^{\left( {r}^{ + }\right) }}\right) ,
$$

假定系统的自旋函数由 ${\chi }_{ + }^{\left( {e}^{ - }\right) }{\chi }_{ - }^{\left( {e}^{ + }\right) }$ 给出.

(a) 在 $A \rightarrow 0,{eB}/{mc} \neq 0$ 的极限下,它是 $H$ 的本征函数吗? 如果它是,能量本征值是什么? 如果它不是, $H$ 的期待值是什么?

(b) 当 ${eB}/{mc} \rightarrow 0, A \neq 0$ 时,求解同样的问题.

3.5 考虑一个自旋为 1 的粒子. 求

$$
{S}_{z}\left( {{S}_{z} + h}\right) \left( {{S}_{z} - h}\right) \;\text{ 和 }{S}_{x}\left( {{S}_{x} + h}\right) \left( {{S}_{x} - h}\right)
$$

的矩阵元.

3.6 设一个刚体的哈密顿量为

---

其中 $\mathbf{K}$ 是本体坐标架中的角动量. 从这个表示式求 $\mathbf{K}$ 的海森伯运动方程,然后在相应的极限下求欧拉运动方程.

3.7 令 $U = {e}^{i{\theta }_{3}a}{e}^{i{\theta }_{2}\beta }{e}^{i{\theta }_{3}\gamma }$ ,其中 $\left( {\alpha ,\beta ,\gamma }\right)$ 是欧拉角. 为了使 $U$ 表示一个转动 $\left( {\alpha ,\beta ,\gamma }\right) ,{G}_{k}$ 必须满足的对易关系是什么? 把 $\mathbf{G}$ 与角动量算符联系起来.

3.8 下列方程式的意义是什么?

$$
{U}^{-1}{A}_{k}U = \sum {R}_{kl}{A}_{l},
$$

其中. $\mathbf{A}$ 的三个分量都是矩阵. 由这个方程式证明. 矩阵元 $\left\langle {m\left| {A}_{k}\right| n}\right\rangle$ 像一个矢量一样变换.

3.9 考虑一个由下式表示的欧拉转动序列

$$
{\mathcal{D}}^{\left( 12\right) }\left( {\alpha ,\beta ,\gamma }\right) = \exp \left( \frac{-i{\sigma }_{3}\alpha }{2}\right) \exp \left( \frac{-i{\sigma }_{2}\beta }{2}\right) \exp \left( \frac{-i{\sigma }_{3}\gamma }{2}\right)
$$

$$
= \left( \begin{matrix} {e}^{-i\left( {\alpha + \gamma }\right) /2}\cos \frac{\beta }{2} & - {e}^{-i\left( {\alpha - \gamma }\right) /2}\sin \frac{\beta }{2} \\ {e}^{i\left( {\alpha - \gamma }\right) /2}\sin \frac{\beta }{2} & {e}^{i\left( {\alpha + \gamma }\right) /2}\cos \frac{\beta }{2} \end{matrix}\right) .
$$

由于转动的群性质,预期这一序列操作等价于绕某个轴转一个 $\theta$ 角的单一转动. 求 $\theta$ .

3. 10 (a) 考虑全同制备的自旋 $\frac{1}{2}$ 系统的一个纯系综. 假定期待值 $\left\langle {S}_{x}\right\rangle$ 和 $\left\langle {S}_{z}\right\rangle$ 已知,而 $\left\langle {S}_{y}\right\rangle$ 的符号也已知. 证明如何确定态矢量. 为什么不必知道 $\left\langle {S}_{y}\right\rangle$ 的大小?

(b) 考虑一个自旋 $\frac{1}{2}$ 系统的混合系综. 假定系综平均值 $\left\lbrack {S}_{x}\right\rbrack ,\left\lbrack {S}_{y}\right\rbrack$ 和 $\left\lbrack {S}_{z}\right\rbrack$ 都是已知的. 证明如何可以构造表征这个系综的 $2 \times 2$ 密度矩阵.

3.11 (a) 证明密度算符 $\rho$ (在薛定谔绘景中) 的时间演化由下式给定

$$
\rho \left( t\right) = u\left( {t,{t}_{0}}\right) \rho \left( {t}_{0}\right) {u}^{ + }\left( {t,{t}_{0}}\right) .
$$

(b) 假定在 $t = 0$ 时有一个纯系综. 证明只要时间演化由薛定谔方程控制,则它不可能演化成一个混合系综.

3.12 考虑自旋为 1 系统的一个系综. 密度矩阵现在是一个 $3 \times 3$ 矩阵. 为了表征这个密度矩阵,需要多少独立的 (实) 参量? 除了 $\left\lbrack {S}_{r}\right\rbrack ,\left\lbrack {S}_{v}\right\rbrack$ 和 $\left\lbrack {S}_{z}\right\rbrack$ . 为了完全表征这个系综还必须知道什么?

3.13 一个角动量本征态 $\left| {j, m = {m}_{\text{最大 }} = j}\right\rangle$ 绕 $y$ 轴转一个无穷小角度 $\varepsilon$ . 不使用 ${d}_{mim}^{\left( t\right) }$ 函数的显式表达式, 求在其原来态上发现这个新的转动后的态的概率表示式,直到 ${\varepsilon }^{2}$ 量级项.

3.14 已知 $3 \times 3$ 矩阵 ${G}_{i}\left( {i = 1,2,3}\right)$ ,其矩阵元由下式给出:

$$
{\left( {G}_{i}\right) }_{jk} = - i\hslash {\varepsilon }_{ijk},
$$

其中 $j$ 和 $k$ 是行和列指标,证明它满足角动量对易关系. 把 ${G}_{i}$ 与比较常用的角动量算符 ${J}_{i}$ ,在 ${J}_{3}$ 取为对角的情况下的 $3 \times 3$ 表示联系起来,实现该联系的变换矩阵的物理 (或几何) 意义是什么? 把得到的结果与无穷小转动下的

$$
\mathbf{V} \rightarrow \mathbf{V} + \widehat{\mathbf{n}}{\delta \phi } \times \mathbf{V}
$$

联系起来. (注: 这个问题可能有助于理解光子的自旋.)

3.15 (a) 令 $\mathbf{J}$ 是角动量. (它可以是轨道角动量 $\mathbf{L}$ ,自旋 $\mathbf{S}$ ,或 ${\mathbf{J}}_{\text{①}}$ .) 利用 ${J}_{t},{J}_{y},{J}_{z}\left( {{J}_{ \pm } \equiv {J}_{t} \pm i{J}_{y}}\right)$ 满足通常角动量对易关系的事实, 证明

$$
{\mathbf{J}}^{2} = {J}_{z}^{2} + {J}_{ + }{J}_{ - } - \hslash {J}_{z}.
$$

(b) 利用 (a) (或其他方式) 推导出在

$$
{J}_{ - }{\psi }_{jm} = {c}_{ - }{\psi }_{j, m - 1}
$$

中的系数 ${c}_{ - }$ 的 “著名” 的表示式.

3.16 证明轨道角动量算符 $\mathrm{L}$ 与算符 ${\mathbf{p}}^{2}$ 和 ${\mathbf{x}}^{2}$ 都对易,即证明 (3.7.2) 式.

3.17 一个受到球对称势 $V\left( r\right)$ 作用的粒子的波函数由下式给出:

$$
\psi \left( \mathbf{x}\right) = \left( {x + y + {3z}}\right) f\left( r\right) .
$$

(a) $\psi$ 是 ${\mathbf{L}}^{2}$ 的一个本征函数吗? 如果是的话, $l$ 值是什么? 如果不是,当测量 ${\mathbf{L}}^{2}$ 时能得到什么样可能的 $l$ 值?

(b) 在各种 ${m}_{t}$ 态上找到该粒子的概率是什么?

(c) 假定以某种方式知道 $\psi \left( \mathbf{x}\right)$ 是一个能量本征函数,本征值为 $E$ . 指出怎样可以找到 $V\left( r\right)$ .

3.18 已知一个在球对称势中的粒子处在 ${\mathbf{L}}^{2}$ 和 ${L}_{z}$ 的本征态,本征值分别为 $l\left( {l + 1}\right) {\hslash }^{2}$ 和 $m\hslash$ . 证明在 $|{lm}\rangle$ 态之间的期待值满足

$$
\left\langle {L}_{x}\right\rangle = \left\langle {L}_{y}\right\rangle = 0,\;\left\langle {L}_{x}^{2}\right\rangle = \left\langle {L}_{y}^{2}\right\rangle = \frac{\left\lbrack l\left( l + 1\right) {\hslash }^{2} - {m}^{2}{\hslash }^{2}\right\rbrack }{2}.
$$

半经典地解释这个结果.

3.19 假定轨道角动量允许有一个半整数的 $l$ 值,比如 $\frac{1}{2}$ . 通常,从

$$
{L}_{ + }{Y}_{1,2,1,2}\left( {\theta ,\phi }\right) = 0,
$$

可以导出

$$
{Y}_{1/2,1/2}\left( {\theta ,\phi }\right) \propto {e}^{{i\phi }/2}\sqrt{\sin \theta }.
$$

现在尝试 (a) 用 ${L}_{ - }$ 作用于 ${Y}_{1/2,1/2}\left( {\theta ,\phi }\right)$ ; (b) 用 ${L}_{ - }{Y}_{1/2, - 1/2}\left( {\theta ,\phi }\right) = 0$ ,构造 ${Y}_{1/2, - 1/2}\left( {\theta ,\phi }\right)$ . 证明这两种做法会导致矛盾的结果. (这给出了轨道角动量不能为一个半奇数 $l$ 值的论据.)

3.20 考虑一个轨道角动量的本征态 $|l = 2, m = 0\rangle$ . 假定这个态绕 $y$ 轴转了 $\beta$ 角. 求在 $m = 0, \pm 1$ 和 $\pm 2$ 的态上找到这个新态的概率. (在附录 $\mathrm{B}$ 的 $\mathrm{B}{.5}$ 节中给出的 $l = 0,1$ 和 2 的球谐函数可能是有用的.)

3.21 本题的目的是借助于笛卡尔本征态 $\left| {n,{n}_{y}{n}_{z}}\right\rangle$ . 确定写成 ${\mathbf{L}}^{2}$ 和 ${L}_{z}$ 本征态的三维各向同性谐振子的简并本征态.

(a) 证明角动量算符由下式给出:

$$
{L}_{i} = {ih}{\varepsilon }_{ijk}{a}_{j}{a}_{k}^{ + }
$$

$$
{\mathbf{L}}^{2} = {\hslash }^{2}\left\lbrack {N\left( {N + 1}\right) - {a}_{k}^{ \dagger }{a}_{k}^{ \dagger }{a}_{j}{a}_{j}}\right\rbrack ,
$$

其中隐含了对于重复指标的求和, ${\varepsilon }_{ijk}$ 是全反对称符号,而 $N \equiv {a}_{i}^{ \dagger }{a}_{j}$ 计数了总的量子数日.

(b) 使用这些关系式,把态 $\left| {{qlm}\rangle = }\right| {01m}\rangle, m = 0, \pm 1$ ,借助于能量简并的三个本征态 $\left| {{n}_{x}{n}_{y}{n}_{z}}\right\rangle$ 表示出来. 在坐标空间中表述你的答案, 并且检查角度和径向依赖关系都是正确的.

(c) 对 $\left| {{qlm}\rangle = }\right| {200}\rangle$ 重复以上两问.

(d) 对 $\left| {{qlm}\rangle = }\right| {02m}\rangle$ ,在 $m = 0,1$ 和 2 的情况下,重复 (a) 和 (b) 两问.

3.22 遵照下列步骤,证明库默尔方程 (3.7.46) 可以用拉盖尔多项式 ${L}_{n}\left( x\right)$ 写出来,后者按照母函数定义为

$$
g\left( {x, t}\right) = \frac{{e}^{-x/\left( {1 - t}\right) }}{1 - t} = \mathop{\sum }\limits_{{n = 0}}^{\infty }{L}_{n}\left( x\right) \frac{{t}^{n}}{n!}
$$

其中 $0 < t < 1.{2.5}$ 节中关于厄米多项式的母函数的讨论将会很有帮助.

(a) 证明 ${L}_{n}\left( 0\right) = n$ ! 和 ${L}_{0}\left( x\right) = 1$ .

(b) 把 $g\left( {x, t}\right)$ 对 $x$ 求微商,证明

$$
{L}^{\prime }{}_{n}\left( x\right) - n{L}^{\prime }{}_{n - 1}\left( x\right) = - n{L}_{n - 1}\left( x\right) ,
$$

并求出前几个拉盖尔多项式.

(c) 把 $g\left( {x, t}\right)$ 对 $t$ 求微商,证明

$$
{L}_{n + 1}\left( x\right) - \left( {{2n} + 1 - x}\right) {L}_{n}\left( x\right) + {n}^{2}{L}_{n - 1}\left( x\right) = 0.
$$

(d) 现在, 证明库默尔方程可以通过推导

$$
x{L}^{\prime \prime }{}_{n}\left( x\right) + \left( {1 - x}\right) {L}^{\prime }{}_{n}\left( x\right) + n{L}_{n}\left( x\right) = 0,
$$

而求解,而且把 $n$ 与氢原子的主量子数联系起来.

(e) 定义缔合拉盖尔多项式为

$$
{L}_{n}^{k}\left( x\right) = {\left( -1\right) }^{k}\frac{{d}^{k}}{d{x}^{k}}\left\lbrack {{L}_{n + k}\left( x\right) }\right\rbrack
$$

证明. 它满足下列方程

$$
x{L}_{n}^{{k}^{ * }}\left( x\right) + \left( {k + 1 - x}\right) {L}_{n}^{{k}^{ * }}\left( x\right) + n{L}_{n}^{k}\left( x\right) = 0
$$

事实上, 可以证明这个方程就是库默尔方程. 这就是说. 氢原子的径向波函数是用缔合拉盖尔多项式表示的. [ (e) 小题是作者在勘误表中要求加上的. 一译者注]

3.23 在角动量的施温格方案中, 算符

$$
{K}_{ + } \equiv {a}_{ + }^{ \dagger }{a}_{ - }^{ \dagger }\;\text{ 和 }\;{K}_{ - } \equiv {a}_{ + }{a}_{ - }
$$

的物理意义是什么? 给出 ${K}_{ \pm }$ 的非零矩阵元.

3.24 通过把 ${j}_{1} = 1$ 和 ${j}_{2} = 1$ 相加,求出所形成的 $j = 2,1,0$ 的所有 9 个 $|j, m\rangle$ 态. 利用简化符号,以 $\pm$ , 0 分别代表 ${m}_{1,2} = \pm 1,0$ ,写出 $|j, m\rangle$ 的显式式,例如

$$
\left| {1,1\rangle = \frac{1}{\sqrt{2}}}\right| + 0\rangle - \frac{1}{\sqrt{2}}|0 + \rangle
$$

可以利用阶梯算符 ${J}_{ \pm }$ ,或递推关系以及正交性. 找一个克莱布什-戈丹系数表用来做比较,检验你的结果.

3.25 (a) 对于任意的 $j$ (整数或半奇数),求

$$
\mathop{\sum }\limits_{{m = - j}}^{j}{\left| {d}_{m{m}^{\prime }}^{\left( j\right) }\left( \beta \right) \right| }^{2}m.
$$

并核对 $j = \frac{1}{2}$ 时所得的答案.

(b) 证明,对于任意的 $j$

$$
\mathop{\sum }\limits_{{m = - j}}^{j}{m}^{2}{\left| {d}_{{m}^{\prime }m}^{\left( j\right) }\left( \beta \right) \right| }^{2} = \frac{1}{2}j\left( {j + 1}\right) {\sin }^{2}\beta + {m}^{\prime 2}\frac{1}{2}\left( {3{\cos }^{2}\beta - 1}\right) .
$$

[提示: 这可以用许多方法证明. 例如,可以利用球 (不可约) 张量语言检查 ${J}_{z}^{2}$ 的转动性质.]

3.26 (a) 考虑一个 $j = 1$ 的系统. 明确地把

$$
\left\langle {j = 1,{m}^{\prime }\left| {J}_{y}\right| j = 1, m}\right\rangle
$$

写成 $3 \times 3$ 矩阵形式.

(b) 证明,只有 $j = 1$ 时,才可以合理地用

$$
1 - i\left( \frac{{J}_{y}}{\hslash }\right) \sin \beta - {\left( \frac{{J}_{y}}{\hslash }\right) }^{2}\left( {1 - \cos \beta }\right)
$$

代替 ${e}^{-i{J}_{y}{\beta }^{i/h}}$ .

(c) 利用 (b) 证明

$$
{d}^{\left( j = 1\right) }\left( \beta \right) = \left( \begin{matrix} \left( \frac{1}{2}\right) \left( {1 + \cos \beta }\right) & - \left( \frac{1}{\sqrt{2}}\right) \sin \beta & \left( \frac{1}{2}\right) \left( {1 - \cos \beta }\right) \\ \left( \frac{1}{\sqrt{2}}\right) \sin \beta & \cos \beta & - \left( \frac{1}{\sqrt{2}}\right) \sin \beta \\ \left( \frac{1}{2}\right) \left( {1 - \cos \beta }\right) & \left( \frac{1}{\sqrt{2}}\right) \sin \beta & \left( \frac{1}{2}\right) \left( {1 + \cos \beta }\right) \end{matrix}\right) .
$$

3. 27 借助

$$
{\mathcal{D}}_{mn}^{\prime }\left( {\alpha \beta \gamma }\right) = \langle {\alpha \beta \gamma } \mid {jmn}\rangle
$$

中的级数把矩阵元 $\left\langle {{\alpha }_{2}{\beta }_{2}{\gamma }_{2}\left| {J}_{3}^{2}\right| {\alpha }_{1}{\beta }_{1}{\gamma }_{1}}\right\rangle$ 表示出来. (作者在勘误表中认为这个题应该删掉,因为 $|{jmn}\rangle$ 代表什么,并不清楚. 一译者注)

3.28 考虑由两个自旋 $\frac{1}{2}$ 的粒子组成的一个系统. 观察者 $\mathrm{A}$ 专门测量其中一个粒子的自旋分量 $\left( {s}_{1z}\right.$ , ${s}_{1, t}$ ,等等),同时观察者 B 测量另一个粒子的自旋分量. 假定已知系统处在自旋单态,即 ${S}_{\& } = 0$ .

(a) 当观察者 $\mathrm{B}$ 不做任何测量时,观察者 $\mathrm{A}$ 得到 ${s}_{1z} = \hslash /2$ 的概率是什么? 对于 ${s}_{1x} = \hslash /2$ 求解同样问题.

(b) 观察者 $\mathrm{B}$ 肯定地确认粒子 2 的自旋处于 ${s}_{2z} = \hslash /2$ 态. 如果观察者 $\mathrm{A}$ (i) 测量 ${s}_{1z}$ ; (ii) 测 ${s}_{1z}$ , 则对观察者 $\mathrm{A}$ 的测量结果能给出的结论是什么? 解释你的答案.

3.29 考虑一个秩为 1 的球张量 (即一个矢量)

$$
{V}_{\pm 1}^{\left( 1\right) } = \mp \frac{{V}_{x} \pm i{V}_{y}}{\sqrt{2}},\;{V}_{0}^{\left( 1\right) } = {V}_{z}.
$$

利用习题 3.26 给出的 ${d}^{\left( j = 1\right) }$ 的表示式,求

$$
\mathop{\sum }\limits_{{q}^{\prime }}{d}_{q{q}^{\prime }}^{\left( 1\right) }\left( \beta \right) {V}_{{q}^{\prime }}^{\left( 1\right) },
$$

并证明这结果正是从绕 $y$ 轴转动时 ${V}_{x, y, z}$ 的变换性质所预期的.

3.30 (a) 用两个不同的矢量 $\mathbf{U} = \left( {{U}_{x},{U}_{y},{U}_{z}}\right)$ 和 $\mathbf{V} = \left( {{V}_{x},{V}_{y},{V}_{z}}\right)$ 构造一个秩为 1 的球张量. 明确地用 ${U}_{x, y, z}$ 和 ${V}_{x, y, z}$ 写出 ${T}_{\pm 1,0}^{\left( 1\right) }$ .

(b) 用两个不同的矢量 $\mathbf{U}$ 和 $\mathbf{V}$ 构造一个秩为 2 的球张量. 明确地用 ${U}_{x, y, z}$ 和 ${V}_{x, y, z}$ 写出 ${T}_{\pm 2, \pm 1,0}^{\left( 2\right) }$ .

3.31 考虑一个无自旋粒子被一个中心力位势束缚于一个固定的中心.

(a) 尽可能只用维格纳-埃卡特定理建立起矩阵元

$$
\left\langle {{n}^{\prime },{l}^{\prime },{m}^{\prime }}\right\rangle \mp \frac{1}{\sqrt{2}}\left( {x \pm {iy}}\right) \left| {n, l, m\rangle \;\text{ 和 }\;\left\langle {{n}^{\prime },{l}^{\prime },{m}^{\prime }}\right\rangle z}\right| n, l, m\rangle
$$

的关系. 肯定地指出在什么样的条件下这些矩阵元非零.

(b) 利用波函数 $\psi \left( \mathbf{x}\right) = {R}_{nl}\left( r\right) {Y}_{l}^{m}\left( {\theta ,\phi }\right)$ 求解同样的问题.

3.32 (a) 把 ${xy},{xz}$ 和 $\left( {{x}^{2} - {y}^{2}}\right)$ 写成一个秩为 2 的球 (不可约) 张量的分量.

(b) 期待值

$$
Q \equiv e\left\langle {\alpha, j, m = j\left| \left( {3{z}^{2} - {r}^{2}}\right) \right| \alpha, j, m = j}\right\rangle
$$

被称为四级矩. 利用 $Q$ 和适当的克莱布什-戈丹系数,求

$$
e\left\langle {\alpha, j,{m}^{\prime }\left| \left( {{x}^{2} - {y}^{2}}\right) \right| \alpha, j, m = j}\right\rangle ,
$$

其中 ${m}^{\prime } = j, j - 1, j - 2,\cdots$ .

3.33 一个处于原点的自旋为 $\frac{3}{2}$ 的原子核受到一个外部非均匀电场的作用. 基本的电四极矩相互作用可以取为

$$
{H}_{\mathrm{{int}}} = \frac{eQ}{{2s}\left( {s - 1}\right) {\hslash }^{2}}\left\lbrack {{\left( \frac{{\partial }^{2}\phi }{\partial {x}^{2}}\right) }_{0}{S}_{x}^{2} + {\left( \frac{{\partial }^{2}\phi }{\partial {y}^{2}}\right) }_{0}{S}_{y}^{2} + {\left( \frac{{\partial }^{2}\phi }{\partial {z}^{2}}\right) }_{0}{S}_{z}^{2}}\right\rbrack ,
$$

其中 $\phi$ 是满足拉普拉斯方程的静电势,而坐标轴的选取. 使得:

$$
{\left( \frac{{\partial }^{2}\phi }{\partial x\;\partial y}\right) }_{0} = {\left( \frac{{\partial }^{2}\phi }{\partial y\;\partial z}\right) }_{0} = {\left( \frac{{\partial }^{2}\phi }{\partial x\;\partial z}\right) }_{0} = 0.
$$

证明相互作用可以写成

$$
A\left( {3{S}_{z}^{2} - {\mathbf{S}}^{2}}\right) + B\left( {{S}_{ + }^{2} + {S}_{ - }^{2}}\right) .
$$

并借助于 ${\left( {\partial }^{2}\phi /\partial {x}^{2}\right) }_{0}$ 等表示 $A$ 和 $B$ . (借助 $|m\rangle$ ,其中 $m = \pm \frac{3}{2}, \pm \frac{1}{2}$ ) 确定能量本征右矢和相应的能量本征值. 有简并存在吗?



	
	
	
	
	
\ifx\allfiles\undefined
\end{document}
	\else
	\fi
