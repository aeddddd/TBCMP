\ifx\allfiles\undefined

% 如果有这一部分另外的package,在这里加上
% 没有的话不需要

\begin{document}
	\else
	\fi
\chapter{李群(Lie Group)与李代数(Lie Algebra)}
\begin{introduction}
	\item Lie群与Lie代数
	\item Casimir算子
	\item 张量
	\item Lie代数的表示
\end{introduction}
这一章大部分内容聚焦在群论,尤其是李群李代数部分,事实上,恐怕到最后的章节也不会继续大篇幅的讲数学内容了,这些数学已经足够支撑你初步的了解这门学科.

虽然这一章的主要内容是数学,但会逐步把相对应的物理内容穿插进来,这样既有助于加深印象,也有助于构建所谓``物理图像".
\section{李群(Lie Group)初步}
\subsection{群与李群(Lie Group)}
我们或许听说过一个说法:``物理学的关键是对称和守恒",而诺特定理给出了对称性与守恒性的联系,例如,时间平移不变性意味着能量守恒;空间平移不变性意味着动量守恒;转动不变性意味着角动量守恒;电势和向量势的规范不变性得出电荷的守恒等.而描述对称性的语言就是群论.

我们可以认为群是一类拥有特殊结构的集合,即满足如下关系的集合\footnote{当然,这里会忽略对于主线无用的那些群论内容,所以如果和数学系的抽象代数对比,你甚至可能会感觉到学的不是同一个东西}:
\begin{definition}[群的定义]
	设$G$是一个集合,若满足下面4个条件,则称$G$为一个群(Group)
	\begin{enumerate}
		\item $G$中存在一种运算规则,对$G$中的任意两个元素$g,h\in G$,存在对应$G$中的一个元素,记为
		\begin{equation}
			k=g\circ h(k=gh)
		\end{equation}
		\item 运算规则满足交换律,对$G$中的任意三个元素$g,h,k\in G$,存在
		\begin{equation}
			(gh)k=g(hk)
		\end{equation}
		\item $G$中存在一个幺元$e$(有时也称单位元),使得对于$G$中任意元素$g$,均有
		\begin{equation}
			ge=eg=g
		\end{equation}
		\item $G$中每一个元素$g$,均存在一逆元$g^{-1}$,使得
		\begin{equation}
			gg^{-1}=g^{-1}g=e
		\end{equation}
	\end{enumerate}
\end{definition}
我们可以发现,群的运算规则通常不满足交换律,特殊的,我们把满足交换律的群称为阿贝尔群(Abel Group)\footnote{关于这个有一个经典笑话:一位美国数学教授来到法国,见路边有一小孩,遂上前问到:“小朋友,你知道1+2等于几吗?”小孩摇摇头说:“不知道.”教授正想感叹法国数学教育如此之落后,却听到小孩接着说:“虽然我不知道1+2等于几,但是我知道1+2等于2+1,因为整数加法群是阿贝尔群!”}.
\begin{definition}[子群的定义]
	设$G$是一个群,$H$为$G$的一个子集($H\subseteq G$),若$H$按照$G$的运算规则仍是一个群,则称$H$是$G$的子群.
\end{definition}
\begin{example}
	全体实数$\mathbb{R}$(或复数$\mathbb{C}$),对加法构成一个阿贝尔群.\\
	我们知道有理数全体是$\mathbb{R}$的子群,而全体偶数是有理数的子群,自然也是$\mathbb{R}$的子群,那么存在一个问题:无理数全体,或奇数全体是否是$\mathbb{R}$的子群?
\end{example}
\begin{solution}
	都不是,首先对于无理数我们注意到$\pi+(-\pi)=0$,而0不是无理数,故无理数不构成加法群.同样的,我们注意到$1+(-1)=0$,0同样也不是奇数,故奇数也不构成加法群.
\end{solution}
\begin{example}
	全体实数除去零$\mathbb{R}/0$或全体复数除去零$\mathbb{C}/0$对乘法构成阿贝尔群.\\
	同样的,我们有个问题:为什么要除去0?
\end{example}
\begin{solution}
	答案是显然的,群中幺元为1,但$0/0$无意义.
\end{solution}
\begin{example}
	$G =\{1,-1,i,-i\}$对复数乘法运算构成一有限阿贝尔群.这里1是$G$的幺元,而-1的逆元就是-1,$i$与$-i$互为逆元.
\end{example}
\begin{example}
	行列式不为零的n阶实矩阵全体对矩阵乘法构成一个群,$n$阶全线性群,其记为$GL(n,R)$,它的元素由$n^2$个独立实参数所确定.其是一个$n^2$维(不可交换)李群,在后面我们会再次讨论它.
\end{example}
\begin{example}
	行列式为 1 的 2 阶实矩阵全体对矩阵乘法构成一个群:二阶(实)特殊线性群$SL(2,R)$.因为二阶实矩阵$\begin{bmatrix}a&b\\c&d\end{bmatrix}$由四个实数$a,b,c,d$ 构成,由于行列式为1的要求,使他们必须满足条件:$ad-bc=1$.所以$SL(2,R)$ 中的元素由3个独立的实参数所确定.按照下面将要给出的定义可见$SL(2,R)$ 是一个三维(不可交换)李群,而且它是$GL(2,R)$的子群.
\end{example}
\begin{example}
	行列式为 1 的 n 阶实矩阵全体对矩阵乘法构成一个群:n阶特殊线性群$SL(n,R)$,这是一个$n^2-1$维(不可交换)李群,而且它是$GL(n,R)$的子群.
\end{example}
\begin{example}
	行列式不为0的 n 阶复矩阵全体对矩阵乘法构成一个群:n阶(复)全线性群$GL(n,C)$,行列式为 1 的 n 阶复矩阵全体对矩阵乘法构成一个群:n阶(复)特殊线性群$SL(n,C)$.\\
	$GL(n,C)$是一个$2n^2$维(不可交换)李群,$SL(n,C)$是一个$2n^2-12$维(不可交换)李群.
\end{example}

我们发现,所举的例子(除第一个外)都存在共同点:元素都是矩阵(实数和复数可看作一阶矩阵),群的运算法则都是矩阵乘法.我们把这类群统称为\textbf{线性群},线性群也是最具代表性的一类李群,今后所使用的李群基本上都是线性群.

\begin{remark}
	\textit{事实上,从现在开始,我们就已经走上物理的道路上了,实际上,哪怕你掌握了这一章的全部内容,可能对于数学上的抽象代数那一套仍非常陌生,但早已足够应付物理上的内容了.在上一段,我们给出了一个断言:``今后所使用的李群基本上都是线性群",实际上,我们完全可以这么说,如果不去碰高能和那些fancy的理论(例如弦论,Ads/CFT等),哪怕仅掌握$U(1),SU(2),SU(4),SO(2),SO(3)$这几个群和其表示论就足够了.}
\end{remark}
下面我们正式进入李群这一部分的内容.
\begin{definition}[Lie群的定义]
	设$G$是一个$r$维流形,同时$G$又是一个群,并将其幺元记为$e$,因$e$又是流形$G$中的一点,所以可取定一个包含$e$的局部坐标邻域$U$;在$U$中取定坐标系$\{U,\varphi\}$.设取$e$为坐标原点,有
	\begin{equation}
		\varphi(e)=(0,0,\cdots,0)
	\end{equation}
	任取$U$中的三个元素$g,h,k$,并设其坐标为
	\begin{equation}
		\begin{aligned}
			\varphi(g)&=(x_1,x_2,\cdots,x_r)\\
			\varphi(h)&=(y_1,y_2,\cdots,y_r)\\
			\varphi(k)&=(z_1,z_2,\cdots,z_r)
		\end{aligned}
	\end{equation}
	而群乘法$k=gh$则可以被定义为以下形式:
	\begin{equation}
		\begin{aligned}
			&z_{1}=f_{1}(x_{1},\cdots,x_{r};y_{1},\cdots,y_{r})\\
			&z_{2}=f_{2}(x_{1},\cdots,x_{r};y_{1},\cdots,y_{r})\\
			&z_{r} =f_r(x_1,\cdots,x_r;y_1,\cdots,y_r) 
		\end{aligned}
	\end{equation}
	我们要求这$r$个函数$f_1,f_2,\cdots,f_r$是无限次可导的(光滑的).我们把这$r$个函数$f_1,f_2,\cdots,f_r$称为$G$的\textbf{乘法函数}.其完全确定了群$G$的结构.我们把这样的群$G$叫做一个$r$维李群.
\end{definition}
我们现在根据群的定义来给出几个自然性质
\begin{enumerate}
	\item 第一个定义是显然的,因为李群的定义建立在这种运算规则上,我们只需要对另外3个条件进行讨论.
	\item 我们现在给出交换律所导出的性质,为方便表述,我们简记群乘法关系为$z=f(x,y)$:
	\begin{equation}
		f(f(x,y),z)=f(x,f(y,z))
	\end{equation}
	\item 对于幺元$e$,其坐标为$(0,0,\cdots,0)$,所以有$ex=xe=x$.
	\begin{equation}
		f(x,0)=f(0,x)=x
	\end{equation}
	\item 对于逆元$g^{-1}$,我们设其坐标为$(\tilde{x}_1,\cdots\tilde{x}_r)$,于是有
	\begin{equation}
		f(x,\tilde{x})=f(\tilde{x},x)=0
	\end{equation}
\end{enumerate}
我们很容易看出,乘法函数是很抽象的,只有乘法函数来研究李群往往是无处下手的(更何况我们是学物理的),于是有了李代数的理论.不过在展开李代数之前,我们使用几个实际的李群的例子来帮助建立对于李群的理解.
\begin{example}
	$T_2=\Big\{\begin{bmatrix}e^{x_1}&x_2\\0&1\end{bmatrix}\Big|x_1,x_2\in\mathbb{R}\Big\}$.这个群的元素由两个独立实参数$x_1,x_2$决定.所以,$T_2$是一个二维流形\footnote{流形:一句话来表述是将一个空间的局部近似为一个欧氏空间,我们把这个欧氏空间称为流形(manifold),你可以把它当做一种空间.}.我们现在来逐个验证其满足群的要求.
\end{example}
\begin{solution}
	\begin{enumerate}
		\item 首先我们验证其封闭性
		\begin{equation}
			\begin{gathered}\begin{bmatrix}e^{x_1}&x_2\\0&1\end{bmatrix}\begin{bmatrix}e^{y_1}&y_2\\0&1\end{bmatrix}=\begin{bmatrix}e^{x_1}e^{y_1}&e^{x_1}y_2+x_2\\0&1\end{bmatrix}\\=\begin{bmatrix}e^{x_1+y_1}&e^{x_1}y_2+x_2\\0&1\end{bmatrix}=\begin{bmatrix}e^{z_1}&z_2\\0&1\end{bmatrix}\in T_2\end{gathered}
		\end{equation}
		并同时写出其乘法函数,不难发现其乘法函数是无限次可微的.
		\begin{equation}
			\begin{aligned}&z_{1}=f_{1}(x_{1},x_{2};y_{1},y_{2})=x_{1}+y_{1},\\&z_{2}=f_{2}(x_{1},x_{2};y_{1},y_{2})=e^{x_{1}}y_{2}+x_{2}.\end{aligned}
		\end{equation}
		\item $T_2$的乘法运算为矩阵乘法,自然满足结合律.
		\item 对于幺元,我们注意到
		\begin{equation}
			\begin{bmatrix}e^0&0\\1&1\end{bmatrix}=\begin{bmatrix}1&0\\0&1\end{bmatrix}
		\end{equation}
		\item 我们注意到有
		\begin{equation}
			\begin{aligned}\begin{bmatrix}e^{-x_1}&-x_2e^{-x_1}\\0&1\end{bmatrix}\begin{bmatrix}e^{x_1}&x_2\\0&1\end{bmatrix}&=\begin{bmatrix}e^{x_1}&x_2\\0&1\end{bmatrix}\begin{bmatrix}e^{-x_1}&-x_2e^{-x_1}\\0&1\end{bmatrix}\\&=\begin{bmatrix}1&0\\0&1\end{bmatrix}\end{aligned}
		\end{equation}
		所以逆元为$\begin{bmatrix}e^{-x_1}&-x_2e^{-x_1}\\0&1\end{bmatrix}$并容易验证其不满足交换律.
	\end{enumerate}
\end{solution}
\begin{example}
	我们的下一个实例是绕定轴转动的旋转群SO(2),显然,我们只需要一个变量(转动角$\theta$)就可以表述一个转动变换,所以我们表示群元为$g(\theta)$,其中$\theta$的取值范围是$[0,2\pi)$.而群的运算法则可以被规定为相继的两个转动,即转动角相加,但需要保持转动角始终在取值范围内.我们可以使用公式表达:
	\begin{equation}
		g(\theta_1)g(\theta_2)=g(\theta_{12}),\qquad\theta_{12}=(\theta_1+\theta_2)\mod 2\pi
	\end{equation}
	我们容易验证其满足对应的4条性质.不过我们在关于线性代数的学习中,我们知道:我们也可以使用旋转矩阵来表述定轴转动.
	\begin{equation}
		\begin{bmatrix}x\\y\end{bmatrix}\xrightarrow{g(\theta)}\begin{bmatrix}x'\\y'\end{bmatrix}=\begin{bmatrix}\cos\theta&-\sin\theta\\\sin\theta&\cos\theta\end{bmatrix}\begin{bmatrix}x\\y\end{bmatrix}=\begin{bmatrix}x\cos\theta-y\sin\theta\\x\sin\theta+y\cos\theta\end{bmatrix}
	\end{equation}
	我们发现,旋转矩阵是SO(2)群的群元.我们在下一个例子会更加深入讨论这部分内容.
\end{example}
\begin{example}
	现在我们需要讨论三维旋转群SO(3),其群元表示三维空间中绕某个固定点的一个转动$g\in SO(3)$,为了方便表述SO(3),我们使用如图所示的欧拉(Euler)角$(\alpha,\beta,\gamma)$来表示一个转动.
	\begin{figure}[tbph]
		\centering
		\includegraphics[width=0.4\linewidth]{"figure/Euler angle"}
		\caption{}
		\label{fig:euler-angle}
	\end{figure}\\
	我们依次写出绕$z$轴旋转$\alpha$角;绕$x$轴旋转$\beta$角;绕$z$轴旋转$\gamma$角的三个群元的矩阵表示:
	\begin{equation}
		g_z^\alpha={\begin{bmatrix}\cos \alpha &-\sin \alpha &0\\\sin \alpha &\cos \alpha &0\\0&0&1\end{bmatrix}};g_x^\beta={\begin{bmatrix}1&0&0\\0&\cos \beta &-\sin \beta \\0&\sin \beta &\cos \beta \end{bmatrix}};g_z^\gamma={\begin{bmatrix}\cos \gamma &-\sin \gamma &0\\\sin \gamma &\cos \gamma &0\\0&0&1\end{bmatrix}}
	\end{equation}
	我们给出最终群元的表示:
	\begin{equation}
		g=g_z^\alpha g_x^\beta g_z^\gamma={\begin{bmatrix}\cos \alpha \cos \gamma -\cos \beta \sin \alpha \sin \gamma &-\cos \alpha \sin \gamma -\cos \beta \sin \alpha \cos \gamma &\sin \beta \sin \alpha \\\sin \alpha \cos \gamma +\cos \beta \cos \alpha \sin \gamma &-\sin \alpha \sin \gamma +\cos \beta \cos \alpha \cos \gamma &-\sin \beta \cos \alpha \\\sin \beta \sin \gamma &\sin \beta \cos \gamma &\cos \beta \end{bmatrix}}
	\end{equation}
	因此,SO(3)的元素可以通过三个独立参量$\alpha,\beta,\gamma$来确定,因此不难验证SO(3)是一个三维李群.
	
	现在我们给出另一种表述SO(3)的方法.\\
	我们对于两个矢量$x=(x_1,x_2,x_3),y=(y_1,y_2,y_3)$给出三维欧式空间$\mathbb{R}_3$的内积:
	\begin{equation}
		\langle x,y\rangle=\sum_{j=1}^{3}x_j y_j=x_1y_1+x_2y_2+x_3y_3
	\end{equation}
	我们定义一个线性变换算符$g=(g_{ij})$,存在关系
	\begin{equation}
		x\xrightarrow{g}x'=gx=\begin{bmatrix}g_{11}&g_{12}&g_{13}\\g_{21}&g_{22}&g_{23}\\g_{31}&g_{32}&g_{33}\end{bmatrix}\begin{bmatrix}x_1\\x_2\\x_3\end{bmatrix}\quad g\in SO(3)\Leftrightarrow\langle gs,gy\rangle=\langle x,y\rangle\quad\forall x,y\in\mathbb{R}_3\quad\text{且}\det g>0
	\end{equation}
	而且我们发现$\langle gs,gy\rangle=\langle x,g^Tgy\rangle$,其中$g^T$表示$g$的转置,即$g_{ij}^T=g_{ij}$,我们根据刚才所给出的关系发现$\langle x,g^Tgy\rangle=\langle x,y\rangle$,即$g^Tg=\textbf{1}$,我们把满足关系$g^Tg=gg^T$的线性变换构成的群称为正交群.
	
	对于3阶矩阵$g$,存在9个元素,但为了满足特殊正交群的特殊性($\det g=1$)和正交性($g^Tg=gg^T$),共有6个方程需要满足.所以,我们可以拿出3个作为独立参数,这再次证明了$g$可以表述SO(3)这个三维李群.
\end{example}
\subsection{指数映射与对数映射}
在前面的部分,我们强调了群的乘法一般不可交换,这直接导致了刻画李群的乘法函数变得非常复杂,这意味着想通过研究乘法函数来研究李群是不现实的.李群之所以能够区分其他群结构的一大主要原因
\subsection{单参数子群}
1
\section{李群与李代数}
\subsection{李氏三定理}
1
\subsection{典型李群和李代数}
1
\section{李代数(Lie Algebra)}
\subsection{基,伴随算符,内积}
1
\subsection{正交补空间}
1
\subsection{李代数的结构}
1
\section{卡西米尔算符(Casimir operator)}
\section{张量}
\subsection{定义,分类,性质}
1
\subsection{不可约张量的分解}
1
\section{李群和李代数的表示及其约化}
	
\section{李代数应用的物理实例}
	
\ifx\allfiles\undefined
\end{document}
	\else
	\fi
