\setchapterstyle{kao}
%\setchapterpreamble[u]{\margintoc}
\chapter{引用}
\labch{引用}

\section{Citations}

\index{citations}

引用信息 \sidecite{Visscher2008,James2013} 是非常简单的:只需使用 \Command{sidecite}\index{\Command{sidecite}} 命令。它目前还没有偏移参数,但未来可能会有。这个命令支持多个条目,正如你所看到的,默认情况下,它在边缘打印引用,并将其添加到文档末尾的参考书目中。请注意,这些引用与文本无关,\sidecite{James2013} 但它们完全是随机的,因为它们仅用于说明该功能。

为了这个设置,我编写了一个单独的包, \Package{kaobiblio} ,你可以在 \Package{styles} 目录中找到并包含在你的主tex文件中。这个包接受所有可以传递给 \Package{biblatex} 的选项,实际上它在后台将这些选项传递给 \Package{biblatex}。此外,它还定义了一些命令,例如 \Command{sidecite},以及可以在 \Class{kao} 书籍中使用的环境。\sidenote[][-.9cm]{因此,您应该始终使用 \Package{kaobiblio} 而不是 \Package{biblatex},但语法和选项完全相同。}

如果您想使用 \Package{bibtex} 而不是 \Package{biblatex},请将选项 \Option{backend=bibtex} 传递给 \Package{kaobiblio}。\Package{kaobiblio} 还支持两个与 \Package{biblatex} 不共享的选项:\Option{addspace} 和 \Option{linkeverything},这两个选项都是布尔选项,意味着它们可以取值为“true”或“false”。如果您在加载 \Package{kaobiblio} 时传递 \Option{addspace=true},则会在引用标记之前自动添加一个空格。如果您传递 \Option{linkeverything=true},则在 authoryear- 和 authortitle- 样式中的作者姓名将成为一个超链接,就像年份一样。\sidenote{作者姓名不是超链接这一事实让不止一个 biblatex 用户感到困扰。有一些\href{https://github.com/plk/biblatex/issues/428}{强有力的论据}\emph{反对}将作者姓名超链接,但在我个人看来,在大多数实际情况下,将作者姓名链接并不会导致任何问题。}

如您所见,\Command{sidecite} 命令将在边缘打印引用。然而,如果没有自定义引用格式的方法,这个命令将毫无用处,因此 \Class{kaobook} 还提供了 \Command{formatmargincitation} 命令。通过\enquote{更新}这个命令,您可以选择哪些项目将在边缘打印。理解它如何工作的最佳方法是查看该命令的实际定义。


\begin{lstlisting}[breaklines,columns=flexible,style=kaolstplain,linewidth=1.5\textwidth]
\newcommand{\formatmargincitation}[1]{%
	\parencite{#1}: \citeauthor*{#1} (\citeyear{#1}), \citetitle{#1}%
}
\end{lstlisting}

\newpage
%在页尾引用超行了,好像处理得不好。添加newpage强制换页,临时解决这个问题。

因此,\Command{formatmargincitation} 接受一个参数,即引用键,并打印出 parencite 后跟冒号,然后是作者,然后是年份(用括号表示),最后是标题。 \sidecite{Battle2014} 现在,假设您希望边缘引用显示年份和作者,后跟标题,最后是一个固定的任意字符串;您需要在文档中添加:


\begin{lstlisting}[breaklines,columns=flexible,style=kaolstplain,linewidth=1.5\textwidth]
\renewcommand{\formatmargincitation}[1]{%
	\citeyear{#1}, \citeauthor*{#1}: \citetitle{#1}; very interesting!%
}
\end{lstlisting}

\renewcommand{\formatmargincitation}[1]{%
	\citeyear{#1}, \citeauthor*{#1}: \citetitle{#1}; very interesting!%
}

上述代码生成的引用看起来如下所示:\sidecite{Zou2005}。当然,当您更改默认的参考文献样式时,更改格式是最有用的。例如,如果您想在参考文献中使用“philosophy-modern”样式,您可能在前言中会有如下内容:

\begin{lstlisting}[breaklines,columns=flexible,style=kaolstplain,linewidth=1.5\textwidth]
\usepackage[style=philosophy-modern]{styles/kaobiblio}
\renewcommand{\formatmargincitation}[1]{%
	\sdcite{#1}%
}
\addbibresource{main.bib}
\end{lstlisting}


\renewcommand{\formatmargincitation}[1]{%
	\parencite{#1}: \citeauthor*{#1} (\citeyear{#1}), \citetitle{#1}%
}

像 \Command{citeyear}、\Command{parencite} 和 \Command{sdcite} 这样的命令只是示例。有关可用命令的完整参考可以在这个 \href{http://tug.ctan.org/info/biblatex-cheatsheet/biblatex-cheatsheet.pdf}{cheatsheet}中找到,位于\enquote{引用}部分。

最后,要编译一个包含引用的文档,您需要使用一个外部工具,对于这个类来说是 biber。您需要运行以下命令(假设您的 tex 文件名为 main.tex):

\begin{lstlisting}[style=kaolstplain]
$ pdflatex main
$ biber main
$ pdflatex main
\end{lstlisting}

\section{Glossaries and Indices}

\index{glossary}

\Class{kaobook} 类加载了 \Package{glossaries} 和 \Package{imakeidx} 包,您可以使用它们为您的书籍添加词汇表和索引。例如,我之前定义了一些词汇条目,现在我将像这样使用它们:\gls{computer}。 \Package{glossaries} 还允许您使用缩略语,例如以下内容:这是完整版本,\acrfull{fpsLabel},这是简短版本 \acrshort{fpsLabel}。这些条目将在附录中的词汇表中出现。

除非您使用 \href{https://www.overleaf.com}{Overleaf} 或其他一些高级的 \LaTeX IDE,否则您需要从终端运行外部命令,以便编译带有词汇表的文档。特别是,需要的命令是:\sidenote{这些是在 UNIX 系统中运行的命令,但请参见 \nrefsec{compiling}; 我 对Windows 的工作方式还没有想法。}

\begin{lstlisting}[style=kaolstplain]
$ pdflatex main
$ makeglossaries main
$ pdflatex main
\end{lstlisting}

注意:您不需要每次编译文档时都运行 \texttt{makeglossaries},而仅在更改词汇表条目时运行。

\index{index} 

要创建索引,您需要在文本中提到“主题”时插入命令 \lstinline|\index{subject}|。例如,在本段的开头,我会写 \lstinline|index{index}|,这样就会在附录中添加一个条目到索引。看看吧!

\marginnote[2mm]{理论上,您也需要为索引运行外部命令,但幸运的是,我们建议的包 \Package{imakeidx} 可以自动编译索引。}

\index{nomenclature}

命名法是一种特殊类型的索引;你可以在这本书的末尾找到一个。要插入命名法,我们使用 \Package{nomencl} 包,并通过 \Command{nomenclature} 命令添加术语。然后我们在希望其出现的位置放置 \Command{printnomenclature}。

此外,使用此包时,我们需要运行一个外部命令来编译文档,否则命名法将不会出现:

\begin{lstlisting}[style=kaolstplain]
$ pdflatex main
$ makeindex main.nlo -s nomencl.ist -o main.nls
$ pdflatex main
\end{lstlisting}

这些包都加载在 \href{style/packages.sty}{packages.sty} 中,这是与此类文件一起提供的文件之一。然而,元素的配置最好在 main.tex 文件中进行,因为每本书将有不同的条目和样式。

请注意,\Package{nomencl} 包在编译文档时造成了问题,因此,为了简化问题,我不得不防止 \Package{scrhack} 加载 \Package{nomencl} 的黑客文件。然而,在 Overleaf 上编译文档时,这个问题似乎消失了。

\marginnote[-19mm]{这一简短的部分绝不是该主题的完整参考,因此您应该查阅上述包的文档,以全面理解它们的工作原理。}



\section{Hyperreferences}
\labsec{hyprefs}

\index{hyperreferences}

为了这个类,我们提供了一个方便的包,帮助您始终以相同的方式引用相同的元素,以确保书籍的一致性。首先,您可以使用特定命令为每个元素标记。例如,如果您想标记一个章节,您可以在 \Command{chapter} 指令后面放置 \lstinline|\labch{chapter-title}|。这只是为了方便,因为 \Command{labch} 实际上只是 \lstinline|\label{ch:chapter-title}| 的别名,因此省去了您写\enquote{ch:}的麻烦。我们为许多通常被标记的元素定义了类似的命令,包括:

\begin{multicols}{2}
\setlength{\columnseprule}{0pt}
\begin{itemize}
	\item Page: \Command{labpage}
	\item Part: \Command{labpart}
	\item Chapter: \Command{labch}
	\item Section: \Command{labsec}
	\item Figure: \Command{labfig}
	\item Table: \Command{labtab}
	\item Definition: \Command{labdef}
	\item Assumption: \Command{labassum}
	\item Theorem: \Command{labthm}
	\item Proposition: \Command{labprop}
	\item Lemma: \Command{lablemma}
	\item Remark: \Command{labremark}
	\item Example: \Command{labexample}
	\item Exercise: \Command{labexercise}
\end{itemize}
\end{multicols}

当然,我们有类似的命令来引用这些元素。
然而,由于引用的样式应依赖于上下文,我们提供不同的命令来引用相同的内容。例如,在某些情况下,您可能希望按名称引用章节,但在其他时候,您只想按数字引用它。一般来说,有四种引用样式,我们称之为普通、变体、名称和完整。

普通样式仅通过数字进行引用。对于章节,可以使用 \lstinline|\refch{chapter-title}| 访问(对于其他元素,语法是类似的)。这样的引用结果为:\refch{引用}。

vario和name样式基于\Package{varioref}包。
它们的语法是\lstinline|\vrefch{chapter-title}|和
\lstinline|\nrefch{chapter-title}|,结果为:
\vrefch{引用},对于vario样式,以及:\nrefch{引用},对于name样式。正如您所看到的,页面在\Package{varioref}样式中被引用。

完整样式引用所有内容。您可以使用它与
\lstinline|\frefch{chapter-title}|,看起来像这样:
\frefch{引用}。

Of course, all the other elements have similar commands (\eg for parts 
you would use \lstinline|\vrefpart{part-title}| or something like that). 
However, not all elements implement all the four styles. The commands 
provided should be enough, but if you want to see what is available or 
to add the missing ones, have a look at the 
\href{styles/kaorefs.sty}{attached package}. 

当然,所有其他元素都有类似的命令(例如,对于部分,你可以使用 \lstinline|\vrefpart{part-title}| 或类似的命令)。然而,并不是所有元素都实现了所有四种样式。提供的命令应该足够,但如果你想查看可用的内容或添加缺失的内容,可以查看附带的包 \href{styles/kaorefs.sty}{attached package}。

为了访问所有这些功能,\Package{kaorefs} 应该在文档的前言中加载。它应该最后加载,或者至少在 \Package{babel}(或 \Package{polyglossia})和 \Package{plaintheorems}(或 \Package{mdftheorems})之后加载。可以像对待任何其他包一样向它传递选项;特别是,可以指定标题的语言。例如,如果你指定\enquote{italian}作为选项,那么它将打印\enquote{Capitolo},这是意大利语的\enquote{Chapter}对应词。如果你知道其他语言,欢迎贡献这些标题的翻译!如需更多详细信息,请随时联系类的作者。

\Package{kaorefs} 包还包括 \Package{cleveref},因此可以使用 \Command{cref},作为所有之前描述的引用命令的补充。

\section{A Final Note on Compilation}
\labsec{compiling}

可能编译 LaTeX 文档最简单的方法是使用 \Package{latexmk} 脚本,因为如果正确配置,它可以处理从参考文献到术语表的所有内容。一般来说,发出的命令是:

\begin{lstlisting}
latexmk [latexmk_options] [filename ...]
\end{lstlisting}

\Package{latexmk} 可以进行广泛的配置(参见 \url{https://mg.readthedocs.io/latexmk.html})。为了方便起见,我在这里打印一个示例配置,该配置将涵盖上述所有步骤。

\begin{lstlisting}
# By default compile only the file called 'main.tex'
@default_files = ('main.tex');

# Compile the glossary and acronyms list (package 'glossaries')
add_cus_dep( 'acn', 'acr', 0, 'makeglossaries' );
add_cus_dep( 'glo', 'gls', 0, 'makeglossaries' );
$clean_ext .= " acr acn alg glo gls glg";
sub makeglossaries {
   my ($base_name, $path) = fileparse( $_[0] );
   pushd $path;
   my $return = system "makeglossaries", $base_name;
   popd;
   return $return;
}

# Compile the nomenclature (package 'nomencl')
add_cus_dep( 'nlo', 'nls', 0, 'makenlo2nls' );
sub makenlo2nls {
    system( "makeindex -s nomencl.ist -o \"$_[0].nls\" \"$_[0].nlo\"" );
}
\end{lstlisting}

然而,如果您不想使用外部包并希望手动完成所有操作,这里有一些提示。\sidenote{由于作者只使用Linux并从命令行编译所有内容,因此他不知道Windows或Mac中的编译是如何工作的。因此,这些提示仅涉及在Linux命令行中的使用。}

\minisec{编译kaobook库中的示例}
要编译位于GitHub上kaobook库的\Path{examples}目录中的示例,特别是文档,请按以下步骤操作。首先,使用命令\lstinline[language=bash]|cd|进入库的根目录,然后运行
\lstinline|pdflatex -output-directory examples/documentation main.tex|。通过这个技巧,您可以使用与该库相关的类文件来编译文档(而不是,例如,您texmf树中的那些)。选项\enquote{-output-directory}也适用于其他与\LaTeX\ 相关的命令,如biber和makeglossaries。

警告说明: 有时 \LaTeX\ 需要多次运行才能正确定位每个元素;这尤其适用于浮动元素的定位,如图形、表格和边注。偶尔,\LaTeX\ 可能需要多达四次重新运行,因此如果边缘元素的对齐看起来奇怪,或者它们渗入主文本中,请尝试再运行一次 pdflatex。 